\documentclass[7pt]{article}

\usepackage{czech}
\usepackage[utf8]{inputenc}   % pro unicode UTF-8
\usepackage{amsmath}
\usepackage{xypic}

\addtolength{\topmargin}{-1in}
\addtolength{\textheight}{1.75in}

\begin{document}

  \section{Poznámky}
  
    \subsection{Vstup algoritmu}
      \paragraph{}$\Lambda$ artinovská algebra nad komutativním tělesem $\mathbb{K}$
      \paragraph{}X $\in$ mod($\Lambda$) nerozložitelný, konečně generovaný
      \paragraph{} V našem případě jde o K-Algebru cest KQ nad konečným quieverem Q. 
      Quiver je konečný, jde tedy o algebru s jednotkou. 
      \paragraph{} Pokud je Q bez cyklů, pak budeme pracovat s algebrou $\Lambda=KQ$.
       \paragraph{} Pokud ale cykly obsahuje, pak využijeme algebru $\Lambda=KQ/I$, 
       kde I je nějaký přípustný (admissible) ideál KQ. Tím zajistíme, že $Q_\Lambda=Q$ 
       stejně jako v případě quiveru bez cyklů.
       \paragraph{} V obou případech je $\Lambda$ je konečně dimensionální 
       a "hereditary" (podmodul projektivního modulu je projektivní) algebra nad $\mathbb{K}$. 
      \paragraph{QPA příklad 1}
\begin{verbatim}
Q := Quiver(2, [[1, 2, "a"], [2, 1, "b"],[1, 1, "c"]]); 
KQ := PathAlgebra(Rationals,Q);    
matrices := [ 
    ["a", [[1,0,0],[0,1,0]]], 
    ["b", [[0,1],[1,0],[0,1]]], 
    ["c", [[0,0],[1,0]]] 
  ]; 
mX := RightModuleOverPathAlgebra(KQ,matrices);
 \end{verbatim}
     \paragraph{QPA příklad 2}
\begin{verbatim}
Q:= Quiver(3, [[1,2,"a"], [1,2,"b"], [2,2,"c"], [2,3,"d"], [3,1,"e"]]);
KQ:= PathAlgebra(Rationals, Q);
gen:= GeneratorsOfAlgebra(KQ);
a:= gen[4];
b:= gen[5];
c:= gen[6];
d:= gen[7];
e:= gen[8];
rels:= [d*e, c^2,a *c*d-b*d, e*a];
I:= Ideal(KQ,rels);
A:= KQ/I;
matrices := [ 
    ["a", [[1,0,0],[0,1,0]]], 
    ["b", [[0,1,1],[1,0,1]]], 
    ["c", [[1,1,1],[0,0,0],[1,1,0]]], 
    ["d", [[3],[2],[1]]], 
    ["e", [[3,2]]] 
  ];
mX := RightModuleOverPathAlgebra(KQ,matrices);
 \end{verbatim}

    \subsection{Výstup algoritmu}
      \paragraph{Generátor} $0\rightarrow DTr(X) \rightarrow E \rightarrow X \rightarrow 0$ 
      množiny $\hat{\Upsilon}_{DTr(X),X}$ všech téměř štěpitelných posloupností končících v X.
    
    
    \subsection{Značení v kódu QPA} 
      \paragraph{} Moduly budu značit velkým písmenem. Dále jim budu přidávat prefix "m", jelikož například "X" je 
      rezervovaný znak a také budu ihned jasné, že jde o modul (například snadno rozeznáme $\Lambda$ jako algebru a modul).
      \paragraph{} Morfismy budou malými písmeny.
      \paragraph{} Naši algebru $\Lambda$ budu značit jednoduše "A".
      \paragraph{} Řecká písmena budu psát jejich opisem latinkou (používaným v 
      TEXu). Např. rho, pi, omega,  mu, nu  ... .
 
   
    \subsection{Výpočet $\Omega$}
      \paragraph{Teorie} 
        Spočteme projektivní pokrytí $P_0$ modulu X \\* 
        \centerline{$Ker(t)\xrightarrow{i} P_{0}\xrightarrow{t}X$}
        a položíme $\Omega:=Ker(t)$. Máme krátkou exaktní 
        posloupnost \\*
        \centerline{$0\rightarrow\Omega\xrightarrow{i} P_{0}\xrightarrow{t}X\rightarrow0$}
        dále označme $P_1$ projektivní pokrytí $\Omega$. Dostaneme minimální 
        projektivní prezentaci modulu X \\*
        \centerline{ \xymatrix{
          P_1 \ar@{->}[rd]_w \ar@{->}[r]^{s=iw} &P_0 \ar@{->}[r]^t &X  \ar@{->}[r] &0 \\
          &\Omega \ar@{->}[u]_i
        } }

      \paragraph{QPA}
\begin{verbatim}
t      := ProjectiveCover(mX);
mP0    := Source(t);
mOmega := Kernel(t);
omega  := ProjectiveCover(mOmega);
i      := KernelInclusion(t);
s      := omega * i;
mP1    := Source(omega);
\end{verbatim}

    \subsection{Modul $Tr(X)$}
      \paragraph{Teorie} 
      \subparagraph{} $mod(\Lambda) \subset Mod(\Lambda)$ podkategorie konečně 
      generovaných modulů
      \subparagraph{} $P_\Lambda(M,N):=\{f\in Hom(M,N)|$ f se faktorizuje skrze 
      projektvní modul \}
      \subparagraph{} $\underline{Hom}_\Lambda(M,N) = Hom_\Lambda(M,N) / 
      P_\Lambda(M,N)$
      \subparagraph{} $\underline{mod}(\Lambda)$ kategorie definovaná: \\*
        \centerline{$Ob(\underline{mod}(\Lambda))=Ob(mod(\Lambda))$} \\*
        \centerline{$Hom_{\underline{mod}(\Lambda)}(A,B)=\underline{Hom}_\Lambda(A,B)$}
      \subparagraph{} $()^*:= Hom(-,\Lambda)$ kontravariantní funktor $mod(\Lambda)\rightarrow mod(\Lambda^{op})$      
      \subparagraph{} Použijeme functor $()^*$ na projektivní pokrytí modulu X  \\*
        \centerline{ \xymatrix{ P_1 \ar@{->}[r]^{s}  &P_0 \ar@{->}[r]^t &X  \ar@{->}[r] &0 } }
        a dostaneme exaktní posloupnost \\*
        \centerline{ \xymatrix{ X^* \ar@{->}[r]^{t^*} &P_0^* \ar@{->}[r]^{s^*} &P_1^* \ar@{->}[r] &Cok(s^*)  \ar@{->}[r] &0 } }      
      \subparagraph{} $Tr(X):=Cok(s^*)$ funktor $\underline{mod}(\Lambda)\rightarrow\underline{mod}(\Lambda^{op})$
      \subparagraph{} $D():=Hom(-,\mathbb{K})$ kontravariantní funktor $\underline{mod}(\Lambda)\rightarrow\underline{mod}(\Lambda^{op})$     
      
      \paragraph{QPA} Implementováno \\*
\begin{verbatim}
mTrX=TransposeOfModule(mX);
mDX=DualOfModule(mX);
mDTrX=DualOfTranspose(mX);
\end{verbatim}

    \subsection{$\Lambda$-homomorfismus $\varphi_{P_0,\Omega}$ (ekvivalěntně lze pro $\varphi_{P_1,\Omega}$)}
      \paragraph{$\Lambda$-homomorfismus $\varphi_{\Lambda, \Omega}$}
        \subparagraph{} Zobrazení \\*        
        \centerline{$\varphi_{\Lambda, \Omega}:Hom_\Lambda(\Lambda,\Omega)\rightarrow 
        Hom_\Lambda(\Lambda,\Lambda)\otimes_\Lambda \Omega$} \\*
        dané předpisem \\*
        \centerline{$\varphi_{\Lambda, \Omega}(g):=1_\Lambda\otimes g(1_\Lambda)$}
        je izomorfismem $\mathbb{K}$-modulů. 
        \subparagraph{} Dále definujme zobrazení $\varphi_{\Lambda, \Omega}^n$  
        tvořené diagonální $n\times n$-maticí jejíž všechny nenulové prvky jsou rovné 
        $\varphi_{\Lambda, \Omega}$:\\*
        \centerline{$
          \varphi_{\Lambda, \Omega}^n:
          Hom_\Lambda(\Lambda,\Omega)^n\rightarrow 
          (Hom_\Lambda(\Lambda,\Lambda)\otimes_\Lambda \Omega)^n
        $}
        dané předpisem \\*
        \centerline{$ \{f_i\}_{i=1}^n\mapsto  \{\varphi_{\Lambda,\Omega}(f_i)\}_{i=1}^n $}
        
      \paragraph{Rozložení $\Lambda^n\simeq P_0\oplus P_0'$ (ekvivaletně lze pro $\Lambda^n\simeq P_1\oplus P_1'$)}
        \subparagraph{} Pokud $\Lambda$ může být zapsána jako 
        $\Lambda=\Lambda e_1\oplus\Lambda e_2\oplus ... \oplus\Lambda e_m$
        pak lze projektivní modul $P_0$ vyjádřit jako
        $P_0\simeq(\Lambda e_1)^{n_1}\oplus(\Lambda e_2)^{n_2}\oplus ... \oplus(\Lambda 
        e_m)^{n_m}$
       Položme $n=max(n_i)$, pak \\*
       \centerline{$\Lambda^n\simeq P_0\oplus(\Lambda e_1)^{n-n_1}\oplus(\Lambda e_2)^{n-n_2}\oplus ... \oplus(\Lambda e_m)^{n-n_m}$} 
       Definujme \\*
       \centerline{$P_0':=(\Lambda e_1)^{n-n_1}\oplus(\Lambda e_2)^{n-n_2}\oplus ... \oplus(\Lambda e_m)^{n-n_m}$} 
       pak \\*
       \centerline{$\Lambda^n\simeq P_0\oplus P_0'$} \\*
       
     \paragraph{QPA} S pomocí funkce $IndecProjectiveModules(A);$ spočteme 
     moduly $\Lambda e_i$ a následně spočteme kolikrát je každý z nich obsažen v 
     rozklad modulu $P$ na direktní součet nerozložitelných projektivních podmodulů. 
     Číslo $n$ bude maximum z těchto čísel.
     \paragraph{} Výsledkem je čtveřice \\*
       \centerline{
         [ [$\Lambda{e_1},$\Lambda{e_2},...,$\Lambda{e_m}$], n, P', [$n_1,n_2,...,n_m$], [$n-n_1,n-n_2,...,n-n_m$] ]
       }\\*
\begin{verbatim}

SuppProjModule := function(P, A)
  local PP, Ps, max, common, i, diff,
        in_multiplicities, ou_multiplicities;

  PP:= IndecProjectiveModules(A);
  Ps:= ZeroModule(A);
  in_multiplicities := [];
  ou_multiplicities := [];
  max := 0;

  for i in [1..Length(PP)] do
    Add(in_multiplicities, 0);

    repeat
      common := CommonDirectSummand(P, PP[i]);
      if IsList(common) then
        in_multiplicities[i] := in_multiplicities[i] + 1;
        P := common[2];
      fi;
    until IsList(common) = false;

    max := Maximum([max, in_multiplicities[i]]);
  od;

  Ps := [];
  for i in [1..Length(PP)] do
    diff := max - in_multiplicities[i];
    ou_multiplicities[i] := diff;

    for j in [1..diff] do
      Add(Ps, PP[i]);
    od;
  od;
  Ps := DirectSumOfModules(Ps);

  return [PP, max, Ps, in_multiplicities, ou_multiplicities];
end;
 \end{verbatim}
  
     \paragraph{Konstrukce isomorfismu $P\oplus P'\simeq\Lambda^n$}
        \subparagraph{}
        Máme následující dva isomorfismy: \\*\\*
          \centerline{$\Lambda^n \simeq (\Lambda e_1 \oplus ... \oplus \Lambda e_m)^n $}    
          \centerline{$P\oplus P' 
            \simeq 
            ((\Lambda e_1)^{n_1}\oplus ... \oplus(\Lambda e_m)^{n_m})
            \oplus
            ((\Lambda e_1)^{n-n_1}\oplus ... \oplus(\Lambda e_m)^{n-n_m})
          $}          
          \\*\\*  
          Pro $i=1,...,m$, $j=1,...,n$ ($n=max(n_i)$) označme kanonické projekce:   \\*
          $p:P \oplus P' \rightarrow P$ \\*
          $p':P \oplus P' \rightarrow P'$ \\*
          $\rho_i:P \rightarrow \Lambda e_i$ \\*
          $\rho_i':P' \rightarrow \Lambda e_i$ \\*
          $\lambda_j:\Lambda^n \rightarrow \Lambda$ \\*
          $\mu_{i,j}: \Lambda \rightarrow \Lambda e_i$ \\*
          
         Podíváme se nejprve na modul $P$, pro $P'$ bude platit ekvivalentní konstrukce. 
         Zvolme libovolné $i=1,...,m$, aby modul $\Lambda e_i$ ležel v rozkladu $P$. 
         Dále zvolme $j=1,..,m$ dostaneme diagram: \\*\\*
         \centerline{\xymatrix{
              P\oplus P'\ar@{->>}[r]^p 
              & P\ar@{->>}[r]^{\rho_i}  
              & \Lambda e_j \\
              \Lambda^n\ar@{->>}[r]_{\lambda_j}
              & \Lambda \ar@{->>}[ru]_{\mu_{i,j}}
          }}   
          Z projektivity $P$  pro každý takovýto diagram existuje Homorfismus    
          $g_{i,j}:P \rightarrow \Lambda^n$ (resp. $g_{i,j}'$ pro $P'$) 
          takový, že vzniklý čtverec komutuje. Dostaneme následující 
          komutativní diagram: \\*\\*
         \centerline{\xymatrix{
              P\oplus P'\ar@{->>}[r]^p \ar@{->}[d]_{d_{i,j}p} 
              & P\ar@{->>}[r]^{\rho_i}  \ar@{->}[dl]^{d_{i,j}} 
              & \Lambda e_j \\
              \Lambda^n\ar@{->>}[r]_{\lambda_i}
              & \Lambda \ar@{->>}[ru]_{\mu_{i,j}}
          }}  
          Homomorfismus  $d_{i,j}p$ (resp. $d_{i,j}'p'$) je isomorfismus dvou 
          modulů $\Lambda e_j$ obsažených v direktním součtu $P \oplus P'$ 
          a direktním součtu $\Lambda^n$. Hledaný isomorfismus $P\oplus P'\simeq\Lambda^n$ 
          je pak součet těchto $n\times m$ homomorfismů tedy 
          $\sum_i\sum_j(d_{i,j})p+\sum_i\sum_j(d_{i,j})'p'$.
                      
     \paragraph{QPA} 
       \begin{verbatim}
# P1 = P
# P2 = P'
# P1_P2 = direktní součin P a P'
IsomorphismProjAndAn := function(P1_P2, An)
  local iso, used,
        proj_P1, proj_P2, proj_P1_P2, proj_P1_and_P2,
        proj_An, proj_A, proj_An_fin;

  proj_P1_P2 := DirectSumProjections(P1_P2);
  proj_P1 := DirectSumProjections( Range(proj_P1_P2[1]) );
  proj_P2 := DirectSumProjections( Range(proj_P1_P2[2]) );

  proj_An_fin := [];  # bude obsahovat všech m*n projekcí \Lambda^n -> \Lambda*e_i
  proj_An := DirectSumProjections(An);
  for f in proj_An do
    proj_A := DirectSumProjections( Range(f) );

    for g in proj_A do
      Add(proj_An_fin, f * g);
    od;
  od;

  iso := ZeroMapping(P1_P2, An);
  used := [1..Length(proj_An_fin)];
  proj_P1_and_P2 := [proj_P1, proj_P2];

  for i in [1,2] do                         # přes moduly P a P'
    for g in proj_P1_and_P2[i] do           # přes projekce P->\Lambda*e_i (resp. P')
      for j in [1..Length(proj_An_fin)] do  # přes projekce \Lambda -> \Lambda*e_i
        f := proj_An_fin[j];

        # pokud mají obě (P->\Lambda e_i   &   \Lambda -> \Lambda*e_i) stejný 
        # Range a druhá nebyla ještě použita, tak je zvedneme a přičteme k  výsledku
        if (not used[j] = true) and Range(f) = Range(g) then
          iso := iso + proj_P1_P2[i] * LiftingMorphismFromProjective(f, g);
          used[j] := true;
          break;
        fi;
      od;
    od;
  od;

  return iso;
end;      
       \end{verbatim}

     \paragraph{$\Lambda$-isomorfismus $\varphi_{P_0,\Omega}$ (ekvivaletně dostaneme $\varphi_{P_1,\Omega}$)} 
         \subparagraph{} Máme diagram \\*          
         \centerline{\xymatrix{            
            (Hom_\Lambda(\Lambda,\Omega))^n  
              \ar[r]^{ \varphi_{\Lambda, \Omega}^n}
            & (Hom_\Lambda(\Lambda,\Lambda)\otimes_\Lambda\Omega)^n
              \ar[d]^{\{g_i\otimes \omega_i\}_{i=1}^n\mapsto\sum_{i=1}^ng_i\rho_i\otimes \omega_i} \\
           Hom_\Lambda(\Lambda^n,\Omega)
              \ar[u]_{f\mapsto \{f\nu_i\}_{i=1}^n}
            & Hom_\Lambda(\Lambda^n,\Lambda)\otimes_\Lambda\Omega
              \ar[d]^{(-\circ\psi)_\Lambda\otimes1_\Omega} \\             
            Hom_\Lambda(P_0\oplus P_0', \Omega)
              \ar[u]_{(-\circ\psi^{-1})_\Omega}
            & Hom_\Lambda(P_0\oplus P_0', \Lambda)\oplus_\Lambda\Omega
              \ar[d]^{((-\circ\mu)_\Lambda\otimes1_\Omega} \\
            Hom_\Lambda(P_0', \Omega)
              \ar[u]_{(-\circ\pi)_\Omega}
              \ar@{.>}[r]_{ \varphi_{P_0, \Omega}}
            & Hom_\Lambda(P_0', \Lambda)\oplus_\Lambda\Omega \\
         } }       
         Pokud půjdeme z levého spodního rohu kolem dokola 
         až do pravého spodního, pak složením homomorfismů dostaneme izomorfismus \\*
         \centerline{$\varphi_{P_0,\Omega}:Hom_\Lambda(P_0,\Omega)\rightarrow Hom_\Lambda(P_0,\Lambda)\otimes_\Lambda \Omega$} 
         s předpisem \\*
         \centerline{$h\mapsto\sum_{i=1}^n \rho_i\psi\mu\otimes h\pi\psi^{-1}\nu_i(1_\Lambda)$}
         kde \\*
         \centerline{$ \rho_i: \Lambda^n\rightarrow\Lambda $ projekce} \\*
         \centerline{$ \nu_i: \Lambda\rightarrow \Lambda^n $ inkluze} \\*
         \centerline{$ \psi: P_0\oplus P_0' \simeq \Lambda^n $ isomorfismus} \\*
         \centerline{$ \mu: P_0\rightarrow P_0\oplus P_0'$ inkluze} \\*
         \centerline{$ \pi: P_0\oplus P_0'\rightarrow P_0$ projekce} \\* \\*
         Ekvivalentně pro modul $P_1 a P_1'$.
         
     \paragraph{QPA} Potřebná kanonická zobrazení spočteme následovně (!!!!!!! pro P1 !!!!!!!):
\begin{verbatim}
supp := SuppProjModule(mP1, A);
mP1s := supp[3];                 # modul P_1'
PP   := supp[1];                 # moduly \Lambda*e_i
n    := supp[2];        
mA   := DirectSumOfModules(PP);  # \Lambda jako modul
mP1_mP1s := DirectSumOfModules([mP1, mP1s]);
---------------------------
# modul \Lambda^n
mAn := [];
for i  in [1..n]  do mAn[i] := mA; od;
mAn := DirectSumOfModules(mAn);
---------------------------
# prvek 1_\Lambda ... pocitame jako sumu vsech e_i
1_mA := Zero(mA);
incl := DirectSumInclusions(mA);
for i in [1..Length(incl)] do
  Ae_i := Source(incl[i]);
  e_i := MinimalGeneratingSetOfModule(Ae_i)[1];
  1_mA := 1_mA + ImageElm(incl[i], e_i);
od;
---------------------------
# hledaná kanonická zpbrazení
rho    := DirectSumProjections(mAn);
nu     := DirectSumInclusions(mAn);
psi_inv:= InverseOfIsomorphism ( IsomorphismProjAndAn(mP1_mP1s, mAn) );
psi    := InverseOfIsomorphism (psi_inv);
mu     := DirectSumInclusions(mP1_mP1s)[1];
pi     := DirectSumProjections(mP1_mP1s)[1];

# a složená
mu_psi_rho := mu * psi * rho;
omega_pi_psi_inv_nu := nu * psi_inv * pi * omega;
  
# omega * pi * psi^(-1) * nu(1_\Lambda)
omega_pi_psi_inv_nu_1_A := List(omega_pi_psi_inv_nu, f -> ImageElm(f, 1_mA));

---------------------------
# Ještě spočteme moduly z pačné algebry, která budou dále potřeba

# s*: P0* -> P1*
# pouze jsem zkopíroval funkci pro výpočet Tr(X) a vrátil 
# homomorfismus s* namísto Tr(X) = Cokernel(s*)
s_star := SStar(mX);     

# ostatní op-moduly
mP1_star := Range(s_star);   # mP1* = Hom(mP1, \Lambda)
mP0_star := Source(s_star);  # mP0* = Hom(mP0, \Lambda)
A_op := OppositePathAlgebra(A);
PP_op:= IndecProjectiveModules(A_op);   # op-moduly k A*e_i
\end{verbatim}


    \subsection{Výpočet homomorfismu $\rho_i\psi\mu$ jako prvku reprezentace P*} 
       \paragraph{Teorie}
         Máme isomorfismy: \\*\\*
         \centerline{
           $Hom_\Lambda (\Lambda e_i, \Lambda)\simeq e_i\Lambda$ daný předpisem $f \mapsto f(e_i)$
         }
         \\*\\*
         \centerline{
           $Hom_\Lambda (P,\Lambda) \simeq 
             Hom_\Lambda (\bigotimes_{i=1}^m\bigotimes_{j=1}^{n_i}\Lambda e_i,\Lambda) \simeq
             \bigotimes_{i=1}^m\bigotimes_{j=1}^{n_i}(Hom_\Lambda (\Lambda e_i,\Lambda))$
         }
         \centerline{
           daný předpisem $f\mapsto\sum_{i=1}^m\sum_{j=1}^{n_i}f(e_i)$
         }
       
        \paragraph{QPA}
          Nejprve definujeme pomocnou funkci, kterou použijeme ještě několikrát později. Funkce vrátí
          pole $algebra\_element$ velikosti m.
          
          Každé $algebra\_element[i]$ spočteme následovně:
          
          Uvažujme bázi $B$ modulu $\Lambda e_i$. Modul $\Lambda e_i$ je 
          generovaný jedním prvkem a to $e_i$, takže všechny ostatní dostaneme 
          působením algebry $\Lambda$ na něj. Položme $algebra\_element[i][j]=\lambda\in\Lambda$, 
          kde $\lambda$ 
          je prvkem algebry $\Lambda$, takovým že $e_i*\lambda=B[j]$ 
          ($*$ značí působení algebry na modulu).
       
          \begin{verbatim}        
ProjectiveBasisVectorGens := function(PP)
  local basis_of_projectives, algebra_element, basis, e_i, A,
        i, j, k, m, a, b;

  A := RightActingAlgebra(PP[1]);

  basis_of_projectives := BasisOfProjectives(A);
  algebra_element := [];
  for i in [1..Length(PP)] do
    basis := BasisVectors(Basis(PP[i]));
    e_i := MinimalGeneratingSetOfModule(PP[i])[1];
    algebra_element[i] := [];

    for j in [1..Length(basis)] do
      for k in [1..Length(basis_of_projectives)] do
        a := basis_of_projectives[i][k];

        for m in [1..Length(a)] do
          b := a[m];

          if not a[m] = 0 then
            if basis[j] = e_i^b then
              algebra_element[i][j] := b;

              break;
            fi;
          fi;
        od;
      od;
    od;
  od;

  return algebra_element;
end;

         \end{verbatim}

         A nyní už můžeme vyjádřit homomorfismus $f:P \rightarrow \Lambda$ jako 
         prvek modulu $P^*$, vsupem je čtveřice $[f, P^*, PP, PP\_op]$, kde $PP$ 
         jsou $\Lambda e\_i$ moduly a $PP\_op$ jejich op-moduly:
         
         
       \begin{verbatim}
                  
 FromHomToPStar := function(f, P_star, PP, PP_op)
  local proj, incl, proj2, image, range, e_i, e_i_j, coeffs, 
        algebra_element, result, el, gen,
        i, j, c;

  proj := DirectSumProjections(mP1);
  incl := DirectSumInclusions(mP1);
  proj2 := DirectSumProjections(mA);
  image := [];

  for  i in [1..Length(proj)] do
    # spocteme f(e_i)
    range := Range(proj[i]);
    e_i := MinimalGeneratingSetOfModule(range)[1];
    e_i := ImageElm(incl[i], e_i);  # nejprve si e_i zobrazíme do P
    e_i := ImageElm(f, e_i);        # a až poté ho zobrazíme morfismem f

    # image[i] obsahuje Sum(f(e_i)) vnorenou do Ae_i
    for j in [1..Length(proj2)] do
      e_i_j := ImageElm(proj2[j], e_i);

      if i = 1 then
        image[j] := e_i_j;
      else
        image[j] := image[j] + e_i_j;
      fi;
    od;
  od;

  # spoctu vzdy souradnice image[i] vzhledem k bazi Ae_i
  coeffs := [];
  for i in [1..Length(PP)] do
    coeffs[i] := Coefficients(Basis(PP[i]), image[i]);
  od;

  # algebra_element[i][j] obsahuje a\in\Lambda, ze j-ty bazovy vektor Ae_i je roven e_i*a
  algebra_element := ProjectiveBasisVectorGens(PP);

  # pro kazdy image[i] spoctu a\in\Lambda , ze image[i] = a
  # a pocitam celkovou sumu OppositePathAlgebraElement(a) vnorenych do P*
  incl := DirectSumInclusions(P_star);
  result := Zero(P_star);
  for i in [1..Length(coeffs)] do
    for j in [1..Length(coeffs[i])] do
      c := coeffs[i][j];
      gen := MinimalGeneratingSetOfModule(PP_op[i])[1];

      if not c = 0 then
        el := c * gen ^ OppositePathAlgebraElement( algebra_element[i][j] );
        result := result + ImageElm(incl[i], el);
      fi;
    od;
  od;

  return result;
end;        
       \end{verbatim}


    \subsection{Tenzorový součin $P^*_0\otimes_\Lambda\Omega$ (resp. $P^*_1\otimes_\Lambda\Omega$)}
      \paragraph{Teorie}
         \subparagraph{}
         Pro $M\in Mod(\Lambda)$ a $e\in \Lambda^{op}$ idempotent máme isomorfismus 
vektorových prostorů:
         \\*\\*
         \centerline{$\Lambda^{op} e\otimes_\Lambda M\simeq eM$} \*
         \centerline{$\lambda\otimes m\mapsto \lambda m$} \\*\\*
         Dále pro moduly $N_i\in Mod(\Lambda^{op}),i=1,..,n$, $M\in Mod(\Lambda)$ platí 
         \\*\\*
         \centerline{$(\bigoplus_{i=1}^n N_i) \otimes_\Lambda M \simeq \bigoplus_{i=1}^n (N_i \otimes_\Lambda 
         M)$}\\*\\*
         Náš modul $P^*$ je projektivní, existuje tedy rozklad  \\*\\*
         \centerline{$P^*\simeq(\Lambda^{op} e_1)^{w_1}\oplus(\Lambda^{op} e_2)^{w_2}\oplus ... \oplus(\Lambda^{op} 
        e_m)^{w_m}$} \\*\\*
        Ten zkombinujeme s dvěmi předchozími isomorfismy a dostaneme isomorfismus 
        \textbf{vektorových prostorů}: \\*\\*
        \centerline{$
        P^*\otimes_\Lambda\Omega 
        \simeq (\Lambda^{op} e_1)^{w_1}\oplus(\Lambda^{op} e_2)^{w_2}\oplus ... \oplus(\Lambda^{op} 
        e_m)^{w_m}\otimes_\Lambda \Omega
        $}
        \centerline{$
        \simeq \bigoplus_{i=1}^m (\Lambda^{op} e_i \otimes_\Lambda \Omega)^{w_i}
        \simeq \bigoplus_{i=1}^m (e_i\Omega)^{w_i}
        $}\\*\\*
       Dále potřebujeme zkonstruovat $\Lambda$-balancované zobrazení \\*\\*
       \centerline{$P^*\times\Omega\rightarrow \bigoplus_{i=1}^m (e_iM)^{w_i}$} 
       \\*\\*
        To bude ze vztahu výše dané předpisem: \\*\\*
       \centerline{$
       p\times\omega=
       \sum_{i=1}^m\sum_{j=1}^{w_i}(\lambda_{i,j}\times\omega)
       \mapsto
       \sum_{i=1}^m\sum_{j=1}^{w_i}(\lambda_{i,j}\omega)
       $, kde $\lambda_{i,j}\in\Lambda^{op}e_i$}\\*\\*
       Prvky $\lambda_{i,j}$ spočteme projekcí $\pi_{i,j}:P^*\rightarrow \Lambda^{op} 
       e_i$. Poté vezmeme ekvivaletní prvek algebry $\Lambda^{op} $ a jemu 
       opačný prvek $\lambda_{i,j}^{op}:=OppsositePathAlgebraElement(\pi_{i,j}(\lambda_{i,j}))$ 
       necháme vždy působit na prvek $\omega$ $\Lambda-modulu$ $\Omega$. Vektory 
       výsledných prvků zobrazíme vnořením vektorových prostorů 
       $\nu_{i,j}:
       e_i\Omega\hookrightarrow 
       \bigoplus_{i=1}^m\bigoplus_{j=1}^{w_i}e_i\Omega$ a sečteme:        
       $\sum_{i=1}^m\sum_{j=1}^{w_i}(\nu_{i,j}(\lambda_{i,j}^{op}*\omega))$.\\*\\*

         \centerline{\xymatrix{
              & \lambda_{1,1}\in\Lambda^{op}e_1 
              &  \ldots spocitame \ldots
              & w*\lambda_{1,1}^{op}\in e_1\Omega
                  \ar@{->>}[rd]^{\nu_{1,1}}  \\
              p\in P^* 
                  \ar@{->>}[ru]^{\pi_{1,1}} 
                  \ar@{->>}[rd]_{\pi_{m,w_m}} 
                   & \vdots & \vdots & \vdots &  
                   \bigotimes_{i=1}^m\bigotimes_{j=1}^{w_i}e_i\Omega \\
              & \lambda_{m,w_m}\in\Lambda^{op}e_m
              &  \ldots spocitame \ldots
              & w*\lambda_{m,w_m}^{op}\in e_m\Omega
                  \ar@{->>}[ru]_{\nu_{m,w_m}}  \\
          }}  
                   
      \paragraph{QPA}
        \begin{verbatim}

@TODO
        \end{verbatim}



    \subsection{Výpočet prvku $\phi_\Omega(w) \in Ker(1_{Tr(X)}\otimes i)$}
      \paragraph{Teorie}
        \subparagraph{}
        Nejprve aplikujeme funktor $()^*$ na posloupnost \\*
        *\centerline{ \xymatrix{ P_1 \ar@{->}[r]^{s}  &P_0 \ar@{->}[r]^t &X  \ar@{->}[r] &0 } }
        a dostaneme následující exaktní posloupnost 
        v $mod(\Lambda^{op})$, kde $\hat{t}$ značí kanonickou projekci $Hom_\Lambda(P_1,\Omega)$ 
        do $Tr(X)$ \\*
        **\centerline{$
          Hom_\Lambda(P_0,\Lambda)\xrightarrow{(-\circ s)_\Lambda}
          Hom_\Lambda(P_1,\Lambda)\xrightarrow{\hat{t}}
          Tr(X) \rightarrow
          0      
        $}        
        Dále použijeme funktory  \\*
        \centerline{$Hom_\Lambda(-,\Omega):mod(\Lambda)\rightarrow mod(R)$} \\*
        \centerline{$-\otimes_\Lambda \Omega:mod(\Lambda^{op})\rightarrow mod(R)$}
        aplikací $-\otimes_\Lambda \Omega$ na posloupnost **
        a $Hom_\Lambda(-,\Omega)$ na posloupnost *
        dostaneme komutativní diagram s exaktnímu řádky \\*\\*
        \centerline{ \xymatrix{
          0  \ar@{->}[r] 
          &Hom(X,\Omega) \ar@{->}[r]^{(-\circ t)_\Omega} 
          &Hom(P_0,\Omega) \ar@{->}[r]^{(-\circ s)_\Omega}  \ar@{->}[d]^{\varphi_{P_0, \Omega}}
          &Hom(P_1,\Omega)  \ar@{->}[d]^{\varphi_{P_1, \Omega}} \\
          &
          &Hom(P_0,\Lambda)\otimes_\Lambda\Omega \ar@{->}[r]^{(-\circ s)_\Lambda\otimes1_\Omega}
          &Hom(P_1,\Lambda)\otimes_\Lambda\Omega \ar@{->}[r]^{\hat{t}\otimes1_\Omega}
          &Tr(X)\otimes_\Lambda\Omega \ar@{->}[r]
          &0
        } }

        \subparagraph{}Položme $\phi_\Omega:=[\hat{t}\otimes1_\Omega]\varphi_{P_1, \Omega}$, 
        máme tedy následující dlouhou exaktní posloupnost \\*
        \centerline{$ 
          0\rightarrow 
          Hom_\Lambda(X,\Omega)\xrightarrow{(-\circ t)}
          Hom_\Lambda(P_0,\Omega)\xrightarrow{(-\circ s)}
          Hom_\Lambda(P_1,\Omega)\xrightarrow{\phi_\Omega}
          Tr(X)\otimes_\Lambda \Omega\rightarrow
          0          
        $} \\*\\*
         a nenulový prvek \phi_\Omega(w) \in Ker(1_{Tr(X)}\otimes i)$ kde $\omega:P_1\rightarrow\Omega$ je 
         projektivní pokrytí modulu $\Omega$ (sekce 1.2). 
         
         \subparagraph{} Zobrazení $\varphi_{P_1,\Omega}$ jsme určili v 
         předchozí části: \\*\\*
         \centerline{$\varphi_{P_1,\Omega}:Hom_\Lambda(P_1,\Omega)\rightarrow Hom_\Lambda(P_1,\Lambda)\otimes_\Lambda \Omega$} 
         předpisem \\*
         \centerline{$h\mapsto\sum_{i=1}^n \rho_i\psi\mu\otimes h\pi\psi^{-1}\nu_i(1_\Lambda)$}
         kde \\*
         \centerline{$ \rho_i: \Lambda^n\rightarrow\Lambda $ projekce} \\*
         \centerline{$ \nu_i: \Lambda\rightarrow \Lambda^n $ inkluze} \\*
         \centerline{$ \psi: P_1\oplus P_1' \simeq \Lambda^n $ isomorfismus} \\*
         \centerline{$ \mu: P_1\rightarrow P_1\oplus P_1'$ inkluze} \\*
         \centerline{$ \pi: P_1\oplus P_1'\rightarrow P_1$ projekce} \\*\\*
         A tedy 
         \centerline{$\phi_\Omega(w)
         =[\hat{t}\otimes1_\Omega]\varphi_{P_1, \Omega}(w)
         =[\hat{t}\otimes1_\Omega]\sum_{i=1}^n \rho_i\psi\mu\otimes w\pi\psi^{-1}\nu_i(1_\Lambda)
         $} \\*\\*         
         \centerline{$
         =\sum_{i=1}^n \hat{t}(\rho_i\psi\mu)\otimes w\pi\psi^{-1}\nu_i(1_\Lambda)
         $}
              
    \subsection{$\mathbb{K}$-Báze $Tr(X)\otimes_\Lambda\Omega$}
      \paragraph{Teorie}
         \subparagraph{} Doplníme prvek $\phi_\Omega(w)$ na bázi modulu $\mathbb{K}$-bázi 
         $B_{Ker(1_{Tr(X)}\otimes i)}=\{\phi_\Omega(w), 
         \omega_2,...,\omega_l\}$. Bázi dále rozšíříme na bázi $B_{Tr(X)\otimes_\Lambda\Omega}=\{\phi_\Omega(w), 
         \omega_2,...,\omega_l\}$ celého $Tr(X)\otimes_\Lambda\Omega$.

    \subsection{$\Lambda$-Homomorfismus $\xi$}
      \paragraph{Teorie}
         \subparagraph{} Definujme $\xi:\Omega\rightarrow DTr(X)$ následovně: pro  $a\in\Omega$ buď
         $\xi(a)$ homomorfismus $\xi(a):Tr(X)\rightarrow \mathbb{K}$ daný předpisem:  \\*
         \centerline{ 
           $q\mapsto$ První $\mathbb{K}$-koeficient prvku $q\otimes a$ vzledem k bázi  $B_{Tr(X)\otimes_\Lambda\Omega}$
         }         
 
    \subsection{Hledané E}
      \paragraph{Teorie}
         \subparagraph{} E spočteme jako pushout homomorfismů $i$ a $\xi$. \\*
        \centerline{ \xymatrix{
          \Omega \ar[r]^\xi \ar[d]_i & DTr(X) \ar@{..}[d] \\ 
          P_0 \ar@{..}[r] & E
        } }

      \paragraph{QPA}   
        \begin{verbatim}E = PullBack(xi, i)\end{verbatim}
      
\end{document} 