\section{Teorie kategorií}

    \begin{dfn}
      Kategorie $\mathcal C$ je trojice $\mathcal C=(Ob(\mathcal C), Hom(\mathcal C), 
      \circ)$, kde $Ob(\mathcal C)$ je nazývána třída objektů $\mathcal C$, $Hom(\mathcal C)$ 
      je nazývána třída morfismů a $\circ$ je binární operace na morfismech 
      splňující:      
      \begin{description}
        \item[(a)] Každým\,\, dvěma\,\, objektům $X,Y\in Ob(\mathcal C)$\,\, přiřadíme\,\, množinu\,\, 
          morfismů $Hom_{\mathcal C}(X,Y)$ nazývanou morfismy z $X$ do $Y$ takovou, že 
          pro $(X,Y)\neq (Z,U)$ je $Hom_{\mathcal C}(X,Y) \cap Hom_{\mathcal C}(Z,U)= 
          \emptyset$.
        \item[(b)] Pro každou trojici $X,Y,Z\in Ob(\mathcal C)$ je operace
          \begin{eqnarray}
             Hom_{\mathcal C}(Y,Z) \times Hom_{\mathcal C}(X,Y)  &\to& Hom_{\mathcal C}(X,Z) \nonumber \\
             (f,g) &\mapsto&  g\circ f \nonumber
          \end{eqnarray}
            nazvaná skládání morfismů, splňující následující dvě podmínky:
          \begin{description}
            \item[(i)] $h\circ(g\circ f)=(h\circ g)\circ f$ pro každou trojici $f\in Hom_{\mathcal 
              C}(X,Y)$, $g\in Hom_{\mathcal C}(Y,Z)$ a $h\in Hom_{\mathcal 
              C}(Z,U)$.
            \item[(ii)] Pro každý objekt $X\in Ob(\mathcal C)$ existuje morfismus $1_X\in Hom_{\mathcal C}(X,X)$,
              nazávaný identický morfismus na $X$, takový, že pro každé $f\in Hom_{\mathcal 
              C}(X,Y)$, $g\in Hom_{\mathcal C}(Z,X)$ je $f\circ 1_X=f$ a $1_X\circ g=g$.         
          \end{description}
      \end{description}
      
      Namísto $f\in Hom_{\mathcal C}(X,Y)$ píšeme často jen $X\xrightarrow{\text{f}}Y$ 
      nebo $f:X\to Y$. Diagram v kategorii $\mathcal C$ nazveme komutativním, 
      pokud každé složení cest se stejným začátkem i koncem si je rovné. 
      Například následující diagram s $X,Y,Z,U\in Ob(\mathcal C)$ je 
      komutativní, pokud $g\circ f=i\circ h$: \\\\
       \centerline{\xymatrix{
         X \ar@{->}[r]^f \ar@{->}[d]^h 
         & Y \ar@{->}[d]^g \\
         X \ar@{->}[r]^i
         & U 
      }}\\\\
    \end{dfn}
    
    \begin{dfn}
      Kategorii $\mathcal C'$ nazveme podkategorií kategorie $\mathcal C$, pokud 
      splňuje následující podmínky:
      \begin{description}
        \item[(a)] $Ob(\mathcal C')$ je podtřída Ob($\mathcal C)$.
        \item[(b)] Pro $X,Y\in Ob(\mathcal C')$ je $Hom_{\mathcal C'}(X,Y)\subseteq Hom_{\mathcal C}(X,Y)$.
        \item[(c)] Skládání morfismů v $\mathcal C'$ je stejné jako v $\mathcal C$. 
        \item[(d)] Pro každý objekt $X\in Ob(\mathcal C')$ je identický 
          morfismus $1_X\in Hom_{\mathcal C'}(X,X)$ stejný jako v kategorii $\mathcal  C$.
      \end{description}
      Podkategorii $\mathcal C'$ nazveme úplnou, pokud 
      $Hom_{\mathcal C'}(X,Y)=Hom_{\mathcal C}(X,Y)$ 
      pro všechny $X,Y\in Ob(\mathcal C')$
    \end{dfn}
    

%    \begin{dfn}
%     Nechť $X,Y\in Ob(\mathcal C')$ a $h:X\to Y$. Pak h nazveme
%      \begin{description}
%        \item[(a)] endomorfismus, pokud $X=Y$.
%        \item[(b)] monomorfismus, pokud pro každé $f,g:Z\to X$ platí: 
%          Je-li $h\circ f=h\circ g$, pak $f=g$.
%        \item[(b)] epimorfismus, pokud pro každé $f,g:Y\to Z$ platí: 
%          Je-li $f\circ h=g\circ h$, pak $f=g$.
%        \item[(b)] izomorfismus, pokud existuje $v:Y\to X$ takové, že $vu=1_X$ a 
%        $uv=1_Y$. Takové $v$ nazýváme inverzem $u$ a značíme ho $u^{-1}$.
%      \end{description}
%
%      
%      Pokud existuje izomorfismus dvou objektů $X,Y\in Ob(\mathcal C')$, pak je 
%      nazýváme izomorfní a značíme je $X\simeq Y$.
%    \end{dfn}

      \begin{dfn}
        Kategorii $\mathcal C$ nazveme aditivní, pokud splňuje následující 
        podmínky:
        \begin{description}
          \item[(a)] Pro každou konečnou množinu $X,Y\in Ob(\mathcal C)$ 
          existuje objekt $X\oplus Y\in \mathcal C$ (nazývaný direktní suma $X$ 
          a $Y$) společně s morfismy 
          $\nu_X\in Hom_{\mathcal C}(X,X\oplus Y)$, 
          $\nu_Y\in Hom_{\mathcal C}(Y,X\oplus Y)$,
          $\pi_X\in Hom_{\mathcal C}(X\oplus Y,X)$ a 
          $\pi_Y\in Hom_{\mathcal C}(X\oplus Y,Y)$ takovými, že platí:          
          \begin{eqnarray}
            \pi_X\nu_X &=& 1_X \nonumber \\
            \pi_Y\nu_Y  &=& 1_Y \nonumber \\
            \pi_X\nu_Y &=& 0 \nonumber \\
            \pi_Y\nu_X &=& 0 \nonumber \\
             \pi_X\nu_X + \pi_Y\nu_Y &=& 1_{X\oplus Y} \nonumber 
          \end{eqnarray}
          Morfismy $\pi_X$ a $\pi_Y$ jsou nazývány kanonické projekce a $\nu_X$ 
          a $\nu_Y$ kanonické inkluze.
          
          \item[(b)] Pro všechny $X,Y\in Ob(\mathcal C)$ má množina $Hom_{\mathcal C}(X,Y)$ 
          strukturu abelovské grupy.
          \item[(c)] Pro každou trojici objektů $X,Y,Z\in Ob(\mathcal C)$ je 
          skládání morfismů \\\\
          \centerline{$\circ:Hom_{\mathcal C}(Y,Z)\times Hom_{\mathcal C}(X,Y)\longrightarrow Hom_{\mathcal C}(X,Z)$} 
          \\\\ bilineární operace.
          \item[(d)] Existuje nulový objekt $0\in Ob(\mathcal C)$ takový, že pro 
          všechny $X\in Ob(\mathcal C)$ je $| Hom_{\mathcal C}(X,0)|=| Hom_{\mathcal 
          C}(0,X)|=1$.
        \end{description}
      \end{dfn}
      
      \begin{dfn}
        Mějme aditivní kategorii $\mathcal C$. Opačná kategorie $\mathcal C^{op}$ 
        ke kategorii $\mathcal C$ má stejné objekty $Ob(\mathcal C^{op})=Ob(\mathcal C)$, 
        ale 
        $Hom_{\mathcal C^{op}}(X,Y)=Hom_{\mathcal C}(Y,X)$ pro všechny 
        $X,Y\in Ob(\mathcal C)$. Sčítání v $Hom_{\mathcal C^{op}}(X,Y)$ je definováno stejně jako
        v $Hom_{\mathcal C}(Y,X)$. Skládání $\circ'$ v $\mathcal C'$ je 
        definováno vztahem $g\circ' f=f\circ g$, kde $\circ$ je skládání v $\mathcal 
        C$.
      \end{dfn}
    
    \begin{dfn}
      Nechť $\mathcal C$ je aditivní kategorie a $f\in Hom_{\mathcal C}(X,Y)$.
      \begin{description}
        \item[(a)] Jádro morfismu $f$ je objekt $Ker(f)\in Ob(\mathcal C)$ 
          společně s morfismem $\nu_f\in Hom_{\mathcal C}(Ker(f),X)$ 
          (nazývaným kanonické vnoření) takovým, že 
          splňuje následující podnínky:
          \begin{description}
            \item[(i)] $f\nu_f=0$.
            \item[(ii)] Pro každé $T\in Ob(\mathcal C)$ a $t\in Hom_{\mathcal C}(T,X)$ 
            takové, že $ft=0$ existuje právě jedno $s\in Hom_{\mathcal C}(T,Ker(f))$ 
            takové, že $t=\nu_f s$. \\\\
            \centerline{\xymatrix{
               Ker(f) \ar@{->}[r]^{\nu_f} 
                 & X \ar@{->}[r]^f 
                 & Y & &  \\
               T \ar@{->}[ru]^t  \ar@{.>}[u]^s         
             }}\\
          \end{description}
        \item[(b)] Kojádro morfismu $f$ je objekt $Cok(f)\in Ob(\mathcal C)$ 
          společně s morfismem $\pi_f\in Hom_{\mathcal C}(X,Cok(f))$
          (nazývaným kanonická projekce) takovým, že 
          splňuje následující podnínky:
          \begin{description}
            \item[(i)] $\pi_ff=0$.
            \item[(ii)] Pro každé $T\in Ob(\mathcal C)$ a $t\in Hom_{\mathcal C}(Y,T)$ 
            takové, že $tf=0$ existuje právě jedno $s\in Hom_{\mathcal C}(Cok(f),T)$ 
            takové, že $t=s\pi_f $. \\\\
            \centerline{\xymatrix{
               X \ar@{->}[r]^f 
                 & Y \ar@{->}[r]^{\pi_f}  \ar@{->}[rd]^t
                 & Cok(f) \ar@{.>}[d]^s   \\
               & & T  & &        
             }}\\
          \end{description}
      \end{description}
    \end{dfn}
      
    \begin{dfn}
      Aditivní kategorii $\mathcal C$ nazveme abelovskou, pokud pro každé $X,Y\in Ob(C)$ 
        a $f\in Hom_C(X,Y)$ existují $Ker(f)$ i $Cok(f)$ a navíc 
        $Cok(\nu_f)\simeq Ker(\pi_f)$, kde $\nu_f:Ker(f)\rightarrow X$ resp. $\pi_f:Y\rightarrow Cok(f)$ 
        je kanonické vnoření resp. kanonická projekce.
    \end{dfn}
    
    \begin{dfn}
      Posloupnost objektů a morfismů \\\\
            \centerline{\xymatrix{
               \cdots \ar@{->}[r]^{f_{n+1}}
                 & X_{n+1} \ar@{->}[r]^{f_n} 
                 & X_n \ar@{->}[r]^{f_{n-1}}
                 & X_{n-1} \ar@{->}[r]^{f_{n-2}}
                 & \cdots
             }}\\\\      v abelovské kategorii $\mathcal C$ nazveme exaktní, pokud pro 
             všechna
      $n$ je $Ker(f_{n-1})=Im(f_n)$. Krátká exaktní posloupnost je exaktní 
      posloupnost tvaru $0\to X\to Y\to Z\to 0$.
    \end{dfn}
    
    \begin{dfn}
      Nechť jsou $\mathcal C$ a $\mathcal C'$ dvě kategorie.  Kovariantní funktor 
      $F:\mathcal C \to \mathcal C'$ přiřazuje každému 
      objektu $X\in Ob(\mathcal C)$ objekt $F(X)\in Ob(\mathcal C')$ a každému 
      morfismu $h:X\to Y$ v $\mathcal C$ morfismus $F(h):F(X)\to F(Y)$ v $\mathcal C'$ 
      takový, že: 
      \begin{description}
        \item[(a)] $T(1_X)=1_{T(X)}$ pro  každý $X\in Ob(\mathcal C)$.
        \item[(b)] pro každou dvoji morfismů $f:X\to Y$ a $g:Y\to Z$ v $\mathcal C$ 
        platí, že $T(g\circ f)=T(g)\circ T(f)$.
      \end{description}
      Kontravariantní funktor $F:\mathcal C \to \mathcal C'$ přiřazuje každému 
      objektu $X\in Ob(\mathcal C)$ objekt $F(X)\in Ob(\mathcal C')$ a každému 
      morfismu $h:X\to Y$ v $\mathcal C$ morfismus $F(h):F(Y)\to F(X)$ v $\mathcal C'$ 
      takový, že: 
      \begin{description}
        \item[(a)] $T(1_X)=1_{T(X)}$ pro  každý $X\in Ob(\mathcal C)$.
        \item[(b)] pro každou dvojici morfismů $f:X\to Y$ a $g:Y\to Z$ v $\mathcal C$ 
        platí, že $T(g\circ f)=T(f)\circ T(g)$.
      \end{description}
    \end{dfn}
    
    \begin{dfn}
      Nechť $T,T':\mathcal C\to \mathcal C'$ jsou dva kovariantní (resp. kontravariantní) 
      funktory. Pak třída 
      morfismů $\Psi=\{\Psi_X:T(X)\to T'(X)\}_{X\in Ob(\mathcal C)}$ je 
      přirozenou transoformací $T$ do $T'$, pokud následující 
      diagram v $\mathcal C'$ komutuje pro každý morfismus $f:X\to Y$: \\\\
            \centerline{\xymatrix{
               T(X) \ar@{->}[r]^{\Psi_X} \ar@{->}[d]_{T(f)}
                 & T'(X) \ar@{->}[d]^{T'(f)} \\
               T(Y) \ar@{->}[r]_{\Psi_Y} 
                 & T'(Y)
             }\\
             resp.\\             
             \xymatrix{
               T(Y) \ar@{->}[r]^{\Psi_Y} \ar@{->}[d]_{T(f)}
                 & T'(Y) \ar@{->}[d]^{T'(f)} \\
               T(X) \ar@{->}[r]_{\Psi_X} 
                 & T'(X)
             }}
             \\\\\\
      Přirozenou transformaci $\Psi$ nazveme přirozenou ekvivalencí 
      (nebo též přirozeným izomorfismem), pokud 
      pro každé $X\in \mathcal C$ je $\Psi_X$ izomorfismus.
      
      Kovariantní funktor $T:\mathcal C\to \mathcal C'$ nazveme ekvivalencí 
      kategorií, pokud existuje funktor $F:\mathcal C'\to \mathcal C$ a 
      přirozené ekvivalence $\Psi:1_{\mathcal C}\to FT$ a $\Phi:1_{\mathcal C'}\to 
      TF$, kde $1_{\mathcal C}$ a $1_{\mathcal C'}$ jsou funktory identity na $C$ resp. $C'$.
      
      Kontravariantní funktor $D:\mathcal C\to \mathcal D$ nazveme dualitou 
      kategorií, pokud indukovaný kovariantní funktor $D:\mathcal C^{op}\to \mathcal D$ 
      je ekvivalence kategorií.
    \end{dfn}
    
    \begin{dfn}
      Nechť $\mathcal C$ a $\mathcal D$ jsou abelovské kategorie a $F:\mathcal C\to\mathcal D$ 
      kovariantní (resp. kontravariantní) funktor. A nechť  \\\\
            \centerline{\xymatrix{
               0 \ar@{->}[r] & A \ar@{->}[r] & B \ar@{->}[r] & C \ar@{->}[r] & 0
             }}\\\\ je exaktní posloupnost v $\mathcal C$. Pak řekneme, že $F$ je
      \begin{description}
        \item[(a)] zleva exaktní, pokud následující posloupnost je exaktní v $\mathcal D$:\\\\
            \centerline{\xymatrix{
               0 \ar@{->}[r] & F(A) \ar@{->}[r] & F(B) \ar@{->}[r] & F(C)
             }}\\\\
             \centerline{(resp. \xymatrix{
               0 \ar@{->}[r] & F(C) \ar@{->}[r] & F(B) \ar@{->}[r] & F(A) 
             })}
        \item[(b)] zprava exaktní, pokud následující posloupnost je exaktní v $\mathcal D$:\\\\
            \centerline{\xymatrix{
               F(A) \ar@{->}[r] & F(B) \ar@{->}[r] & F(C) \ar@{->}[r] & 0
             }}\\\\
             \centerline{(resp. \xymatrix{
               F(C) \ar@{->}[r] & F(B) \ar@{->}[r] & F(A) \ar@{->}[r] & 0) 
             })}
        \item[(c)] exaktní, pokud následující posloupnost je exaktní v $\mathcal D$:\\\\
            \centerline{\xymatrix{
               0 \ar@{->}[r] & F(A) \ar@{->}[r] & F(B) \ar@{->}[r] & F(C) \ar@{->}[r] & 0
             }}\\\\
             \centerline{(resp. \xymatrix{
               0 \ar@{->}[r] & F(C) \ar@{->}[r] & F(B) \ar@{->}[r] & F(A) \ar@{->}[r] & 0) 
             })}
      \end{description}
    \end{dfn}
    
    \begin{dfn}
      \begin{description}
        \item
        \item[(a)] $Set:=$ kategorie množin, kde morfismy jsou množinová 
        zobrazení.
        \item[(b)] $Ab:=$ kategorie abelovských grup, kde 
        morfismy jsou homomorfismy abelovských grup. 
        \item[(c)] $Mod(S):=$ kategorie $S$-modulů okruhu $S$, kde morfismy jsou 
        homomorfismy $S$-modulů.
      \end{description}
    \end{dfn}
    
    \begin{lem}
      Nechť $\mathcal C$ je kategorie a $X\in Ob(\mathcal C)$, pak
      \begin{description}
        \item[(a)] máme kovariantní funktor \\\\
          \centerline{$Hom_{\mathcal C}(X,-):{\mathcal C}\rightarrow Set$,}\\\\
          daný pro $Y,Z\in Ob({\mathcal C})$ a $f\in Hom_{\mathcal C}(Y,Z)$ předpisem:\\\\
          \centerline{\begin{aligned}
            Hom_{\mathcal C}(X,-)(Y) &:= & Hom_{\mathcal C}(X,Y)  \\
            Hom_{\mathcal C}(X,-)(f)  &:= & Hom_{\mathcal C}(X,Y)&\rightarrow Hom_{\mathcal C}(X,Z) \\
            & & g & \mapsto fg
          \end{aligned}}
        \item[(b)] máme kontravariantní funktor \\\\
          \centerline{$Hom_{\mathcal C}(-,X):{\mathcal C}\rightarrow Set$,}\\\\
          daný pro $Y,Z\in Ob({\mathcal C})$ a $f\in Hom_{\mathcal C}(Y,Z)$ předpisem:\\\\
          \centerline{\begin{aligned}
            Hom_{\mathcal C}(-,X)(Y) &:= & Hom_{\mathcal C}(Y,X)  \\
            Hom_{\mathcal C}(-,X)(f)  &:= & Hom_{\mathcal C}(Z,X)&\rightarrow Hom_{\mathcal C}(Y,X) \\
            & & g & \mapsto gf
          \end{aligned}}
        \item[(c)] pokud je ${\mathcal C}$ abelovská, pak jsou $Hom_{\mathcal C}(X,-)$ i $Hom_{\mathcal C}(-,X)$ 
          zleva exaktní funktory z ${\mathcal C}$ do $Ab$ (Kategorie abelovských grup).
      \end{description}
      
      Pro střučnost budeme v dalším textu zapisovat homomorfismus 
      $Hom_{{\mathcal C}}(X,-)(f)$ jako $(f\circ-)_X$ a $Hom_{{\mathcal C}}(-,X)(f)$ jako 
      $(-\circ f)_X$.
    \end{lem}
    
    \begin{proof}
      \begin{description}
        \item
        \item[(a)] Zřejmě $Hom_{\mathcal C}(X,Y)\in Set$ a \\\\
          \centerline{$Hom_{\mathcal C}(X,-)(1_Y)=[g\mapsto 1_Yg=g]=1_{Hom_{\mathcal C}(X,Y)}$} 
          \\\\
          pro každé $Y\in \mathcal C$. Dále je-li $Y,Z,W\in \mathcal C$, $f_1\in Hom_{\mathcal C}(Y,Z)$ 
          a $f_2\in Hom_{\mathcal C}(Z,W)$, pak platí 
          \begin{eqnarray}
            (Hom_{\mathcal C}(X,-)(f_2f_1))(g) 
            &=& f_2f_1g \nonumber \\
            &=& f_2(f_1g) \nonumber \\
            &=& (Hom_{\mathcal C}(X,-)(f_2))(f_1g) \nonumber \\
            &=& Hom_{\mathcal C}(X,-)(f_2)Hom_{\mathcal C}(X,-)(f_1)(g) \nonumber
          \end{eqnarray}
          pro každé $Hom_{\mathcal C}(X,Y)$ a tedy \\\\
          \centerline{$Hom_{\mathcal C}(X,-)(f_2f_1)=Hom_{\mathcal C}(X,-)(f_2)Hom_{\mathcal 
          C}(X,-)(f_1)$.}\\
        \item[(b)] Dokáže se podobně jako (a).
        \item[(c)] Že $Hom_{\mathcal C}(X,-)$ i $Hom_{\mathcal C}(-,X)$ jsou 
          funktory $\mathcal C\to Ab$ je zřejmé, dokážeme exaktnost zleva $Hom_{\mathcal 
          C}(X,-)$. Exaktnost zleva funktoru $Hom_{\mathcal C}(-,X)$ se dokáže ekvivalentně. 
          
          Mějme tedy krátkou exaktní posloupnost v $\mathcal C$: \\\\
         \centerline{\xymatrix{
           0 \ar@{->}[r] 
           & A \ar@{->}[r]^f  
           & B \ar@{->}[r]^g  
           & C \ar@{->}[r] 
           & 0
        }.}\\\\
        Potřebujeme dokázat, že následující posloupnost\\\\
         \centerline{\xymatrix{
           0 \ar@{->}[r] 
           & Hom_{\mathcal C}(X,A) \ar@{->}[r]^{(f\circ -)_X} 
           & Hom_{\mathcal C}(X,B) \ar@{->}[r]^{(g\circ -)_X}  
           & Hom_{\mathcal C}(X,C) \ar@{->}[r] 
        }}\\\\
        je exaktní v $Ab$. Nejprve ukážeme, že $(f\circ -)_X$ je monomorfismus v 
        $Ab$, což je v $Ab$ to samé jako jako injektivní. Nechť $h\in Hom_{\mathcal C}(X,A)$ 
        je takové, že \\\\
        \centerline{$(f\circ -)_X(h)=fh=0$} \\\\
        Pak protože $f$ je monomorfismus, musí být $h=0$. Nechť $h\in \mathcal C}(X,B)$, pak \\\\
        \centerline{$(f\circ -)_X(g\circ - )_X(h)=(f\circ -)_X(gh)=fgh=(fg\circ -)_X(h)$} 
        \\\\
        a tedy \\\\
        \centerline{$(f\circ -)_X(g\circ -)_X=(fg\circ -)_X$.}\\\\
        Z čehož plyne, že \\\\
        \centerline{$Im((f\circ -)_X)\subseteq Ker((g\circ -)_X)$.}\\\\        
        Nyní dokážeme opačnou inkluzi \\\\
        \centerline{$Ker((g\circ -)_X) \subseteq Im((f\circ -)_X)$.}\\\\
        Buď $h\in Ker((g\circ -)_X)$. Pak $gh=0$ a protože $f$ je jádro $g$, pak 
        se $h$ faktorizuje skrze $f$, neboli existuje $j\in Hom_{\mathcal C}(X,A)$ 
        takové, že \\\\
        \centerline{$h=fj=(f\circ -)_X(j)$.} \\\\
        A tedy $h\in Im((f\circ -)_X)$.
      \end{description}
    \end{proof} 
    
     \begin{pzn}
       Buď $\mathcal C$ abelovská kategorie a $A,B,C,D\in Ob(\mathcal C)$ a $u,v,f,g$ 
       momorfismy takové, že následující diagram komutuje: \\\\
       \centerline{\xymatrix{
       A \ar@{->}[r]^f \ar@{->}[d]^u & B \ar@{->}[d]^v  \\
       C \ar@{->}[r]^g & D
      }}\\\\\\
      Protože je kategorie $\mathcal C$ abelovská, tak existují jádra i kojádra morfismů $f$ 
      a $g$. A dá se dokázat, že existují i jednoznačně určené morfismy 
      $u_{ker}:Ker(f)\to Ker(g)$ a $v_{cok}:Ker(f)\to Ker(g)$ takové, že následující diagram komutuje: \\\\
       \centerline{\xymatrix{
       0 \ar@{->}[r] 
         & Ker(f) \ar@{->}[r]^{\nu_f} \ar@{.>}[d]^{u_{ker}}
         & A \ar@{->}[r]^f \ar@{->}[d]^u 
         & B \ar@{->}[d]^v \ar@{->}[r]^{\pi_f}
         & Cok(f) \ar@{->}[r] \ar@{.>}[d]^{v_{cok}}
         & 0 \\
       0 \ar@{->}[r] 
         & Ker(g)\ar@{->}[r]^{\nu_g} 
         & C \ar@{->}[r]^g 
         & D \ar@{->}[r]^{\pi_f}
         & Cok(g) \ar@{->}[r] 
         & 0
      }}\\\\           
    \end{pzn}
    
    \begin{dfn}
      Zobrazení $u_{ker}$ (resp. $v_{cok}$) z předchozí poznámky nazýváme jádrový 
      morfismus $u$ (resp. kojádrový morfismus $v$). Budeme je takto značit, ačkoli je toto značení 
      nepopisuje jednoznačně, protože vždy závisí na diagramu, ke kterému se 
      vztahují. V dalším textu bude ale vždy z kontextu jasné, o který doagram se jedná.
    \end{dfn}
      \clearpage




 