%\documentclass[12pt,a4paper]{report}
\documentclass[12pt, a4paper, twoside, openright]{report}

\usepackage{czech}
\usepackage[utf8]{inputenc}   % pro unicode UTF-8
\usepackage{amsmath}
\usepackage{amssymb}
\usepackage{xypic}
\usepackage{comment}
\usepackage{amsthm}
\usepackage{hyperref}
\usepackage{fancyvrb} 
\usepackage{graphicx}
\usepackage{pgfplots}
\usepackage{emptypage}



% Toto makro definuje kapitolu, která není očíslovaná, ale je uvedena v obsahu.
\def\chapwithtoc#1{
\chapter*{#1}
\addcontentsline{toc}{chapter}{#1}
}

% metadata
\hypersetup{pdftitle=Algoritmy v teorii reprezentací}
\hypersetup{pdfauthor=Marek Trunkát}

% pravidla MFF
% Okraje: levý 40mm, pravý 25mm, horní a dolní 25mm
% (ale pozor, LaTeX si sám přidává 1in)
\setlength\textwidth{145mm}
\setlength\textheight{247mm}
\setlength\oddsidemargin{15mm}
\setlength\evensidemargin{15mm}
\setlength\topmargin{0mm}
\setlength\headsep{0mm}
\setlength\headheight{0mm}
% \openright zařídí, aby následující text začínal na pravé straně knihy
\let\openright=\clearpage

%%%%% Základní nastavení pro oboustranný tisk:
%%%%% ----------------------------------------------------
 \setlength\textwidth{145mm}
 \setlength\textheight{247mm}
 \setlength\oddsidemargin{15mm}
 \setlength\evensidemargin{0mm}
 \setlength\topmargin{0mm}
 \setlength\headsep{0mm}
 \setlength\headheight{0mm}
 \let\openright=\cleardoublepage


% Tato makra přesvědčují mírně ošklivým trikem LaTeX, aby hlavičky kapitol
% sázel příčetněji a nevynechával nad nimi spoustu místa. Směle ignorujte.
\makeatletter
\def\@makechapterhead#1{
  {\parindent \z@ \raggedright \normalfont
   \Huge\bfseries \thechapter. #1
   \par\nobreak
   \vskip 20\p@
}}
\def\@makeschapterhead#1{
  {\parindent \z@ \raggedright \normalfont
   \Huge\bfseries #1
   \par\nobreak
   \vskip 20\p@
}}
\makeatother

\theoremstyle{plain}
\newtheorem{thm}{Věta}[chapter]
\newtheorem{lem}[thm]{Lemma}

\theoremstyle{definition}
\newtheorem{dfn}[thm]{Definice}
\newtheorem{pzn}[thm]{Poznámka}
\newtheorem{dsl}[thm]{Důsledek}
\newtheorem{pr}[thm]{Příklad}

% Zalamovat za sekci, radek. Zrusime to v sekci Implementace.
\let\oldsection = \section
\renewcommand{\section}[1]{
	\pagebreak
	\oldsection{#1}
}

\hypersetup{
   colorlinks=false,       % false: boxed links; true: colored links
    linkcolor=blue,          % color of internal links (change box color with linkbordercolor)
    citecolor=magenta,        % color of links to bibliography
    filecolor=magenta,      % color of file links
    urlcolor=cyan           % color of external links
}

\begin{document} 
  
  \thispagestyle{empty}
\begin{center}

\large

Univerzita Karlova v Praze

\medskip

Matematicko-fyzikální fakulta

\vfill

{\bf\Large DIPLOMOVÁ PRÁCE}

\vfill

\centerline{\mbox{\includegraphics[width=60mm]{logo.eps}}}

\vfill
\vspace{5mm}

{\LARGE Marek Trunkát}

\vspace{15mm}

% Název práce přesně podle zadání
{\LARGE\bfseries Algoritmy v teorii reprezentací}

\vfill

% Název katedry nebo ústavu, kde byla práce oficiálně zadána
% (dle Organizační struktury MFF UK)
Katedra algebry

\vfill

\begin{tabular}{rl}

Vedoucí diplomové práce: & RNDr. Jan Šťovíček, Ph.D. \\
\noalign{\vspace{2mm}}
Studijní program: & Matematika \\
\noalign{\vspace{2mm}}
Studijní obor: & Matematické struktury \\
\end{tabular}

\vfill

% Zde doplňte rok
Praha 2013

\end{center} 
  \include{empty}
  \thispagestyle{empty}
\vglue 0pt plus 1fill

Na tomto místě bych chtěl poděkovat vedoucímu mé diplomové práce 
RNDr. Janu Šťovíčkovi, Ph.D. za nespočet cenných rad, připomínek a trpělivost
při vedení mé práce.

\vspace{20mm}
\newpage 
  \include{empty}
  \include{prohlaseni}
  \include{empty}
  \thispagestyle{empty}

\vbox to 0.5\vsize{
\setlength\parindent{0mm}
\setlength\parskip{5mm}

Název práce:
Algoritmy v teorii reprezentací

Autor:
Marek Trunkát

Katedra:  % Případně Ústav:
Katedra algebry
% dle Organizační struktury MFF UK

Vedoucí diplomové práce:
RNDr. Jan Šťovíček, Ph.D., Katedra algebry
% dle Organizační struktury MFF UK, případně plný název pracoviště mimo MFF UK

Abstrakt:
% abstrakt v rozsahu 80-200 slov; nejedná se však o opis zadání diplomové práce
Práce se zabývá implementací algoritmu pro nalezení generátoru
skoro štěpitelných posloupností nerozložitelného a neprojektivního 
modulu algebry cest nad konečným toulcem.
Algoritmus je zde implementován v 
algebraickém systému GAP (Groups, Algorithms, Programming) 
s využitím doplňujícího balíku QPA (Quivers and Path Algebras).

Klíčová slova:
% 3 až 5 klíčových slov
skoro štěpitelné posloupnosti, teorie reprezentací, algoritmus, QPA

\vss}\nobreak\vbox to 0.49\vsize{
\setlength\parindent{0mm}
\setlength\parskip{5mm}

Title:
Algorithms in Representation Theory

Author:
Marek Trunkát

Department:
Department of Algebra
% dle Organizační struktury MFF UK v angličtině

Supervisor:
RNDr. Jan Šťovíček, Ph.D., Department of Algebra
% dle Organizační struktury MFF UK, případně plný název pracoviště
% mimo MFF UK v angličtině

Abstract:
% abstrakt v rozsahu 80-200 slov v angličtině; nejedná se však o překlad
% zadání diplomové práce
This thesis deals with the implementation of an algorithm to compute a generator of almost split sequences which end in an indecomposable nonprojective module of path algebra over finite quiver. The algorithm is implemented in the algebra system GAP (Groups, Algorithms, Programming) with the additional package QPA (Quivers and Path Algebras).

Keywords:
% 3 až 5 klíčových slov v angličtině
almost split sequences, representation theory, algorithm, QPA

\vss}
  \include{empty} 

\pagenumbering{gobble}
  \tableofcontents
\cleardoublepage
\pagenumbering{arabic}
  \include{empty} 
  \include{empty} 
  
  \clearpage
\setcounter{page}{1} 

\chapter{Úvod}

\paragraph{ } Cílem této práce je implementace algoritmu pro nalezení generátoru skoro 
štěpitelných posloupností v systému \cite{GAP4} \textbf{Groups, Algorithms, Programming -
a System for Computational Discrete Algebra} s využitím balíku \cite{QPA} 
\textbf{Quivers and Path Algebras}.  

První část obsahuje všechnu potřebnou teorii. Nejprve krátký úvod do teorie kategorií,
poté teorii nutnou pro konstrukci algoritmu 
a nakonec teorii potřebnou pro implementaci algoritmu v teorii reprezentací.

V druhé části nalezneme konstrukci algoritmu, tak jak byla popsána v 
diplomové práci \cite{3} \textbf{Computing almost split sequences} norského studenta 
\textbf{Tea Sormbroen Lian}.

V třetí části algoritmus implementujeme a na závěr jsou umístěny příklady 
použití a srovnání rychlosti s algoritmem, který je v \cite{QPA} aktuálně 
implementován.


 
  \chapter{Teorie}

  \paragraph{}Tato kapitola obsahuje potřebnou teorii pro konstrukci algoritmu pro výpočet 
  generátoru skoro štěpitelných posloupností i jeho následnou implementaci. Je 
  rozdělena do dvou částí.
  
  Nejprve si připomeneme základní pojmy z teorie kategorií, které budou nutné pro naši další práci. 
  Teorie je zde čerpána především z \cite{3} a dále \cite{1}, kde je možné nalézt podrobnější informace.  
   
  Druhá část se zabývá algebrami nad komutativním okruhem, jejich 
  moduly a funktory v kategorii modulů. Dále jsou zde popsány vlastnosti
  skoro štěpitelných posloupností. Teorie je kompilovaná z několika 
  zdrojů - tří knih (\cite{2}, \cite{4} a \cite{5}) a dále diplomové práce \cite{3}.
  
  Třetí část obsahuje základy teorie reprezentací artinovských algeber, které 
  využijeme v samotné implementaci algoritmu v knihovně \cite{QPA}. Teorie je čerpána převážně z knih \cite{1} 
  a \cite{2}. \\\\
  \clearpage
  \section{Teorie kategorií}

    \begin{dfn}
      Kategorie $\mathcal C$ je trojice $\mathcal C=(Ob(\mathcal C), Hom(\mathcal C), 
      \circ)$, kde $Ob(\mathcal C)$ je nazývána třída objektů $\mathcal C$, $Hom(\mathcal C)$ 
      je nazývána třída morfismů a $\circ$ je binární operace na morfismech 
      splňující:      
      \begin{description}
        \item[(a)] Každým\,\, dvěma\,\, objektům $X,Y\in Ob(\mathcal C)$\,\, přiřadíme\,\, množinu\,\, 
          morfismů $Hom_{\mathcal C}(X,Y)$ nazývanou morfismy z $X$ do $Y$ takovou, že 
          pro $(X,Y)\neq (Z,U)$ je $Hom_{\mathcal C}(X,Y) \cap Hom_{\mathcal C}(Z,U)= 
          \emptyset$.
        \item[(b)] Pro každou trojici $X,Y,Z\in Ob(\mathcal C)$ je operace
          \begin{eqnarray}
             Hom_{\mathcal C}(Y,Z) \times Hom_{\mathcal C}(X,Y)  &\to& Hom_{\mathcal C}(X,Z) \nonumber \\
             (f,g) &\mapsto&  g\circ f \nonumber
          \end{eqnarray}
            nazvaná skládání morfismů, splňující následující dvě podmínky:
          \begin{description}
            \item[(i)] $h\circ(g\circ f)=(h\circ g)\circ f$ pro každou trojici $f\in Hom_{\mathcal 
              C}(X,Y)$, $g\in Hom_{\mathcal C}(Y,Z)$ a $h\in Hom_{\mathcal 
              C}(Z,U)$.
            \item[(ii)] Pro každý objekt $X\in Ob(\mathcal C)$ existuje morfismus $1_X\in Hom_{\mathcal C}(X,X)$,
              nazávaný identický morfismus na $X$, takový, že pro každé $f\in Hom_{\mathcal 
              C}(X,Y)$, $g\in Hom_{\mathcal C}(Z,X)$ je $f\circ 1_X=f$ a $1_X\circ g=g$.         
          \end{description}
      \end{description}
      
      Namísto $f\in Hom_{\mathcal C}(X,Y)$ píšeme často jen $X\xrightarrow{\text{f}}Y$ 
      nebo $f:X\to Y$. Diagram v kategorii $\mathcal C$ nazveme komutativním, 
      pokud každé složení cest se stejným začátkem i koncem si je rovné. 
      Například následující diagram s $X,Y,Z,U\in Ob(\mathcal C)$ je 
      komutativní, pokud $g\circ f=i\circ h$: \\\\
       \centerline{\xymatrix{
         X \ar@{->}[r]^f \ar@{->}[d]^h 
         & Y \ar@{->}[d]^g \\
         X \ar@{->}[r]^i
         & U 
      }}\\\\
    \end{dfn}
    
    \begin{dfn}
      Kategorii $\mathcal C'$ nazveme podkategorií kategorie $\mathcal C$, pokud 
      splňuje následující podmínky:
      \begin{description}
        \item[(a)] $Ob(\mathcal C')$ je podtřída Ob($\mathcal C)$.
        \item[(b)] Pro $X,Y\in Ob(\mathcal C')$ je $Hom_{\mathcal C'}(X,Y)\subseteq Hom_{\mathcal C}(X,Y)$.
        \item[(c)] Skládání morfismů v $\mathcal C'$ je stejné jako v $\mathcal C$. 
        \item[(d)] Pro každý objekt $X\in Ob(\mathcal C')$ je identický 
          morfismus $1_X\in Hom_{\mathcal C'}(X,X)$ stejný jako v kategorii $\mathcal  C$.
      \end{description}
      Podkategorii $\mathcal C'$ nazveme úplnou, pokud 
      $Hom_{\mathcal C'}(X,Y)=Hom_{\mathcal C}(X,Y)$ 
      pro všechny $X,Y\in Ob(\mathcal C')$
    \end{dfn}
    

%    \begin{dfn}
%     Nechť $X,Y\in Ob(\mathcal C')$ a $h:X\to Y$. Pak h nazveme
%      \begin{description}
%        \item[(a)] endomorfismus, pokud $X=Y$.
%        \item[(b)] monomorfismus, pokud pro každé $f,g:Z\to X$ platí: 
%          Je-li $h\circ f=h\circ g$, pak $f=g$.
%        \item[(b)] epimorfismus, pokud pro každé $f,g:Y\to Z$ platí: 
%          Je-li $f\circ h=g\circ h$, pak $f=g$.
%        \item[(b)] izomorfismus, pokud existuje $v:Y\to X$ takové, že $vu=1_X$ a 
%        $uv=1_Y$. Takové $v$ nazýváme inverzem $u$ a značíme ho $u^{-1}$.
%      \end{description}
%
%      
%      Pokud existuje izomorfismus dvou objektů $X,Y\in Ob(\mathcal C')$, pak je 
%      nazýváme izomorfní a značíme je $X\simeq Y$.
%    \end{dfn}

      \begin{dfn}
        Kategorii $\mathcal C$ nazveme aditivní, pokud splňuje následující 
        podmínky:
        \begin{description}
          \item[(a)] Pro každou konečnou množinu $X,Y\in Ob(\mathcal C)$ 
          existuje objekt $X\oplus Y\in \mathcal C$ (nazývaný direktní suma $X$ 
          a $Y$) společně s morfismy 
          $\nu_X\in Hom_{\mathcal C}(X,X\oplus Y)$, 
          $\nu_Y\in Hom_{\mathcal C}(Y,X\oplus Y)$,
          $\pi_X\in Hom_{\mathcal C}(X\oplus Y,X)$ a 
          $\pi_Y\in Hom_{\mathcal C}(X\oplus Y,Y)$ takovými, že platí:          
          \begin{eqnarray}
            \pi_X\nu_X &=& 1_X \nonumber \\
            \pi_Y\nu_Y  &=& 1_Y \nonumber \\
            \pi_X\nu_Y &=& 0 \nonumber \\
            \pi_Y\nu_X &=& 0 \nonumber \\
             \pi_X\nu_X + \pi_Y\nu_Y &=& 1_{X\oplus Y} \nonumber 
          \end{eqnarray}
          Morfismy $\pi_X$ a $\pi_Y$ jsou nazývány kanonické projekce a $\nu_X$ 
          a $\nu_Y$ kanonické inkluze.
          
          \item[(b)] Pro všechny $X,Y\in Ob(\mathcal C)$ má množina $Hom_{\mathcal C}(X,Y)$ 
          strukturu abelovské grupy.
          \item[(c)] Pro každou trojici objektů $X,Y,Z\in Ob(\mathcal C)$ je 
          skládání morfismů \\\\
          \centerline{$\circ:Hom_{\mathcal C}(Y,Z)\times Hom_{\mathcal C}(X,Y)\longrightarrow Hom_{\mathcal C}(X,Z)$} 
          \\\\ bilineární operace.
          \item[(d)] Existuje nulový objekt $0\in Ob(\mathcal C)$ takový, že pro 
          všechny $X\in Ob(\mathcal C)$ je $| Hom_{\mathcal C}(X,0)|=| Hom_{\mathcal 
          C}(0,X)|=1$.
        \end{description}
      \end{dfn}
      
      \begin{dfn}
        Mějme aditivní kategorii $\mathcal C$. Opačná kategorie $\mathcal C^{op}$ 
        ke kategorii $\mathcal C$ má stejné objekty $Ob(\mathcal C^{op})=Ob(\mathcal C)$, 
        ale 
        $Hom_{\mathcal C^{op}}(X,Y)=Hom_{\mathcal C}(Y,X)$ pro všechny 
        $X,Y\in Ob(\mathcal C)$. Sčítání v $Hom_{\mathcal C^{op}}(X,Y)$ je definováno stejně jako
        v $Hom_{\mathcal C}(Y,X)$. Skládání $\circ'$ v $\mathcal C'$ je 
        definováno vztahem $g\circ' f=f\circ g$, kde $\circ$ je skládání v $\mathcal 
        C$.
      \end{dfn}
    
    \begin{dfn}
      Nechť $\mathcal C$ je aditivní kategorie a $f\in Hom_{\mathcal C}(X,Y)$.
      \begin{description}
        \item[(a)] Jádro morfismu $f$ je objekt $Ker(f)\in Ob(\mathcal C)$ 
          společně s morfismem $\nu_f\in Hom_{\mathcal C}(Ker(f),X)$ 
          (nazývaným kanonické vnoření) takovým, že 
          splňuje následující podnínky:
          \begin{description}
            \item[(i)] $f\nu_f=0$.
            \item[(ii)] Pro každé $T\in Ob(\mathcal C)$ a $t\in Hom_{\mathcal C}(T,X)$ 
            takové, že $ft=0$ existuje právě jedno $s\in Hom_{\mathcal C}(T,Ker(f))$ 
            takové, že $t=\nu_f s$. \\\\
            \centerline{\xymatrix{
               Ker(f) \ar@{->}[r]^{\nu_f} 
                 & X \ar@{->}[r]^f 
                 & Y & &  \\
               T \ar@{->}[ru]^t  \ar@{.>}[u]^s         
             }}\\
          \end{description}
        \item[(b)] Kojádro morfismu $f$ je objekt $Cok(f)\in Ob(\mathcal C)$ 
          společně s morfismem $\pi_f\in Hom_{\mathcal C}(X,Cok(f))$
          (nazývaným kanonická projekce) takovým, že 
          splňuje následující podnínky:
          \begin{description}
            \item[(i)] $\pi_ff=0$.
            \item[(ii)] Pro každé $T\in Ob(\mathcal C)$ a $t\in Hom_{\mathcal C}(Y,T)$ 
            takové, že $tf=0$ existuje právě jedno $s\in Hom_{\mathcal C}(Cok(f),T)$ 
            takové, že $t=s\pi_f $. \\\\
            \centerline{\xymatrix{
               X \ar@{->}[r]^f 
                 & Y \ar@{->}[r]^{\pi_f}  \ar@{->}[rd]^t
                 & Cok(f) \ar@{.>}[d]^s   \\
               & & T  & &        
             }}\\
          \end{description}
      \end{description}
    \end{dfn}
      
    \begin{dfn}
      Aditivní kategorii $\mathcal C$ nazveme abelovskou, pokud pro každé $X,Y\in Ob(C)$ 
        a $f\in Hom_C(X,Y)$ existují $Ker(f)$ i $Cok(f)$ a navíc 
        $Cok(\nu_f)\simeq Ker(\pi_f)$, kde $\nu_f:Ker(f)\rightarrow X$ resp. $\pi_f:Y\rightarrow Cok(f)$ 
        je kanonické vnoření resp. kanonická projekce.
    \end{dfn}
    
    \begin{dfn}
      Posloupnost objektů a morfismů \\\\
            \centerline{\xymatrix{
               \cdots \ar@{->}[r]^{f_{n+1}}
                 & X_{n+1} \ar@{->}[r]^{f_n} 
                 & X_n \ar@{->}[r]^{f_{n-1}}
                 & X_{n-1} \ar@{->}[r]^{f_{n-2}}
                 & \cdots
             }}\\\\      v abelovské kategorii $\mathcal C$ nazveme exaktní, pokud pro 
             všechna
      $n$ je $Ker(f_{n-1})=Im(f_n)$. Krátká exaktní posloupnost je exaktní 
      posloupnost tvaru $0\to X\to Y\to Z\to 0$.
    \end{dfn}
    
    \begin{dfn}
      Nechť jsou $\mathcal C$ a $\mathcal C'$ dvě kategorie.  Kovariantní funktor 
      $F:\mathcal C \to \mathcal C'$ přiřazuje každému 
      objektu $X\in Ob(\mathcal C)$ objekt $F(X)\in Ob(\mathcal C')$ a každému 
      morfismu $h:X\to Y$ v $\mathcal C$ morfismus $F(h):F(X)\to F(Y)$ v $\mathcal C'$ 
      takový, že: 
      \begin{description}
        \item[(a)] $T(1_X)=1_{T(X)}$ pro  každý $X\in Ob(\mathcal C)$.
        \item[(b)] pro každou dvoji morfismů $f:X\to Y$ a $g:Y\to Z$ v $\mathcal C$ 
        platí, že $T(g\circ f)=T(g)\circ T(f)$.
      \end{description}
      Kontravariantní funktor $F:\mathcal C \to \mathcal C'$ přiřazuje každému 
      objektu $X\in Ob(\mathcal C)$ objekt $F(X)\in Ob(\mathcal C')$ a každému 
      morfismu $h:X\to Y$ v $\mathcal C$ morfismus $F(h):F(Y)\to F(X)$ v $\mathcal C'$ 
      takový, že: 
      \begin{description}
        \item[(a)] $T(1_X)=1_{T(X)}$ pro  každý $X\in Ob(\mathcal C)$.
        \item[(b)] pro každou dvojici morfismů $f:X\to Y$ a $g:Y\to Z$ v $\mathcal C$ 
        platí, že $T(g\circ f)=T(f)\circ T(g)$.
      \end{description}
    \end{dfn}
    
    \begin{dfn}
      Nechť $T,T':\mathcal C\to \mathcal C'$ jsou dva kovariantní (resp. kontravariantní) 
      funktory. Pak třída 
      morfismů $\Psi=\{\Psi_X:T(X)\to T'(X)\}_{X\in Ob(\mathcal C)}$ je 
      přirozenou transoformací $T$ do $T'$, pokud následující 
      diagram v $\mathcal C'$ komutuje pro každý morfismus $f:X\to Y$: \\\\
            \centerline{\xymatrix{
               T(X) \ar@{->}[r]^{\Psi_X} \ar@{->}[d]_{T(f)}
                 & T'(X) \ar@{->}[d]^{T'(f)} \\
               T(Y) \ar@{->}[r]_{\Psi_Y} 
                 & T'(Y)
             }\\
             resp.\\             
             \xymatrix{
               T(Y) \ar@{->}[r]^{\Psi_Y} \ar@{->}[d]_{T(f)}
                 & T'(Y) \ar@{->}[d]^{T'(f)} \\
               T(X) \ar@{->}[r]_{\Psi_X} 
                 & T'(X)
             }}
             \\\\\\
      Přirozenou transformaci $\Psi$ nazveme přirozenou ekvivalencí 
      (nebo též přirozeným izomorfismem), pokud 
      pro každé $X\in \mathcal C$ je $\Psi_X$ izomorfismus.
      
      Kovariantní funktor $T:\mathcal C\to \mathcal C'$ nazveme ekvivalencí 
      kategorií, pokud existuje funktor $F:\mathcal C'\to \mathcal C$ a 
      přirozené ekvivalence $\Psi:1_{\mathcal C}\to FT$ a $\Phi:1_{\mathcal C'}\to 
      TF$, kde $1_{\mathcal C}$ a $1_{\mathcal C'}$ jsou funktory identity na $C$ resp. $C'$.
      
      Kontravariantní funktor $D:\mathcal C\to \mathcal D$ nazveme dualitou 
      kategorií, pokud indukovaný kovariantní funktor $D:\mathcal C^{op}\to \mathcal D$ 
      je ekvivalence kategorií.
    \end{dfn}
    
    \begin{dfn}
      Nechť $\mathcal C$ a $\mathcal D$ jsou abelovské kategorie a $F:\mathcal C\to\mathcal D$ 
      kovariantní (resp. kontravariantní) funktor. A nechť  \\\\
            \centerline{\xymatrix{
               0 \ar@{->}[r] & A \ar@{->}[r] & B \ar@{->}[r] & C \ar@{->}[r] & 0
             }}\\\\ je exaktní posloupnost v $\mathcal C$. Pak řekneme, že $F$ je
      \begin{description}
        \item[(a)] zleva exaktní, pokud následující posloupnost je exaktní v $\mathcal D$:\\\\
            \centerline{\xymatrix{
               0 \ar@{->}[r] & F(A) \ar@{->}[r] & F(B) \ar@{->}[r] & F(C)
             }}\\\\
             \centerline{(resp. \xymatrix{
               0 \ar@{->}[r] & F(C) \ar@{->}[r] & F(B) \ar@{->}[r] & F(A) 
             })}
        \item[(b)] zprava exaktní, pokud následující posloupnost je exaktní v $\mathcal D$:\\\\
            \centerline{\xymatrix{
               F(A) \ar@{->}[r] & F(B) \ar@{->}[r] & F(C) \ar@{->}[r] & 0
             }}\\\\
             \centerline{(resp. \xymatrix{
               F(C) \ar@{->}[r] & F(B) \ar@{->}[r] & F(A) \ar@{->}[r] & 0) 
             })}
        \item[(c)] exaktní, pokud následující posloupnost je exaktní v $\mathcal D$:\\\\
            \centerline{\xymatrix{
               0 \ar@{->}[r] & F(A) \ar@{->}[r] & F(B) \ar@{->}[r] & F(C) \ar@{->}[r] & 0
             }}\\\\
             \centerline{(resp. \xymatrix{
               0 \ar@{->}[r] & F(C) \ar@{->}[r] & F(B) \ar@{->}[r] & F(A) \ar@{->}[r] & 0) 
             })}
      \end{description}
    \end{dfn}
    
    \begin{dfn}
      \begin{description}
        \item
        \item[(a)] $Set:=$ kategorie množin, kde morfismy jsou množinová 
        zobrazení.
        \item[(b)] $Ab:=$ kategorie abelovských grup, kde 
        morfismy jsou homomorfismy abelovských grup. 
        \item[(c)] $Mod(S):=$ kategorie $S$-modulů okruhu $S$, kde morfismy jsou 
        homomorfismy $S$-modulů.
      \end{description}
    \end{dfn}
    
    \begin{lem}
      Nechť $\mathcal C$ je kategorie a $X\in Ob(\mathcal C)$, pak
      \begin{description}
        \item[(a)] máme kovariantní funktor \\\\
          \centerline{$Hom_{\mathcal C}(X,-):{\mathcal C}\rightarrow Set$,}\\\\
          daný pro $Y,Z\in Ob({\mathcal C})$ a $f\in Hom_{\mathcal C}(Y,Z)$ předpisem:\\\\
          \centerline{\begin{aligned}
            Hom_{\mathcal C}(X,-)(Y) &:= & Hom_{\mathcal C}(X,Y)  \\
            Hom_{\mathcal C}(X,-)(f)  &:= & Hom_{\mathcal C}(X,Y)&\rightarrow Hom_{\mathcal C}(X,Z) \\
            & & g & \mapsto fg
          \end{aligned}}
        \item[(b)] máme kontravariantní funktor \\\\
          \centerline{$Hom_{\mathcal C}(-,X):{\mathcal C}\rightarrow Set$,}\\\\
          daný pro $Y,Z\in Ob({\mathcal C})$ a $f\in Hom_{\mathcal C}(Y,Z)$ předpisem:\\\\
          \centerline{\begin{aligned}
            Hom_{\mathcal C}(-,X)(Y) &:= & Hom_{\mathcal C}(Y,X)  \\
            Hom_{\mathcal C}(-,X)(f)  &:= & Hom_{\mathcal C}(Z,X)&\rightarrow Hom_{\mathcal C}(Y,X) \\
            & & g & \mapsto gf
          \end{aligned}}
        \item[(c)] pokud je ${\mathcal C}$ abelovská, pak jsou $Hom_{\mathcal C}(X,-)$ i $Hom_{\mathcal C}(-,X)$ 
          zleva exaktní funktory z ${\mathcal C}$ do $Ab$ (Kategorie abelovských grup).
      \end{description}
      
      Pro střučnost budeme v dalším textu zapisovat homomorfismus 
      $Hom_{{\mathcal C}}(X,-)(f)$ jako $(f\circ-)_X$ a $Hom_{{\mathcal C}}(-,X)(f)$ jako 
      $(-\circ f)_X$.
    \end{lem}
    
    \begin{proof}
      \begin{description}
        \item
        \item[(a)] Zřejmě $Hom_{\mathcal C}(X,Y)\in Set$ a \\\\
          \centerline{$Hom_{\mathcal C}(X,-)(1_Y)=[g\mapsto 1_Yg=g]=1_{Hom_{\mathcal C}(X,Y)}$} 
          \\\\
          pro každé $Y\in \mathcal C$. Dále je-li $Y,Z,W\in \mathcal C$, $f_1\in Hom_{\mathcal C}(Y,Z)$ 
          a $f_2\in Hom_{\mathcal C}(Z,W)$, pak platí 
          \begin{eqnarray}
            (Hom_{\mathcal C}(X,-)(f_2f_1))(g) 
            &=& f_2f_1g \nonumber \\
            &=& f_2(f_1g) \nonumber \\
            &=& (Hom_{\mathcal C}(X,-)(f_2))(f_1g) \nonumber \\
            &=& Hom_{\mathcal C}(X,-)(f_2)Hom_{\mathcal C}(X,-)(f_1)(g) \nonumber
          \end{eqnarray}
          pro každé $Hom_{\mathcal C}(X,Y)$ a tedy \\\\
          \centerline{$Hom_{\mathcal C}(X,-)(f_2f_1)=Hom_{\mathcal C}(X,-)(f_2)Hom_{\mathcal 
          C}(X,-)(f_1)$.}\\
        \item[(b)] Dokáže se podobně jako (a).
        \item[(c)] Že $Hom_{\mathcal C}(X,-)$ i $Hom_{\mathcal C}(-,X)$ jsou 
          funktory $\mathcal C\to Ab$ je zřejmé, dokážeme exaktnost zleva $Hom_{\mathcal 
          C}(X,-)$. Exaktnost zleva funktoru $Hom_{\mathcal C}(-,X)$ se dokáže ekvivalentně. 
          
          Mějme tedy krátkou exaktní posloupnost v $\mathcal C$: \\\\
         \centerline{\xymatrix{
           0 \ar@{->}[r] 
           & A \ar@{->}[r]^f  
           & B \ar@{->}[r]^g  
           & C \ar@{->}[r] 
           & 0
        }.}\\\\
        Potřebujeme dokázat, že následující posloupnost\\\\
         \centerline{\xymatrix{
           0 \ar@{->}[r] 
           & Hom_{\mathcal C}(X,A) \ar@{->}[r]^{(f\circ -)_X} 
           & Hom_{\mathcal C}(X,B) \ar@{->}[r]^{(g\circ -)_X}  
           & Hom_{\mathcal C}(X,C) \ar@{->}[r] 
        }}\\\\
        je exaktní v $Ab$. Nejprve ukážeme, že $(f\circ -)_X$ je monomorfismus v 
        $Ab$, což je v $Ab$ to samé jako jako injektivní. Nechť $h\in Hom_{\mathcal C}(X,A)$ 
        je takové, že \\\\
        \centerline{$(f\circ -)_X(h)=fh=0$} \\\\
        Pak protože $f$ je monomorfismus, musí být $h=0$. Nechť $h\in \mathcal C}(X,B)$, pak \\\\
        \centerline{$(f\circ -)_X(g\circ - )_X(h)=(f\circ -)_X(gh)=fgh=(fg\circ -)_X(h)$} 
        \\\\
        a tedy \\\\
        \centerline{$(f\circ -)_X(g\circ -)_X=(fg\circ -)_X$.}\\\\
        Z čehož plyne, že \\\\
        \centerline{$Im((f\circ -)_X)\subseteq Ker((g\circ -)_X)$.}\\\\        
        Nyní dokážeme opačnou inkluzi \\\\
        \centerline{$Ker((g\circ -)_X) \subseteq Im((f\circ -)_X)$.}\\\\
        Buď $h\in Ker((g\circ -)_X)$. Pak $gh=0$ a protože $f$ je jádro $g$, pak 
        se $h$ faktorizuje skrze $f$, neboli existuje $j\in Hom_{\mathcal C}(X,A)$ 
        takové, že \\\\
        \centerline{$h=fj=(f\circ -)_X(j)$.} \\\\
        A tedy $h\in Im((f\circ -)_X)$.
      \end{description}
    \end{proof} 
    
     \begin{pzn}
       Buď $\mathcal C$ abelovská kategorie a $A,B,C,D\in Ob(\mathcal C)$ a $u,v,f,g$ 
       momorfismy takové, že následující diagram komutuje: \\\\
       \centerline{\xymatrix{
       A \ar@{->}[r]^f \ar@{->}[d]^u & B \ar@{->}[d]^v  \\
       C \ar@{->}[r]^g & D
      }}\\\\\\
      Protože je kategorie $\mathcal C$ abelovská, tak existují jádra i kojádra morfismů $f$ 
      a $g$. A dá se dokázat, že existují i jednoznačně určené morfismy 
      $u_{ker}:Ker(f)\to Ker(g)$ a $v_{cok}:Ker(f)\to Ker(g)$ takové, že následující diagram komutuje: \\\\
       \centerline{\xymatrix{
       0 \ar@{->}[r] 
         & Ker(f) \ar@{->}[r]^{\nu_f} \ar@{.>}[d]^{u_{ker}}
         & A \ar@{->}[r]^f \ar@{->}[d]^u 
         & B \ar@{->}[d]^v \ar@{->}[r]^{\pi_f}
         & Cok(f) \ar@{->}[r] \ar@{.>}[d]^{v_{cok}}
         & 0 \\
       0 \ar@{->}[r] 
         & Ker(g)\ar@{->}[r]^{\nu_g} 
         & C \ar@{->}[r]^g 
         & D \ar@{->}[r]^{\pi_f}
         & Cok(g) \ar@{->}[r] 
         & 0
      }}\\\\           
    \end{pzn}
    
    \begin{dfn}
      Zobrazení $u_{ker}$ (resp. $v_{cok}$) z předchozí poznámky nazýváme jádrový 
      morfismus $u$ (resp. kojádrový morfismus $v$). Budeme je takto značit, ačkoli je toto značení 
      nepopisuje jednoznačně, protože vždy závisí na diagramu, ke kterému se 
      vztahují. V dalším textu bude ale vždy z kontextu jasné, o který doagram se jedná.
    \end{dfn}
      \clearpage




 
  \section{Algebry a moduly}\label{algebry-moduly}

  \paragraph{ }V této části budeme pracovat s pevně zvoleným
  asociativním okruhem s jednotkou $R$, který navíc bude 
  komutativní, lokální a artinovský. Nejprve si tyto definice připomeneme:
    
  \begin{dfn}
    Komutativní okruh $R$ je lokální, pokud má právě jeden maximální pravý ideál a ten je nenulový.
  \end{dfn}
  
  Pokud je $R$ je lokální s maximálním pravým ideálem $\underline 
  m$, pak $\underline m$ je oboustranný ideál a je zároveň maximálním 
  levým ideálem dle \cite{1} Lemma I.4.6.
  
  My ale budeme pracovat s komutativním okruhem $R$, ten je 
  lokální jednoduše právě tehdy, 
  když má právě jeden maximální nenulový ideál.
  
  \begin{lem}\label{faktor-lokalniho-lokalni}
    Pokud je $R$ lokální okruh, pak $R/J$ je lokální okruh pro každý vlastní 
    ideál $J$ okruhu $R$.
  \end{lem}
  \begin{proof}
    Buď $R$ lokální okruh, $J$ jeho vlastní ideál a nechť $\underline{m}$ 
    značí maximální ideál okruhu $R$. Dokážeme, že $J/\underline{m}$ je 
    maximální jednoznačně určený ideál okruhu $R/J$.
    
    Mějme libovolný ideál $Y$ okruhu $R/J$, pak $Y$ je tvaru $Y=X/J$ pro nějaký 
    ideál $X$ okruhu $R$ takový, že $X\subsetneq Y\subsetneq R$. 
    Jelikož ideál $\underline{m}$ 
    je maximální, musí být $X\subseteq \underline{m}$ a tedy \\\\
    \centerline{$Y=X/J \subseteq \underline{m}/J$.}
  \end{proof}
  
  \begin{dfn}
    Okruh $R$ je zleva (resp. zprava) artinovský, pokud se každý klesající 
    řetězec jeho levých (resp. pravých) ideálů zastaví. Neboli máme-li klesající řetězec 
    levých (resp. pravých) ideálů okruhu $R$ \\\\
    \centerline{$R=I_0\supseteq I_1\supseteq I_2\supseteq\ldots$,} \\\\
    pak existuje $n\in \mathbb{N}$ takové, že $I_j=I_i$ pro každé $i,j\geq n$.
  \end{dfn}
  
  \begin{dfn}
     Jakobsonův radikál $rad(R)$ okruhu R definujeme jako průnik všech 
     maximálních ideálů $R$.    
  \end{dfn}
  
    Dále tedy nechť $R$ značí pevně zvolený komutativní, lokální a artinovský okruh.
    Nyní zavedeme klíčový pojem $R$-algebry.
        
    \begin{dfn}
      $R$-algebra $A$ je okruh, který je zároveň $R$-modulem takovým, že pro 
      $\alpha,\beta,\lambda\in A$ a $r,s\in R$ platí:
      \begin{description}
        \item[(a)] $(r\alpha+s\beta)\lambda=r(\alpha\lambda)+s(\beta\lambda)$ 
        \item[(b)] $\alpha(r\beta+s\lambda)=r(\alpha\beta)+s(\alpha\lambda)$ 
      \end{description}      
      Artinovská $R$-algebra je $R$-algebra, která je konečně generovaná jako 
      $R$-modul.  
      Opačná algebra $A^{op}$ k algebře $A$ je algebra se stejnou 
      modulovou strukturou,  která má ale okruhové násobení definováno opačně: \\\\
      \centerline{$a\cdot_{A^{op}}b:=b\cdot_A a$}  \\\\
      Kategorii levých $A$-modulů značíme $Mod(A)$. O pravých $A$-modulech
      budeme referovat jako o $A^{op}$-modulech. 
      Poznamenejme, že v části \hyperref[teorie-reprezentaci]{části \ref*{teorie-reprezentaci}}  
      zabývající se teorií reprezentací a poté v 
      \hyperref[teorie-reprezentaci]{Kapitole \ref*{implementace}}  zabývájící se implementací algoritmu budeme pracovat s pravýmu 
      moduly a $Mod(A)$ bude tedy značit kategorii pravých modulů. Na vše později upozorníme.
      
      $R$-podmodul $I$ 
      $R$-algebry $A$ je pravým (resp. levým) ideálem $A$, pokud $xa\in I$ (resp. $ax\in I$)
      pro každý $x\in I$ a $a\in A$. Oboustranný ideál je ideál, který je 
      zároveň levým i pravým ideálem.  
    \end{dfn}
    
    Dále bude $A$ značit pevně zvolenou artinovskou $R$-algebru a pojďme se 
    podívat na vztah $R$-modulů a $A$-modulů.
        
    \begin{dfn}
      Nechť $S$ je okruh. Označme $mod(S)$ úplnou podkategorii $Mod(S)$ konečně 
      generovaných modulů.
    \end{dfn}
    
    \begin{lem}\label{lem-mod-Mod} Platí:
      \begin{description}
        \item[(a)] $Mod(A)\subseteq Mod(R)$
        \item[(b)] $Hom_A(M,N)\subseteq Hom_R(M,N)$ pro $M,N\in Mod(A)$ 
        \item[(c)] $mod(A)\subseteq mod(R)$   
        \item[(d)] Pokud $M\in Mod(A)\cap mod(R)$, pak $M\in mod(A)$.  
      \end{description}
      Poznamenejme, že v případech (a) i (c) se jedná o podmnožinu, ne o úplnou 
      podkategorii.
    \end{lem}
    \begin{proof}
      \begin{description}
        \item
        \item[(a)] Pro $M\in Mod(A)$ definujeme násobení $R\times M\rightarrow M$ následovně: 
          \\\\
          \centerline{$(r,m)\mapsto (r\cdot 1_A)m$}\\\\
          Protože $r\cdot 1_A\in A$, je násobení dobře definované. 
        \item[(b)] Nechť $M,N\in Mod(A)$ a $f\in Hom_A(M,N)$. Pak pro $m_1,m_2\in M$ 
          $a\in A$ máme \\
          \centerline{$f(am_1+m_2)=af(m_1)+f(m_2)$.}\\\\
          A tedy pro každé $r\in R$ dle bodu (a) platí, že
          $f(rm_1+m_2)=f((r1_A)m_1+m_2)=(r1_A)f(m_1)+f(m_2)=rf(m_1)+f(m_2)$.
        \item[(c)] Nechť $M\in mod(A)$, pak pro nějaké $n\in\mathbb{N}$ existuje
          $A$-modulový epimorfismus \\
          \centerline{$g:A^n\rightarrow M$.}\\\\
          Ten je dle bodu (b) také $R$-modulovým epomorfismem. Algebra $A$ je 
          artinovská, tedy konečně generovaná jako $R$-modul. Pak existuje  
          $R$-modulový epimorfismus \\
          \centerline{$h:R^m\rightarrow A$}\\\\
          a také $R$-modulový epimorfismus \\\\
          \centerline{$h^n:R^{mn}\rightarrow A^n$.}\\\\
          Složením $g$ a $h^n$ dostaneme $R$-modulový epimorfismus \\\\
          \centerline{$gh^n:R^{mn}\rightarrow M$,}\\\\
          z jehož existence plyne, že $M\in mod(R)$
        \item[(d)] Struktura $R$-modulu je na $M$ definována následovně:\\\\
          \centerline{$rm=(r1_A)m$} \\\\
          Nechť $\{m_1,m_2,\cdots,m_n\}\subseteq M$  generuje $M$ jako $R$-modul. 
          Pak pro každé $m\in M$ platí: 
          \begin{eqnarray}
            m &=& \sum_{i=1}^n r_i m_i=\sum_{i=1}^n \underbrace{(r_i1_A)}_{\in A} m_i\subseteq\sum_{i=1}^n Am_i 
            \nonumber
          \end{eqnarray} 
          A tedy množina $\{m_1,m_2,\cdost,m_n\}\subseteq M$  generuje $M$ i jako 
          $A$-modul.
      \end{description}    
    \end{proof}
    
    \begin{thm}\label{dir-sum-hom} Nechť $S$ je okruh, $\bigoplus_{i=1}^nX_i$ je direktní součet 
       $S$-modulů a pro $i=1,\ldots,n$ jsou $\nu_i:X_i\rightarrow \bigoplus_{i=1}^nX_i$
       kanonické inkluze a $\rho_i:\bigoplus_{i=1}^nX_i\rightarrow X_i$ 
       kakonické projekce. Pak máme pro $Y\in Mod(S)$ následující izomorfismus: \\\\
       \centerline{$\xi:Hom_S(\bigoplus_{i=1}^nX_i,Y)\rightarrow \bigoplus_{i=1}^nHom_S(X_i,Y)$}\\\\
       daný předpisem $f\mapsto\{f\nu_i\}_{i=1}^n$ a jeho inverz\\\\
       \centerline{$\xi^{-1}:\bigoplus_{i=1}^nHom_S(X_i,Y)\rightarrow Hom_S(\bigoplus_{i=1}^nX_i,Y)$}\\\\
       daný předpisem $\{f_i\}_{i=1}^n\mapsto \sum_{i=1}^nf_i\rho_i$.    
       
       Pokud buď $Y=S$, nebo pro všechny $n=1,2,\ldots,n$ je $X_i=S$, pak $\xi$ 
       a $\xi^{-1}$ jsou izomorfismy $S$-modulů.
     \end{thm}
     
     \begin{proof}
       Dokážeme pouze první část, druhá část je přímočaré ověření modulových axiomů. 
       
       Nechť tedy $f\in Hom_S(\bigoplus_{i=1}^nX_i,Y)$. Pak 
       \begin{eqnarray}
         \xi^{-1}\xi(f) 
         &=& \xi^{-1}(\{f\nu_i\}_{i=1}^n) \nonumber \\
         &=& \sum_{i=1}^n f\nu_i\rho_i \nonumber \\
         &=& f \sum_{i=1}^n \nu_i\rho_i \nonumber \\
         &=& f 1_{\otimes_{i=1}^n X_i} \nonumber \\
         &=& f.  \nonumber
       \end{eqnarray}
       
       Opačně nechť $\{f_i\}_{i=1}^n\in {\bigoplus_{i=1}^n Hom_S(X_i,Y)$, pak
       \begin{eqnarray}
         \xi\xi^{-1}(\{f_i\}_{i=1}^n)
         &=& \xi \left( \sum_{i=1}^n f_i\rho_i \nonumber \right)\\
         &=& \left\{  \left( \sum_{i=1}^n f_i\rho_i \right) \nu_j \right\}_{j=1}^n \nonumber \\
         &=& \left\{ \sum_{i=1}^n f_i(\rho_i \nu_j) \right\}_{j=1}^n  \nonumber \\
         &=& \left\{ \sum_{i=1}^n f_i \delta_{i,j} \right\}_{j=1}^n  \nonumber \\
         &=& \left\{ f_j \}_{j=1}^n  \nonumber
       \end{eqnarray}
     \end{proof}
    
    Dále budeme pracovat s pojmem tenzorového součinu, ten zde nebudeme zavádět. 
    Definici a všechny potřebné vlastnosti je možné nalézt například v \cite{5} 
    Kapitole 2.2, nebo v \cite{3} Kapitole 2.8. 
    
    Připomeňme pouze, že pro $M\in Mod(A^{op})$, $N\in Mod(A)$
    tenzorový součin $M\otimes_A N$ vždy existuje, má strukturu $R$-modulu a 
    máme následující dva funktory
    \\\\ \centerline{$M\otimes_A -: Mod(A)\to Mod(R)$} \\\\ 
    \centerline{$-\otimes_A N: Mod(A^{op})\to Mod(R)$} \\\\ 
     z nichž první je zleva exaktní. 
          
     \begin{thm}\label{thm-adjunkce}
       Nechť $M\in Mod(A^{op})$, $N\in Mod(A)$ a $L\in Mod(R)$. Pak máme 
       následující izomorfismus abelovských grup 
       \\\\
       \centerline{$\theta_{M,N,L}:Hom_R(M\otimes_A N, L)\rightarrow Hom_A(N,Hom_R(M,L))$} 
       \\\\
       daný předpisem \\
       \centerline{$\theta_{M,N,L}(f):=[n\mapsto f(-\otimes n)]$}
       \\\\
       pro $f\in Hom_R(M\otimes_A N,L)$. Navíc $\theta_{M,N,L}$ je přirozený v 
       $M$, $N$ i $L$.
     \end{thm}
     \begin{proof}
       \cite{5} Theorem 2.75, 2.76.
     \end{proof}     
     
     \begin{lem}[Lemma five]\label{lemma-five} Uvažujme následující diagram 
       v $Mod(A)$ s exaktními řádky:\\
       \centerline{\xymatrix{
       A_1 \ar@{->}[r]_{f_1} \ar@{->}[d]_{t_1} 
         & A_2 \ar@{->}[r]_{f_2} \ar@{->}[d]_{t_2} 
         & A_3 \ar@{->}[r]_{f_3} \ar@{->}[d]_{t_3} 
         & A_4 \ar@{->}[r]_{f_4} \ar@{->}[d]_{t_4} 
         & A_5 \ar@{->}[d]_{t_5}  \\
       B_1 \ar@{->}[r]_{h_1}
         & B_2 \ar@{->}[r]_{h_2} 
         & B_3 \ar@{->}[r]_{h_3}  
         & B_4 \ar@{->}[r]_{h_4} 
         & B_5  \\
      }}\\\\
      Pak platí:
      \begin{description}
        \item[(a)] Jsou-li $t_2$ a $t_4$ epimorfismy a $t_5$ monomorfismus, pak 
        je $t_3$ epimorfismus.
        \item[(b)] Jsou-li $t_2$ a $t_4$ monomorfismy a $t_1$ epimorfismus, pak 
        je $t_3$ monomorfismus.
        \item[(c)] Jsou-li $t_1$, $t_3$, $t_4$ a $t_5$ izomorfismy, pak je i $t_3$ 
        izomorfismus.
      \end{description}
     \end{lem}
     
     \begin{proof}
       Tvrzení (c) jasně plyne z (a) a (b). Dokážeme zde pouze (a), jelikož 
       (b) se dokáže analogicky.
       
       Nechť $b_3\in B_3$. Protože $t_4$ je epimorfismus, tak existuje $a_4\in A_4$ 
       takové, že $t_4(a_4)=h_3(b_3)$. Z komutativity posledního čtverce a 
       exaktnosti spodního řádku v $B_4$ dostaneme \\
       \centerline{$t_5h_4(a_4)=h_4t_4(a_4)=h_4h_3(b_3)=0$.} \\\\
       Protože $f_5$ je monomorfismus musí být $f_4(a_4)=0$ a $a_4\in 
       Ker(f_4)=Im(f_3)$. Tedy existuje $a_3\in A_3$ takové, že $a_4=f_3(a_4)$ a 
       tedy platí, že \\\\
       \centerline{$h_3t_3(a_3)=t_4f_3(a_3)=t_4(a_4)=h_3(b_3)$.} \\\\
       Pak $b_3-t_3(a_3)\in Ker(h_3)=Im(h_2)$ a tedy existuje $b_2\in B_2$ takové, že 
       $h_2(b_2)=b_3-t_3(a_3)$. Protože $t_2$ je epimorfismus, musí existovat $a_2\in A_2$, 
       pro které $t_2(a_2)=b_2$. Z komutativity diagramu plyne \\\\
       \centerline{$t_3f_2(a_2)=h_2t_2(a_2)=h_2(b_2)=b_3-t_3(a_3)$} \\
       a \\
       \centerline{$t_3(f_2(a_2)+a_3)=b_3$.} \\\\     
       Homomorfismus $t_3$ je tedy epimorfismem.  
     \end{proof}
     
     \begin{lem}\label{lem-komut-schod}
       Nechť $S$ je okruh a nechť \\\\
       \centerline{\xymatrix{
         0 \ar@{->}[r] 
           & M \ar@{->}[r]^f
           & N \ar@{->}[r]^g \ar@{->}[d]_\alpha^\simeq
           & L \ar@{->}[d]_\beta^\simeq \\
        & & N' \ar@{->}[r]^u
           & L' \ar@{->}[r]^v
           & K' \ar@{->}[r]
           & 0   
      }}\\\\\\
      je komutativní diagram s exaktními řádky v $Mod(S)$. Pak\\\\
       \centerline{\xymatrix{
         0 \ar@{->}[r] 
           & M \ar@{->}[r]^f
           & N \ar@{->}[r]^g 
           & L \ar@{->}[r]^{v\beta}
           & K' \ar@{->}[r]
           & 0   
      }}\\\\      
      je exaktní posloupnost v $Mod(S)$.
     \end{lem}
     
     \begin{proof}
       Exaktnost v $M$ a $N$ plyne z exaktnosti původního diagramu. 
       Protože $v$ a $\beta$ jsou epimorfismy, pak je $v\beta$ také epimorfismem, takže 
       posloupnost je exaktní i v $K'$. 
       
       Protože $(v\beta)g=(vu)\alpha=0$ máme $Im(g)\subseteq Ker(v\beta)$. Zbývá dokázat 
       opačnou inkluzi.
       
       Nechť $c\in Ker(v\beta)$, pak $v\beta(c)=0$ a tedy $\beta(c)\in 
       Ker(v)=Im(u)$. Potom existuje $b'\in N'$ takové, že \\\\
       \centerline{$u(b')=\beta(c)$.}\\\\
       Pak $\alpha^{-1}(b')\in N$ a
       \\\\
       \centerline{$\beta g \alpha^{-1}(b')=u(b')=\beta(c)$.}\\\\
       Protože $\beta$ je epimorfismus, tak nám tato úvaha dává
       \\\\
       \centerline{$g(\alpha^{-1}(b'))=c$}\\\\
       a tedy $c\in Im(g)$.
     \end{proof}
     
   \subsection{Projektivní moduly}
   
     \begin{dfn}
       $A$-modul $P$ nazveme projektivním, pokud pro každé $A$-moduly $M,N$, 
       epimorfismus $h\in Hom_A(M,N)$ a homomorfismus $f\in Hom_A(P,N)$ existuje 
       $f'\in Hom(P,M)$ takový, že následující diagram komutuje:\\\\
       \centerline{\xymatrix{
         & P \ar@{.>}[ld]_{f'} \ar@{->}[d]^f \\
         M \ar@{->}[r]_h & N \ar@{->}[r] & 0     
      }}
     \end{dfn}
   
     \begin{lem}\label{lem-faktorizuje-skrze-proj}
       Nechť $M,N\in mod(A)$, $P\in Mod(A)$, $f\in Hom_A(M,N)$ se faktorizuje skrze projektivní modul 
       a $g\in Hom_A(P,N)$ je \textbf{na}. Pak $f$ se faktorizuje skrze $P$.
     \end{lem}
     
     \begin{proof}
       Uvažujme následující diagram v $Mod(A)$:\\\\
       \centerline{\xymatrix{
         & 
           & M \ar@{->}[dd]^f \ar@{->}[dl] \\
         & P' \ar@{->}[rd] \ar@{.>}[ld]_{\exists h} \\
         P \ar@{->}[rr]_g
           & 
           & N \ar@{->}[r]
           & 0      
      }}\\\\
      Homomorfismus $f$ se faktorizuje skrze nějaký projektivní modul $P'$. 
      Protože $P'$ je projektivní a $g$ je \textbf{na}, existuje $h:P'\rightarrow P$ 
      takové, že diagram výše komutuje a $f$ se tedy faktorizuje skrze $P$.
     \end{proof}
        
     \begin{dfn}
       Prvek $e\in A$ nazveme idempotentem, pokud $e^2=e$. 
       Množinu idempotentů $\{e_1,e_2,\ldots,e_n\}$
       nazveme ortogonální, pokud pro každé $i\neq j$ je $e_ie_j=e_je_i=0$. Primitivní 
       idempotent je takový, který není možné zapsat jako součet alespoň 
       dvou nenulových ortogonálních idempotentů. 
       Idempotent je centrální, pokud $ea=ae$  pro každý prvek $a\in A$.
     \end{dfn}
     
     \begin{thm}\label{rozklad-A-na-proj}
       Nechť $A$ je Artinovská algebra. Pak $A=Ae_1\oplus Ae_2\oplus\ldots\oplus 
       Ae_n$ právě tehdy, když $e_1,$ $e_2,\ldots,e_n$ je množina po dvou ortogonálních 
       primitivních idempotentů $A$ a navíc $1=e_1+e_2+\ldots+e_n$.
     \end{thm}
     
     \begin{proof}
       Každá algebra má alespoň dva idempotenty a to $0$ a $1$. Nemá-li žádné 
       jiné, pak máme množinu nenulových primitivních ortogonálních idempotentů 
       $\{1\}$ a rozklad $A=A1$.
       
       Existuje-li netriviální idempotent $e$ algebry $A$, pak je netriviálním 
       idempotentem také $1-e$, protože 
       $(1-e)^2=1-2e+e^2=1-2e+e=1-e$. Ty jsou 
       navíc ortogonální, protože $e(1-e)=e-e^2=e-e=0$. 
       Algebru $A$ tedy můžeme 
       rozložit na $A=Ae\oplus A(1-e)$ a $1=e + (1-e)$.
       
       Protože algebra $A$ je konečně generovaná, můžeme ji rozložit na direktní součet 
       nerozložitelných levých ideálů $A=P_1\oplus P_2\oplus\ldots\oplus P_n$. 
       Ty musejí být z diskuze výše tvaru $P_i=Ae_i$ pro $e_1,e_2,\ldots,e_n$ po 
       dvou ortogonální primitivní idempotenty takové, že $1=e_1+e_2+\ldots+e_n$.
       
       Naopak máme-li $1=e_1+e_2+\ldots+e_n$ rozklad $1$ na součet po dvou 
       ortogonálních primitivních idempotentů je existence rozkladu $A=Ae_1\oplus Ae_2\oplus\ldots\oplus 
       Ae_n$ zřejmá.
     \end{proof}
       
     \begin{dfn}
       Množinu po dvou ortogonálních primitivních idempotentů $e_1,$ $e_2,\ldots,e_n$ 
       takových, že $A=Ae_1\oplus Ae_2\oplus\ldots\oplus Ae_n$ nazveme úplnou 
       množinou primitivních ortogonálních idempotentů $A$.
     \end{dfn}
                       
     \begin{thm}\label{rozklad-proj}
       Nechť $e_1,e_2,\ldots,e_n$ je úplná množina primitivních ortogonálních idempotentů $A$ 
       a $P$ projektivní $A-modul$. Pak existují $m_1,m_2,\ldots,m_n\in\mathbb N$ taková, že\\
       \centerline{$P=(Ae_1)^{m_1}\plus (Ae_2)^{m_2}\oplus\ldots\oplus (Ae_n)^{m_n}$.}       
     \end{thm}
     \begin{proof}
       Dle \cite{5} Theorem 3.5 je modul $P$ projektivní právě tehdy, když existuje $n\in\mathbb 
       N$ takové, že $P$ je direktním sčítancem $A^n$. Výsledný tvar dostáváme 
       aplikací \hyperref[rozklad-A-na-proj]{Věty \ref*{rozklad-A-na-proj}} na $A^n$.
     \end{proof}
     
    \begin{dsl}\label{Ae-projektivni}
       Prvek $e\in A$ je primitivní idempotent právě tehdy, když $Ae$ je 
       nerozložitelný projektivní $A$-modul.
     \end{dsl} 
     
     \begin{thm}\label{izo-hom-aei}
       Buď $e\in A$ je idempotent, pak zobrazení \\\\
       \centerline{$Hom_A(Ae,A)\simeq eA$}\\
       dané předpisem\\
       \centerline{$f\mapsto f(e)$} \\\\
       je izomorfismem $A$-modulů.
     \end{thm}
     \begin{proof}
       Že se jedná o homomorfismus je zřejmé.
       Označme si tento homomorfismus $Hom_A(Ae,A)\simeq eA$ jako $\psi_1$.  
       Definujme zobrazení
       \begin{eqnarray}
         \psi_2:eA &\simeq& Hom_A(Ae,A) \nonumber \\
         ea &\mapsto& [a'e\mapsto a'ea].  \nonumber
       \end{eqnarray}
       Pak pro $f\in Hom_A(Ae,A)$ a $a\in A$ máme
       \begin{eqnarray}
         \psi_1\psi_2(ea)&=&\psi_1([a'e\mapsto a'ea]) \nonumber \\
         &=&ea \nonumber
       \end{eqnarray}
       a
       \begin{eqnarray}
         \psi_2\psi_1(f)(ae)&=&\psi_2(f(e))(ae) \nonumber\\
         &=&\psi_2(ef(e))(ae) \nonumber\\
         &=&[xe\mapsto xe(fe)] \nonumber\\
         &=&aef(e) \nonumber \\
         &=&f(ae). \nonumber
       \end{eqnarray}
       Jde tedy o izomorfismus.
     \end{proof}
     
     \begin{thm}\label{rozklad-A-na-P}
       Buď $P$ projektivní $A$-modul. Pak existuje druhý projektivní $A$-modul $P'$ 
       a $n\in \mathbb N$ takové, že $P\oplus P'\simeq A$.
     \end{thm}
     \begin{proof}
       Dle \hyperref[rozklad-A-na-proj]{Věty \ref*{rozklad-A-na-proj}} můžeme algebru $A$ vyjádřit jako \\\\
        \centerline{$A=A e_1\oplus A e_2\oplus\ldots\oplus A e_m$,} \\\\
        kde $A e_i$ jsou 
        nerozložitelné projektivní $A$-moduly takové, že $Ae_i\not\simeq Ae_j$ 
        pro $i,j\in{1\ldots m}$ a $i\neq j$. Pak lze projektivní modul $P$  dle
        \hyperref[rozklad-proj]{Věty \ref*{rozklad-proj}} vyjádřit jako 
        \\\\
        \centerline{$P\simeq(A e_1)^{n_1}\oplus(A e_2)^{n_2}\oplus\ldots\oplus(A 
        e_m)^{n_m}$.} \\\\       
       Položme $n=max_{1\leq i \leq m}(n_i)$, pak \\\\
       \centerline{$
         A^n\simeq P_0
           \oplus(A e_1)^{n-n_1}
           \oplus(A e_2)^{n-n_2}
           \oplus\ldots
           \oplus(A e_m)^{n-n_m}
       $.} \\\\
       Definujme projektivní $A$-modul \\\\ 
       \centerline{$
         P_0':= 
           (A e_1)^{n-n_1}
           \oplus(A e_2)^{n-n_2}
           \oplus\ldots 
           \oplus(A e_m)^{n-n_m}
       $,} \\\\ 
       pak $P_0\oplus P_0'\simeq A^n$.
     \end{proof}
     
   \subsection{Funktor Ext}
   
     \begin{dfn}
       Injektivní rezolventa E modulu $M\in Mod(R)$ je exaktní posloupnost  \\
       \centerline{\xymatrix{
       \textbf E: 0 \ar@{->}[r] 
         & M \ar@{->}[r]^{\eta}
         &  E^0 \ar@{->}[r]^{d^0}
         &  E^1 \ar@{->}[r]^{d^1}
         &  E^2 \ar@{->}[r]^{d^2}
         &  \cdots,
      }}\\\\
      kde každé $E^n$ je injektivní. 
    \end{dfn}
     
     \begin{lem}
       Pro každý modul $M\in mod(R)$ existuje injektivní rezolventa.
     \end{lem}
     \begin{proof}
       Dle \cite{5} Theorem 3.38 může být každý $R$-modul vnořený jako 
       podmodul do injektivního $R$-modulu. Tedy existuje injektivní $R$-modul $E^0$ a 
       kanonické vnoření $\nu^0:M\rightarrow E^0$. Dostáváme exaktní posloupnost\\\\
       \centerline{\xymatrix{
         0 \ar@{->}[r]
         & M \ar@{->}[r]^{\nu^0}
         & E^0 \ar@{->}[r]^{\pi^0}
         & Cok(\nu^0) \ar@{->}[r]
         & 0  
      },}\\\\
      kde $\pi^0$ je kanonická projekce. Tento postup můžeme iterativně opakovat 
      pro $Cok(\nu^0)$ namísto M a dále s 
      $Cok(\nu^1),$ $Cok(\nu^2),$ $\ldots$:\\\\
       \centerline{\xymatrix{
         0 \ar@{->}[r]
         & M \ar@{->}[r]^{\nu^0}
         & E^0 \ar@{->}[rr]^{d^0:=\nu^1\pi^0} \ar@{->}[rd]_{\pi^0}
         & & E^1 \ar@{->}[rr]^{d^0:=\nu^2\pi^1} \ar@{->}[rd]_{\pi^1}
         & & \ldots \\
         &
         &
         & Cok(\nu^0) \ar@{->}[ru]_{\nu^1}
         &
         & Cok(\nu^1) \ar@{->}[ru]_{\nu^2}
      }}     
    \end{proof}     
    
    \begin{dfn}
      Buď $\mathcal A$ libovolná abelovská katergorie. Komplex $C$ v kagorii $\mathcal A$ je 
      (konečná či nekonečná) posloupnost morfismů a objektů\\\\
       \centerline{\xymatrix{
       \textbf C:  \cdots \ar@{->}[r]^{d_{n+2}}
         &  C_{n+1} \ar@{->}[r]^{d_{n+1}}
         &  C_{n} \ar@{->}[r]^{d_{n}}
         &  C_{n-1} \ar@{->}[r]^{d_{n-1}}
         &  \cdots
      },}\\\\
      kde $d_nd_{n+1}=0$ pro všechna $n\in\mathbb Z$. Dále položme:
      \begin{eqnarray}
        Z_n(C) &:=& Ker(d^n) \nonumber \\
        B_n(C) &:=& Im(d^{n+1}) \nonumber\\
        H_n(C) &:=& Z_n(E)/B_n(E) = Ker(d^n)/Im(d^{n+1}) \nonumber
      \end{eqnarray} 
     \end{dfn}
     
     \begin{dfn}
       Nechť $M,N\in Mod(R)$ a zvolme libovolně injektivní rezolventu modulu $N$\\
       \centerline{\xymatrix{
       \textbf E: 0 \ar@{->}[r] 
         & N \ar@{->}[r]^{\eta}
         &  E^0 \ar@{->}[r]^{d^0}
         &  E^1 \ar@{->}[r]^{d^1}
         &  E^2 \ar@{->}[r]^{d^2}
         &  \cdots
      }.}\\\\
      Aplikujeme-li kovariantní funktor $Hom(M,-)$ na injektivní rezolventu $\textbf 
      E$, dostaneme komplex $Hom_R(M,\textbf E)$:\\\\
       \centerline{\xymatrix{
         0\ar@{->}[r]
         &  Hom_R(M,E^0) \ar@{->}[r]^{d^0_*}
         &  Hom_R(M,E^1) \ar@{->}[r]^{d^1_*}
         &  Hom_R(M,E^2) \ar@{->}[r]^{d^2_*}
         &  \cdots
      }.}\\\\
      Definujme nyní \\\\
      \centerline{$Ext^n_R(M,N):=H^n(Hom_R(M,\textbf E))=\frac{Ker(d_*^n)}{Im(d_*^{n-1})}$,}\\
      kde \\
      \centerline{$d_*^n:Hom_R(M,E^n)\rightarrow Hom_R(M,E^{n+1})$}\\\\
      je dáno předpisem $f\mapsto d^nf$.
     \end{dfn}
     
     Nebudeme zde dokazovat podrobně vlastnosti funktoru $Ext$, vše je obsaženo 
     například v \cite{5} kapitole 6.2.3.
     Pouze bez důkazu uvedeme následující tvrzení: 
     
     \begin{thm}
       \begin{description}
         \item 
         \item[(a)] $Ext^n_R(M,N)$ je nezávislý na volbě injektivní rezolventy 
           modulu $N$.
         \item[(b)] Pro $R$ komutativní je $Ext^n_R(M,-)$ funktor $Mod(R)\rightarrow 
           Mod(R)$.
       \end{description}
     \end{thm}

   \subsection{Funktor $D$}
     
     \begin{dfn}
       Nechť $S$ je okruh.
       \begin{description}
         \item[(a)] Nechť  navíc $M,N\in Mod(S)$ a $M\subseteq N$. Řekneme, že $N$ je esenciálním rozšířením 
         $M$, pokud $X\cap M\neq0$ pro každý podmodul $0\neq X\subseteq N$. 
         \item[(b)] Injektivní obal $I$ modulu $N\in Mod(S)$ je injektivní $S$-modul $I$ společně s 
         monomorfismem $i:M\rightarrow I$, kde $I$ je esenciální rozšíření $Im(i)$. 
       \end{description}
     \end{dfn}
     
     Náš okruh $R$ je komutativní a lokální, má tedy jednoznačně určený maximální ideál, který 
     budeme značit $\underline{m}$. 
     Jako $I$ budeme dále značit injektivní obal 
     jednoduchého $R$-modulu $R/\underline{m}$ (jednoduchost plyne z maximality $\underline{m}$).
     
     \begin{dfn}
       Definujme funktor $D:=Hom_R(-,I)$. Funktor D se nazývá duál.
     \end{dfn}
     
     \begin{thm}\label{lem-dual}
       Funktor $D$ je exaktní a kontravariantní funktor 
       \begin{description}
         \item[(a)] $D:mod(R)\rightarrow mod(R)$ 
         \item[(b)] $D:mod(A)\rightarrow mod(A^{op})$ 
         \item[(c)] V obou předchozích případech je funktor $D$ navíc dualita kategorií.
       \end{description}
     \end{thm}
     \begin{proof}
         Víme, že v obou případech $D$ je zleva exaktní kontravariantní funktor 
         v kategorii abelovských grup. Navíc $D$ je exaktní, protože
       modul $I$  je injektivní dle \cite{5} Proposition 3.25. Dále dokážeme, že 
       $D$ je zároveň funktorem v obou výše uvedených kategoriích:
        
       \begin{description}
         
         \item[(a)] Nechť $M\in Mod(R)$, pak $Hom_M(R,I)\in mod(R)$. Dokážeme,
         že máme-li navíc libovolný modul $M'\in Mod(R)$ a $h\in Hom_R(M,M')$, pak $Dh\in 
         Hom_R(DM,DM')$.
         Nechť tedy $f\in DM'$ a $r\in R$. Pak
         \begin{eqnarray}
           Dh(rf')(m) &=& (-\circ h)_I(rf')(m)\nonumber \\
           &=& (rf'h)(m') \nonumber \\ 
           &=& r(f'h)(m')\nonumber \\
           &=& r(-\circ h)_I(f')(m)\nonumber \\
           &=& rDh(f')(m)\nonumber
         \end{eqnarray}
         pro každé $m\in M$ a tedy \\\\
         \centerline{$Dh(rf')=rDh(f')$.}\\\\
         Platnost rovnosti \\\\
         \centerline{$Dh(f_1'+f_2')=Dh(f_1')+Dh(f_2')$}\\\\
         pro každé $f_1',f_2'\in DM'$ dokazovat nebudeme a přenecháme ji 
         čtenáři.
                  
         \item[(b)] Nechť $M\in mod(A)$, pak $Hom_R(M,I)\in Mod(A^{op})$ a navíc 
         je $Hom_R(M,I)$ konečně generovaný jako $R$-modul a tedy $Hom_R(M,I)\in 
         mod(A^{op})$. 
         Ukážeme pouze, že $D$ zobrazí homomorfismus $A$-modulů na homomorfismus 
         $A^{op}$-modulů, zbytek důkazu opět ponecháme čtenáři. 
         Nechť $M,M'\in mod(A)$ a $h\in Hom_A(M,M')$. Pak pro $f'\in DM'$ 
         a $\lambda\in A$ máme
         \begin{eqnarray}
           Dh(f'\lambda)(m)&=& (-\circ)_I(f'\lambda)(m)  \nonumber   \\    
           &=& (f'\lambda)(\underbrace{h(m)}_{\in M'}) \nonumber \\      
           &=& f'(\lambda h (m)) \nonumber       \\
           &=& f'(h(\lambda m)) \nonumber       \\
           &=& (-\circ h)_I (f')(\lambda m) \nonumber       \\
           &=& \underbrace{Dh(f')}_{\in Hom_R(M,I)}(\lambda m) \nonumber       \\
           &=& (Dh(f')\lambda)(m) \nonumber         
         \end{eqnarray}
         pro každé $m\in M$ a tedy \\\\
         \centerline{$Dh(f'\lambda)=Dh(f')\lambda$.}
         
         \item[(c)] Dokážeme, že  zobrazení $\alpha: 1_{mod(R)}\to D^2$,
         definované pro $M\in mod(R)$ 
         \begin{eqnarray}
           \alpha_M: M &\to& Hom_R(Hom_R(M,I),I) \nonumber \\
           m&\mapsto&[f\mapsto f(m)] \nonumber
         \end{eqnarray}
         je přirozenou ekvivalencí funktorů. Zvolme pevně $M\in mod(R)$. Zřejmě máme 
         $\alpha_M(m)\in  Hom_R(Hom_R(M,I),I)$ 
         pro každé $m\in M$. Mějme 
         $f_1,f_2\in Hom_R(M,I)$ a $r\in R$ , pak         
         \begin{eqnarray}
           \alpha_M(m)(rf_1+f_2) &=& (rf_1+f_2)(m) \nonumber \\
           &=& rf_1(m)+f_2(m) \nonumber \\
           &=& r\alpha_M(f_1) + \alpha_M(f_2).\nonumber 
         \end{eqnarray}                 
         Navíc pro $m_1,m_2\in M$ a $r\in R$ je 
         \begin{eqnarray}
           \alpha_M(rm_1+m_2)(f)&=& f(rm_1+m_2)\nonumber \\
           &=& rf(m_1)+f(m_2)\nonumber \\
           &=& r\alpha_M(m_1)(f_1)+\alpha_M(m_2))f\nonumber \\
           &=& (r\alpha_M(m_1)+\alpha_M(m_2))(f)\nonumber 
         \end{eqnarray}
         pro každé $f\in Hom_R(M,I)$, tedy platí, že         
         \begin{eqnarray}
           \alpha_m(rm_1+m_2) &=& r\alpha_M(m_1) + \alpha_M(m_2) \nonumber
         \end{eqnarray}
         a $\alpha_M$ je tedy $R$-modulovým homomorfismem. Dále dokážeme, že jde 
         o přirozenou transformaci. Nechť $M,M'\in Mod(R)$ a $h\in Hom_R(M,M')$. 
         Pak 
         \begin{eqnarray}
           D^2(h) &=& Hom_R(-,I)Hom_R(-,I)(h)\nonumber \\
           &=& Hom_R(-,I)((-\circ h)_I)\nonumber\\
           &=& (-\circ(-\circ h)_I)_I.\nonumber 
         \end{eqnarray}
         Musíme dokázat, že následující diagram komutuje:
         \\\\
         \centerline{$\xymatrix{
           M \ar[rrr]^{\alpha_M} \ar[d]^h
          &&& Hom_R(Hom_R(M,I),I) 
           \ar[d]^{(-\circ(-\circ h)_I)_I}\\
           M' \ar[rrr]^{\alpha_{M'}}
          &&& Hom_R(Hom_R(M',I),I) \\
         }$}\\\\
         Nechť $m\in M$ a uvažujme následující diagram:
         \\\\
         \centerline{$\xymatrix{
           m \ar@{|->}[rrr] \ar@{|->}[d]
          &&& [f\mapsto f(m)] 
           \ar@{|.>}[d]^?\\
           h(m) \ar@{|->}[rrr]
          &&& [f'\mapsto f'h(m)] \\
         }$}\\\\
         Musíme dokázat, že \\\\
         \centerline{$
         (-\circ(-\circ h)_I)_I([f\mapsto f(m)])= [f'\mapsto f'h(m)].
         $}\\\\
         Pro $f'\in Hom_R(Hom_R(M',I),I)$ platí, že 
         \begin{eqnarray}
           (-\circ(-\circ h)_I)_I([f\mapsto f(m)])(f')
           &=& [f\mapsto f(m)]\circ (-\circ h)_I (f')\nonumber \\
           &=& [f\mapsto f(m)](f'h)\nonumber \\
           &=& f'h(m)\nonumber 
         \end{eqnarray}
         a $\alpha$ je přirozenou transformací. Zbývá dokázat, že pro každé $M\in mod(R)$ 
         je $\alpha_M$ izomorfismem $R$-modulů. Začneme s důkazem, že jde o 
         monomorfismus, neboli že prokaždé $0\neq m\in M$ existuje $f\in Hom_R(M,I)$, že $\alpha_M(m)(f)=f(m)\neq 
         0$.
         Uvažujme homomorfismus $R$-modulů 
         \begin{eqnarray}
           \hat f: Rm&\to& I \nonumber \\
           rm &\mapsto& r+\underline m. \nonumber
         \end{eqnarray}
         Připomeňme, že $\underline m$ značí maximální ideál $R$. Protože $Rm\subseteq M$ 
         je podmodul, tak inkluze $i: Rm\to M$ je monomorfismem $R$-modulů. Z 
         injektivity $I$ existuje $R$-modulový homomorfismus $f:M\to I$ takový, 
         že následující diagram komutuje:  \\\\
         \centerline{$\xymatrix{
           Rm \ar[r]^i  \ar[d]_{\hat f} & M \ar@{.>}[ld]^f \\
           I
         }$}\\\\
         Pak \\\\
         \centerline{$f(m)=fi(m)=\hat f(m)=\hat f(1_Rm)=1_R+ \underline m\neq 0$} 
         \\\\
         a $\alpha_M$ je tedy monomorfismem. Že jde zároveň o epimorfismus a tedy i o izomorfismus se 
         dokáže na základě tvrzení \cite{2} Proposition 1.4 (Kapitola 1). 
         Zároveň vynecháme i důkaz, že $\alpha_M$ s výše uvedenou definicí 
         je pro $M\in mod(A)$ izomorfismem $A$-modulů. 
         Ten se provede přímočaře a přenecháme ho tedy čtenáři.
       \end{description}
     \end{proof}
     
     \begin{dsl}\label{lem-dual-teleso}
       Pro $R=K$ těleso je funktor $D$ daný vztahem $D=Hom_K(-,K)$
       jako funktor $mod(K)\rightarrow mod(K)$.
     \end{dsl}
     \begin{proof}
       Pokud $R$ je těleso, pak  je jeho maximální ideál $\underline{m}=0$, tedy 
       \\\\
       \centerline{$R=R/\underline{m}=K$.}\\\\
       Navíc $R$ jakožto $R$-modul je injektivní a jelikož každý nenulový 
       podmodul $R$ má s $R$ nenulový průnik, je $R$ svým vlastním injektivním 
       obalem a tedy \\\\
       \centerline{$I=R=K$.}
     \end{proof}
     
     \begin{dfn}
       Buď $V\in mod(K)$, kde $K$ je libovolné těleso, a nechť \\\\
       \centerline{$B_V:=\{v_1,v_2,\ldots,v_l\}$}\\\\
       je K-báze V. Definujme zobrazení \\\\
       \centerline{$d_{B_V}:B_V\rightarrow D(V)$} \\\\
       předpisem $d_{B_V}(v_i)(v_j):=\delta_{i,j}$, kde $\delta$ je Kroneckerova 
       delta.
     \end{dfn}
     
     \begin{lem}
       Buď $V\in mod(K)$, kde $K$ je libovolné těleso, a nechť \\\\
       \centerline{$B_V:=\{v_1,v_2,\ldots,v_n\}$}\\\\
       je $K$-báze $V$. Položme \\\\
       \centerline{$dB_V:=\{d_{B_V}(v_1),d_{B_V}(v_2),\ldots,d_{B_V}(v_l)\},$}  \\\\
       lak $dB_V$ je $K$-báze D(V).  Nazveme ji duální bází $D(V)$ 
       vzhledem k $B_V$.
     \end{lem}
     
     \begin{proof}
       Nejprve dokážeme, že jsou prvky $dB_V$ lineárně nezávislé. 
       Nechť $\alpha_1,\ldots,\alpha_l \in K$ jsou takové, že 
       $\sum_{i=1}^l \alpha_i d_{B_V}(v_i)=0$. To znamená, že 
       $\sum_{i=1}^l \alpha_i d_{B_V}(v_i)(v)=0$ pro každé $v\in V$, neboli,  že 
       pro každé $j=1,2,\ldots,l$ máme \\\\
       \centerline{$0
         =\sum_{i=1}^l \alpha_i d_{B_V}(v_i)(v_j)
         =\sum_{i=1}^l \alpha_i \delta_{i,j}
         =\alpha_j$.} \\\\
       A tedy $dB_V$ je lineárně nezávislá množina.
       
       Nyní dokážeme, že libovolné $f\in D(V)$ vyjádříme jako lineární kombinaci 
       prvků $dB_V$. Máme vztah 
       \begin{eqnarray}
         d_{B_V}(v_i)\left(\sum_{j=1}^l\alpha_jv_j\right)
         =\sum_{j=1}^l\alpha_jd_{B_V}(v_i)(v_j)       
         =\alpha_i.  \nonumber          
       \end{eqnarray}
       Pro každé $v=\sum_{j=1}^l\alpha_jv_j$ pak platí
       \begin{eqnarray}
         f(v)
         &=& f\left(\sum_{j=1}^l\alpha_jv_j\right)  \nonumber \\        
         &=& \sum_{j=1}^l\alpha_j f(v_j)  \nonumber \\        
         &=& \sum_{j=1}^l d_{B_V}(v_j)(v) f(v_j)  \nonumber \\     
         &=& \sum_{j=1}^l f(v_j) d_{B_V}(v_j)(v)  \nonumber \\              
         &=& \left( \sum_{j=1}^l f(v_j) d_{B_V}(v_j) \right)(v) \nonumber         
       \end{eqnarray}
       Z toho plyne, že $f=\sum_{j=1}^l f(v_j) d_{B_V}(v_j)$ a tím jsme hotovi.
     \end{proof}
     
     \begin{dsl}
       Nechť $B_V$ je báze vektorového prostoru $V$.
       Koeficienty $f\in D(V)$ vzhledem k bázi $dB_V$ pak spočteme 
       jako obrazy korespondujících prvků báze $B_V$ při zobrazení $f$.
     \end{dsl}
         
     \begin{lem}\label{lem-baze-dual-xi}
       Nechť $V,W\in mod(K)$, kde $K$ je\, těleso.\,\, Dále \, nechť $\xi \in Hom_K(V,W)$ 
       je izomorfismus,  \\\\
       \centerline{$B_V:=\{v_1,v_2,\ldots,v_l\}$}\\\\
       je K-báze V a  \\\\
       \centerline{$\xi(B_V):=\{\xi(v_1),\xi(v_2),\ldots,\xi(v_l)\}$}\\\\
       je korespondující K-báze W. Pak pro každé $1\leq j \leq l$ máme: \\\\
       \centerline{$(D\xi^{-1})(d_{B_V}(v_j))=d_{\xi(B_V)}(\xi(v_j))$}\\
     \end{lem}
     \begin{proof}
       Nechť $1\leq j \leq l$ a $\sum_{i=1}^l\alpha_i\xi(v_i)\in W$. Pak
       \begin{eqnarray}
         (D\xi^{-1})(d_{B_V}(v_j)) \left( \sum_{i=1}^l\alpha_i\xi(v_i) \right)
         &=& d_{B_V}(v_j) \circ \xi^{-1} \left( \sum_{i=1}^l\alpha_i\xi(v_i) \right) \nonumber \\      
         &=& d_{B_V}(v_j) \left( \sum_{i=1}^l\alpha_i v_i \right) \nonumber \\      
         &=& \alpha_j \nonumber \\      
         &=& d_{\xi(B_V)}(\xi(v_j)) \left( \sum_{i=1}^l\alpha_i\xi(v_i) \right) \nonumber        
       \end{eqnarray}
       Z toho již přímo plyne hledaná rovnost.
     \end{proof}
     
   \subsection{Funktor $()^*$}
     
     \begin{dfn}       
       \begin{description} \item
         \item[(a)] Nechť P(A) značí kategorii, jejíž objekty jsou konečně generované 
            projektivní $A$-moduly a $Hom_{P(A)}(P,P')=Hom_A(P,P')$ pro $P,P'\in P(A)$. 
         \item[(b)] Pro $A,B\in mod(A)$ definujeme esenciální epimorfismus $t\in Hom_A(A,B)$
           jako epimorfismus $A$-modulů takový, že platí: Pro každé $M\in mod(A)$ a 
           $u\in Hom_A(M,A)$ takové, že složený 
           homomorfismus $tu$ je epimorfismus, je $u$ epimorfismus.  
           \\\\      
           \centerline{\xymatrix{           
              A
                \ar@{->}[r]^t
              & B  \\
              M
                \ar[ru]_{tu}
                \ar[u]^u
           }}\\
         \item[(c)] Projektivní pokrytí modulu $X$ je projektivní $A$-modul $P$ 
           spolu s esenciálním epimorfismem $t:P\rightarrow X$. Poznamenejme, že 
           někdy budeme o projektivním pokrytí referovat jako o dvojici 
           $(P,t)$, jindy pouze jako o modulu či epimorfismu, pokud nebude druhé 
           v našich úvahach potřeba.
       \end{description}       
     \end{dfn}
     
     \begin{lem}\label{lem-proj-pokryti}
       Nechť $X\in mod(A)$, pak
       \begin{description}
         \item[(a)] je-li $P$ projektivní pokrytí $X$, pak $P\in P(A)$ a $P$ je tedy konečně generovaný. 
         \item[(b)] projektivní pokrytí modulu $X$ je dáno jednoznačně až na izomorfismus. 
         \item[(c)] projektivní pokrytí modulu $X$ existuje.
       \end{description} 
     \end{lem}
     \begin{proof}
       \begin{description}
         \item
         \item[(a)] Buď $P$ projektivní pokrytí $X\in mod(A)$. Pak máme 
         esenciální epimorfismus $t\in Hom_A(P,X)$ a navíc jelikož je $X$ konečně generovaný,  
         existuje $n\in \mathbb{N}$ a epimorfismus $u\in Hom_A(A^n,X)$.\\\\      
           \centerline{\xymatrix{           
               & A^n \ar[d]^u \ar@{.>}[ld]_{\exists v} \\
             P \ar[r]_t
               & X \ar[r] \ar[d] 
               & 0 \\
             & 0  
           }} \\\\ 
         Pak existuje $v\in Hom_A(A^n,P)$ takové, že diagram výše komutuje. Protože $tv=u$ 
         je epimorfismus a $t$ je esenciální epimorfismus, musí být také $v$ 
         epimorfismem a modul $P$ konečně generovaný.
         \item[(b)] Buďte $P$ a $P'$ dvě projektivní pokrytí modulu $X$ a $t,t'$ 
         jejich esenciální epimorfismy. Pak z vlastností projektivního pokrytí 
         existují homomorfismy $h\in Hom_A(P,P')$ a $h'\in Hom_A(P',P)$ takové, 
         že \\\\
         \centerline{$t'h=t$ a $th'=t'$.}\\\\      
         \centerline{\xymatrix{           
               & P \ar[d]^t \ar@/_0.3pc/[ld]_{h} \\
             P' \ar[r]_{t'} \ar@/_0.3pc/[ru]_{h'}
               & X \ar[r] \ar[d] 
               & 0 \\
             & 0  
         }} \\\\ 
         Protože $t$ a $t'$ jsou esenciální epimorfismy, $th'$ a $t'h$ 
         epimorfismy, tak jsou i zobrazení $h$ a $h'$ epimorfismy. Dále máme \\\\
         \centerline{$t'(hh')=(t'h)h'=th'=t'$} 
         \\\\
         a $hh'$ je také epimorfismus. Modul $P'$ je konečně generovaný a $hh'$ 
         je identita. To nám implikuje, že $h$ je monomorfismus a tedy 
         i izomorfimus modulů $P$ a $P'$.
         \item[(c)] Plyne z \cite{2} Theorem 4.2, jelikož $A$ je artinovská $R$-algebra.
       \end{description} 
     \end{proof}
     
     \begin{pzn}
       Připomeňme, že homomorfismus konečně generovaných $A$-modulů má jádro v 
       $mod(A)$, které je určeno jednoznačně až na izomorfismus.
     \end{pzn}
     
     \begin{dfn}
       Minimální projektivní prezentace modulu $X\in mod(A)$ je exaktní 
       posloupnost 
      \\\\      
      \centerline{\xymatrix{     
         P_1 \ar@{->}[r]^s 
           & P_0 \ar@{->}[r]^t 
           & X \ar@{->}[r]
         & 0
      },}\\\\ 
      kde $P_0$ je projektivní pokrytí modulu $X$ a $P_1$ je projektivní 
      pokrytí modulu $Ker(t)\subseteq P_0$. Přesněji $s=iw$, kde $i:Ker(t)\rightarrow P_0$ 
      je kanonická inkluze jádra homomorfismu a $w:P_1\rightarrow Ker(t)$ je 
      projektivní pokrytí modulu $Ker(t)$.\\\\
       \centerline{\xymatrix{
         P_1 \ar@{->}[rr]^{s=iw} \ar@{->}[rd]_w
           & & P_0 \ar@{->}[r]^t 
           & X \ar@{->}[r]
           & 0 \\
         & Ker(t) \ar@{->}[ru]_i
      },}
     \end{dfn}
     
     \begin{lem} 
       Minimální projektivní prezentace modulu $X\in mod(A)$ je dána jednoznačně 
       až na izomorfismus.
     \end{lem}
     \begin{proof}
      Mějme následující dvě projektivní prezentace modulu $X$:\\\\
       \centerline{\xymatrix{
         P_1 \ar@{->}[r]^{s} 
         & P_0 \ar@{->}[r]^t 
         & X \ar@{->}[r]
         & 0 \\
      }}\\\\
       \centerline{\xymatrix{
         P_1' \ar@{->}[r]^{s'} 
         & P_0' \ar@{->}[r]^{t'} 
         & X \ar@{->}[r]
         & 0 \\
      }}\\\\
      Uvažujme inkluze
      \begin{eqnarray}
          i:Ker(t) &\to& P_0 \nonumber \\
         i':Ker(t') &\to& P_0' \nonumber 
      \end{eqnarray}
      a projekce
      \begin{eqnarray}
          \pi:P_1 &\to& Im(s)=Ker(t) \nonumber \\
          \pi':P_1' &\to& Im(s')=Ker(t') \nonumber 
      \end{eqnarray}
       Máme následující diagram:
      \\\\      
      \centerline{\xymatrix{     
         P_1 \ar@{->}[rr]^s \ar@{->}[rd]^{\pi} \ar@{.>}[ddd]^{h_1}
           & & P_0 \ar@{->}[r]^t \ar@{->}[ddd]^{h_0}_{\simeq}
           & X \ar@{->}[r] \ar@{=}[ddd]
         & 0 \\
         & Ker(t) \ar@{->}[ru]_{i} \ar@{.>}[ddd]^{u}_{\simeq} \\
         \\
         P_1' \ar@{->}[rr]^{s'} \ar@{->}[rd]^{\pi'} \ar@{.>}[ddd]^{h_1'} 
           & & P_0' \ar@{->}[r]^{t'} \ar@{->}[ddd]^{h_0'}_{\simeq} 
           & X' \ar@{->}[r] \ar@{=}[ddd]
         & 0 \\
         & Ker(t') \ar@{->}[ru]_{i'} \ar@{.>}[ddd]^{u'}_{\simeq}\\
         \\
         P_1 \ar@{->}[rr]^s \ar@{->}[rd]^{\pi}
           & & P_0 \ar@{->}[r]^t 
           & X \ar@{->}[r]
         & 0 \\
         & Ker(t) \ar@{->}[ru]_{i}\\
      }}\\\\ 
      Izomorfismy $h_0$ a $h_0'$ existují dle 
      \hyperref[lem-proj-pokryti]{Lemma \ref*{lem-proj-pokryti}}. Existence 
      izomorfismů $u$ a $u'$ takových, že 
      \begin{eqnarray}
        h_0i &=& i'u \nonumber \\
        h_0'i'&=&iu'.  \nonumber
      \end{eqnarray}
      plyne z \hyperref[lemma-five]{Lemma \ref*{lemma-five}}.
      
      Protože $\pi'$ je epimorfismus a $P_1$ projektivní, existuje
      $h_1\in Hom_A(P_1,P_1')$ takové, že\\
      \centerline{$u\pi=\pi'h_1$.}\\\\
      Navíc protože je $u\pi$ epimorfismus a $\pi'$ esenciální epimorfismus,
      musí být i $h_1$ epimorfismus. Ekvivalentně existuje epimorfosmus 
      $h_1\in Hom_A(P_1,P_1')$ takový, že\\\\
      \centerline{$u'\pi'=\pi h_1'$.}\\\\
      Složené zobrazení $h_1'h_1$ je epimorfismus a jelikož je modul $P_1$ 
      dle \hyperref[lem-proj-pokryti]{Lemma \ref*{lem-proj-pokryti}} 
      konečně generovaný, je $h_1'h_1$ izomorfismus. To implikuje, že 
      homomorfismus $h_1$ je monomorfismem a tedy i izomorfismem.
     \end{proof}
     
     \begin{dfn}
       Definujme hom funktor $()^* := Hom_A(-,A)$. 
     \end{dfn}
     
     \begin{thm}\label{lem-star}
       Funktor $()^*$ je
       \begin{description}
         \item[(a)] kontravariantním zleva exaktním funktorem $mod(A)\rightarrow mod(A^{op})$. 
         \item[(b)] kontravariantním funktorem $P(A)\rightarrow P(A^{op})$.
       \end{description}
     \end{thm}
     \begin{proof}
       Víme, že $Hom_A(-,A)$ je kontravariantním zleva exaktním funktorem 
       abelovských grup.
       \begin{description}
         \item[(a)] Buď $M\in mod(A)$, pak $Hom_A(M,A)\in mod(R)$. Definujme na 
         $Hom_A(M,A)$ násobení prvky $A$ zprava předpisem 
         \begin{eqnarray} 
           Hom_A(M,A)\times A&\to& Hom_A(M,A)\nonumber \\
            f\lambda &\mapsto& [f\lambda(m) \mapsto f(m)\lambda]. \nonumber 
         \end{eqnarray}          
         Že je násobení dobře definované a asociativní ke sčítání homomorfismů je zřejmé.
         Ověříme, že $()^*$ přenaší homomorfismus $A$-modulů na homomorfismus 
         $A^{op}$-modulů. Nechť $M,M'\in mod(A)$, $h\in Hom_A(M,M')$, $f'\in M^*'$ 
         a $\lambda\in A$. Pak
         \begin{eqnarray}
           (h^*(f'\lambda))(m)&=& ((f'\lambda)h)(m)\nonumber \\
           &=& (f'\lambda)(h(m)) \nonumber \\
           &=& f'(h(m))\lambda \nonumber \\
           &=& (f'h)(m)\lambda\nonumber \\
           &=& ((f'h)\lambda)(m)\nonumber \\
           &=& (h^*(f')\lambda)(m)\nonumber 
         \end{eqnarray}
         pro všechna $m\in M$ a tedy \\\\
         \centerline{$h^*(f'\lambda)=h^*(f')\lambda$.}\\\\
         Obdobně se dokáže rovnost \\\\
         \centerline{$h^*(f'_1 + f'_2)=h^*(f'_1 + f'_2)$.}\\\\
         Pak $Hom_A(M,A)\in Mod(A^{op})\cap mod(R)$ a tedy dle 
           \hyperref[lem-mod-Mod]{Lemma \ref*{lem-mod-Mod}} máme $Hom_A(M,A)\in 
           mod(A)$.
           
         \item[(b)] Buď $P$ projektivní $A$-modul. Pak dle 
         \hyperref[rozklad-A-na-P]{Věty \ref*{rozklad-A-na-P}} 
         existuje projektivní $A$-modul  $P'$ a 
         $n\in \mathbb N$ takové, že $A^n=P\oplus P'$. Potom máme následující izomorfismus 
         $A^{op}$-modulů: \\\\
         \centerline{$ Hom_A(A^n,A)\simeq Hom_A(P\oplus P', A)$.} \\\\
         Navíc dle 
         \hyperref[dir-sum-hom]{Věty \ref*{dir-sum-hom}} následující izomorfismy 
         $A^{op}$-modulů:\\\\
         \centerline{$ Hom_A(P, A) \oplus Hom_A(P', A) \simeq Hom_A(A, A)$.} \\\\
         \centerline{$ Hom_A(A^n, A) \simeq Hom_A(A, A)^n$.} \\\\
         Pak s využitím \hyperref[izo-hom-aei]{Věty \ref*{izo-hom-aei}} 
         dostáváme izomorfismus  $A^{op}$-modulů\\\\
         \centerline{$ Hom_A(P, A) \oplus Hom_A(P', A) \simeq Hom_A(A^n, A) \simeq (A^{op})^n$.} \\\\
         Z toho již plyne, že $P^*=Hom_A(P,A)$ je direktním sčítancem $(A^{op})^n$ 
         a tedy projektivní.
       \end{description}
     \end{proof}
     
   \subsection{Funktor $Tr$}
    
     \begin{lem}\label{lemma-cok*}
       Nechť $X\in mod(A)$ a posloupnost  \\\\
       \centerline{\xymatrix{
       P_1 \ar@{->}[r]^s 
         & P_0 \ar@{->}[r]^t 
         & X \ar@{->}[r]
         & 0
      },}\\\\ je minimální projektivní prezentace modulu $X$. Pak
       \begin{description}
         \item[(a)] Následující posloupnost je exaktní v $mod(A^{op})$: \\\\
         \centerline{\xymatrix{
           0 \ar@{->}[r] 
             & X^* \ar@{->}[r]^{t^*} 
             & P_0^* \ar@{->}[r]^{s^*} 
             & P_1^* \ar@{->}[r] 
             & Cok(s^*) \ar@{->}[r]
             & 0
          }}       
         \item[(b)] Je-li navíc  \\
         \centerline{\xymatrix{
           P_1' \ar@{->}[r]^{s'} 
             & P_0' \ar@{->}[r]^{t'} 
             & X \ar@{->}[r]
             & 0
          }} \\\\
          druhá projektivní prezentace modulu $X$, pak máme izomorfismus 
          $A^{op}$-modulů: \\\\
          \centerline{$Cok(s^*)\simeq Cok(s'^*)$}
       \end{description}
     \end{lem}
     \begin{proof}
       \begin{description}
         \item
         \item[(a)] Plyne z \hyperref[lem-star]{Lemma \ref*{lem-star}} a z toho, 
           že kojádro morfismu konečně generovaných modulů je konečně 
           generované.
         \item[(b)] Dle \hyperref[lem-proj-pokryti]{Lemma \ref*{lem-proj-pokryti}}
           existují izomorfismy $h_0\in Hom_A(P_0,P_0')$ a $h_1\in Hom_A(P_1,P_1')$ 
           takové, že následující diagram komutuje: \\\\
           \centerline{\xymatrix{
             0 \ar@{->}[r] 
               & P_1 \ar@{->}[r]^s \ar@{.>}[d]_{h_1} 
               & P_0 \ar@{->}[r]^t \ar@{.>}[d]_{h_0} 
               & X \ar@{->}[r] \ar@{=}[d] 
               & 0 \\
             0 \ar@{->}[r] 
               & P_1' \ar@{->}[r]^{s'} 
               & P_0' \ar@{->}[r]^{t'} 
               & X \ar@{->}[r]
               & 0
           }}\\\\\\       
           Aplikujeme-li funktor $()^*$ dostaneme následující komutativní 
           diagram:\\\\
           \centerline{\xymatrix{
             0 \ar@{->}[r]
               & X^{*} \ar@{->}[r]^{t^{'*}} \ar@{=}[d]
               & P_0^{'*} \ar@{->}[r]^{s^{'*}} \ar@{->}[d]^{h_0^*}
               & P_1^{'*} \ar@{->}[r]  \ar@{->}[d]^{h_1^*}
               & Cok(s^{'*}) \ar@{->}[r]  \ar@{.>}[d]^{\simeq}
               & 0 \\
             0 \ar@{->}[r]
               & X^{*} \ar@{->}[r]^{t^{*}} 
               & P_0^{*} \ar@{->}[r]^{s^{*}} 
               & P_1^{*} \ar@{->}[r] 
               & Cok(s^{*}) \ar@{->}[r]
               & 0
           }}\\\\\\       
           Z \hyperref[lemma-five]{Lemma \ref*{lemma-five}} pak plyne hledaný $S^{op}$-modulový 
           izomorfismus $Cok(s^{*})\simeq Cok(s^{*'})$.
         \end{description} 
     \end{proof}
     
     \begin{dfn}
       Na základě předchozího \hyperref[lemma-cok*]{Lemma \ref*{lemma-cok*}} 
       definujme zobrazení \\\\  
       \centerline{$Tr:mod(A)\rightarrow mod(A^{op})$} \\\\
       předpisem $Tr(X):=Cok(s^*)$, kde \\\\
       \centerline{\xymatrix{
         P_1 \ar@{->}[r]^s 
           & P_0 \ar@{->}[r]^t 
           & X \ar@{->}[r]
           & 0
       },}\\\\
       je libovolná projektivní prezentace modulu $X$.       
     \end{dfn}
     
     \begin{dfn}
       \begin{description} \item
         \item[(a)] Pro $M,N\in mod(A)$ položme: 
           \\\\
           $P_A(M,N):=\{f\in Hom_A(M,N)|f$ se faktorizuje skrze projektivní modul$\}$ \\
           \\
          \centerline{$\underline{Hom}_A(M,N):=Hom_A(M,N)/P_A(M,N)$} \\
         \item[(b)] Definujme kategorii $\underline{mod}(A)$: \\\\
           \centerline{$Ob(\underline{mod}(A)):=Ob(mod(A))$}\\\\
           \centerline{$Hom_{\underline{mod}(A)}(M,N):=\underline{Hom}_A(M,N)$}\\ 
       \end{description}       
     \end{dfn}
     
     \begin{thm}
       Pro dva moduly $X,X'\in mod(A)$ a jejich dvě projektivní prezentace 
       \\\\      
       \centerline{\xymatrix{
         P_1 \ar@{->}[r]^s 
           & P_0 \ar@{->}[r]^t 
           & X \ar@{->}[r]
           & 0
       }}\\\\       
       \centerline{\xymatrix{
         P_1' \ar@{->}[r]^{s'} 
           & P_0' \ar@{->}[r]^{t'} 
           & X' \ar@{->}[r]
           & 0
       },}\\\\\\
       dodefinujme $Tr:\underline{Hom}_A(X,X')\rightarrow 
       \underline{Hom}_{A^{op}}(Tr(X'),Tr(X))$ předpisem \\\\
       \centerline{$h+P_A(X,X')\mapsto (h_1^*)_{Cok}+P_{A^{op}}(Tr(X'),Tr(X))$,}
       \\\\kde $h_0\in Hom_A(P_0,P_0')$ a $h_1\in Hom_A(P_1,P_1')$ jsou 
       libovolně zvolené homomorfismy takové, že následující diagram komutuje: 
       \\\\
           \centerline{\xymatrix{
             X^{'*} \ar@{->}[r]^{t^{'*}} \ar@{->}[d]^{h^*}
               & P_0^{'*} \ar@{->}[r]^{s^{'*}} \ar@{->}[d]^{h_0^*}
               & P_1^{'*} \ar@{->}[r]  \ar@{->}[d]^{h_1^*}
               &Tr(X') \ar@{->}[r]  \ar@{->}[d]^{(h_1^*)_{Cok}}
               & 0 \\
             X^{*} \ar@{->}[r]^{t^{*}} 
               & P_0^{*} \ar@{->}[r]^{s^{*}} 
               & P_1^{*} \ar@{->}[r] 
               & Tr(X)\ar@{->}[r]
               & 0
           },}\\\\\\     
       Pak je $Tr$ kontravariantním funktorem $Tr: \underline{mod}(A)\rightarrow 
       \underline{mod}(A^{op})$. 
     \end{thm}
     \begin{proof}
       Důkaz rozdělíme na 3 části:
       \begin{description}
         \item[(1)] Pro morfismy $g_0,h_0\in Hom_A(P_1,P_1')$ a $g_0,h_0\in Hom_A(P_0,P_0')$ 
         takové, že následující diagramy komutují, \\\\
         \centerline{\xymatrix{
         P_1 \ar@{->}[r]^s  \ar[d]^{h_1}
           & P_0 \ar@{->}[r]^t   \ar[d]^{h_0}
           & X \ar@{->}[r] \ar[d]^{h}
           & 0 \\
         P_1' \ar@{->}[r]^{s'} 
           & P_0' \ar@{->}[r]^{t'} 
           & X' \ar@{->}[r]
           & 0
       } \\ \xymatrix{
         P_1 \ar@{->}[r]^s  \ar[d]^{g_1}
           & P_0 \ar@{->}[r]^t   \ar[d]^{g_0}
           & X \ar@{->}[r] \ar[d]^{h}
           & 0 \\
         P_1' \ar@{->}[r]^{s'} 
           & P_0' \ar@{->}[r]^{t'} 
           & X' \ar@{->}[r]
           & 0
       } \\}\\\\\\
       platí, že \\\\
       \centerline{$(h_1^*)_{Cok}-(g_1^*)_{Cok}\in P_{A^{op}}(Tr(X'),Tr(X))$.} 
       \\\\
       Neboli $(h_1^*)_{Cok}$ a $(g_1^*)_{Cok}$ reprezentují stejný prvek z 
       $\underline{Hom}_{A^{op}}(Tr'(X),Tr(X))$.
       
         \item[(2)] Pro $h\in P_A(X,X')$ je \\\\
         \centerline{$(h_1^*)_{Cok}\in P_{A^{op}}(Tr(X'),Tr(X))$.}\\\\
         Neboli prvek $\underline{Hom}_{A}(X',X)$ se zobrazuje na prvek 
         $\underline{Hom}_{A^{op}}(Tr'(X),Tr(X))$
         nezávisle na zvoleném reprezentantu.

         \item[(3)] Funktor $Tr$ kontravariantním funktorem $\underline{mod}(A)\rightarrow 
       \underline{mod}(A^{op})$.
       \end{description}
       
       
       \begin{description}
         \item[(1)] Uvažujme následující diagram: \\\\
         \centerline{\xymatrix{
         P_1 \ar@{->}[rr]^s  \ar@/^1pc/[dd]^{h_1} \ar@/_1pc/[dd]_{g_1}
           && P_0 \ar@{->}[rr]^t   \ar@/_1pc/[dd]_{g_0}  \ar@/^1pc/[dd]^{h_0}  \ar@{.>}[lldd]_u
           && X \ar@{->}[rr] \ar[dd]^{h}
           && 0 \\\\
         P_1' \ar@{->}[rr]^{s'} \ar[rd]
           && P_0' \ar@{->}[rr]^{t'} 
           && X' \ar@{->}[rr]
           && 0 \\
           & Ket(t') \ar[ru]
         }\\}\\\\\\
         Protože $ht=t'h_0=t'g_0$, pak $t'(h_0-g_0)=0$ a tedy $h_0-g_0$ se 
         faktorizuje skrze $Ker(t')$. Pak z projektivity $P_0$ existuje 
         kanonická projekce $P_1'$ na $Im(s')=Ker(t')$ a $(g_0-h_0)$ se 
         faktorizuje skrze $P_1'$. Proto existuje $u\in Hom_A(P_0,P_1')$ takové, 
         že \\\\
         \centerline{$s'u=g_0-h_0$.}\\\\
         Aplikujeme-li funktor $()^*$, dostaneme následující diagram v 
         $mod(A^{op})$:\\\\
         \centerline{\xymatrix{
         P_0'^* \ar@{->}[rr]^{s'^*}  \ar@/^1pc/[dd]^{h_0'^*} \ar@/_1pc/[dd]_{g_0'^*}
           && P_1'^* \ar@{->}[rr]^{\hat t'}   \ar@/_1pc/[dd]_{g_1'^*}  \ar@/^1pc/[dd]^{h_1'^*}  
           \ar[lldd]_{u^*}
           && Tr(X') \ar@{->}[rr] \ar@/^1pc/[dd]^{(h_1^*)_{Cok}} \ar@/_1pc/[dd]_{(g_1^*)_{Cok}}
           \ar@{.>}[lldd]_{v}
           && 0 \\\\
         P_0^* \ar@{->}[rr]^{s^*}
           && P_0^* \ar@{->}[rr]^{\hat t} 
           && X' \ar@{->}[rr]
           && 0 \\
           & \quad\quad &
           & \quad\quad &
         }\\}\\
         Protože platí 
         \\\\\centerline{$u^*s'^*=g_0^*-h_0^*$,} \\\\
         pak 
         \\\\\centerline{$s^*u^*s'^*=s^*(g_0^*-h_0^*)=(g_1^*-h_1^*)s'^*$}\\\\
          a tedy
         \\\\\centerline{$(g_1^*-h_1^*-s^*u^*)s'^*=0$.}\\\\
         Pak se $(g_1^*-h_1^*-s^*u^*)$ faktorizuje skrze kojádro $s'^*$, 
         konkrétně $\hat t'$. To znamená, že existuje $u\in Hom_{mod(A^{op})}(Tr(X'),P_1^*)$ 
         takové, že 
         \\\\\centerline{$v \hat t'=g_1^*-h_1^*-s^*u^*$.}\\\\
         Přenásobíme-li poslední řádek $\hat t$, dostaneme
         \\\\\centerline{$\hat t v \hat t'=\hat t (g_1^*-h_1^*-s^*u^*)=((g_1^*)_{Cok} - (h_1^*)_{Cok})\hat t '$,}\\\\
         a protože $\hat t'$ je epimorfismus, pak 
         \\\\\centerline{$\hat t v=(g_1^*)_{Cok}-(h_1^*)_{Cok}$.}\\\\
         A tedy $(g_1^*)_{Cok}-(h_1^*)_{Cok}$ se faktorizuje skrze $P_1^*$ a 
         \\\\\centerline{$(g_1^*)_{Cok}-(h_1^*)_{Cok} \in P_{A^{op}}(Tr(X'),Tr(X))$.}
                  
         \item[(2)] Uvažujme diagram:\\\\
         \centerline{\xymatrix{
         P_1 \ar@{->}[rr]^s   \ar[dd]_{h_1}
           && P_0 \ar@{->}[rr]^t  \ar[dd]^{h_0} \ar@{.>}[lldd]_v  \ar@{.>}[lddd]_v
           && X \ar@{->}[rr] \ar[dd]^{h}  \ar@{.>}[lldd]_u
           && 0 \\\\
         P_1' \ar@{->}[rr]^{s'} \ar[rd]
           && P_0' \ar@{->}[rr]^{t'} 
           && X' \ar@{->}[rr]
           && 0 \\
           & Ket(t') \ar[ru]
         }\\}\\\\\\
         Předpokládejme, že se $h$ faktorizuje skrze projektivní $A$-modul $P$. 
         Pak protože $t'$ je epimorfismus, tak se $h$ faktorizuje i skrze $t'$. 
         Neboli existuje $u'\in Hom_A(X,P_0)$ takové, že 
         \\\\\centerline{$t'u=h$.}\\\\
         Uvažujme $A$-homomorfismus $(h_0-ut)$, ten se faktorizuje skrze 
         $Ket(t')$. Přenásobením $t'$ zleva 
         dostaneme rovnost
         \\\\\centerline{$t'(h_0-ut)=t'h_0-t'ut=t'h_0-ht=0$.}\\\\
         Pak protože $P_0$ je projektivní modul a z existence kanonické 
         projekce $P_1'$ na $Im(s')=Ket(t')$, se musí $h_0-ut$ faktorizovat 
         skrze $P_1'$. Tedy existuje $v\in Hom_A(P_0,P_1')$ takové, že
         \\\\\centerline{$s'v=h_0-ut$.}\\\\
         Aplikací funktoru $()^*$ získáme rovnost
         \\\\\centerline{$v^*s'^*=h_0^*-t^*u^*$}\\\\
         a následující diagram komutuje:\\\\
         \centerline{\xymatrix{
          0 \ar[r]
           & X'^* \ar@{->}[rr]^{t'^*}  \ar[dd]^{h^*}
           && P_0'^* \ar@{->}[rr]^{s'^*}  \ar[dd]^{h_0^*} \ar@{->}[lldd]_{u^*}
           && P_1'^* \ar@{->}[rr]^{\hat t'} \ar[dd]^{h_1^*}  \ar@{->}[lldd]_{v^*}
           && Tr(X')  \ar@{->}[r] \ar[dd]^{(h_1^*)_{Cok}}  \ar@{.>}[lldd]_{w}
           & 0 \\\\
         0 \ar[r]
           & X^* \ar@{->}[rr]^{t^*}
           && P_0^* \ar@{->}[rr]^{s^*} 
           && P_1 \ar@{->}[rr]^{\hat t}           
           && Tr(X) \ar@{->}[r]
           & 0
         }\\}\\\\\\
         Ukážeme, že se $(h_1^*-s^*v^*)$ faktorizuje skrze $Cok(s'^*)=Tr(X'):$
         \begin{eqnarray}
           (h_1^*-s^*v^*)s'^* &=& h^*s'^*-s^*v^*s'^*   \nonumber \\
           &=&  h^*s'^*-s^*(h_0^*-t^*u^*)  \nonumber \\
           &=&  h^*s'^*-s^* h_0^*-s^*t^*u^*    \nonumber \\
           &=& 0.   \nonumber 
         \end{eqnarray}
         Pak existuje $w\in Hom_A(Tr(X'),P_1^*)$ takové, že
         \\\\\centerline{$w\hat t'=h_1^*-s^*v^*$,}\\\\
         a my konečně vidíme, že
         \\\\\centerline{$\hat w\hat t'=\hat t(h_1^*-s^*v^*)=\hat t h_1^*=(h_1^*)_{Cok}\hat t'$.}\\\\
         A protože $\hat t'$ je epimorfismus, musí být
         \\\\\centerline{$(h_1^*)_{Cok}=\hat t w$.}\\\\
         Tedy $(h_1^*)_{Cok}\in P_{A_{op}}(Tr(X'),Tr(X))$.
         \item[(3)] Již víme, že zobrazení
           \\\\\centerline{$Tr:Ob(\underline{mod}(A))\to Ob(\underline{mod}(A^{op}))$}\\\\
           je dobře definované a z (1) a (2) navíc, že pro $X,X'\in \underline{mod}(A)$ 
           je dobře definované i zobrazení
           \\\\\centerline{$Tr: \underline{Hom}_A(X,X') \to \underline{Hom}_{A^{op}}(Tr(X'),Tr(X))$.}\\\\
           Zbývá tedy dokázat, že $Tr$ je kompatibilní se skládáním morfismů a 
           že zachovává identitu. Nechť tedy $X,Y,Z\in \underline{mod}(A)$, $f\in Hom_A(X,Y)$ 
           a $g\in Hom_A(Y,Z)$. Chceme ukázat, že 
           \\\\\centerline{$Tr(gf)=Tr(f)Tr(g)$.}\\\\
           Máme následující  komutativní diagram v $mod(A)$, kde jednotlivé 
           řádky jsou minimální  projektivní prezentace modulů $X$, $Y$ a $Z$:
           \\\\\centerline{$\xymatrix{
                P_{X,1} \ar[r]^{s_X} \ar[d]^{f_1}
             & P_{X,0} \ar[r]^{t_X} \ar[d]^{f_0}
             & X \ar[r]  \ar[d]^{f}
             & 0 \\
                P_{Y,1} \ar[r]^{s_Y} \ar[d]^{g_1}
             & P_{Y,0} \ar[r]^{t_Y} \ar[d]^{g_0}
             & Y \ar[r]  \ar[d]^{g}
             & 0 \\
                P_{Z,1} \ar[r]^{s_Z} 
             & P_{Z,0} \ar[r]^{t_Z}
             & Z \ar[r]
             & 0
           }$} \\\\\\
           Dle bodu (1) si můžeme při hledání $Tr(gf)$ zvolit libovolné dva 
           homomorfismy $(gf)_0$ a $(gf)_1$ takové, že následující diagram 
           komutuje:
           \\\\\centerline{$\xymatrix{
                P_{X,1} \ar[r]^{s_X} \ar[d]^{(gf)_1}
             & P_{X,0} \ar[r]^{t_X} \ar[d]^{(gf)_0}
             & X \ar[r]  \ar[d]^{gf}
             & 0 \\
                P_{Z,1} \ar[r]^{s_Z} 
             & P_{Z,0} \ar[r]^{t_Z}
             & Z \ar[r]
             & 0
           }$} \\\\\\
           Tuto podmínku jasně splňuje i volba:
           \begin{eqnarray}
            (gf)_0 &:=& g_0f_0 \nonumber \\
            (gf)_1 &:=& g_1f_1.\nonumber
           \end{eqnarray}
           Aplikací funktoru $()^*$ dostaneme následující diagram v $mod(A^{op})$, 
           na který je možné zároveň nahlížet jako na diagram v 
           $\underline{mod}(A^{op})$, kde jednotlivé homomorfismy jsou zástupci 
           svých tříd ekvivalence:
           \\\\\centerline{$\xymatrix{
                P_{Z,0}^* \ar[r]^{s_Z^*} \ar[d]^{g_0^*}
             & P_{Z,1}^* \ar[r]^{\hat t_Z} \ar[d]^{g_1^*}
             & Tr(Z) \ar[r]  \ar[d]^{Tr(g)} \ar@/^6pc/[dd]^{Tr(gf)}
             & 0 \\
                P_{Y,0}^* \ar[r]^{s_Y^*} \ar[d]^{f_0^*}
             & P_{Y,1}^* \ar[r]^{\hat t_Y} \ar[d]^{f_1^*}
             & Tr(Y) \ar[r]  \ar[d]^{Tr(f)}
             & 0 \\
                P_{X,0}^* \ar[r]^{s_X^*} 
             & P_{X,1}^* \ar[r]^{\hat t_X}
             & Tr(X) \ar[r]
             & 0
           }$} \\\\\\
          V tomto diagramu vidíme následující rovnost
           \\\\\centerline{$
             Tr(gf)\hat t_Z
             =\hat t_X f_1^*g_1^*
             =\hat t_X f_1^*g_1^*
             = Tr(f) \hat t_Y g_1^*
             = Tr(f)Tr(g)\hat t_Z$}\\\\
           a tedy, protože $\hat t_Z$ je epimorfismus, máme
           \\\\\centerline{$Tr(gf)=Tr(f)Tr(g)$.}\\\\
           Nyní zbývá ukázat, že
           \\\\\centerline{$Tr(1_X)=1_{Tr(X)}$,}\\\\
           pro každé $X\in\underline{mod}(A)$. Uvažme diagram
           \\\\\centerline{$\xymatrix{
                P_{1} \ar[r]^{s} \ar[d]^{1_{P_0}}
             & P_{0} \ar[r]^{t} \ar[d]^{1_{P_1}}
             & X \ar[r]  \ar[d]^{1_X}
             & 0 \\
                P_{1} \ar[r]^{s} 
             & P_{0} \ar[r]^{t}
             & X \ar[r]
             & 0
           }$} \\\\\\
           Aplikujeme-li funktor $()^*$ dostaneme
           \\\\\centerline{$Tr(1_X)=(1_{P_1}^*)_{Cok}=1_{Tr(X).}$}\\\\
       \end{description}
     \end{proof}
     
     \begin{pzn}
       Zmiňme bez důkazu ještě několik užitečných vlastností funktoru $Tr$, 
       jejichž důkaz a věškeré podrobnosti lze nalézt hned v několika zdrojích, 
       například \cite{2}:
       \begin{description}
         \item[(a)] $Tr^2=1_{\underline{mod}(A)}$.
         \item[(b)] $Tr(\bigoplus_{i=1}^n M_i)=\bigoplus_{i=1}^n Tr(M_i)$. 
         \item[(c)] $Tr(M)=0$ $\Leftrightarrow$ $M$ je projektivní. 
         \item[(d)] $Tr(M)\simeq$ neprojektivní části $M$.
       \end{description}
     \end{pzn}
     
   \subsection{Funktory $\delta^*$ a $\delta_*$} 
     
     \begin{dfn}\label{def-delta-*}
       Nechť $X\in mod(A)$ a $\delta$ značí následující exaktní posloupnost \\\\
       \centerline{\xymatrix{
         0 \ar@{->}[r] 
           & M \ar@{->}[r]^f 
           & N \ar@{->}[r]^g 
           & L \ar@{->}[r]
           & 0
       }.}\\
       \begin{description}
         \item[(a)] Definujme $\delta_*(X)$ exaktností následující posloupnosti $R$-modulů: \\\\
           \centerline{\xymatrix{
             0 \ar@{->}[r] 
               & Hom_A(L,X) \ar@{->}[r]^{(-\circ g)_X}
               & Hom_A(N,X) \ar@{->}[r]^{(-\circ f)_X}
               & Hom_A(M,X) \ar@{->}[r]
               & \delta_*(X) \ar@{->}[r]
               & 0 &\,&             
             }}\\ 
         \item[(b)] Definujme $\delta^*(X)$ exaktností následující posloupnosti $R$-modulů: \\\\
           \centerline{\xymatrix{
             0 \ar@{->}[r] 
               & Hom_A(X,M) \ar@{->}[r]^{(f\circ -)_X}
               & Hom_A(X,N) \ar@{->}[r]^{(g\circ -)_X}
               & Hom_A(X,L) \ar@{->}[r]
               & \delta^*(X) \ar@{->}[r]
               & 0 &\,&
           }}  
       \end{description}
     \end{dfn}
     
     \begin{thm}\label{thm-delta}
       Nechť $X,X'\in mod(A)$ a $h\in Hom_A(X,X')$ a $\delta$ je jako v definici 
       \hyperref[def-delta-*]{Definici \ref*{def-delta-*}}. Pak platí:
       \begin{description}
         \item[(a)] Položme $\delta_*(h):=((h \circ -)_M)_{Cok}$ jako na následujícím diagramu:  \\\\
           \centerline{\xymatrix{
             0 \ar@{->}[r] 
               & Hom_A(L,X) \ar@{->}[r]^{(-\circ g)_X} \ar@{->}[d]^{(h\circ -)_L}
               & Hom_A(N,X) \ar@{->}[r]^{(-\circ f)_X} \ar@{->}[d]^{(h\circ -)_N}
               & Hom_A(M,X) \ar@{->}[r] \ar@{->}[d]^{(h\circ -)_M}
               & \delta_*(X) \ar@{->}[r] \ar@{->}[d]^{((h \circ -)_M)_{Cok}}
               & 0\\
             0 \ar@{->}[r] 
               & Hom_A(L,X') \ar@{->}[r]^{(-\circ g)_X'}
               & Hom_A(N,X') \ar@{->}[r]^{(-\circ f)_X'}
               & Hom_A(M,X') \ar@{->}[r]
               & \delta_*(X') \ar@{->}[r]
               & 0 &\,&\\
           }}\\\\\\
           Spolu s tímto zobrazením je $\delta_*$ kovariantní funktor $mod(A)\rightarrow mod(R)$.

         \item[(b)] Položme $\delta^*(h):=((-\circ h)_L)_{Cok}$ jako na následujícím diagramu: \\\\
          \centerline{\xymatrix{
             0 \ar@{->}[r] 
               & Hom_A(X',M) \ar@{->}[r]^{(f\circ -)_X'} \ar@{->}[d]^{(-\circ h)_M}
               & Hom_A(X',N) \ar@{->}[r]^{(g\circ -)_X'} \ar@{->}[d]^{(-\circ h)_N}
               & Hom_A(X',L) \ar@{->}[r] \ar@{->}[d]^{(-\circ h)_L}
               & \delta^*(X') \ar@{->}[r] \ar@{->}[d]^{((-\circ h)_L)_{Cok}}
               & 0 \\
             0 \ar@{->}[r] 
               & Hom_A(X,M) \ar@{->}[r]^{(f\circ -)_X}
               & Hom_A(X,N) \ar@{->}[r]^{(g\circ -)_X}
               & Hom_A(X,L) \ar@{->}[r]
               & \delta^*(X) \ar@{->}[r]
               & 0 &\,&
           }}\\\\\\
           Spolu s tímto zobrazením je $\delta^*$ kontravariantní funktor $mod(A)\rightarrow mod(R)$.
       \end{description}
     \end{thm}
     
     \begin{proof}
       Dokážeme pouze (a), část (b) se dokáže analogicky. Máme 
       $Hom_A(N,X)$, $Hom_A(M,X)\in mod(R)$ a tedy i $\delta_*(X)\in mod(R)$ 
       jakožto kojádro homomorfismu konečně generovaných $R$-modulů. 
       
       Hom 
       funktory nám zobrazují $A$-homomorfismy na $R$-homomorfismy a stejně tak 
       funktor $\delta_*$, jelikož je definovaný jako kojádro zobrazení 
       $R$-homomorfismů. Dále je zřejmé, že $\delta_*(1_X)=1_{\delta_*(X)}$.
     
       Zvolme si pevně $X,X',X''\in mod(A)$, $h\in Hom_A(X,X')$ a $h'\in 
       Hom_A(X',X'')$. Dokážeme, že $\delta_*(h'h)=\delta_*(h')\delta_*(h)$. 
       Máme následující komutativní diagram: \\\\
       \centerline{
           \xymatrix{
             Hom_A(M,X) \ar@{->}[rr]^{\pi} \ar@{.>}[d]^{(h\circ -)_M} \ar@/_5pc/[dd]_{(h'h\circ-)_M}
               & & \delta_*(X) \ar@{.>}[d]_{\delta_*(h)} \ar@/^5pc/[dd]^{\delta_*(h'h)} \\
             Hom_A(M,X') \ar@{->}[rr]^{\pi'} \ar@{.>}[d]^{(h'\circ -)_M}
               & & \delta_*(X') \ar@{.>}[d]_{\delta_*(h')} \\   
             Hom_A(M,X') \ar@{->}[rr]^{\pi''}
               & & \delta_*(X') \\           
           }}\\\\\\
       Pro každé $u\in Hom_A(M,X)$ platí
       \begin{eqnarray}
         (h'h\circ-)_M(u)
         &=& h'hu \nonumber \\
         &=& (h'\circ-)_M(hu) \nonumber \\
         &=& (h'\circ-)_M(h\circ-)_M(u), \nonumber
       \end{eqnarray}
       pak $(h'h\circ-)_M=(h'\circ-)_M(h\circ-)_M$. Z komutativity diagramu 
       plyne 
       \begin{eqnarray}
         \delta_*(h'h)\pi
         &=& \pi''(h'h\circ-)_M \nonumber \\
         &=& \pi''(h'\circ-)_M(h\circ-)_M  \nonumber \\
         &=& \delta_*(h')\pi'(h\circ-)_M \nonumber \\
         &=& \delta_*(h')\delta_*(h)\pi,  \nonumber 
       \end{eqnarray}
       a protože $\pi$ je epimorfismus, tak nám tato rovnost implikuje \\\\
       \centerline{ $\delta_*(h'h)=\delta_*(h')\delta_*(h)$.}
     \end{proof}
     
     \begin{dfn}
       Funktor \\
       \centerline{$\delta_*:mod(A)\rightarrow mod(R)$}\\\\
       se nazývá kovariantní defekt funktor a funktor  \\\\
       \centerline{$\delta^*:mod(A)\rightarrow mod(R)$} \\\\ 
       se nazývá kontravariantní defekt funktor.
     \end{dfn}
     
  \subsection{Skoro štěpitelné posloupnosti}
     
     \begin{lem}\label{lem-almost-split-def}
       Pro následující exaktní posloupnost\\\\
       \centerline{\xymatrix{
         0 \ar@{->}[r] 
           & M \ar@{->}[r]^f 
           & N \ar@{->}[r]^g 
           & L \ar@{->}[r]
           & 0
       }}\\\\ 
       v $mod-A$ jsou následující tvrzení ekvivalentní:
       \begin{description}
         \item[(a)] Existuje $f'\in Hom_A(N,M)$ takové, že $f'f=1_M$.
         \item[(b)] Existuje $g'\in Hom_A(L,N)$ takové, že $gg'=1_L$.  
       \end{description}        
     \end{lem}
     \begin{proof}
       Dokážeme, že z (a) plyne (b). Opačná implikace je analogická. Nechť tedy $f'\in Hom_A(N,M)$ 
       je takový, že $f'f=1_M$. Uvažujme následující komutativní diagram v 
       $mod(A)$:\\\\
       \centerline{\xymatrix{
         0 \ar@{->}[r] 
           & M \ar@{->}[r]^f \ar@{=}[d] 
           & N \ar@{->}[r]^g \ar@{=}[d] \ar@{->}[ld]_{f'}
           & L \ar@{->}[r]      \ar@{.>}[d]^{1_C} \ar@{.>}[ld]_{g'}
           & 0 \\
         0 \ar@{->}[r] 
           & M \ar@{->}[r]^f 
           & N \ar@{->}[r]^g 
           & L \ar@{->}[r]
           & 0
       }}\\\\\\
       Položme $h:=1_N-ff'$, pak \\\\
       \centerline{$hf=f-(ff')f=f-f(f'f)=0$,} \\\\
       neboli $h$ se faktorizuje skrze $Cok(f)=L$. Což znamená, že existuje homomorfismus $g'\in Hom_A(L,N)$ 
       takový, že \\\\
       \centerline{$g'g=h$.} \\\\
       Pak $(gg')g=g(g'g)=gh=g1_B-g(ff')=g-(gf)f'=1_Lg$, a protože $g$ je 
       epimorfismus, musí být \\\\
       \centerline{$gg'=1_C$.}
     \end{proof}
     
     \begin{dfn}
       \begin{description} \item
         \item[(a)] Nechť $M,N\in mod(A)$ a $f\in Hom_A(M,N)$ je epimorfismus. 
         Řekneme, že $f$ je štěpitelný epimorfismus, pokud existuje $f'\in mod_A(N,M)$
         takový, že \\\\ 
         \centerline{$ff'=1_N$.}
         \item[(b)] Nechť $\delta$ je následující exaktní posloupnost v $mod(A)$\\\\
           \centerline{\xymatrix{
             0 \ar@{->}[r] 
               & M \ar@{->}[r]^f 
               & N \ar@{->}[r]^g 
               & L \ar@{->}[r]
               & 0,
               &
               &
           }}\\\\
           pak řekneme, že $\delta$ je štěpitelná posloupnost, pokud 
             splňuje jednu z ekvivalentních podmínek z 
             \hyperref[lem-almost-split-def]{Lemma \ref*{lem-almost-split-def}}.      
       \end{description}         
     \end{dfn}
     
     \begin{dfn}
         Nechť $\delta$ je následující exaktní posloupnost v $mod(A)$\\\\
           \centerline{\xymatrix{
             0 \ar@{->}[r] 
               & M \ar@{->}[r]^f 
               & N \ar@{->}[r]^g 
               & L \ar@{->}[r]
               & 0,
               &
               &
           }}\\\\
           pak řekneme, že $\delta$ je skoro štěpitelná posloupnost, pokud 
         splňuje následující dvě podmínky:                
         \begin{description}  
           \item[($g$ je zprava minimální)] Pokud $h\in End_A(M)$ a $gh=g$, pak $g$ je izomorfismus.
           \item[($g$ je zprava skoro štěpitelný)]  Homomorfismus $g$ není štěpitelný epimorfismus a pro 
             každé $Y\in mod(A)$ a $h\in 
             Hom_A(Y,L)$, které není štěpitelný epimorfismus, existuje $u\in Hom_A(Y,N)$ 
             takové, že $h=gu$.\\\\
             \centerline{\xymatrix{
               & & & Y \ar@{->}[d]^h \ar@{-->}[ld]_u \\             
               0 \ar@{->}[r] 
                 & M \ar@{->}[r]^f 
                 & N \ar@{->}[r]^g 
                 & L \ar@{->}[r]
                 & 0
                 &
                 &
                 &
             }}\\
       \end{description}         
     \end{dfn}     
     
     Poznamenejme, že \cite{2} Proposition 1.14 (Kapitola V) popisuje řadu 
     ekvivalentních definic skoro štěpitelné posloupnosti.
     
     \begin{thm}
       \begin{description} \item
         \item[(a)] Všechny skoro štěpitelné posloupnosti v $mod(A)$ jsou tvaru 
           \\\\
           \centerline{\xymatrix{
             0 \ar@{->}[r] 
               & DTr(X) \ar@{->}[r] 
               & E \ar@{->}[r] 
               & X \ar@{->}[r]
               & 0
               &
           }}\\\\
           kde $E\in mod(A)$ a $X\in mod(A)$ je nerozložitelý a neprojektivní.
         \item[(b)] Pro každý $X\in mod(A)$ nerozložitelný a neprojektivní modul
           existuje skoro štěpitelná posloupnost 
           \\\\
           \centerline{\xymatrix{
             0 \ar@{->}[r] 
               & DTr(X) \ar@{->}[r] 
               & E \ar@{->}[r] 
               & X \ar@{->}[r]
               & 0
               & 
           }} \\\\
           v $mod(A)$.
       \end{description}      
     \end{thm}
     \begin{proof}
        \begin{description}
           \item
           \item[(a)]
           Pokud je \\\\
             \centerline{\xymatrix{
               0 \ar@{->}[r] 
                 & Y \ar@{->}[r]^f 
                 & E \ar@{->}[r]^g
                 & X \ar@{->}[r]
                 & 0
                 &
             }}\\\\
         skoro štěpitelná posloupnost, pak dle \cite{2} Proposition. 1.14 (Kapitola 5) 
         je $X$ nerozložitelný a $Y\simeq DTr(X)$. Pokud by $X$ bylo projektivní, 
         pak by existovalo $g'\in Hom_A(X,E)$ takové, že $gg'=1_E$ a posloupnost 
         by byla štěpitelná.         
         \item[(b)] Tvrzení plyne z \cite{2} Theorem 1.15 (Kapitola 5).
       \end{description}      
     \end{proof}
     
     \begin{dfn}
       Nechť $U,V\in mod(A)$. Označme následující dvě množiny:
       \begin{description}
         \item[(a)] $\Upsilon_{U,V}:=\{$Krátké exaktní posloupnosti vedoucí z $U$ do $V\}$
         \item[(b)] $\hat{\Upsilon}_{U,V}:=\{$Skoro štěpitelné posloupnosti vedoucí z $U$ do $V\}\subseteq \Upsilon_{U,V}$  
       \end{description}      
       Definujme relaci ekvivalence $\sim$ na $\Upsilon_{U,V}$ respektive na $\hat{\Upsilon}_{U,V}$ 
       tak, že dvě posloupnosti \\
           \centerline{\xymatrix{
             0 \ar@{->}[r] 
               & U \ar@{->}[r]
               & E \ar@{->}[r]
               & V \ar@{->}[r]  
               & 0 \\
             0 \ar@{->}[r] 
               & U \ar@{->}[r]
               & E' \ar@{->}[r]
               & V \ar@{->}[r]
               & 0 
           }} \\\\\\
       jsou ekvivalentní, pokud existuje $e\in Hom_A(E,E')$ takové, že 
       následující diagram komutuje:  \\
           \centerline{\xymatrix{
             0 \ar@{->}[r] 
               & U \ar@{->}[r] \ar@{=}[d] 
               & E \ar@{->}[r] \ar@{->}[d]_e 
               & V \ar@{->}[r] \ar@{=}[d] 
               & 0 \\
             0 \ar@{->}[r] 
               & U \ar@{->}[r] 
               & E' \ar@{->}[r] 
               & V \ar@{->}[r]
               & 0 
           }} \\\\
           
          Poznamenejme, že symetrie této relace plyne z 
          \hyperref[lemma-five]{Lemma \ref*{lemma-five}}, 
          zatímco tranzitivita a reflexivita jsou zřejmé. Zároveň bychom 
          měli relaci $\sym$ indexovat koncovými moduly  posloupností 
          $U,V$, z kontextu je ale vždy zřejmé, k jakým modulům se vztahuje.
          
         Na třídách ekvivalence $\Upsilon_{U,V}/\sim$ nyní zavedeme sčítání 
         (nazýváné Baerova suma), s nímž bude mít $\Upsilon_{U,V}/\sim$ strukturu  
         abelovské grupy.
         
         Mějme tedy dvě krátké exaktní posloupnosti vedoucí z $U$ do $V$\\\\
           \centerline{\xymatrix{
             \xi_1 :0 \ar@{->}[r] 
               & U \ar@{->}[r]^{\beta_1}
               & E \ar@{->}[r]^{\alpha_1}
               & V \ar@{->}[r]  
               & 0 \\
             \xi_2 :0 \ar@{->}[r] 
               & U \ar@{->}[r]^{\beta_2} 
               & E' \ar@{->}[r]^{\alpha_2} 
               & V \ar@{->}[r]
               & 0 
           }} \\\\\\
         Nechť $[\xi_1]$ resp. $[\xi_2]$ značí jejich třídy ekvivalence. Definujme 
         $[\xi_1]+[\xi_2]\in \Upsilon_{U,V}/\sim$
         následovně: Nechť $f:V\to V\oplus V$ je dáno předpisem $f(v)=(v,v)$ pro všechna 
         $v\in V$ a $g:U \oplus U\to U$ je dáno předpisem $g(u_1,u_2)=u_1+u_2$ 
         pro všechna $u_1,u_2\in U$. Nechť $\xi_1\oplus\xi_2$ je suma \\\\
           \centerline{\xymatrix{
             \xi_1\oplus\xi_2 :0 \ar@{->}[r] 
               & U\oplus U \ar@{->}[r]^{(\beta_1,\beta_2)}
               & E\oplus E \ar@{->}[r]^{(\alpha_1,\alpha_2)}
               & V\oplus V \ar@{->}[r]  
               & 0
           }} \\\\
        a položme $[\xi_1]+[\xi_2]$ rovno třídě ekvivalence exaktní posloupnosti\\\\
           \centerline{\xymatrix{
             0 \ar@{->}[r] 
               & U \ar@{->}[r]
               & \tilde E \ar@{->}[r]
               & V \ar@{->}[r]  
               & 0
           },} \\\\
           kterou spočteme následovně jedním ze dvou možných postupů: 
           \begin{description}
             \item[(a)]
               Modul $E_1$ dostaneme jako pushout $g$ a $(\beta_1,\beta_2)$. 
               Modul $\tilde E$ následně položíme rovno pullbacku $h_1$ a $f$. \\\\             
               \centerline{\xymatrix{
                 0 \ar@{->}[r] 
                   & U\oplus U \ar@{->}[rr]^{(\beta_1,\beta_2)} \ar@{->}[d]_g
                   & & E\oplus E \ar@{->}[rr]^{(\alpha_1,\alpha_2)} \ar@{->}[d]
                   & & V\oplus V \ar@{->}[r] \ar@{=}[d]
                   & 0 \\
                 0 \ar@{->}[r] 
                   & U \ar@{->}[rr] \ar@{=}[d]
                   & &  E_1 \ar@{->}[rr]^{h_1} 
                   & & V\oplus V \ar@{->}[r]   
                   & 0 \\
                 0 \ar@{->}[r] 
                   & U \ar@{->}[rr]
                   & & \tilde E \ar@{->}[rr] \ar@{->}[u]
                   & & V \ar@{->}[r]  \ar@{->}[u]_f
                   & 0 & &
                 }}\\     
             \item[(b)]
               Modul $E_2$ dostaneme jako pullback $f$ a $(\alpha_1,\alpha_2)$. 
               Modul $\tilde E$ následně položíme rovno pushoutu $h_2$ a $g$. \\\\             
               \centerline{\xymatrix{
                 0 \ar@{->}[r] 
                   & U\oplus U \ar@{->}[rr]^{(\beta_1,\beta_2)} \ar@{=}[d]
                   & & E\oplus E \ar@{->}[rr]^{(\alpha_1,\alpha_2)} 
                   & & V\oplus V \ar@{->}[r] 
                   & 0  \\
                 0 \ar@{->}[r] 
                   & U\oplus U \ar@{->}[rr]^{h_2} \ar@{->}[d]_g
                   & &  E_2 \ar@{->}[rr] \ar@{->}[d] \ar@{->}[u]
                   & & V \ar@{->}[r] \ar@{->}[u]_f
                   & 0 \\
                 0 \ar@{->}[r] 
                   & U \ar@{->}[rr]
                   & & \tilde E \ar@{->}[rr] 
                   & & V \ar@{->}[r]  \ar@{=}[u]
                   & 0 & &
                 }}\\
           \end{description}
           
           Oběma postupy dospějeme ke stejnému výsledku, jak je znázorněno na 
           následujícím komutativním diagramu:
           \\\\
           \centerline{\xymatrix{
             0 \ar@{.>}[rr] 
               & & U\oplus U \ar@{->}[rr]^{(\beta_1,\beta_2)} \ar@{=}[dd] \ar@{->}[ld]_g
               & & E\oplus E \ar@{->}[rr]^{(\alpha_1,\alpha_2)} \ar@{->}[ld]
               & & V\oplus V \ar@{.>}[r]  \ar@{=}[ld]
               & 0 \\
             0 \ar@{.>}[r] 
               & U \ar@{->}[rr] \ar@{=}[dd]
               & & E_1 \ar@{->}[rr] \ar@{->}[dd]
               & & V\oplus V \ar@{.>}[rr]  
               & & 0 \\
             0 \ar@{.>}[rr] 
               & & U\oplus U \ar@{->}[rr] \ar@{->}[ld]_g
               & & E_2 \ar@{->}[rr] \ar@{->}[uu] \ar@{->}[ld]
               & & V \ar@{.>}[r]  \ar@{->}[uu]_f \ar@{=}[dd] \ar@{=}[ld]
               & 0 \\
             0 \ar@{.>}[r] 
               & U \ar@{->}[rr]
               & & \tilde E \ar@{->}[rr]
               & & V \ar@{.>}[rr]  \ar@{->}[uu]_f
               & & 0 \\
           }} \\\\\\
         Nulový prvek $\Upsilon_{U,V}/\sim$ bude exaktní posloupnost \\\\
           \centerline{\xymatrix{
             0 \ar@{->}[r] 
               & U \ar@{->}[r]
               & U\oplus V \ar@{->}[r]
               & V \ar@{->}[r]  
               & 0
           },} \\\\
         a inverzní prvek k posloupnosti \\\\
           \centerline{\xymatrix{
             0 \ar@{->}[r] 
               & U \ar@{->}[r]^{\beta_1}
               & E \ar@{->}[r]^{\alpha_1}
               & V \ar@{->}[r]  
               & 0
           }} \\\\ je \\\\
           \centerline{\xymatrix{
             0 \ar@{->}[r] 
               & U \ar@{->}[r]^{\beta_1}
               & E \ar@{->}[r]^{-\alpha_1}
               & V \ar@{->}[r]  
               & 0
           }.} \\
                    
         To, že oba postupy výpočtu Baerovy sumy zaručují stejný výsledek, 
         že je Baerova suma 
         správně definovaná a splňuje všechny axiomy, aby
         $\Upsilon_{U,V}/\sim$ spolu s ní tvořilo abelovskou grupu, zde 
         dokazovat nebudeme. Prodrobnou konstrukci a důkaz je možné nalézt v 
         \cite{2} Kapitole I Sekci 5.
     \end{dfn}
          
     Zvolme si nyní pevně libovolný nerozložitelný a neprojektivní $A$-modul 
     $X$. Budeme s ním pracovat po zbytek této kapitoly.
  
     \begin{dfn}
       Něchť $\Gamma:=\underline{End}_A(X)$.
     \end{dfn}
     
     \begin{lem}\label{lem-D-na-jednoduchych}
     \begin{description}
       \item
       \item[(a)] Artinovská $R$-algebra $\Gamma$ je lokální okruh. 
       \item[(b)] Pokud $M\in mod(\Gamma)$ je nenulový, pak $D(M)\in mod(\Gamma^{op})$ je nenulový. 
       \item[(c)] Pokud $M\in mod(\Gamma)$ je jednoduchý, pak $D(M)\in mod(\Gamma^{op})$ je jednoduchý. 
      \end{description} 
     \end{lem}
     \begin{proof}
       \begin{description}
         \item
         \item[(a)] Dle \cite{2} Theorem 2.2 (Kapitola 2) je $End_A(X)$ 
           lokální okruh. Pak je dle \hyperref[faktor-lokalniho-lokalni]{Lemma \ref*{faktor-lokalniho-lokalni}}
           lokálním okruhem i $\Gamma$. 
           
         \item[(b)] Nechť $M\in mod(\Gamma)$ a $m\in M$ je nenulový prvek. 
           Připomeňme, že $M$ je zároveň $R$-modulem a $I$ je injektivní obal $R/\underline{m}$, 
           kde $\underline{m}$ je maximální ideál $R$. Nechť $\hat f\in Hom_R(Rm,I)$ 
           je daný předpisem \\\\
           \centerline{$\hat f (rm):=r+\underline m$} \\\\
           a $i$ je kanonická inkluze \\\\
           \centerline{$i:Rm\rightarrow M$.} \\\\
           Protože $I$ je injektivní, existuje $f\in Hom_R(M,I)$  takové, že 
           \\\\
           \centerline{$fi=\hat f$} \\\\
           a \\\\
           \centerline{$f(m)=fi(m)=\hat f (m)=1_R+ \underline m\neq 0$.} \\\\
           A tedy $f\in D(M)$ je nenulový prvek.           
         
         \item[(c)] Nechť $M\in mod(\Gamma)$ je jednoduchý. Pokud $M=0$, pak $D(M)=0$. 
           Nechť tedy je $M$ nenulový modul.  Předpokládejme pro spor, že $D(M)$ není jednoduchý 
           $\Gamma^{op}$-modul. Pak $D(M)$ obsahuje netriviální podmodul $U$. 
           To nám dává následující exaktní posloupnost $mod(\Gamma^{op})$:
           \\\\
           \centerline{\xymatrix{
             0 \ar@{->}[r] 
               & D(D(M)/U) \ar@{->}[r] 
               & M \ar@{->}[r] 
               & D(U) \ar@{->}[r] 
               & 0 \\
           }} \\\\
           Připomeňme, že funktor $D=Hom_R(-,I):mod(\Gamma)\rightarrow mod(\Gamma^{op})$ 
           je dualita a tedy máme $\Gamma$-modulový izomorfismus $D^2(N)\simeq N$ 
           pro každý $N\in mod(\Gamma)$. Aplikujeme-li tedy funktor $D$ na naši exaktní posloupnost, 
           dostaneme následující posloupnost v $mod(\Gamma)$:          
           \\\\
           \centerline{\xymatrix{
             0 \ar@{->}[r] 
               & U \ar@{->}[r] 
               & D(M) \ar@{->}[r] 
               & D(M)/U \ar@{->}[r] 
               & 0 \\
           }} \\\\
           Protože $U\neq D(M)$, tak $D(M)/U\neq 0$. Pak jelikož je $M$  
           jednoduchý, tak $D(D(M)/U)\simeq D^2(M)$ a z exaktnosti naší 
           posloupnosti vidíme, že  $D(U)=0$. Pak ale i $D=0$ dle bodu (a), což 
           je spor s naším předpokladem, že $U$ je netriviální podmodul $D(M)$.
       \end{description}
     \end{proof}
     
     \begin{dfn}
       Nechť $S$ je artinovský okruh, $M\in mod(S)$.  Definujme
       \begin{description}
         \item[(a)] $Top_S(M):=M/rad(S)M$.
         \item[(b)] $Soc_S(M):=\sum\{U|U$ je jednoduchý $S$-podmodul  $M\}$.
       \end{description} 
     \end{dfn}
     
     \begin{lem}\label{lem-soc-top}
       Platí:
       \begin{description}
         \item[(a)] $Top_\Gamma(\Gamma)$ je jednoduchý $\Gamma$-modul. 
         \item[(b)] $Soc_\Gamma(D\Gamma)\simeq DTop_{\Gamma^{op}}(\Gamma)$ jako 
         $\Gamma$-moduly.
         \item[(c)] $Soc_\Gamma(D\Gamma)$ je jednoduchý $\Gamma$-modul.
       \end{description} 
     \end{lem}
     
     \begin{proof}
       \begin{description}
         \item
         \item[(a)] Máme izomorfismus $End_\Gamma(\Gamma)=Hom_\Gamma(\Gamma,\Gamma)\simeq 
           \Gamma$. Dle \hyperref[lem-D-na-jednoduchych]{Lemma \ref*{lem-D-na-jednoduchych}} 
           je $End_\Gamma(\Gamma)$ lokální okruh. Protože $\Gamma$ je artinovská 
           $R$-algebra, je také artinovským okruhem. Navíc $\Gamma$ je 
           projektivní $\Gamma$-modul. Z toho plyne, že $rad(\Gamma)\Gamma$ je 
           jediný maximální podmodul $\Gamma$.
           
           Ukážeme nyní, že $Top_\Gamma(\Gamma)=\Gamma/rad(\Gamma)\Gamma$ je 
           jednoduchý $\Gamma$-modul. Nechť $M$ je nenulový podmodul 
           $\Gamma/rad(\Gamma)\Gamma$. Pak $M$ je tvaru \\\\
           \centerline{$M=N/rad(\Gamma)\Gamma$} \\\\
           pro nějaký $\Gamma$-modul $N$ takový, že \\\\
           \centerline{$rad(\Gamma)\Gamma\subseteq N\subseteq \Gamma$.}\\\\
           Protože $M$ je nenulový je $rad(\Gamma)\neq N$. Protože $rad(\Gamma)\Gamma$ 
           je maximální podmodul $\Gamma$, musí být $N=\Gamma$ a tedy \\\\
           \centerline{$M=\Gamma/rad(\Gamma)\Gamma$.}
           
         \item[(b)] Uvažujme následující exaktní posloupnost 
           $\Gamma^{op}$-modulů: \\\\
           \centerline{$0
             \longrightarrow rad(\Gamma)  
             \longrightarrow \Gamma   
             \longrightarrow \Gamma/rad(\Gamma)  
             \longrightarrow 0$} \\\\
           Aplikací funktoru $D$ dostaneme exaktní posloupnost $\Gamma$-modulů:\\\\
           \centerline{$0
             \longrightarrow D(\Gamma/rad(\Gamma))  
             \longrightarrow D(\Gamma)   
             \longrightarrow D(rad(\Gamma))  
             \longrightarrow 0$} \\\\
           Dle \cite{2} Proposition 3.1 (Kapitola 1) je $\Gamma/rad(\Gamma)$ 
           polojednoduchý $\Gamma^{op}$-modul. Tedy z 
           \hyperref[lem-D-na-jednoduchych]{Lemma \ref*{lem-D-na-jednoduchych}} 
           a z komutativity $D$ s konečnými direktními sumami (plyne z 
           \hyperref[dir-sum-hom]{Lemma \ref*{dir-sum-hom}}) je  
           $D(\Gamma/rad(\Gamma))$ polojednoduchý podmodul $D\Gamma$. To 
           znamená, že \\\\
           \centerline{$D(\Gamma/rad(\Gamma))\subseteq Soc_\Gamma(D\Gamma)$.} 
           \\\\
           Ze stejného důvodu je $DSoc_\Gamma(D\Gamma)$ polojednoduchý 
           $\Gamma^{op}$-modul. Navíc \\\\
           \centerline{$Soc_\Gamma(D\Gamma)\subseteq D\Gamma$} 
           \\\\
           a dosáváme následující komutativní diagram v $mod(\Gamma)$:\\\\
           \centerline{\xymatrix{
               & 0 \ar@{->}[d] \\
             0 \ar@{->}[r]
               & D(\Gamma/rad(\Gamma)) \ar@{->}[r] \ar@{->}[d]
               & D\Gamma \ar@{=}[d] \\
             0 \ar@{->}[r]
               & Soc_\Gamma(D\Gamma) \ar@{->}[r]
               & D\Gamma \ar@{->}[r]
               & D\Gamma/Soc_\Gamma(D\Gamma) \ar@{->}[r]
               & 0
           }}\\\\\\
           Aplikací $D$ dostaneme následující komutativní diagram z exaktními 
           řádky v $mod(\Gamma^{op})$:\\\\
           \centerline{\xymatrix{
              0 \ar@{->}[r]
                & D(D\Gamma/Soc_\Gamma(D\Gamma)) \ar@{->}[r]
                & \Gamma \ar@{=}[d] \ar@{->}[rr]^b
                & & D(Soc_\Gamma(D\Gamma))  \ar@{->}[r] \ar@/^2pc/[d]^d
                & 0 \\
              0 \ar@{.>}[r]
                & rad(\Gamma) \ar@{.>}[r]^a
                & \Gamma \ar@{->}[rr]^c
                & & \Gamma/rad(\Gamma) \ar@{->}[r] \ar@{->}[d] \ar@/^2pc/[u]^e
                & 0 \\
              & & & & 0
           }}\\\\           
           Víme, že $rad(\Gamma)$ je jádro $c$, tedy ho doplníme do 
           diagramu a dostaneme exaktní řádek. Protože $rad(\Gamma)$ anihiluje 
           polojednoduché \,$\Gamma^{op}$-moduly, \,je $b1_\Gamma a=0$. Navíc
           $\Gamma/rad(\Gamma)$ je kojádro $a$ a tedy musí existovat 
           homomorfismus
           $e\in Hom_{\Gamma^{op}}(\Gamma/rad(\Gamma), D(Soc_\Gamma(D\Gamma)))$ 
           takový, že \\\\
           \centerline{$ec=b1_\Gamma=b$.}\\\\
           Navíc protože $b$ je epimorfismus, tak je jím také $e$. 
           Složení homomorfismů $de\in End_{\Gamma^{op}}(\Gamma/rad(\Gamma))$ je tedy podle 
           \cite{2} Proposition 1.4 (Kapitola 1) izomorfismus, protože $\Gamma/rad(\Gamma)$ 
           je konečně generovaný modul. Z toho plyne že $e$ je navíc 
           epimorfismem, neboli \\\\
           \centerline{$D(Soc_\Gamma(D\Gamma))\simeq \Gamma/rad(\Gamma)=Top_{\Gamma^{op}}(\Gamma)$} \\\\
           jakožto $\Gamma^{op}$-moduly. A ekvivalentně \\\\
           \centerline{$Soc_\Gamma(D\Gamma)\simeq DTop_{\Gamma^{op}}(\Gamma)$} 
           \\\\
           jako $\Gamma$-moduly.     
                    
         \item[(c)] Plyne z (a), pokud budeme nahlížet na $\Gamma$ jako na 
           $\Gamma^{op}$-modul. Protože $Top_{\Gamma^{op}}(\Gamma)$ je jednoduchý 
           $\Gamma^{op}$-modul, plyne z 
           \hyperref[lem-D-na-jednoduchych]{Lemma \ref*{lem-D-na-jednoduchych}} a ze 
           vztahu $Soc_\Gamma(D\Gamma)\simeq DTop_{\Gamma^{op}}(\Gamma)$, že je 
           jednoduchý $\Gamma$-modul.
       \end{description}
     \end{proof}

     \begin{lem}\label{lem-jednoduchy-modul-gen}
       Každý jednoduchý modul $M$ okruhu $S$ může být vygenerován jakýmkoli svým 
       nenulovým prvkem.
     \end{lem}
     \begin{proof}
       Pokud $m\in M\backslash\{0\}$, pak $m$ generuje nenulový podmodul $M$, 
       který musí být z jednoduchosti $M$ roven celému $M$.
     \end{proof}
     
     \begin{thm}\label{ekvivalence-upsilon-ext}
       \begin{description} \item 
         \item[(a)] Nechť $U,V \in mod(A)$. Pak $\Upsilon_{U,V}/\sim$ je 
           abelovská grupa a máme izomorfismus abelovských grup: \\\\ 
           \centerline{$Ext_A^1(V,U)\simeq (\Upsilon_{U,V}/\sim)$}
         \item[(b)] Nechť $X\in mod(A)$ je nerozložitelný a neprojektivní. Potom 
           můžeme definovat na $\Upsilon_{DTr(X),X}/\sim$ takovou strukturu, že:
           \begin{description}
             \item[(i)] $(\Upsilon_{DTr(X),X}/\sim)\in mod(\Gamma)$,
             \item[(ii)] 
             $Soc_\Gamma(\Upsilon_{DTr(X),X}/\sim)=(\hat{\Upsilon}_{DTr(X),X}/\sim)$.
           \end{description}
       \end{description}      
     \end{thm}
     \begin{proof}
       \begin{description}
         \item
         
         \item[(a)] \cite{5} Theorem. 7.21.
           
         \item[(b)] Nebudeme zde provádět kompletní důkaz. Všechny podrobnosti 
         je možné nalézt v \cite{2} Kapitole 5. Dále pouze naznačíme důkaz části (i).
         
         \begin{description}
         \item[(i)]
         Základem je pozorování, že $(\Upsilon_{DTr(X),X}/\sim)\in Mod(\Gamma)$ 
         Důkaz tohoto pozorování provedeme v části věnované konstrukci algoritmu v 
         \hyperref[upsilon-je-modul]{Lemma \ref*{upsilon-je-modul}}. 
         Potom izomorfismus z (a) přenáší $\Gamma$-modulovou strukturu i na
         $Ext_A^1(X,DTr(X))$ a stává se $\Gamma$-izomorfismem.         
         V důkazu \hyperref[thm-omega-x]{Věty \ref*{thm-omega-x}} ukážeme, že
         $Ext_A^1(X,DTr(X))=\delta_*(DTr(X))$ pro určitou krátkou exaktní posloupnost $\delta$
         a protože dle
         \hyperref[thm-delta]{Věty \ref*{thm-delta}} je
         $\delta_*(DTr(X))\in mod(R)$, pak i $Ext_A^1(X,DTr(X))\in mod(R)$.
         A tedy dle \hyperref[lem-mod-Mod]{Lemma \ref*{lem-mod-Mod}} vidíme, že 
         oba $Ext_A^1(X,DTr(X))$ i $(\Upsilon_{DTr(X),X}/\sim)$ jsou konečně generované jako $\Gamma$-moduly.
         \end{description}         
       \end{description}
     \end{proof}
  \clearpage
    \section{Teorie reprezentací}\label{teorie-reprezentaci}
  
    \paragraph{ }Nyní se přesuneme k Algebrám cest. Namísto 
    okruhu $R$ budeme nyní pracovat s pevně zvoleným zvoleným komutativním
    tělesem $K$. $K$-algebra tak ke struktuře $K$-modulu dostává 
    navíc strukturu vektorového prostoru nad  $K$.
    
    Tělěso $K$ má právě dva ideály $0$ a $K$, je tedy lokálním i artinovským okruhem 
    a veškerá teorie z předchozí části je aplikovatelná i pro $R=K$.
  
  \subsection{Toulec a algebra cest}
  
    \begin{dfn}
      Řekneme, že $K$-algebra $A$ je 
      konečné dimenze, pokud dimenze $dim_K A$ vektorového prostoru $A$
      nad $K$ je konečná. 
      
      $K$-vektorový podprostor $B$ $K$-algebry $A$ je $K$-podalgebrou $A$, pokud 
      $1_A\in B$ a $bb'\in B$ pro každý $b,b'\in B$. 
      
      Pokud jsou $A$ a $B$ dvě $K$-algebry, pak okruhový
      homomorfismus $f:A\rightarrow B$ takový, že $f$ je $K$-linerání, nazveme 
      homomorfismem $K$-algeber.
                      
      Algebra $A$ je souvislá, pokud není direktním součtem dvou algeber 
      (nebo ekvivalentně, pokud 0 a 1 jsou jediné centrální idempotenty $A$). 
    \end{dfn}  
    
    \begin{dfn}
      Buď $A$ $K$-algebra a ${e_1,e_2,\ldots,e_m}$ úplná množina primitivních 
      ortogonálních idempotentů $A$. Pak řekneme, že $A$ je základní, pokud $A$-moduly 
      $e_iA\not\simeq e_jA$ pro každé $i\neq j$.
    \end{dfn}
    
    V této části budeme pracovat s pravými $A$-moduly. O levých modulech budeme 
    referovat jako o $A^{op}$-modulech.
    
    \begin{pzn}
      Nechť $A$ je $K$-algebra, pak $A$-modul $M$ je vektorový prostor nad $K$.      
    \end{pzn}
    
    \begin{dfn}
      $A$-modul $M$ nazveme konečně dimenzionálním, pokud je $dim_K M$ konečná.
    \end{dfn}
    
    Nyní si zavedeme pojem toulce. Toulec bude obecnější forma grafu, kde 
    připustíme smyčky (cesty nulové délky) a existenci více cest mezi stejnými 
    body (takový graf bývá také běžně označován jako multigraf). Budeme ho značit 
    písmenem $Q$ z anlického slova quiver (=toulec).
  
    \begin{dfn}
      Toulec $Q$ je čtveřice $(Q_0,Q_1,s,t)$, kde:      
      \begin{description}
        \item[(a)] $Q_0$ je množina, jejíž elementy jsou nazývány body či 
        vrcholy.
        \item[(b)] $Q_1$ je množina, jejíž elementy jsou nazývány šipky.
        \item[(c)] $s,t$ jsou dvě zobrazení $Q_1\rightarrow Q_0$, která každé šipce
        $\alpha\in Q_1$ přiřadí její $s(\alpha)\in Q_0$ počáteční a
        $t(\alpha)\in Q_0$ koncový vrchol.
      \end{description}   
      
      Podtoulec toulce $Q$ je toulec $Q'=(Q_0',Q_1',s',t')$ takový, že $Q_0'\subseteq 
      Q_0$, $Q_1'\subseteq Q_1$ a $s'=s|_{Q_1'}$, $t'=t|_{Q_1'}$. 
      
      Toulec $Q$ je souvislý, pokud jeho vrcholy a šipky, při neuvažování jejich orientace, 
      tvoří souvislý graf. Toulec $Q$ je konečný, pokud $Q_0$ a $Q_1$ jsou konečné
      množiny.
      
      Nechť $Q=(Q_0,Q_1,s,t)$ a $a,b\in Q_0$. Cestou délky $l$ z $a$ do $b$ nazýváme posloupnost 
      $(a|\alpha_1,\alpha_2,\ldots,\alpha_l|b)$, kde $\alpha_k\in Q_1$ pro každé 
      $1\leq k \leq l$, $s(\alpha_1)=a$, $t(\alpha_k)=s(\alpha_{k+1})$ pro každé 
      $1\leq k < l$ a $t(\alpha_l)=b$. Množinu všech cest délky $l$ budeme 
      značit $Q_l$. Dále každý bod $a\in Q_0$ ztotožněme s cestou nulové délky, 
      tu budeme nazývat triviální a značit $\epsilon_a=(a||a)$. Cestu délky $l\geq 1$ 
      nazveme cyklem, pokud počáteční vrchol a koncový vrchol splývají. Cyklus 
      délky 1 nazveme smyčkou. Toulec je acyklický, pokud neobsahuje žádné 
      cykly.  
    \end{dfn}    
    
    \begin{dfn}
      Nechť $Q$ je toulec a $K$ libovolné těleso. Algebra cest $KQ$ toulce $Q$ je $K$-algebra, jejíž báze je 
      tvořena všemi cestami $(a|\alpha_1,\alpha_2,\ldots,\alpha_l|b)$ délky $l\geq 0 $ 
      v $Q$ a součin bázových vektorů je definován následovně: \\\\
      \centerline{
      $(a|\alpha_1,\alpha_2,\ldots,\alpha_l|b)
        (c|\beta_1,\beta_2,\ldots,\beta_k|d)        
        =\delta_{bc}(a|\alpha_1,\alpha_2,\ldots,\alpha_l,\beta_1,\beta_2,\ldots,\beta_k|b)$,
      }\\\\
      kde $\delta_{bc}$ je Krocnekerova delta.
    \end{dfn}
    
    \begin{pr}
      Nechť $Q$ je toulec \\
      \centerline{\xymatrix{
        \circ^1 \ar@{->}[r]_\alpha \ar@/^3pc/[rr]^\gamma
          & \circ^2 \ar@{->}[r]_\beta
          & \circ^3
      }}\\\\
      Báze $KQ$ pak je množina 
      $\{\epsilon_1,\epsilon_2,\epsilon_3,\alpha,\beta,\gamma,\alpha\beta\}$ a 
      násobení bázových prvků je dané tabulkou: \\\\
      \centerline{\begin{tabular}{ c | c c c c c c c  }
        & \epsilon_1 & \epsilon_2 &\epsilon_3 & \alpha & \beta & \gamma & \alpha\beta  \\ \hline
        \epsilon_1 & \epsilon_1 & 0 & 0 & \alpha & 0 & \gamma & $\alpha\beta$ \\
        \epsilon_2 & 0 & \epsilon_2 & 0 & 0 & \beta & 0 & 0 \\
        \epsilon_3 & 0 & 0 & \epsilon_3 & 0 & 0 & 0 & 0 \\
        \alpha & 0 & \alpha & 0 & 0 & $\alpha\beta$ & 0 & 0 \\
        \beta & 0 & 0 & \beta & 0 & 0 & 0 & 0 \\
        \gamma & 0 & 0 & \gamma & 0 & 0 & 0 & 0 \\
        \alpha\beta & 0 & 0 & \alpha\beta & 0 & 0 & 0 & 0 \\
      \end{tabular}}\\\\      
    \end{pr}

    \begin{pzn}
      Násobení bázových elementů $KQ$ je rozšířeno na všechny prvky $KQ$ s pomocí 
      distributivity vůči sčítání. Platí: \\\\
      \centerline{$KQ=KQ_0\oplus KQ_1\oplus\ldots\oplus KQ_l\oplus\ldots$,}\\\\
      kde $KQ_i$ je podprostor $KQ$ generovaný množinou $Q_i$ všech cest délky 
      $i$.
    \end{pzn}
    
    \begin{lem}\label{quiver-kq-lemma}
      Nechť $Q$ je toulec a $KQ$ jeho algebra cest. Pak platí:
      \begin{description}
        \item[(a)] $KQ$ je asociativní algebra. 
        \item[(b)] $KQ$ obsahuje jednotku právě tehdy, když je $Q_0$ konečná množina.
        \item[(c)] $KQ$ je konečně dimenzionální právě, když $Q$ je konečný a acyklický.
      \end{description}      
    \end{lem}
    \begin{proof}
      \begin{description}
        \item 
        \item[(a)]Součin bázových vektorů je skládání cest a to je asociativní vůči 
      sčítání v $KQ$.
        \item[(b)]Triviální cesta $\epsilon_a=(a||a)$ je idempotentem $KQ$ a tedy pro $Q_0$ 
      konečnou je $\sum_{a\in Q_0}\epsilon_a$ jednotka $KQ$. Opačně nechť je $Q_0$ 
      nekonečná a nechť \\\\
      \centerline{$1=\sum_{i=1}^m \lambda_i \omega_i$ ($\lambda_i\in K$ a $\omega_i\in Q_1$)} 
      \\\\
      je jednotka $KQ$. Množina $Q_0'$ počátečních vrcholů  cest $\omega_i$ 
      má nejvýše m prvků. Můžeme tedy zvolit $a\in Q_0\backslash
      Q_0'\neq\emptyset$. Pak ale $\epsilon_a 1=0$, což je spor.

        \item[(c)]Pokud je $Q$ nekonečný, pak je nekonečná také báze $KQ$. Pokud máme 
      cyklus $w=\alpha_1,\alpha_2,\ldots,\alpha_l$, pak máme nekonečně mnoho bázových 
      vektorů tvaru $w^t=(\alpha_1,\alpha_2,\ldots,\alpha_l)^t$ a $KQ$ tedy nemůže být 
      konečně dimenzionální. Opačně, pokud $Q$ je konečně dimenzionální, pak 
      obsahuje pouze konečně mnoho cest a tedy $KQ$ je konečně dimenzionální.
      \end{description}
    \end{proof}
    
    \begin{dsl}\label{dsl-quiver-idem}
      Nechť $Q$ je konečný toulec. Pak prvek $1=\sum_{a\in Q_0}\epsilon_a$ je 
      jednotka $KQ$ a množina $\{\epsilon_a=(a||a)|a\in Q_0\}$ všech triviálních cest  
      je úplná množina primitivních ortogonálních idempotentů $KQ$.
    \end{dsl}
    \begin{proof}
      Plyne z definice násobení, že $\epsilon_a$ jsou ortogonální idempotenty 
      $KQ$. Protože množina $Q_0$ je konečná, prvek $1=\sum_{a\in Q_0}\epsilon_a$ 
      je jednotka $KQ$. 
      
      Zbývá ukázat, že idempotenty $\epsilon_a$ $KQ$ jsou primitivní, neboli že 0 a $\epsilon_a$ 
      jsou jedinými idempotenty $\epsilon_a(KQ)\epsilon_a$. Každý idempotent $\epsilon_a(KQ)\epsilon_a$ 
      může být zapsán ve tvaru $\lambda \epsilon_a + w$, kde $\lambda\in K$ a $w$ 
      je lineární kombinace cest skrze $a$ délky alespoň 1. Rovnost \\\\
      \centerline{$0=\epsilon_a^2 - \epsilon_a=(\lambda^2-\lambda)\epsilon_a + (2\lambda-1)w+w^2$} 
      \\\\
      nám dává $w=0$ a $\lambda ^2=\lambda $ a tedy $\lambda=0$ nebo 
      $\lambda=1$. V prvním případě je $\epsilon=0$ a druhém 
      $\epsilon=\epsilon_a$.
    \end{proof}   
     
    \begin{dfn}
      Nechť $Q$ je konečný a souvislý toulec. Oboustranný ideál algebry cest $KQ$ 
      generovaný šipkami $Q$ nazýváme šipkový ideál $KQ$ a značíme $R_Q$.
    \end{dfn}
    
    \begin{pzn}
      Ideál $R_Q$ můžeme rozložit na direktní součet: \\\\
      \centerline{ $R_Q=KQ_1\oplus KQ_2 \oplus\ldots\oplus KQ_l\oplus\ldots$ } 
      \\\\
      Dále pro každé $l\geq1$ máme $R^l_Q=\bigoplus_{m\geq l}KQ_m$ a tedy $R_Q^l$ 
      je ideál $KQ$ generovaný jako vektorový prostor množinou všech cest 
      délky $\geq l$.
    \end{pzn}
    
    \begin{lem}\label{lem-q-kq-souvisle}
      Nechť $Q$ je konečný toulec. Pak algebra cest $KQ$ je souvislá, právě když 
      $Q$ je souvislý. 
    \end{lem}
    \begin{proof}
      Předpokládejme, že $Q$ není souvislý a $Q'$ buď souvislá komponenta $Q$. 
      Nechť $Q''$ je úplný podtoulec $Q$ mající množinu vrcholů 
      $Q_0''=Q_0\backslash Q_0'$. Dle předpokladu $Q_0'$ ani $Q_0''$ nejsou prázdné. 
      Zvolme libovolné $a\in Q_0'$ a $b\in Q_0''$. Protože $Q$ není souvislý, 
      musí být každá cesta $w$ v $Q$ celá obsažena v $Q_0'$ nebo $Q_0''$. V 
      prvním případě je $w\epsilon_b=0$ a tedy $\epsilon_a w \epsilon_b=0$ a 
      tedy $\epsilon_a (KQ) \epsilon_b=0$. V druhém případě dostaneme, že $\epsilon_b (KQ) 
      \epsilon_a=0$. Tedy $KQ$ není ani v jednom případě souvislá.
      
      Předpokládejme nyní pro spor, že $Q$ je souvislý, ale $KQ$ není. Algebra $KQ$ 
      je direktním součtem dvou algeber, existuje tedy rozklad $Q_0$ na 
      disjunktní podmnožiny $Q_0'$ a $Q_0''$ takové, že pro všechna $a\in Q_0'$ 
      a $b\in Q_0''$ je \\\\
      \centerline{$\epsilon_a(KQ)\epsilon_b=\epsilon_b(KQ)\epsilon_a=0$.}\\\\  
      Protože $Q$ je souvislý, 
      můžeme zvolit taková $a\in Q_0'$  a $b\in Q_0''$, že budou spojena šipkou $\alpha:a\to b$. 
      To je ale spor, protože\\\\
      \centerline{$\alpha=\epsilon_a\alpha\epsilon_b\in 
      \epsilon_a(KQ)\epsilon_b=0$.}
    \end{proof}
    
  \subsection{Přípustný ideál}
    
    \begin{dfn}
      Nechť $Q$ je konečný toulec a $R_Q$ je šipkový ideál algebry cest $KQ$. 
      Řekneme, že oboustranný ideál $I$ algebry $KQ$ je přípustný, pokud 
      existuje $m\geq 2$ takový, že \\\\
      \centerline{$R_Q^m\subseteq I \subseteq R_Q^2$.}\\
      
      Pokud $I$ je přípustný ideál $KQ$, pak dvojici $(Q,I)$ nazveme omezeným 
      toulcem. Algebru $KQ/I$ nazveme algebrou omezeného toulce $(Q,I)$.
    \end{dfn}
    
    \begin{pzn}
      Definice jednoduše říká, že ideál $I$ je přípustný, pokud existuje $m\geq 2$ 
      takové, že $I$ obsahuje včechny cesty délky alespoň $m$ a neobsahuje šipky 
      (cesty délky 1) ani triviální cesty.
      
      V případě acyklického toulce bude přípustný každý ideál obsažený v ideálu 
      $R_Q^2$.
    \end{pzn}
    
    \begin{pr}
      Nechť $Q$ je toulec   \\
      \centerline{\xymatrix{
        \circ^1 \ar@{->}[r]_\alpha \ar@/^3pc/[rr]^\gamma
          & \circ^2 \ar@{->}[r]_\beta
          & \circ^3 \ar@(ur,dr)[]^{\delta}
      }}\\\\
      Ideál $I=\{\alpha\beta-\gamma\delta,\delta^2\}$ je přípustný, ale idál 
      $J=\{\alpha\beta-\gamma\delta\}$ přípustný není, protože neobsahuje například 
      cesty $\delta^k\in R_Q^k$ pro žádné $k\geq 2$.
    \end{pr} 
    
    \begin{dfn}
      Nechť $Q$ je toulec. Relace v $Q$ s koeficienty v $K$ je $K$-lineární 
      kombinace cest délky alespoň dva se stejným počátkem a koncem.       
      Relaci tvaru $w_1-w_2$, kde $w_1$ a $w_2$ jsou dvě cesty, nazveme relací komutativity. 
      
      Pokud množina relací $M$ generuje přípustný ideál, pak řekneme, že je toulec 
      $Q$ omezený množinou relací $M$.
    \end{dfn}
    
    \begin{lem}\label{lem-kq-jako-soucin}
      Nechť $Q$ je konečný toulec a $R_Q$ šipkový ideál $KQ$ a $\epsilon_a=(a||a)$ 
      pro každé $a\in Q_0$. Množina $\{\bar\epsilon_a=\epsilon_a+R_Q|$ $a\in Q_0\}$ 
      je úplná množina ortogonálních idempotentů $KQ/R_Q$ a $KQ/R_Q$ je 
      izomorfní direktnímu součtu $K\oplus K\oplus\ldots\oplus K$ 
      kopií tělesa $K$.
    \end{lem}
    
    \begin{proof}
      Máme rozklad na direktní sumu \\\\
      \centerline{$KQ/R_Q=\bigoplus_{a,b\in Q_0}\bar\epsilon_a(KQ/R_Q)\bar\epsilon_b$,}\\\\
      který, protože $R_Q$ obsahuje všechny cesty délky alespoň jedna, můžeme 
      ještě zjednodušit na tvar\\\\
      \centerline{$KQ/R_Q=\bigoplus_{a\in Q_0}\bar\epsilon_a(KQ/R_Q)\bar\epsilon_a$.}\\\\
      $KQ/R_Q$ je pak generováno jako $K$-vektorový prostor třídami ekvivalence 
      cest délky 0, tedy množinou $\{\bar\epsilon_a=\epsilon_a+R_Q|$ $a\in 
      Q_0\}$. To je množina primitivních ortogonálních idempotentů algebry 
      $KQ/R_Q$. Navíc pro každé $a\in Q_0$ je algebra $\bar\epsilon_a(KQ/R_Q)\bar\epsilon_a$ 
      generovaná jedním prvkem $\bar\epsilon_a$ jako $K$-vektorový prostor a 
      tedy izomorfní jako $K$-algebra  tělěsu $K$. Tedy $KQ/R_Q$ je izomorfní 
      součtu $K\oplus K\oplus\ldots\oplus K$.
    \end{proof}
            
    \begin{lem}\label{lem-nilpotent-je-radikal}
      Pokud je $I$ oboustranným nilpotentním ideálem algebry $A$, pak $I\subseteq rad(A)$. 
      Pokud je navíc algebra $A/I$ izomorfní direktnímu součtu kopií tělesa $K$, pak $I=rad(A)$. 
    \end{lem} 
    
    \begin{proof}
      \cite{1} Corollary I.1.4.
    \end{proof}
    
    \begin{thm}\label{mega-veta-toulec}
      Nechť $Q$ je konečný toulec, $I$ přípustný ideál $KQ$ a $R_Q$ je šipkový ideál $KQ$. Pak platí:
      \begin{description}
        \item[(a)] Množina $\{e_a=\epsilon_a+I|a\in Q_0\}$ je úplná množina 
        primitivních ortogonálních idempotentů algebry $KQ/I$.
        \item[(b)] Algebra $KQ/I$ je souvislá, právě když Q je souvislý toulec.
        \item[(c)] Algebra $KQ/I$ je konečně dimenzionální. 
        \item[(d)] $I$ je konečně generovaný. 
        \item[(e)] Existuje konečná množina relací $\{\rho_1,\ldots,\rho_m\}$ 
        taková, že $I=<\rho_1,\ldots,\rho_m>$.
        \item[(f)] $rad(KQ/I)=R_Q/I$.
      \end{description}
    \end{thm}
    \begin{proof}
      \begin{description}
        \item 
        \item[(a)]
          Protože $e_a$ je obraz $\epsilon_a$ kanonickým homomorfismem $KQ\to 
          KQ/I$, plyne z \hyperref[dsl-quiver-idem]{Důsledku 
          \ref*{dsl-quiver-idem}}, že daná množina je úplnou množinou 
          ortogonálních idempotentů. Musíme ještě ověřit, že každé $e_a$ je 
          primitivní, neboli že jedinými idempotenty $e_a(KQ/I)e_a$ jsou 0 a 
          $e_a$. Každý idempotent $e\in e_a(KQ/I)e_a$ může být zapsán jako $e=\lambda 
          e_a+w+I$, kde $\lambda\in K$ a $w$ je lineární kombinace cyklů skrze $a$ 
          délky alespoň 1. Rovnost $e^2=e$ nám dává \\\\
          \centerline{$(\lambda^2-\lambda)e_a+(2\lambda-1)w+w^2\in I$.} \\\\
          Protože $I\subseteq R^2_Q$, musí být 
          $\lambda^2-\lambda=0$ a tedy $\lambda=0$ nebo $\lambda=1$. 
          Nechť $\lambda=0$, pak $e=w+I$, kde $w$ je idempotent modulo $I$. 
          Na druhou stranu protože $R^m_Q\subseteq I$ pro nějaké $m\geq 2$, musí 
          být $w^m\in I$ a $w$ je tedy také nilpotent modulo $I$. Pak $w\in I$ a $e=0$.
          Nechť tedy $\lambda=1$, pak $e_a-e=-w+I$ je také idempotent v $e_a(KQ/I)e_a$ 
          a $w$ je opět idempotent modulo $I$. Protože je opět nilpotentem 
          modulo $I$, musí náležet do $I$. Pak $e_a=e$.
          
        \item[(b)] Dokážeme jednu implikaci po druhé:
          \begin{description}
            \item[\Rightarrow] Nechť $KQ/I$ je souvislá. 
              Pokud $Q$ není souvislý toulec, pak $KQ$ není 
              souvislá algebra dle \hyperref[lem-q-kq-souvisle]{Lemma 
              \ref*{lem-q-kq-souvisle}}. Tedy $KQ$ obsahuje centrální idempotent 
              $\gamma$ různý od $0$ a $1$, ten můžeme dle důkazu \cite{1} Lemma II.1.6 zvolit jako 
              sumu cest nulové délky, tedy vrcholů. Pak ale $c=\gamma+I\neq I$. 
              Na druhou stranu $c=1+I$ implikuje $1-\gamma\in I$, což je také 
              nemožné, protože $I\subseteq R^2_Q$. Prvek $c$ je tedy centrálním 
              idempotentem $KQ/I$ a ta není souvislá.
            
            \item[\Leftarrow] Předpokládejme nyní pro spor, že $Q$ je souvislý, ale $KQ/I$ není. 
              Pak je algebra $KQ/I$ 
               direktním součtem dvou algeber, existuje tedy rozklad $Q_0$ na 
              disjunktní podmnožiny $Q_0'$ a $Q_0''$ takové, že pro všechna $a\in Q_0'$ 
              a $b\in Q_0''$ je \\\\
              \centerline{$\epsilon_a(KQ/I)\epsilon_b=\epsilon_b(KQ/I)\epsilon_a=0$.}\\\\  
              Protože $Q$ je souvislý, 
              můžeme zvolit taková $a\in Q_0'$  a $b\in Q_0''$, že budou spojena 
              šipkou $\alpha:a\to b$. 
              To je ale spor, protože $\alpha=\epsilon_a\alpha\epsilon_b$ a tedy pro $\bar\alpha=\alpha+I$ platí \\\\
              \centerline{$\bar\alpha=\epsilon_a\bar\alpha\epsilon_b\in 
              \epsilon_a(KQ/I)\epsilon_b=0$.}
          \end{description}
        
        \item[(c)] Protože $I$ je přípustný ideál, existuje $m\geq 2$ takové, že 
          $R_Q^m\subseteq I$, pak existuje surjektivní homomorfismus $K$-algeber 
          $KQ/R_Q^m\to KQ/I$. Tedy stačí dokázat, že $KQ/R_Q^m$ je konečně 
          dimenzionální. Pak třídy ekvivalence cest délky menší než $m$ tvoří 
          bázi $KQ/R_Q^m$ jako $K$-vektorového prostoru. Těch je ale konečně 
          mnoho, protože $Q$ je konečný.
          
        \item[(d)] Nechť $m\geq 2$ takové, že $R_Q^m\subseteq I$. Máme 
          následující krátkou exaktní posloupnost $KQ$-modulů: \\\\
          \centerline{$0\longrightarrow R_Q^m_Q \longrightarrow I \longrightarrow I/R_Q \longrightarrow 
          0$}\\\\
          Stačí tedy dokázat, že $R_Q^m$ a $I/R_Q^m$ jsou konečně generované jako 
          $KQ$-moduly. Z definice je $R_Q^m$ generovaný cestami délky $m$ a těch 
          je konečně mnoho. Na druhou stranu $I/R_Q^m$ je dle bodu (c) ideál konečně 
          dimenzionální algebry $KQ/R_Q^m$.  Tedy $I/R_Q^m$ je konečně dimenzionální 
          $K$-vektorový prostor a tedy konečně generovaný $KQ$-modul.
        
        \item[(e)] Dle\,\, bodu\,\, (d)\, je\, ideál\, $I$\, algebry\, $KQ$\, generován\, konečnou\, 
          množinou $\{\sigma_1,\sigma_2,\ldots,\sigma_t\}$. Prvky $\sigma_i$ nejsou obecně 
          relace, protože cesty obsažené v $\sigma_i$ nemusí mít stejný počátek 
          a konec. Na druhou stranu pro všechny $a,b\in Q_0$ je $\epsilon_a \sigma_i \epsilon_b$ 
          relace a $\sigma_i=\sum_{a,b\in Q_0}  \epsilon_a \sigma_i \epsilon_b$. 
          Pak $\{ \epsilon_a \sigma_i \epsilon_b|$ $a,b\in Q_0,$ $i=1,2,\ldots t\}$ 
          generuje $I$.
                  
        \item[(f)] Protože $I$ je přípustný ideál $KQ$ existuje $m\geq 2$ takové, že $R_Q^m\subseteq 
          I$. Tedy $(R_Q/I)^m=0$ a $R_Q/I$ je nilpotentní ideál $KQ/R_Q$. Na 
          druhou stranu dle \hyperref[lem-kq-jako-soucin]{Lemma 
          \ref*{lem-kq-jako-soucin}} je algebra \\\\
          \centerline{$(KQ/I)/(R_Q/I)\simeq KQ/R_Q$} \\\\
          izomorfní direktnímu součinu kopií tělesa $K$.
          Pak z  \hyperref[lem-nilpotent-je-radikal]{Lemma 
          \ref*{lem-nilpotent-je-radikal}} plyne, že $R_Q/I=rad(KQ/R_Q)$.
          
        
      \end{description}
    \end{proof}
    
    \begin{dsl}\label{mega-veta-toulec-dsl}
      Buď $Q$ konečný toulec, $R_Q$ šipkový ideál $KQ$ a $I$ přípustný ideál $KQ$. 
      Pak platí:
      \begin{description}
         \item[(a)]Pro každé $l\geq1$ je $rad^l(KQ/I)=(R_Q/I)^l$ a tedy platí, že  \\\\
           \centerline{$rad(KQ/I)/rad^2(KQ/I)=(R_Q/I)/(R_Q/I)^l\simeq R_Q/R_Q^2$.\\}
         
         \item[(b)]        
          Algebra omezeného quiveru $KQ/I$ je potom souvislá, konečně 
          dimenzionální s jednotkou, mající $R_Q/I$ jako radikál a $\{e_a:=\epsilon_a+I|$ $a\in Q_0\}$ 
          úplnou množinu primitivních ortogonálních idempotentů.
      \end{description}
    \end{dsl} 
    
    \begin{proof}
      Přímý důsledek \hyperref[mega-veta-toulec]{Věty \ref*{mega-veta-toulec}}, konkrétně bodů (a), (b) a (c).
    \end{proof}
    
    \begin{dfn}
      Nechť $A$ je základní, souvislá a konečně dimenzionální K-algebra a $\{e_1,e_2,\ldots,e_n\}$ 
      úplná množina primitivních ortogonálních idempotentů $A$. Definujme vlastní 
      toulec $Q_A$ algebry A následovně:
      \begin{description}
        \item[(a)] Body $Q_A$ budou čísla $1,2,\ldots,n$ která jsou v 
        bijektivní korespondenci s idempotenty $e_1,e_2,\ldots,e_n$.
        \item[(b)] Pro každé dva body $a,b\in (Q_A)_0$ budou šipky $\alpha:a\rightarrow b$ 
        v bijektivní korespondenci s vektory v bázi K-vektorového prostoru $e_a(radA/rad^2A)e_b$.
      \end{description}
    \end{dfn}
    
    \begin{pzn}      
      Protože $A$ je konečně dimenzionální, jsou konečně dimenzionální i 
      K-vektorové prostory $e_a(radA/rad^2A)e_b$ pro každé $a,b\in(Q_A)_0$ 
      a toulec $Q_A$ je tedy konečný.    
     \end{pzn}    
     
    \begin{dfn}
      Řekneme, že $K$-algebra $A$ je elementární, pokud je 
      $K$-algebra $A/rad(A)$ izomorfní direktnímu součtu $K\oplus K\oplus\ldots\oplus K$ 
      kopií tělesa $K$.
    \end{dfn}
    
    \begin{pzn}
      Poznamenejme, že v případě algebraicky uzavřeného tělesa je $K$-algebra 
      elementární, právě když je základní.
    \end{pzn}
      
    \begin{lem}\label{lem-velke-algebra-repre}
      Nechť $A$ je konečně dimenzionální, elementární, základní a souvislá $K$-algebra. Pak platí 
      následující:
      \begin{description}
        \item[(a)] Toulec $Q_A$ algebry A nezávisí na volbě úplné množiny
          ortogonálních idempotentů $A$.
        \item[(b)] Pro každý pár $e_a,e_b$ primitivních ortogonálních 
          idempotentů $A$ je K-lineární zobrazení 
          \begin{eqnarray}
            \xi: e_a(radA)e_b/e_a(rad^2A)e_b &\to& e_a(radA/rad^2A)e_b \nonumber \\
            e_axe_b+e_a(rad^2A)e_b &\mapsto& e_a(x+rad^2A)e_b \nonumber
          \end{eqnarray}
          izomorfismus.
        \item[(c)] Pro každou šipku $\alpha:i\rightarrow j$ v $(Q_A)_1$ buďte $x_\alpha\in e_i(radA)e_j$
         takové, že $\{x_\alpha+rad^2 A|\alpha:i\rightarrow j\}$ je báze $e_i(radA/rad^2A)e_j$ 
         (viz. bod (a)). Pak           
          \begin{description}
            \item[(i)] pro každé dva body $a,b\in(Q_A)_0$ můžeme každý prvek $x\in e_a(radA)e_b$ 
            napsat ve tvaru \\
            \centerline{$x=\sum x_{\alpha_1}x_{\alpha_2}\ldots 
            x_{\alpha_l}\lambda_{\alpha_1\alpha_2\ldots\alpha_l}$,}\\\\
            kde $\lambda_{\alpha_1\alpha_2\ldots\alpha_l}\in K$ a suma je 
            počítána přes všechny cesty $\alpha_1\alpha_2\ldots\alpha_l$ v $Q_A$ 
            z $a$ do $b$.
            \item[(ii)] pro každou šipku $\alpha:i\rightarrow j$ prvek $x_\alpha$ 
            určuje jednoznačně nenulový homomorfismus (který není izomorfismem) 
            $\tilde{x_\alpha}\in Hom_A(e_jA,e_iA)$ takový, že 
            $\tilde{x_\alpha}(e_j)=x_\alpha$, $Im\tilde{x_\alpha}\subseteq e_i(radA)$ 
            a $Im\tilde{x_\alpha}\not\subseteq e_i(rad^2A)$. 
          \end{description}
        \item[(d)] Toulec $Q_A$ algebry $A$ je souvislý. 
      \end{description}
    \end{lem}
    
    \begin{proof}
      \begin{description}
        \item
        
        \item[(a)] Počet vrcholů $Q_A$ je určen jednoznačně, protože je roven 
          počtu nerozložitelných direktních sčítanců $A_A$ a to je dáno jednoznačně dle \cite{1} I.4.10.
          Dle stejné věty jsou tyto direktní sčítance určeny jednoznačně až na 
          izomorfismus a jejich pořadí. Mějme tedy dva libovolné rozklady $A_A$ 
          \begin{eqnarray}
            A_A &=& \bigoplus_{a=1}^n e_aA = \bigoplus_{b=1}^n e'_bA \nonumber
          \end{eqnarray}
          kde $e_aA \simeq e'_aA$. Musíme ukázat, že pro každou dvojici $a,b$ 
          je \\\\
          \centerline{$dim_K e_a(rad(A)/rad^2(A))e_b=dim_K e'_a(rad(A)/rad^2(A))e'_b$.} 
          \\\\
          $K$-lineární zobrazení 
          \begin{eqnarray}
            e_a (rad(A)) &\to& e_a (rad(A)/rad^2(A)) \nonumber \\
            e_ax &\mapsto& e_a(x+rad^2A)  \nonumber
          \end{eqnarray}
          má jádro $e_a (rad^2(A))$. Pak \\\\
          \centerline{$e_a (rad(A)/rad^2(A)) \simeq e_a (rad(A))/e_a(rad^2(A)) \simeq (e_a rad(A))/(e_a rad^2(A))$\\\\\\}\\\\
          Máme tedy izomorfismus vektorových prostorů
          \begin{eqnarray}
            e_a (rad(A)/ rad^2(A)) e_b
            &=& [ rad(e_aA)/ rad^2(e_aA)]e_b \nonumber \\
            &=& Hom_A(e_bA,rad(e_aA)/rad^2(e_aA))  \nonumber \\
            &=& Hom_A(e'_bA,rad(e'_aA)/rad^2(e'_aA))  \nonumber \\
            &=& [ rad(e'_aA)/ rad^2(e'_aA)]e'_b  \nonumber \\
            &=& e'_a (rad(A)/ rad^2(A)) e'_b  \nonumber 
          \end{eqnarray}
        
        \item[(b)] Je zřejmé, že $K$-lineární zobrazení 
          \begin{eqnarray}
            e_a (rad(A)) e_b &\to&   e_a (rad(A)/rad^2(A)) e_b \nonumber \\
            e_axe_b &\mapsto&   e_a(x+rad^2A)e_b \nonumber
          \end{eqnarray}
          má jádro $e_a (rad^2(A)) e_b$. A tedy náš homomorfismus $\psi$ je 
          izmorfismem.
        
        \item[(c-i)] Protože jako $K$-vektorový prostor je \\\\
          \centerline{$rad(A)\simeq rad(A)/rad^2(A)\oplus rad^2(A)$,} \\\\
          máme i izomorfismus  \\\\
          \centerline{$e_a(rad(A))e_b\simeq e_a(rad(A)/rad^2(A))e_b\oplus e_a(rad^2(A))e_b$.} \\\\
          A tedy $x$ může být zapsáno jako\\\\
          \centerline{$
            x = \sum_{\alpha:a\to b}x_\alpha \lambda_\alpha $ modulo $  e_a(rad^2(A))e_b
          $,}\\\\  
          kde $\lambda_a\in K$ pro $a\in Q_1$. Nebo více formálně jako \\\\
          \centerline{$x'=x-\sum_{\alpha:a\to b}x_\alpha \lambda_\alpha\in e_a(rad^2(A))e_b$.} \\\\      
          Rozklad $rad(A)=\bigoplus_{i,j}e_i(rad(A))e_j$ nám implikuje, že \\\\
          \centerline{$e_a(rad^2(A))e_b=\sum_{c\in(Q_A)_0}[e_a(rad(A))e_c][e_c(rad(A))e_b]$.} 
          \\\\
          Tedy $x'=\sum_{c\in(Q_A)_0}x'_cy'_c$, kde $x'_c\in e_a(rad(A))e_c$ a $y'_c\in 
          e_c(rad(A))e_b$. Z předchozí diskuze vyplývá, že 
          $x'_c=\sum_{\beta:a\to c}x_\beta\lambda_\beta$ a
          $y'_c=\sum_{\gamma:c\to b}x_\gamma\lambda_\gamma$
          modulo $rad^2(A)$, kde $\lambda_\beta,\lambda_\gamma\in K$. Tedy \\\\
          \centerline{$
            \sum_{\alpha:a\to b}x_\alpha \lambda_\alpha + 
            \sum_{\beta:a\to c}
            \sum_{\gamma:c\to b}
            x_\beta x_\gamma \lambda_\beta \lambda_\gamma
            $ modulo $  e_a(rad^3(A))e_b 
          $.}\\\\
          Dále pokračujeme indukcí do $n$ takového, že  $rad^n(A)=0$. To 
          existuje z nilpotentnosti $rad(A)$.
                    
        \item[(c-ii)] Dle předpokladu je prvek $x_\alpha\in e_i(rad(A))e_j$ 
          nenulový a zobrazuje se na nenulový prvek $\tilde x_\alpha$ 
          $K$-lineárním izomorfismem \\\\
          \centerline{$e_i(rad(A))e_j\simeq Hom_A(e_j A,e_i(rad(A)))$.} \\\\
          Z toho plyne, že $\tilde x_\alpha(e_j)=x_\alpha$, $Im \tilde x_\alpha\subseteq e_i(rad(A))$ 
          a $Im \tilde x_\alpha\not\subseteq e_i(rad^2(A))$.
        
        \item[(d)] Nechť $Q_A$ není souvislý. Pak lze množinu $(Q_A)_0$ rozdělit na 
          dvě disjunktní neprázdné množiny $Q_0'$ a $Q_0''$ takové, že mezi nimi 
          nejsou žádné šipky. Pokud $i\in Q_0'$ a $j\in Q_0''$, pak 
          $e_iAe_j=e_jAe_i=0$, pak dle \cite{1} Lemma II.1.6 $A$ 
          není souvislá, což je spor.
        
      \end{description}
    \end{proof}
    
    \begin{lem}
      Nechť $Q$ je konečný a souvislý toulec, $I$ přípustný ideál $KQ$ a $A=KQ/I$. 
      Pak $Q_A=Q$.
    \end{lem}
    
    \begin{proof}
      Dle \hyperref[mega-veta-toulec]{Věty \ref*{mega-veta-toulec}} (a) 
      je $\{e_a=\epsilon_a|$ $a\in Q_0\}$ úplná množina ortogonálních 
      idempotentů $A=KQ/I$. Tedy vrcholy v $Q_A$ jsou v bijektivní korespondenci 
      s vrcholy v $Q$. 
      A dle \hyperref[mega-veta-toulec-dsl]{Důsledku \ref*{mega-veta-toulec-dsl}} 
      jsou šipky z $a$ do $b$ v $Q$ v bijektivní korespondenci s vektory v bázi 
      $K$-vektorového prostoru $e_a(rad(A)/rad^2(A))e_b$  a tedy s šipkami z $a$ do $b$ v $Q_A$.
    \end{proof}
    
    \begin{thm}
      Nechť $A$ je základní, souvislá, elementární a konečně dimenzionální K-algebra. Pak 
      existuje přípustný ideál $I$ algebry $KQ_A$ takový, že $A\simeq KQ_A/I$.
    \end{thm}
    \begin{proof}
      Pro každou šipku $\alpha: i\to j$ v $(Q_A)_1$ buď $x_\alpha\in rad (A)$ takové, že 
      $\{x_\alpha+rad^2(A)|\alpha:i\to j\}$ tvoří bázi $e_i(rad(A)/rad^2(A))e_j$. 
      Definujme dvě zobrazení:
      \begin{eqnarray}
         \varphi_0:(Q_A)_0 &\to& A \nonumber \\
         a &\mapsto& e_a \nonumber \\ \nonumber \\
         \varphi_1:(Q_A)_1 &\to& A \nonumber \\
         \alpha &\mapsto& x_\alpha \nonumber
      \end{eqnarray}
      Pak prvky $\varphi_0(a)$ tvoří úplnou množinu primitivních ortogonálních 
      idempotentů $A$ a pokud $\alpha:a\to b$, pak máme 
      \\\\\centerline{$
        \varphi_0(a)\varphi_1(\alpha)\varphi_0(b)=e_ax_\alpha e_b=x_\alpha=\varphi_1(\alpha)
      $.}\\
      
      Dle \cite{1} Proposition II.1.8 lze $\varphi_0$ a $\varphi_1$ jednoznačně rozšířit na 
      homomorfismus  $K$-algeber 
      \\\\\centerline{$\varphi: KQ_A\to A$.}\\\\ 
      Tato vlastnost se nazývá 
      univerzální vlastnost $K$-algeber. 
      
      Ukážeme, že $\varphi$ je epimorfismus. Jeho obraz je generovaný prvky $e_a$ 
      $(a\in (Q_A)_0)$ a $x_\alpha$ $(\alpha\in (Q_A)_1)$ a tedy stačí ukázat, 
      že tyto prvky generují celé $A$. Protože $K$ je algebraicky uzavřené, plyne 
      z Wedderburn-Malcevovi věty (\cite{1} I.1.6), že je kanonický homomorfismus 
      $A \to A/rad (A)$  štěpitelný a $A$ je štěpitelné rozšíření polojednoduché 
      algebry $A/rad( A)$ podle $rad (A)$. Protože $A/rad( A)$ je generované 
      prvky $e_a$, zbývá dokázat, že každý prvek $rad(A)$ může být zapsán jako 
      polynom v $x_\alpha$. A to plyne z 
      \hyperref[lem-velke-algebra-repre]{Lemma \ref*{lem-velke-algebra-repre} (c)}.
      
      Zbývá ukázat, že $I=Ker\varphi$ je přípustný. Nechť $R$ značí šipkový 
      ideál algebry $KQ_A$. Z definice $\varphi$ máme $\varphi(R)\subseteq 
      rad(A)$ a tedy $\varphi(R^l)\subseteq rad^l(A)$ pro $l\geq 1$. Protože $rad(A)$ 
      je nilpotentní, existuje $m\geq 1$ takový, že $rad^m(A)=0$ a tedy $R^m\subseteq 
      Ker\varphi=I$. Nyní dokážeme, že $I\subseteq R^2$. Buď $x\in I$, pak 
     můžeme $x$ zapsat jako
     \\\\\centerline{$
       x
       =\sum_{a\in (Q_A)_0}\epsilon_a\lambda_a
       +\sum_{\alpha\in (Q_A)_1}\alpha\mu_\alpha+y
     $,}\\\\
     kde $\lambda_a,\mu_\alpha\in K$ a $y\in R^2$. Pokud $\vaprhi(x)=0$, pak 
     \\\\\centerline{$
       0
       =\sum_{a\in (Q_A)_0}e_a\lambda_a
       +\sum_{\alpha\in (Q_A)_1}x_\alpha\mu_\alpha+\varphi(y)
     $.}\\\\
     Pak ale máme
     \\\\\centerline{$
       \sum_{a\in (Q_A)_0}e_a\lambda_a
       =-\sum_{\alpha\in (Q_A)_1}x_\alpha\mu_\alpha-\varphi(y)
       \in rad(A)
     $.}\\\\
     Protože $Rad(A)$ je nilpotentní a $e_a$ jsou ortogonální idempotenty, musí být $\lambda_a=0$ 
     pro každé $a\in (Q_A)_0$. Podobně máme 
     \\\\\centerline{$
       \sum_{\alpha\in (Q_A)_1}x_\alpha\mu_\alpha
       =-\varphi(y)
       \in rad^2(A)
     $}\\\\
     a tedy $\sum_{\alpha\in (Q_A)_1}(x_\alpha+rad^2(A))\mu_\alpha=0$ platí v 
     $rad(A)/rad^2(A)$. Ale množina $\{x_\alpha+rad^2(A)|\alpha\in (Q_A)_1\}$ je 
     z  báze $rad(A)/rad^2(A)$ a tedy $\mu_\alpha=0$ pro každé 
     $\alpha\in (Q_A)_1$ a tím pádem $x=y\in R^2$.
    \end{proof}

  \subsection{Reprezentace a moduly}
  
    \begin{dfn}
      Nechť $Q$ je toulec. K-lineární reprezentaci M toulce Q definujeme 
      následovně:
      \begin{description}
        \item[(a)] Každému bodu $a\in Q_0$ přiřadíme K-vektorový prostor $M_a$. 
        \item[(b)] Každé šipce $\alpha: a \rightarrow b \in Q_1$ přiřadíme 
        K-lineární zobrazení $\varphi_\alpha:M_a\rightarrow M_b$. 
      \end{description}
      Takovou reprezentaci budeme značit $M=(M_a,\varphi_\alpha)_{a\in Q_0,\alpha\in 
      Q_1}$, nebo jednoduše $M=(M_a,\varphi_\alpha)$. Nazveme ji konečnou, pokud 
      každý vektorový prostor $M_a$  je konečně dimenzionální.
      
      Nechť $M=(M_a,\varphi_\alpha)$ a $M'=(M_a',\varphi'_\alpha)$ jsou dvě 
      reprezentace toulce Q. Morfismus reprezentací $f:M\rightarrow M'$ je 
      množina $f=(f_a)_{a\in Q_0}$ K-lineárních zobrazení $f_a:M_a\rightarrow M_a'$ 
      takových, že jsou kompatibilními se zobrazeními $\varphi_a$ a 
      $\varphi'_a$, neboli pro každou šipku $\alpha: a\rightarrow b$ máme 
      $\varphi'_\alpha f_a=f_b\varphi_\beta$ a tedy následující diagram komutuje: \\\\
      \centerline{\xymatrix{
        M_a  
            \ar@{->}[r]^{\varphi_a}  
            \ar@{->}[d]^{f_a}
          & M_b 
             \ar@{->}[d]^{f_b} \\
        M'_a \ar@{->}[r]^{\varphi'_a} 
          & M'_b
      }}\\\\
      
      Nechť jsou $f:M\rightarrow M'$ a  $g:M'\rightarrow M''$ dva morfismy 
      reprezentací Q, kde $f=(f_a)_{a\in Q_0}$ a $g=(g_a)_{a\in Q_1}$. Jejich 
      složení definujme jako množinu $gf=(g_af_a)_{a\in Q_1}$. Složením dvou morfismů 
      reprezentací vznikne opět morfismus reprezentací. 
      
      K-lineární reprezentace toulce Q nám tedy spolu s jejich morfismy a skládáním morfismů 
      tvoří kategorii $Re_K(Q)$. Jako $rep_k(Q)$ označíme její úplnou podkategorii 
      sestávající z konečně dimenzionálních reprezentací.
    \end{dfn}
    
    \begin{lem}
      Nechť $Q$ je konečný toulec, pak jsou $Rep_K(Q)$ a $rep_K(Q)$ abelovské 
      K-kategorie.
    \end{lem}
    \begin{proof}
      \cite{1} Lemma III.1.3.
    \end{proof}
    
    \begin{dfn}
      Nechť Q je konečný toulec a $M=(M_a,\varphi_\alpha)$ reprezentace $Q$.  Pro 
      každou netriviální cestu $\omega=\alpha_1\alpha_2\ldots\alpha_l$ z $a$ do $b$ 
      definujme vyhodnocení M na cestě $\omega$ jako K-lineární zobrazení z $M_a$ 
      do $M_b$ definované: \\\\
      \centerline{$\varphi_\omega:\varphi_{\alpha_l}\varphi_{\alpha_l-1}\ldots\varphi_{\alpha_1}$.} 
      \\\\
      Definici vyhodnocení dále rozšíříme na K-lineární kombinace cest se 
      stejným počátkem a koncem. Nechť \\
      \centerline{$\rho=\sum_{i=1}^m\lambda_i\omega_i$} 
      \\\\
      je taková kombinace, kde pro každé $i$ je $\lambda_i\in K$ a $\omega_i$ je cesta v 
      Q, pak \\\\
      \centerline{$\varphi_\rho=\sum_{i=1}^m\lambda_i\rho_{\omega_i}$.}\\
    \end{dfn}
    
    \begin{dfn}
      Nechť  $Q$ je konečný toulec a $I$ přípustný ideál $KQ$. Řekneme, že reprezentace 
      $M=(M_a,\varphi_\alpha)$ toulce Q je omezená ideálem $I$, nebo že splňuje všechny relace 
      z $I$, pokud \\
      \centerline{$\varphi_\rho=0$, pro každou relaci $\rho\in I$}\\\\ 
      Úplnou podkategorii $Rep_K(Q)$ (resp. $rep_K(Q)$) sestávající z 
      reprezentací Q omezených ideálem I budeme značit $Rep_K(Q,I)$ (resp. 
      $rep_K(Q,I)$).    
    \end{dfn}
    
    \begin{pr}
      Nechť Q je toulec \\
      \centerline{\xymatrix{
        & & \circ^3 \ar@{->}[ld]_\beta \\
        \circ^1 
          & \circ^2 \ar@{->}[l]_\lambda
          & 
          & \circ^5 \ar@{->}[lu]_\alpha \ar@{->}[ld]^\gamma \\
        & & \circ^4 \ar@{->}[lu]^\delta
      }}\\\\\\ omezený relací komutativity $\alpha\beta=\gamma\delta$. Pak obě 
      následující reprezentace jsou touto relací také omezeny: \\\\
      \centerline{\xymatrix{
        & & K \ar@{->}[ld]_{[^1_0]} \\
        K
          & K^2 \ar@{->}[l]_{[1,1]}
          & 
          & 0 \ar@{->}[lu] \ar@{->}[ld] \\
        & & K \ar@{->}[lu]^{[^0_1]}
        } 
        \rightaligned{
        \space\space\space\space\space\space\space\space\space\space
        \xymatrix{
        & & K \ar@{->}[ld]_1 \\
        K
          & K \ar@{->}[l]_1
          & 
          & K \ar@{->}[lu]_1 \ar@{->}[ld]^1 \\
        & & K \ar@{->}[lu]^1
      }}}\\\\
      Naopak následující reprezentace touto relací omezená není: \\\\
      \centerline{\xymatrix{
        & & 0 \ar@{->}[ld] \\
        K
          & K \ar@{->}[l]_1
          & 
          & K \ar@{->}[lu] \ar@{->}[ld]^1 \\
        & & K \ar@{->}[lu]^1
      }}
      
    \end{pr}
    
    \begin{thm}\label{ekvivalence-rep-a-mod}
      Nechť $A=KQ/I$, kde $Q$ je konečný souvislý toulec a I přípustný ideál KQ. 
      Pak existuje K-lineární ekvivalence kategorií \\\\
      \centerline{\xymatrix{
        F:ModA 
          \ar@{->}[r]^{\simeq} 
        & Rep_K(Q,I)
      }}\\\\
      a její restrikce na konečné moduly a reprezentace \\\\
      \centerline{\xymatrix{
        F:modA 
          \ar@{->}[r]^{\simeq} 
        & rep_K(Q,I)
      }.}
    \end{thm}
    
    \begin{proof}
      \begin{description}
        \item 
        
        \item[(1)] Nejprve zkonstruujeme funktor $F:mod(A)\to rep_K(Q,I)$. Nechť 
          $M_A$ je $A$-modul. Definujme $K$-lineární reprezentaci
           $F(M)=(M_a, \varphi_\alpha)_{a\in Q_0, \alpha\in Q_1}$
            toulce $Q$ 
          následujícím způsobem: 
          
            \begin{description}
              \item[(i)] Pro $a\in Q_0$ buď $M_a:=Me_a$, kde $e_a=\epsilon_a+I$ 
              je idempotent $A=KQ/I$.
              
              \item[(ii)] Pro $\varphi:a\to b$ v $Q_1$ definujme 
                \begin{eqnarray}
                  \varphi_\alpha: M_a &\to& M_b \nonumber \\
                  x &\mapsto& x\overline \alpha(=xe_a\overline\alpha e_b), \nonumber
                \end{eqnarray}
              kde $\overline\alpha=\alpha+I$ je třída reprezentantů $\alpha$ 
              modulo $I$.
            \end{description}
            Protože $M$ je $A$-modul, tak $\varphi_\alpha$ je $K$-lineární 
            zobrazení. Pak $F(M)$ je omezené ideálem $I$, protože pro relaci 
            $\rho=\sum_{i=1}^m \lambda_iw_i$ z $a$ do $b$ v $I$, kde 
            $w_i=\alpha_{i,1}\alpha_{i,2}\ldots\alpha_{i,l_i}$, máme 
            \begin{eqnarray}
              \varphi_\rho(x) &=& \sum_{i=1}^m \lambda_i \varphi_{w_i}(x) \nonumber \\
              &=& \sum_{i=1}^m \lambda_i \varphi_{\alpha_{i,l_i}}\ldots \varphi_{\alpha_{i,1}}(x)  \nonumber \\
              &=& \sum_{i=1}^m \lambda_i(x \overline \alpha_{i,1} \ldots \overline \alpha_{i,l_i}) \nonumber \\
              &=&  x\cdot \sum_{i=1}^m \lambda_i(\overline \alpha_{i,1} \ldots \overline \alpha_{i,l_i}) \nonumber \\
              &=& x\cdot \overline \rho = x0 = 0. \nonumber
            \end{eqnarray}
            Tím je definován náš funktor na objektech. Nechť $f:M_A\to M'_A$ je 
            homomorfismus $A$-modulů. Chceme definovat morfismus $F(f):F(M_A)\to 
            F(M'_A)$ v $Rep_K(Q,I)$. Pro $a\in Q_0$ a $x=xe_a\in Me_a=M_a$ máme 
            \\\\\centerline{$f(xe_a)=f(xe_a^2)=f(xe_a)e_a\in M'e_a=M'_a$.}\\\\
            Tedy restrikce $f_a$ homomorfismu $f$ na $M_a$ je $K$-lineární zobrazení $f_a:M_a\to 
            M'_a$.  Položme $F(f):(f_a)_{a\in Q_0}$. Zbývá ověřit, že pro každou šipku $\alpha:a\to b$ 
            máme $\varphi'_\alpha f_a=f_b\varphi_\alpha$, z čehož plyne, že $F(f)$ 
            je morfismem reprezentací. Nechť $x\in M_a$, pak
            \\\\\centerline{$f_b\varphi_\alpha(x)=f_b(x\overline \alpha)=f(x)\overline\alpha=f_a(x)\overline\alpha=\varphi'_\alpha f_a(x)$.}\\\\
            Ověření toho, že $F$ je $K$-lineární funktor $Mod(A)\to Rep_K(Q,I)$ a
            že se restriktuje na $K$-lineární funktor $mod(A)\to rep_K(Q,I)$ je již
             přímořaré a přenecháme ho 
            čtenáři.
            
        \item[(2)]
        Nyní zkonstruujeme $K$-lineární funktor  $G:rep_K(Q,I)\to mod(A)$. Nechť 
        $(M_a,\varphi_\alpha)$ je objekt $Rep_K(Q,I)$. Položme 
        \\\\\centerline{$G(M):=\bigoplus_{a\in Q_0}M_a$}\\\\
        a definujme strukturu $K$-vektorového prostoru následujícím způsobem. 
        Protože $A=KQ/I$, můžeme nejprve na $G(M)$ definovat $KQ$-modulovou 
        strukturu a poté ukázat, že je anihilovaný ideálem $I$. Nechť tedy $(x_a)_{a\in Q_0}$ 
        je z $G(M)$. Stačí nám definovat součin tvaru $xw$, kde $w$ je libovolná 
        cesta v $Q$. Pro $w=\epsilon_a$ stacionární cestu v $a$ položíme 
        \\\\\centerline{$xw=x\epsilon_a=x_a$.}\\\\
        Nechť $w=\alpha_1\alpha_2\ldots\alpha_l$ je netriviální cesta z $a$ do $b$ 
        a uvažujme $K$-lineární zobrazení $\varphi_w=\varphi_{\alpha_l}\ldots\varphi_{\alpha_1}:M_a\to 
        M_b$. Položme
        \\\\\centerline{$(xw)_c=\delta_{bc}\varphi_w(x_a)$,}\\\\
        kde $\delta$ značí Kroneckerovu deltu. Jednoduše řečeno, $xw$ je prvkem $G(M)=\bigoplus_{a\in 
        Q_0}M_a$, jehož jedinou nenulovou složkou je $(xw)_b=\varphi_w(x_a)\in M_b$. Tím jsme ukázali, 
        že $G(M)$ je $KQ$-modul. Navíc plyne z definice $G(M)$, že pro každou $\rho\in I$ 
        a $x\in G(M)$ je $x\rho=0$. Tedy $G(M)$ je i $KQ/I$-modulem při 
        ztotožnění 
        \\\\\centerline{$x(v+I)=xv$,}\\\\
        pro $x\in G(M)$ a $v\in KQ$. Tím je dán náš 
        funktor na objektech. Nechť $(f_a)_{a\in Q_0}$ je morfismus z $M=(M_a,\varphi_\alpha)$ 
        do $M'=(M'_a,\varphi'_\alpha)$ v $Rep_K(Q,I)$. Chceme zkonstruovat 
        morfismus $A$-modulů $f:G(M)\to G'(M)$. Protože $G(M)=\bigoplus_{a\in 
        Q_0}M_a$ a $G(M)=\bigoplus_{a\in  Q_0}M'_a$ jako $K$-vektorové prostory, 
        existuje $K$-lineární zobrazení
        \\\\\centerline{$f=\bigoplus_{a\in 
        Q_0}f_a:G(M)\to G(M')$.}\\\\
        Ukážeme, že $f$ je homomorfismus $A$-modulů, neboli, že pro každé $x\in G(M) $ 
        $w\in KQ/I$ platí $f(xw)=f(x)w$. Postačí nám to dokázat pro $x=x_a\in M_a$ 
        a $w=w+I$, kde $w$ je cesta z $a$ do $b$ v $Q$. Pak
        \begin{eqnarray}
            f(xw) &=& f(x_aw) \nonumber \\
            &=& f_b\varphi_w(x_a) \nonumber \\
            &=& \varphi'_wf_a(x_a) \nonumber \\
            &=& f_a(x_a)w \nonumber \\
            &=& f(x)w.\nonumber 
        \end{eqnarray}
        Nyní je již přímočaré dokázat, že $G$ je $K$-lineární funktor, který se 
        restriktuje na $K$-lineární funktor $mod A\to rep_K(Q,I)$. A je 
        jednoduché ověřit, že $FG\simeq 1_{Rep_K(Q,I)}$ a $GF\simeq 1_{Mod(A)}$.

        \item[(3)]        
        Druhá část tvrzení plyne z toho, že protože $Q$ je konečný, tak pro 
        $K$-lineární reprezentaci $M=(M_a, \varphi_\alpha)$ omezeného toulce 
        $(Q,I)$, je
        \\\\\centerline{$dim_K(\bigoplus_{a\in Q_0}M_a)<\infty$}\\\\
        právě tehdy, když $dim_KM_a<\infty$ pro každé $a\in Q_0$.
        
      \end{description}
    \end{proof}
    
    \begin{dsl}
      Nechť $Q$ je konečný, souvislý a acyklický toulec. Pak existuje ekvivalence 
      kategorií $ModKQ\simeq Rep_K(Q)$ a její restrikce $modKQ\simeq rep_K(Q)$. 
    \end{dsl}
    \begin{proof}
      Jelikož je $Q$ konečný, tak algebra $KQ$ je konečně dimenzionální dle 
      \hyperref[quiver-kq-lemma]{Lemma \ref*{quiver-kq-lemma}}. 
      Tvrzení plyne z \hyperref[ekvivalence-rep-a-mod]{Věty \ref*{ekvivalence-rep-a-mod}}, 
      položíme-li $I=0$.
    \end{proof}
    
    \begin{pzn}
      Soubor $\{e_a|a\in Q_0\}$ je úplná množina primitivních ortogonálních 
      idempotentů algebry $A$. Rozklad $A_A=\bigoplus_{a\in Q_0}e_a A$ je 
      rozkladem $A_A$ na direktní součet po dvou neizomorfních nerozložitelných 
      projektivních $A-$modulů. 
      
      V následující větě popíšeme projektivní moduly $P(a)=e_a A$ jako reprezentace 
      toulce Q.
    \end{pzn}
    
    \begin{thm}\label{lem-proj-prezentace}
      Nechť $(Q,I)$ je omezený toulec, $A=KQ/I$ a $P(a)=\epsilon_a A$, kde $a\in 
      Q_0$. Pokud $P(a)=(P(a)_b,\varphi_\beta)$ jako reprezentace $Q$, pak $P(a)_b$ je K-vektorový 
        prostor s množinou generátorů \{$\bar{\omega}=\omega+I|$ $\omega$ je 
        cesta z $a$ do $b$\} a pro šipku $\beta:b\rightarrow c$ je K-lineární 
        zobrazení $\varphi_\beta:P(a)_b\rightarrow P(a)_c$ definované jako 
        násobení zprava $\bar{\beta}=\beta+I$.
    \end{thm}
    
    \begin{proof}
      Z definice funktoru ekvivalence ve
      \hyperref[ekvivalence-rep-a-mod]{Větě \ref*{ekvivalence-rep-a-mod}} plyne, 
      že pro korespondující reprezentaci A-modulu $P(a)_A=e_aA$ máme pro každé $b\in 
      Q_0$:\\\\
      \centerline{$
        P(a)_b=P(a)e_b=e_aAe_b=e_a(KQ/I)e_b=(\epsilon_a(KQ)\epsilon_b)/(\epsilon_a I \epsilon_b)
      $.}\\\\
      Navíc je-li $(\beta:b\rightarrow c)\in Q_1$, pak $\varphi_\beta:e_aAe_b\rightarrow e_aAe_c$ 
      je dáno násobením zprava $\bar{\beta}=\beta+I$, neboli máme-li $\bar{\omega}=\omega+I$ 
      třídu cest z a do b, pak 
      $\varphi_\beta(\bar{\omega})=\bar{\omega}\bar{\beta}$.
    \end{proof}
    
    \begin{dfn}
      Definujme pro každé $a\in Q_0$ reprezentaci $S(a)$ toulce $Q$ následovně:
      \begin{description}
        \item[(a)] Položme vektorový prostor $S(a)_a=K$ a všechny ostatní vektorové 
        prostory $S(a)_b$, kde $b\neq a$, položme rovné nule.
        \item[(b)] Všem šipkám z $Q_1$ přiřaďme nulová zobrazení.
      \end{description} 
    \end{dfn}
    
    \begin{pzn}
      Poznamenejme, že pro každé $a\in Q_0$ je $S(a)$ jednoduchý modul.
    \end{pzn}
  
  \chapter{Algoritmus}\label{kapitola-algoritmus}

  \praragraph{}\paragraph{ }V této kapitole popíšeme algoritmus pro nalezení generátoru skoro štěpitelných 
   posloupností končících v nerozložitelném a neprojektivním modulu $X$. Konstrukce algoritmu je 
   vychází z diplomové práce \cite{3} Tea Sormbroen Lian: Computing almost split sequences. 
   Co se týče teorie, tak navážeme na \hyperref[algebry-moduly]{části \ref*{algebry-moduly}}  a $A$-modulem budeme rozumět levý modul.
    
   V první části zkonstruujeme čtyři potřebné izomorfismy pro obecný komutativní,
   artinovský, lokální okruh $R$.    
   V druhé části konečně popíšeme algoritmus pro jednodušší případ, kdy $R=K$ bude těleso.
    \clearpage
      
  \section{Konstrukce potřebných izomorfismů$}
    \praragraph{ }\paragraph{ } V celé této části budeme pracovat nad pevně zvoleným komutativním, artinovským, 
    lokálním okruhem 
    $R$ a artinovskou $R$-algebrou $A$. 
    Dále nechť $X$ je pevně zvolený, neprojektivní, nerozložitelný $A$-modul
    a zvolme pevně jeho projektivní prezentaci\\
      \centerline{\xymatrix{
        P_1 \ar@{->}[r]^s 
          & P_0 \ar@{->}[r]^t 
          & X \ar@{->}[r] 
          & 0
      }}\\\\
      a jako $\delta$ označme libovolnou krátkou exaktní posloupnost A-modulů \\\\
      \centerline{\xymatrix{
        0 \ar@{->}[r] 
          & M \ar@{->}[r]^f
          & N \ar@{->}[r]^g 
          & L \ar@{->}[r] 
          & 0
      }.}\\\\
  
    \subsection{Izomorfismus $\varphi_{P,Y}$}
      Naším cílem je zkonstruovat $R$-izomorfismus \\\\ 
      \centerline{$\varphi_{P,Y}:Hom_A(P,Y)\rightarrow Hom_A(P,A)\otimes_A Y$,} 
      \\\\
      kde $P,Y\in mod(A)$ a $P$ je projektivní.
            
      \begin{lem}\label{lemma-alpha-P-Y}
        Nechť $Y\in mod(A)$ a $P\in P(A)$. Definujme zobrazení\\\\
        \centerline{$\alpha_{P,Y}:Hom_A(P,A)\otimes_A Y\rightarrow Hom_A(P,Y)$} 
        \\\\
        předpisem \\
        \centerline{$\alpha_{P,Y}(f\otimes y):=[p\mapsto f(p)y]$.}\\\\
        Zobrazení $\alpha_{P,Y}$ je homomorfismem R-modulů přirozeným ve 
        složkách $P$ i $Y$.
      \end{lem}
      \begin{proof}
        Vidíme, že $[p\mapsto f(p)y]\in Hom_A(P,Y)$ pro libovolný $f\in Hom_A(P,Y)$ 
        a $y\in Y$, protože \\\\
        \centerline{$f(p+\lambda p')y=f(p)y+\labda f(p')y$} \\\\
        pro každé $p,p'\in P$ a $\lambda\in A$. Nechť navíc $r\in R$, $f\in Hom_A(P,A)$ 
        a $y\in Y$. Pak
        \begin{eqnarray}
          r\cdot \alpha_{P,Y}(f\otimes y)
           &=& r[p\mapsto f(p)y]  \nonumber \\
           &=& [p\mapsto r(f(p)y)]  \nonumber \\
           &=& [p\mapsto (rf(p))y]  \nonumber \\
           &=& [p\mapsto (rf)(p)y]  \nonumber \\
           &=& \alpha_{P,Y}(rf\otimes y)  \nonumber \\
           &=& \alpha_{P,Y}(r\cdot f\otimes y)  \nonumber 
        \end{eqnarray}
        a $\alpha_{P,Y}$ je homomorfismem $R$-modulů.
        
        Dokážeme nyní, že $\alpha_{P,Y}$ je přirozené v $P$. Mějme libovolné $P,P'\in P(A)$ 
        a $h\in Hom_A(P,P')$. Musíme dokázat, že následující diagram komutuje:\\\\      
         \centerline{\xymatrix{            
           Hom_A(P', A)\otimes_A Y \ar[rr]^{\alpha_{P',Y}} \ar[d]_{(-\circ h)_A\otimes 1_Y}
           & & Hom_A(P', Y) \ar[d]^{(-\circ h)_Y} \\
           Hom_A(P, A)\otimes_A Y \ar[rr]^{\alpha_{P,Y}}
           & & Hom_A(P, Y)
         }}\\\\\\
        Nechť $f \otimes y \in Hom_A(P', A)\otimes_A Y$, pak \\\\
        \centerline{$
          (-\circ h)_A\otimes \alpha_{P',Y}(f\otimes y)
          = (-\circ h)_A\otimes ([p' \mapsto f(p')y])
          = [p\mapsto f(h(p))y]
        $}\\
        a\\
        \centerline{$
          \alpha_{P,Y}\circ[(-\circ h)_A\otimes 1_Y](f\otimes y)
          = \alpha_{P,Y}(fh\otimes y)
          = [p\mapsto f(h(p))y]
        $.}\\\\
        Diagram tedy komutuje. 
        
        Dále dokážeme přirozenost $\alpha_{P,Y}$ v $Y$. Nechť tedy $Y,Y'\in 
        mod(A)$ a $g\in Hom_A(Y,Y')$. Musíme dokázat, že následující diagram komutuje:\\\\      
         \centerline{\xymatrix{            
           Hom_A(P, A)\otimes_A Y \ar[rr]^{\alpha_{P,Y}} \ar[d]_{1_{Hom_A(P,A)}\otimes g}
           & & Hom_A(P, Y) \ar[d]^{(g\circ -)_P} \\
           Hom_A(P, A)\otimes_A Y' \ar[rr]^{\alpha_{P,Y'}}
           & & Hom_A(P, Y')
         }}\\\\\\
         Nechť $f\otimes y \in Hom_A(P, A)\otimes_A Y$. Pak \\\\
         \centerline{$
           (g\circ-)_P\circ\alpha_{P,Y}(f\otimes y)
           = (g\circ-)_P([p\mapsto f(p)\cdot y])
           =[p\mapsto g(f(p)\cdot y)]
         $} \\
         a \\
         \centerline{$
           \alpha_{P,Y'}\circ[1_{Hom_A(P,A)}\otimes g](f\otimes y)
           = \alpha_{P,Y'}(f\otimes g(y))
           = [p\mapsto f(p)\cdot g(y)]
         $.}\\\\
         Protože $f(p)\in A$ a pro každé $p\in P$ je $g$ $A$-modulovým 
         homomorfismem, tak máme rovnost \\\\
         \centerline{$g(f(p)\cdot y)=f(p)\cdot g(y)$.}\\\\         
         Diagram tedy komutuje i tímto směrem a náš homomorfismus je přirozený v obou proměných $P$ 
         i $Y$.         
      \end{proof}
      
      \begin{lem}\label{lem-varphi-a-y-izo}
        Nechť $Y\in mod(A)$. Definujme zobrazení\\\\
        \centerline{$\varphi_{A,Y}: Hom_A(A,Y) \rightarrow Hom_A(A,A)\otimes_A Y$} 
        \\\\
        předpisem \\
        \centerline{$\varphi_{A,Y}(g):=1_A\otimes g(1_A)$.}\\\\
        Zobrazení $\varphi_{A,Y}$ je homomorfismem R-modulů a je inverzní k $\alpha_{A,Y}$.
      \end{lem}
      \begin{proof}
        Jasně $1_A\otimes g(1_A)\in Hom_A(A,A)\otimes_A Y$. Nejprve ukážeme, že $\varphi_{A,Y}$ 
        je homomorfismus $R$-modulů. Nechť $r\in R$ a $g\in Hom_A(A,Y)$. Pak
        \begin{eqnarray}
          \varphi_{A,Y}(rg)
            &=& 1_A\otimes (rg)(1_A) \nonumber \\
            &=& 1_A\otimes (r1_A)(g(1_A)) \nonumber \\
            &=& r1_A\otimes (g(1_A)) \nonumber \\
            &=& r\cdot 1_A\otimes (g(1_A)) \nonumber \\
            &=& r\varphi_{A,Y}(g). \nonumber
        \end{eqnarray}
        Dále ukážeme, že platí následující dvě rovnosti:
        \begin{eqnarray}
          \varphi_{A,Y} \alpha_{A,Y} &=& 1_{Hom_A(A,A)\otimes_A Y} \nonumber \\
          \alpha_{A,Y} \varphi_{A,Y} &=& 1_{Hom_A(A,Y)} \nonumber 
        \end{eqnarray}
        Nechť $f\in Hom_A(A,A)$,  $g\in Hom_A(A,Y)$ a $y\in Y$, pak 
        \begin{eqnarray}
          \varphi_{A,Y} \alpha_{A,Y}(f\otimes g)
          &=& \varphi_{A,Y}([\lambda\mapsto f(\lambda)y]) \nonumber \\
          &=& 1_A \otimes f(1_A)y5 \nonumber \\
          &=& 1_Af(1_A)\otimes Y \nonumber \\
          &=& f\otimes y \nonumber 
        \end{eqnarray}
        a 
        \begin{eqnarray}
          \alpha_{A,Y} \varphi_{A,Y}(g)
          &=& \alpha_{A,Y} (1_A \otimes g(1_A))  \nonumber \\
          &=& [\lambda\mapsto 1_A(\lambda)g(1_A)]  \nonumber \\
          &=& [\lambda\mapsto \lambda g (1_A)=g(\lambda)]=g.  \nonumber 
        \end{eqnarray}
        Pak $\varphi_{A,Y} $ a $\alpha_{A,Y}$ jsou vzájemně inverzní homomorfismy.
      \end{proof}
      
      \begin{lem}
        Nechť $P,Y\in mod(A)$, pak platí:
        \begin{description}
          \item[(a)] $P^*=Hom_A(P,A)\in mod(A^{op})$ 
          \item[(b)] $P^*\otimes Y =Hom_A(P,A)\otimes_A Y \in mod(R)$, kde násobení prvky $R$ definujeme následovně: 
          \\
            \centerline{$r\cdot f\otimes y:=(rf)\otimes y$}  
          \item[(c)] $Hom_A(P,Y)\in mod(R)$, kde násobení prvky $R$ definujeme následovně: 
          \\\\
            \centerline{$(rf)(p):=r(f(p))$}  
        \end{description}
      \end{lem}
      
      \begin{proof}
        To že je modulová struktura u (b) a (c) dobře definovaná je zřejmé. Ověříme 
        pouze u všech tří případů, že se jedná o konečně generované moduly.
        
        \begin{description}          
          \item[(a)] Funktor $()^*$ je funktorem $P(A)\to P(A^{op})$ a tedy $P^*$ 
            je projektivní modul, který je dle \hyperref[rozklad-proj]{Věty \ref*{rozklad-proj}} 
            direktním součtem konečného počtu $A$-modulů generovaných jedním prvkem. 
            Je tedy sám konečně generovaný.
            
          \item[(b)] Tenzorový součin $P^*\otimes Y$ je faktor $R$-modulu kartézského součinu 
             dvou konečně generovaných $R$-modulů a je tedy sám konečně 
             generovaný.
           
          \item[(c)] Plyne z (b) a z izomorfismu z \hyperref[lem-varphi-a-y-izo]{Lemma 
            \ref*{lem-varphi-a-y-izo}}.          
        \end{description}
      \end{proof}      
      
        Homomorfismy $\varphi_{A,Y}$ a $\alpha_{A,Y}$ jsou vzájemně inverzní, 
        jde tedy o izomorfismy. Označme $\varphi_{A,Y}^n$ diagonální $n\times n$ matici, jejíž všechny 
        nenulové prvky jsou rovny $\varphi_{A,Y}$.  Ta definuje zobrazení \\\\
        \centerline{$\varphi_{A,Y}^n:Hom_A(A,Y)^n\rightarrow(Hom_A(A,A)\otimes_A Y)^n$} 
        \\\\
        předpisem \\
        \centerline{$\{f_i\}_{i=1}^n\mapsto \{\varphi_{A,Y}(f_i)\}_{i=1}^n$.}\\\\
        Víme, že $\varphi_{A,Y}$ je izomorfismus, pak je jím také 
        $\varphi^n_{A,Y}$.
        
        Nyní se již pustíme do konstrukce cílového homomorfismu $\varphi_{P,Y}$.
        Následující diagram ilustruje postup, kterým ve třech krocích nalezneme hledaný izomorfismus 
        $\varphi_{P,Y}:Hom_A(P,Y)\rightarrow Hom_A(P,A)\otimes_A Y$: \\\\      
         \centerline{\xymatrix{            
            (Hom_A(A,Y))^n  
              \ar[r]^{ \varphi_{A, Y}^n}
            & (Hom_A(A,A)\otimes_A Y)^n
              \ar[d]\\
           Hom_A(A^n,Y)
              \ar[u]
              \ar@{.>}[r]_{ \varphi_{A^n, Y}}
            & Hom_A(A^n,A)\otimes_A Y
              \ar[d] \\             
            Hom_A(P\oplus P', Y)
              \ar[u]
              \ar@{.>}[r]_{ \varphi_{P\oplus P', Y}}
            & Hom_A(P\oplus P', A)\otimes_A Y
              \ar[d]\\
            Hom_A(P, Y)
              \ar[u]
              \ar@{.>}[r]_{ \varphi_{P, Y}}
            & Hom_A(P, A)\otimes_A Y 
         }}\\\\
        Budeme tímto diagramem postupovat odshora, od nám již známého 
        izomorfismu $\varphi^n_{A,Y}$ k hledanému izomorfismu 
        $\varphi_{P,Y}$.
        
       \paragraph{Krok 1:} 
       
       Buď $\{\nu_i:A\rightarrow A^n\}_{i=1}^n$ množina kanonických inkluzí 
       a $\{\rho_i:A^n\rightarrow A\}_{i=1}^n$ množina kanonických projekcí. 
       Definujme homomorfismus $A^{op}$-modulů \\\\
       \centerline{$\xi_1:Hom_A(A^n,Y)\rightarrow (Hom_A(A,Y))^n$} \\\\
       předpisem $f\mapsto\{f\nu_i\}_{i=1}^n$
       a homomorfismus $R$-modulů \\\\
       \centerline{$\xi_2:(Hom_A(A,A)\otimes_A Y)^n\rightarrow Hom_A(A^n,A)\otimes_A 
       Y$}\\\\
       předpisem $\{g_i\otimes y_i\}_{i=1}^n\mapsto \sum_{i=1}^n g_i\rho_i\otimes y_i$. Složíme-li nyní 
       $\xi_1$ a $\xi_2$ s izomorfismem $\varphi_{A,Y}^n$ na 
       $\varphi_{A^n,Y}:=\xi_2\circ \varphi_{A,Y}^n\circ \xi_1$, dostaneme diagram:\\\\   
         \centerline{\xymatrix{            
            (Hom_A(A,Y))^n  
              \ar[r]^{ \varphi_{A, Y}^n}
            & (Hom_A(A,A)\otimes_A Y)^n
              \ar[d]^{\xi_2:\{g_i\otimes \omega_i\}_{i=1}^n\mapsto\sum_{i=1}^ng_i\rho_i\otimes \omega_i} \\
           Hom_A(A^n,Y)
              \ar[u]_{\xi_1:f\mapsto \{f\nu_i\}_{i=1}^n}
              \ar@{.>}[r]_{ \varphi_{A^n, Y}}
            & Hom_A(A^n,A)\otimes_A Y              
         }}\\\\
       Pro $h\in Hom_A(A^n,Y)$ platí:        
       \begin{eqnarray}
         \varphi_{A^n, Y}(h) &=& \xi_2\circ \varphi_{A,Y}^n\circ \xi_1(h)  \nonumber \\
         &=& \xi_2\circ \varphi_{A,Y}^n(\{h\nu_i\}_{i=1}^n) \nonumber \\
         &=& \xi_2(\{\varphi_{A,Y}(h\nu_i)\}_{i=1}^n) \nonumber \\
         &=& \xi_2(\{1_A\otimes h\nu_i(1_A)\}_{i=1}^n)  \nonumber \\
         &=& \sum_{i=1}^n \rho_i\otimes h\nu_i(1_A) \nonumber
       \end{eqnarray}
       
       \paragraph{Krok 2:} 
       
       Nyní přejdeme k dalšímu kroku. 
       Dle \hyperref[rozklad-A-na-P]{Věty \ref*{rozklad-A-na-P}} 
       existuje $n\in \mathbb N$ a projektivní $A$-modul $P'_0$ takové, že
       $P_0\oplus P'_0\simeq A^n$. Označme tento $A$-modulový izomorfismus 
       $\psi: P_0\oplus P_0'\simeq A^n$. Izomorfismus $\psi$ nám definuje 
       další dva $R$-modulové homomorfismy:  
       \begin{eqnarray}         
          (-\circ \psi^{-1})_Y&:&
          Hom_A(P\oplus P',Y
          )\rightarrow Hom_A(A^n,Y) \nonumber \\
         (-\circ\psi)_A\otimes 1_Y&:&
         Hom_A(A^n,A)\otimes_A Y
         \rightarrow 
         Hom_A(P\oplus P',A)\otimes_A Y \nonumber 
       \end{eqnarray}
       Položme nyní $
         \varphi_{P\oplus P',Y}:=
         [(-\circ\psi)_A\otimes 1_Y] \circ
         \varphi_{A^n,Y} \circ
         [(-\circ \psi^{-1})_Y]
       $. Celý krok ilustruje následující diagram:\\\\   
         \centerline{\xymatrix{            
           Hom_A(A^n,Y)
              \ar@{.>}[r]_{ \varphi_{A^n, Y}}
            & Hom_A(A^n,A)\otimes_A Y
              \ar[d]^{(-\circ\psi)_A\otimes1_Y} \\             
            Hom_A(P\oplus P', Y)
              \ar[u]_{(-\circ\psi^{-1})_Y}
              \ar@{.>}[r]_{ \varphi_{P\oplus P', Y}}
            & Hom_A(P\oplus P', A)\otimes_A Y
         }}\\\\\\
       Pak pro $h\in Hom_A(P\oplus_A P',Y)$ platí: 
       \begin{eqnarray}
          \varphi_{P\oplus P',Y}(h)&=&[(-\circ\psi)_A\otimes 1_Y]\circ\varphi_{A^n,Y}\circ[(-\circ\psi^{-1})_Y](h) \nonumber \\
          &=& [(-\circ\psi)_A\otimes 1_Y]\circ\varphi_{A^n,Y}(h\psi^{-1})  \nonumber \\
          &=& [(-\circ\psi)_A\otimes 1_Y](\sum_{i=1}^n\rho_i\otimes h\psi^{-1}\nu_i(1_A))  \nonumber \\
          &=& \sum_{i=1}^n\rho_i\psi\otimes h\psi^{-1}\nu_i(1_A)  \nonumber
       \end{eqnarray}
       
       \paragraph{Krok 3:}
       
       Postupme nyní k poslednímu třetímu kroku konstrukce. Uvažujme kanonickou 
       projekci \\\\
       \centerline{$\pi : P\oplus P'\rightarrow P$} \\\\
       modulu $P\oplus P'$ na $P$ a kanonickou inkluzi \\\\
       \centerline{$\mu : P\rightarrow P\oplus P'$} \\\\
       modulu $P$ do $P\oplus P'$. Ty nám definují dva $R$-modulové 
       homomorfismy: 
       \begin{eqnarray}
         (-\circ\pi)_Y &:& Hom_A(P,Y)\rightarrow Hom_A(P\oplus P', Y) \nonumber \\
         (-\circ\mu)\otimes 1_Y &:& Hom_A(P\oplus P', A)\otimes_A Y \to Hom_A(P, A)\otimes_A Y \nonumber
       \end{eqnarray}
       Definujme homomorfismus $\varphi_{P,Y}:Hom_A(P,Y)\rightarrow Hom_A(P,A)\otimes_A Y$ vztahem: \\\\
       \centerline{$\varphi_{P,Y}:= [(-\circ\mu)\otimes 1_Y] \circ \varphi_{P\otimes P',Y} \circ (-\circ\pi)_Y$}      
       \\\\  
         \centerline{\xymatrix{           
            Hom_A(P\oplus P', Y)
              \ar@{.>}[r]_{ \varphi_{P\oplus P', Y}}
            & Hom_A(P\oplus P', A)\otimes_A Y
              \ar[d]^{((-\circ\mu)_A\otimes1_Y} \\
            Hom_A(P, Y)
              \ar[u]_{(-\circ\pi)_Y}
              \ar@{.>}[r]_{ \varphi_{P, Y}}
            & Hom_A(P, A)\otimes_A Y 
         }}\\\\\\
       Pro $h\in Hom_A(P,Y)$ pak platí:
       \begin{eqnarray}
         \varphi_{P,Y}(h) &=& [(-\circ\mu)\otimes 1_Y] \circ \varphi_{P\otimes P',Y} \circ (-\circ\pi)_Y (h)  \nonumber \\
         &=& [(-\circ\mu)\otimes 1_Y]\circ\varphi_{P\otimes P',Y} (h\pi)  \nonumber \\
         &=& [(-\circ\mu)\otimes 1_Y][\sum_{i=1}^n\rho_i\psi\otimes h\pi\psi^{-1}\nu_i(1_A)]  \nonumber \\
         &=& \sum_{i=1}^n\rho_i\psi\mu\otimes h\pi\psi^{-1}\nu_i(1_A) \nonumber
       \end{eqnarray}

       \begin{thm}\label{varphi-izomorfismus}
         Zobrazení $\varphi_{P,Y}:Hom_A(P,Y)\rightarrow Hom_A(P,A)\otimes_A Y$, 
         definované \\\\
         \centerline{$\varphi_{P,Y}(h):=\sum_{i=1}^n\rho_i\psi\mu\otimes h\pi\psi^{-1}\nu_i(1_A)$} 
         \\\\
         pro $h\in Hom_A(P,Y)$, je izomorfismem $R$-modulů, který je přirozený v 
         $P$ a $Y$.
       \end{thm}
       \begin{proof}
         Zobrazení $\varphi_{P,Y}$ je konstruováno pouhým skládáním homomorfismů 
         $R$-modulů, je tedy také $R$-modulovým homomorfismem.
         
         Abychom dokázali, že se jedná o izomorfismus, dokážeme, že je inverzní k 
         homomorfismu $\alpha_{P,Y}$ definovanému v \hyperref[lemma-alpha-P-Y]{Lemma 
         \ref*{lemma-alpha-P-Y}}. Nechť tedy $h\in Hom_A(P,Y)$, pak
         
         \begin{eqnarray}
           \alpha_{P,Y} \varphi_{P,Y}(h) &=& \alpha_{P,Y} \left(  \sum_{i=1}^n\rho_i\psi\mu\otimes h\pi\psi^{-1}\nu_i(1_A) \right) \nonumber \\
           &=& 
             \left[ 
               p \mapsto \left( 
                 \sum_{i=1}^n
                 \underbrace{ \rho_i\psi\mu(p) }_{\in A} 
                 \cdot 
                 \underbrace{ h\pi\psi^{-1}\nu_i }_{A-hom.}
                 (1_A) 
               \right) \right] \nonumber \\
           &=& \left[ p \mapsto \left( \sum_{i=1}^n h\pi\psi^{-1}\nu_i\rho_i\psi\mu(p) \right) \right] \nonumber \\
           &=& 
             \left[ 
               p \mapsto h\pi\psi^{-1} 
               \underbrace{ \left( \sum_{i=1}^n \nu_i\rho_i \right) }_{=1_{A^n}}
               \psi\mu(p) 
             \right]\nonumber \\
           &=& [p\mapsto h(p)] \nonumber \\
           &=& h. \nonumber
         \end{eqnarray}         
         Opačně nechť $f\otimes y\in Hom_A(P,A)\otimes_A Y$, pak
         
         \begin{eqnarray}
           \varphi_{P,Y} \alpha_{P,Y} (f\otimes y) &=& \varphi_{P,Y}([p\mapsto f(p)\cdot y]) \nonumber \\
           &=& \sum_{i=1}^n \rho_i\psi\mu \otimes [p\mapsto f(p)\cdot y](\pi\psi^{-1}\nu_i(1_A)) \nonumber \\
           &=& \sum_{i=1}^n \rho_i\psi\mu \otimes \underbrace{f\pi\psi^{-1}\nu_i(1_A)}_{\in A} \cdot y  \nonumber \\
           &=& \sum_{i=1}^n \rho_i\psi\mu \cdot \underbrace{f\pi\psi^{-1}\nu_i(1_A)}_{\in A} \otimes y \nonumber \\
           &=& \left[  
             p \mapsto \sum_{i=1}^n 
               \underbrace{\rho_i\psi\mu(p)}_{\in A}
               \cdot 
               \underbrace{f\pi\psi^{-1}\nu_i}_{A-hom.}
               (1_A)
           \right] \otimes y \nonumber \\
           &=& \left[  
             p \mapsto \sum_{i=1}^n 
               f\pi\psi^{-1}\nu_i
               ( \rho_i\psi\mu(p) )
           \right] \otimes y \nonumber \\
           &=& 
             f\pi\psi^{-1} 
             \left( 
               \underbrace{\sum_{i=1}^n\nu_i\rho_i}_{=1_{A^n}}
             \right)
             \psi\mu \otimes y
             \nonumber \\
           &=& f\otimes y.  \nonumber
         \end{eqnarray}
        Homomorfismus $\varphi_{P,Y}$ je tedy inverzní k $\alpha_{P,Y}$ a tedy 
        izomorfimus. Přirozenost v $P$ a $Y$ plyne z přirozenosti $\alpha_{P,Y}$ v $P$ a $Y$ 
        dokázané v \hyperref[lemma-alpha-P-Y]{Lemma \ref*{lemma-alpha-P-Y}}. 
       \end{proof}


   \subsection{Izomorfismus $\sigma_{\delta,X}$}\label{alg-sigma}
   
      Naším cílem je zkonstruovat $\underline{End}_A(X)^{op}$-izomorfismus \\\\
      \centerline{$\sigma_{\delta,X}:\delta^*(X)\rightarrow Ker(1_{Tr(X)}\otimes f)$.} \\\\
      Aplikujme funktor $()^*$ na minimální projektivní prezentaci modulu $X$.  
      Dostaneme následující krátkou exaktní posloupnost v $mod(A^{op})$, kde $\hat{t}$ 
      značí kanonickou projekci $Hom_A(P_1,Y)\rightarrow Tr(X)=Cok((-\circ 
      s)_A)$:\\\\
      \centerline{\xymatrix{
        Hom_A(P_0,A) \ar@{->}[r]^{(-\circ s)_A}
          & Hom_A(P_1,A) \ar@{->}[r]^{\hat{t}}
          & Tr(X) \ar@{->}[r] 
          & 0
      }}*\\\\
      Dále připomeňme následující funktory, kde $Y\in mod(A):$\\\\
      \centerline{$Hom_A(-,Y):mod(A)\rightarrow mod(R)$}\\\\
      \centerline{$-\otimes_A Y: Mod(A^{op})\rightarrow Mod(R)$}\\\\
      Aplikujeme-li funktor $Hom_A(-,Y)$ na minimální projektivní prezentaci 
      modulu $X$ a funktor $-\otimes_A Y$ na posloupnost *, dostaneme 
      následující komutativní diagram s exaktními řádky v $mod(R)$ (moduly ve spodním 
      řádku jsou konečně generované jakožto izomorfní obrazy konečně 
      generovaných modulů v horním řádku):\\\\
      \centerline{\xymatrix{
        0 \ar[r]
          & Hom_A(X,Y) \ar@{->}[r]^{(-\circ t)_Y}
          & Hom_A(P_0,Y) \ar@{->}[r]^{(-\circ s)_Y} \ar@{->}[d]^{\varphi_{P_0,Y}}
          & Hom_A(P_1,Y) \ar@{->}[d]^{\varphi_{P_1,Y}} \\
        & & Hom_A(P_0,A)\otimes_A Y \ar@{->}[r]^{(-\circ s)_A\otimes 1_Y}
          & Hom_A(P_1,A)\otimes_A Y \ar@{->}[r]^{\hat{t}\otimes 1_Y}
          & Tr(X)\otimes_A Y \ar@{->}[r] 
          & 0
      }}\\\\
      
      \begin{dfn}\label{def-phi} 
        Nechť $Y\in mod(A)$. Definujme homomorfismus $R$-modulů \\\\
        \centerline{$\phi_Y:Hom_A(P_1,Y)\rightarrow Tr(X)\otimes_A Y$} \\\\
        vztahem \\\\
        \centerline{$\phi_Y:=[\hat{t}\otimes 1_Y]\circ\varphi_{P_1,Y}$.}\\
      \end{dfn}
      
      \begin{pzn}
        Pro $h\in Hom_A(P_1,Y)$ máme        
        \begin{eqnarray}
          \phi_Y(h) &=& [\hat{t}\otimes 1_Y]\circ\varphi_{P_1,Y}(h) \nonumber \\
          &=& 
            [\hat{t}\otimes 1_Y] \left(  
              \sum_{i=1}^n\rho_i\psi\mu \otimes h\pi\psi^{-1}\nu_i(1_A)
            \right) \nonumber \\
          &=& \sum_{i=1}^n\hat{t}(\rho_i\psi\mu) \otimes h\pi\psi^{-1}\nu_i(1_A) \nonumber
        \end{eqnarray}
      \end{pzn}
      
      \begin{lem}\label{lem-exakt-radek-phi-Y}
        Homomorfismus $R$-modulů $\phi_Y$ je přirozený v $Y$ a následující 
        posloupnost $R$-modulů je exaktní: \\\\
      \centerline{\xymatrix{
        0 \ar@{->}[r]
          & Hom_A(X,Y) \ar@{->}[r]^{(-\circ t)_Y}
          & Hom_A(P_0,Y) \ar@{->}[r]^{(-\circ s)_Y} 
          & Hom_A(P_1,Y) \ar@{->}[r]^{\phi_Y}
          & Tr(X)\otimes_A Y \ar@{->}[r] 
          & 0
      }}\\
      \end{lem}
      \begin{proof}
        Diagram komutuje díky přirozenosti $\varphi_{P,Y}$.
        Exaktnost posloupnosti plyne z definice $\phi_Y$ a \hyperref[lem-komut-schod]{Lemma 
        \ref*{lem-komut-schod}}.
        
        Ukážeme nyní, že $\phi_Y$ je přirozené v $Y$. Nejprve ukážeme, že 
        homomorfismus $\hat t\otimes 1_Y$ 
        je v $Y$ přirozený. Nechť $Y,Y'\in mod(A)$ a nechť $h\in Hom_A(Y,Y')$. 
        Pak \\\\
        \centerline{\xymatrix{
          Hom_A(P_1,Y)\otimes_A Y \ar@{->}[rr]^{\hat t\otimes 1_Y} \ar@[->][d]_{1_{Hom_A(P_1,Y)}\otimes h}
            & & Tr(X)\otimes_A Y  \ar@[->][d]^{1_{Tr(X)}\otimes h} \\
          Hom_A(P_1,Y)\otimes_A Y' \ar@{->}[rr]^{\hat t\otimes 1_Y'} 
            & & Tr(X)\otimes_A Y' 
        }}\\\\\\
        zřejmě komutuje, protože \\\\
        \centerline{$
          [1_{Tr(X)}\otimes h]\circ [\hat t \otimes 1_Y]
          = \hat t \otimes h
          = [\hat t\otimes 1'_Y] \circ [1_{Hom_A(P_1,A)}\otimes h]
        $.} \\\\
        Navíc protože dle \hyperref[varphi-izomorfismus]{Věty \ref*{varphi-izomorfismus}} 
        je $\varphi_{P_1, Y}$ přirozený v $Y$, je jejich složení $\phi_Y$ také 
        přirozené v $Y$.        
      \end{proof}
      
      Připomeňme, že jsme si jako na začátku této kapitoly zvolili pevně exaktní 
      posloupnost $A$-modulů $\delta$\\\\
      \centerline{\xymatrix{
        0 \ar@{->}[r] 
          & M \ar@{->}[r]^f
          & N \ar@{->}[r]^g 
          & L \ar@{->}[r] 
          & 0
      }.}\\
      
      \begin{thm}\label{mega-diagram}
        Následující diagram je komutativní a jeho řádky i sloupce jsou exaktní 
        posloupnosti $R$-modulů:\\\\
      \centerline{\xymatrix{
        & 
          & 
          & 
          & 0 \ar@{->}[d] 
          & \\
        & 0 \ar@{->}[d]
          & 0 \ar@{->}[d] 
          & 0 \ar@{->}[d] 
          & Ker(1_{Tr(X)}\otimes f) \ar@{_{(}->}[d]
          & \\
        0 \ar@{->}[r]
          & Hom_A(X,M) \ar@{->}[r]^{(-\circ t)_M} \ar@{->}[d]_{(f\circ-)_X} 
          & Hom_A(P_0,M) \ar@{->}[r]^{(-\circ s)_M} \ar@{->}[d]_{(f\circ-)_P_0} 
          & Hom_A(P_1,M) \ar@{->}[r]^{\phi_M} \ar@{->}[d]_{(f\circ-)_P_1} 
          & Tr(X)\otimes_A M \ar@{->}[r] \ar@{->}[d]_{1_{Tr(X)}\otimes f} 
          & 0 \\
        0 \ar@{->}[r]
          & Hom_A(X,N) \ar@{->}[r]^{(-\circ t)_N} \ar@{->}[d]_{(g\circ-)_X} 
          & Hom_A(P_0,N) \ar@{->}[r]^{(-\circ s)_N} \ar@{->}[d]_{(g\circ-)_P_0} 
          & Hom_A(P_1,N) \ar@{->}[r]^{\phi_N} \ar@{->}[d]_{(g\circ-)_P_1} 
          & Tr(X)\otimes_A N \ar@{->}[r] \ar@{->}[d]_{1_{Tr(X)}\otimes g} 
          & 0 \\
        0 \ar@{->}[r]
          & Hom_A(X,L) \ar@{->}[r]^{(-\circ t)_L} \ar@{->}[d] 
          & Hom_A(P_0,L) \ar@{->}[r]^{(-\circ s)_L} \ar@{->}[d] 
          & Hom_A(P_1,L) \ar@{->}[r]^{\phi_L} \ar@{->}[d] 
          & Tr(X)\otimes_A L \ar@{->}[r] \ar@{->}[d] 
          & 0 \\
        & \delta^*(X) \ar@{->}[d] 
          & 0
          & 0
          & 0
          & \\
        & 0
      }}\\
      \end{thm}
      \begin{proof}
        Exaktnost řádků a komutativita pravých čtverců plyne z 
        \hyperref[lem-exakt-radek-phi-Y]{Lemma \ref*{lem-exakt-radek-phi-Y}}. 
        Zbytek diagramu komutuje z asociativity skládání homomorfismů 
        $R$-modulů.
        Exaktnost levého sloupce plyne z exaktnosti zleva funktoru $Hom_A(X,-)$ a 
        definice $\delta^*(X)$. Prostřední dva sloupce jsou exaktní z 
        projektivity $A$-modulů $P_0$ a $P_1$. A konečně pravý sloupec je 
        exaktní, protože $Tr(X)\otimes_A-$ je zprava exaktní funktor.      
      \end{proof}
      
      \begin{dfn}
        Definujme zobrazení $\sigma_{\delta,X}:\delta^*(X)\rightarrow Ker(1_{Tr(X)}\otimes f)$ 
        následujícím algoritmem:     
        
        \begin{description}
          \item[\space\space\space Vstup:] $\bar{h}\in \delta^*(X)$
          
          \item[\space\space\space Výstup:] $\sigma_{\delta,X}(\bar{h})$
          
         \item[\space\space\space Průběh:] 
            \begin{description} 
               \item[]
               \item[(a)] Nejprve zvolme vzor $h\in Hom_A(X,L)$ prvku $\bar{h}$.
               \item[(b)] Zvolme libovolné $u\in Hom_A(P_0,N)$ takové, že $gu=ht$.
               \item[(c)] Nalezněme $v\in Hom_A(P_1,M)$ takové, že $fv=us$.
               \item[(d)] Položme $\sigma_{\delta,X}(\bar{h}):=\phi_M(v)$.
            \end{description}     
        \end{description}\,\,     
      \end{dfn}
      
      \begin{thm}\label{thm-sigma-delta-x}
        Zobrazení $\sigma_{\delta,X}$ je 
        izomorfismus $\underline{End}_A(X)^{op}$-modulů \\\\
        \centerline{$\sigma_{\delta,X}:\delta^*(X)\rightarrow Ker(1_{Tr(X)}\otimes f)$,} \\\\
        který je přirozený v $\delta$ a $X$.
      \end{thm}
      \begin{proof}
        Nebudeme zde podrobně dokazovat, že $\sigma_{\delta,X}$ je dobře 
        definovaný (nezávislý na volbě $h$, $u$ a $v$),
         zobrazuje do $Ker(1_{Tr(X)}\otimes f)$ a že je přirozený v $\delta$ i $X$. 
         Podrobný důkaz je možné nalézt v \cite{3} Proposition 76.
        
        Naznačíme zde alespoň jakým způsobem jsou obě strany 
        $\underline{End}_A(X)^{op}$-moduly, čímž lépe porozumíme jejich 
        struktuře pro naši další práci. Struktura $\underline{End}_A(X)^{op}$-modulu na $\delta^*(X)$ 
         je dána následujícím způsobem
         \begin{eqnarray}
           \delta^*(X)\times \underline{End}_A(X) &\to& \delta^*(X) \nonumber \\
           (\bar h,\bar e) &\mapsto& \overline{h\circ e}, \nonumber
         \end{eqnarray}
         kde $h\in Hom_A(X,L)$  a $e\in End_A(X)$ jsou libovolní reprezentanti 
         prvků $\bar h$ a $\bar e$.
         A v druhém případě je
         $\underline{End}_A(X)^{op}$-modulová struktura na $Ker(1_{Tr(X)}\otimes f)$ 
         dána zobrazením
         \begin{eqnarray}
           Ker(1_{Tr(X)}\otimes f) \times \underline{End}_A(X)^{op} &\to& Ker(1_{Tr(X)}\otimes f) 
           \nonumber \\
           (q\otimes a, \bar e) &\mapsto& 
           (Tr(e)\otimes 1_M)_{Ker}\underbrace{(q\otimes a)}_{\in Ker(1_{Tr(X)}\otimes 
           f)}= \nonumber \\
            && =(Tr(e)\otimes 1_M)\underbrace{(q\otimes a)}_{\in Tr(X)\otimes_A M}
           \nonumber
         \end{eqnarray}
         jak je ilustrováno na následujícím diagramu: \\\\
         \centerline{\xymatrix{
           0 \ar[r] 
             & Ker(1_{Tr(X)}\otimes f) \ar[r] \ar[d]^{(Tr(e)\otimes 1_M)_{Ker}}
             & Tr(X)\otimes_A M \ar[r] \ar[d]^{Tr(e)\otimes 1_M}
             & Tr(X)\otimes_A N \ar[r] \ar[d]^{Tr(e)\otimes 1_N}
             & Tr(X)\otimes_A L \ar[r] \ar[d]^{Tr(e)\otimes 1_L}
             & 0 \\
           0 \ar[r] 
             & Ker(1_{Tr(X)}\otimes f) \ar[r]
             & Tr(X)\otimes_A M \ar[r]
             & Tr(X)\otimes_A N \ar[r]
             & Tr(X)\otimes_A L \ar[r]
             & 0
         }.}\\\\\\
        Zbytek důkazu  tedy vynecháme.        
      \end{proof}
      
    \subsection{Izomorfismus $\gamma_{\delta,X}$}
        
        Nejprve připomeňme izomorfismus abelovských group z \hyperref[thm-adjunkce]{Věty \ref*{thm-adjunkce}} 
        \\\\
        \centerline{$\theta_{M,N,L}:Hom_R(M\otimes_A N,L)\rightarrow Hom_A(N,Hom_R(M,L))$,} 
        \\\\
        kde $M\in Mod(A^{op})$, $N\in Mod(A)$ a $L\in Mod(R)$, přirozený ve všech složkách. Ten využijeme ke 
        konstrukci izomorfismu 
        \\\\
        \centerline{$\gamma_{\delta, X}:D(Ker(1_{Tr(X)}\otimes f)) \rightarrow \delta_*(DTr(X))$,}
        \\\\
        kterému poté dodefinujeme strukturu $\underline{End}_A(X)$-modulového 
        homomorfismu.
        
        Budeme opět pracovat s diagramem z \hyperref[mega-diagram]{Věty 
        \ref*{mega-diagram}} a naší posloupností $\delta$. 
        Připomeňme si funktor duálu $D=Hom_R(-,I)$ 
        a uvažujme endomorfismus $h\in End_A(X)$. Potom je $Tr(h)\in End_R(Tr(X))$. 
        Máme následující komutativní diagram:\\
        \centerline{\xymatrix{
          Ker(1_{Tr(X)}\otimes f) \ar@{->}[r] \ar@{->}[d]_{(Tr(h)\otimes 1_M)_{ker}} 
            & Tr(X) \otimes_A M  \ar@{->}[rr]^{1_{Tr(X)}\otimes f} \ar@{->}[d]_{Tr(h)\otimes 1_M}
            & & Tr(X) \otimes_A N \ar@{->}[d]_{Tr(h)\otimes 1_N} \\
          Ker(1_{Tr(X)}\otimes f) \ar@{->}[r]   
            & Tr(X) \otimes_A M  \ar@{->}[rr]^{1_{Tr(X)}\otimes f}
            & & Tr(X) \otimes_A N
        }}\\\\\\
        Aplikujeme-li na jeho levý čverec funktor $D$, dostaneme následující komutativní 
        diagram:\\\\
        \centerline{\xymatrix{
          DKer(1_{Tr(X)}\otimes f) = Hom_R(Ker(1_{Tr(X)}\otimes f),I)
              \ar@{->}[r] 
              \ar@{->}[d]_{(-\circ (Tr(h)\otimes 1_M)_{ker})_I} 
            & Hom_R(Tr(X) \otimes_A M,I)  \ar@{->}[d]_{(-\circ Tr(h)\otimes 1_M)_I} \\
          DKer(1_{Tr(X)}\otimes f) = Hom_R(Ker(1_{Tr(X)}\otimes f),I)
            \ar@{->}[r]   
            & Hom_R(Tr(X) \otimes_A M,I)  
        }}\\\\\\
        Na základě tohoto diagramu zformulujeme následující lemma.
                 
        \begin{lem}
          $R$-modul $DKer(1_{Tr(X)}\otimes f)$ je spolu s násobením \\\\
          \centerline{$\underline{End}_A(X)\times DKer(1_{Tr(X)}\otimes f) \rightarrow DKer(1_{Tr(X)}\otimes f)$} 
          \\\\
          definovaným  \\\\
          \centerline{$\bar{h}\cdot\bar{z}:=(-\circ(Tr(h)\otimes1_M)_{Ker})_I(\bar{z})=\bar{z}\circ(Tr(h)\otimes1_M)_{Ker}$} 
          \\\\
          $\underline{End}_A(X)$-modulem.
        \end{lem}
        \begin{proof}
           Nejprve ukážeme, že \\\\           
           \centerline{$\bar{h}\cdot\bar{z}:=\bar{z}\circ(Tr(h)\otimes1_M)_{Ker}=0$}\\\\
           pro všechny $h\in P_A(X,X)$ a $z\in DKer(1_{Tr(X)}\otimes f)$. V  
           důkazu \hyperref[thm-sigma-delta-x]{Věty \ref*{thm-sigma-delta-x}} 
           jsme viděli, že $(Tr(h)\otimes 1_M)_{Ker}=0$ pro všechny $h\in P_A(X,X)$. 
           Tím jsme hotovi.
           
           Dále, protože $I$ je injektivní modul, pro každé $\bar z\in DKer(1_{Tr(X)}\otimes f)$ 
           existuje $z\in D(Tr(X)\otimes_A M)$ takové, že \\
           \centerline{$\bar z = zi$.}\\\\
           Uvažujme následující diagram v $mod(R)$: \\\\
        \centerline{\xymatrix{
          Ker(1_{Tr(X)}\otimes f) \ar@{->}[r]^i \ar@{->}[d]_{(Tr(h)\otimes 1_M)_{Ker}}
            & Tr(X) \otimes_A M \ar@{->}[d]^{Tr(h)\otimes 1_M}  \ar@{->}[rr]^{1_{Tr(X)\otimes f}}
            & & Tr(X) \otimes_A N \\
          Ker(1_{Tr(X)}\otimes f) \ar@{->}[r]^i \ar@{->}[d]_{\bar z}
            & Tr(X) \otimes_A M \ar@{.>}[ld]_z \ar@{->}[rr]^{1_{Tr(X)\otimes f}}
            & & Tr(X) \otimes_A N \\
          I
         }}\\\\\\
         Vidíme, že \\\\
         \centerline{$\bar h \bar z=\bar z \circ(Tr(h)\otimes 1_M)=z\circ (Tr(h)\otimes 1_M)\circ 
         i$}*\\\\\
         pro každé $z$ splňující $\bar z = zi$.
         
           Pro $q\otimes a \in Ker(1_{Tr(X)}\otimes f)$ platí
            \begin{eqnarray}
            (\bar h\bar z)(q\otimes a)
             &=& \bar z((Tr(h)\otimes 1_M)_{Ker})(q\otimes a)  \nonumber \\
             &=& z(Tr(h)\otimes 1_M)i(q\otimes a)   \nonumber \\
             &=& z(Tr(h)(q)\otimes a).  \nonumber
          \end{eqnarray}
        
          Nyní ověříme, že násobení splňuje axiom asociativity. Nechť $\bar z\in DKer(1_{Tr(X)}\otimes 
          f)$, $\bar h_1,\bar h_\in\bar{End}_A(X)$, předpokládejme, že $z\in D(Tr(X)\otimes_A M)$ 
          splňuje podmínku $\bar z = zi$ a $Tr(h_1)$ a $Tr(h_2)$ jsou 
          reprezentanti $Tr(\bar h_1)$ a  $Tr(\bar h_2)$. Pak
          \begin{eqnarray}
            \bar h_1(\bar h_2 \bar z)
              &\overset{*}{=}& \bar h_1 (\underbrace{z\circ (Tr(h_2)\otimes 1_M)\circ i}_{\in DKer(1_{Tr(X)}\otimes f)})  \nonumber \\
              &\overset{*}{=}& z \circ (Tr(h_2)\otimes 1_M) \circ (Tr(h_1)\otimes 1_M) \circ i   \nonumber \\
              &=& z (Tr(h_2)Tr(h_1)\otimes 1_M) \circ i \nonumber \\
              &=& z (Tr(h_2h_1)\otimes 1_M) \circ i \nonumber \\
              &\overset{*}{=}& (\bar h_1 \bar h_2)\bar z.  \nonumber
          \end{eqnarray}
        
          Zbytek axiomů ověřovat nebudeme, jejich ověření je přímočaré a ponecháme ho čtenáři.
        \end{proof}       
          
        \begin{lem}\label{upsilon-je-modul}
          Definujme na množině $\Upsilon_{DTr(X),L}/\sim$ tříd ekvivalence krátkých exaktních posloupností 
          vedoucích z $DTr(X)$ do $L$ násobení \\\\
          \centerline{$\underline{End}_A(X)\times (\Upsilon_{DTr(X),L}/\sim) \rightarrow (\Upsilon_{DTr(X),L}/\sim)$} 
          \\\\
          předpisem  \\\\
          \centerline{$
            \bar{h} 
            \cdot 
            (0\rightarrow DTr(X)\rightarrow E\rightarrow L\rightarrow 0)
            \mapsto
            (0\rightarrow DTr(X)\rightarrow E'\rightarrow L\rightarrow 0)
          $,} 
          \\\\
          kde modul $E'$ je pushout diagramu: \\\\
          \centerline{\xymatrix{
          DTr(X) \ar@{->}[r] \ar@{->}[d]^{DTr(h)}        
            &E 
            \\
          DTr(X)  
          }}\\\\\\
          Pak je $\Upsilon_{DTr(X),L}/\sim$ spolu s výše definovaným násobením  
          $\underline{End}_A(X)$-modulem.
        \end{lem}
        \begin{proof}
           Nejprve nechť $h_1,h_2\in \underline{End}_A(X)$. Pak \\\\
           \centerline{$\bar h_1(\bar h_2(0\to DTr(X)\to E\to L\to 0))$} \\\\
           nám dává následující diagram:  \\\\
          \centerline{\xymatrix{
          0 \ar@{->}[r]        
            & DTr(X) \ar@{->}[r]  \ar@{->}[d]^{DTr(h_2)}
            & E \ar@{->}[r] \ar@{->}[d]
            & L \ar@{->}[r] \ar@{=}[d]        
            & 0  \\
          0 \ar@{->}[r]  
            & DTr(X) \ar@{->}[r]  \ar@{->}[d]^{DTr(h_1)}
            & E' \ar@{->}[r] \ar@{->}[d]
            & L \ar@{->}[r]  \ar@{=}[d]     
            & 0\\
          0 \ar@{->}[r]  
            & DTr(X) \ar@{->}[r] 
            & E'' \ar@{->}[r] 
            & L \ar@{->}[r]       
            & 0
          }}\\\\\\
          Potřebujeme dokázat, že spodní řádek diagramu obdržíme i jako:\\\\
           \centerline{$(\bar h_1\bar h_2)(0\to DTr(X)\to E\to L\to 0)=(\overline{h_1 h_2})(0\to DTr(X)\to E\to L\to 0)$} \\\\
           Neboli, že $E''$ je zároveň pushoutem diagramu i následujícího diagramu: \\\\
          \centerline{\xymatrix{
            DTr(X) \ar@{->}[r]  \ar@{->}[d]^{DTr(h_1h_2)}
              & E  \\
            DTr(X)
          }}\\\\\\
          Protože jsou $D$ i $Tr$ kontravariantní funktory, pak \\\\
          \centerline{$DTr(h_1h_2)=DTr(h_1)DTr(h_2)$.}\\\\
          Z vlastností pushoutu je zřejmé, že $E''$ společně s korespondujícím 
          homomorfismem z $Hom_A(DTr(X),E'')$ a složením pushoutových morfismů z 
          $Hom_A(E,E')$ a $Hom_A(E,E'')$ z původního diagramu splní první 
          vlastnost pushoutu, neboli udělá následující diagram komutativní:\\\\
          \centerline{\xymatrix{
            DTr(X) \ar@{->}[r]  \ar@{->}[d]^{DTr(h_1h_2)} 
              & E \ar@{->}[d] \\
            DTr(X) \ar@{->}[r]
              & E''
          }}\\\\\\
          Ještě je třeba dokázat univerzální vlastnost pushoutu. Využijeme toho, 
          že $E'$  a $E''$ jsou pushouty následujících diagramů:\\\\
          \centerline{\xymatrix{
            DTr(X) \ar@{->}[r]  \ar@{->}[d]^{DTr(h_2)}
              & E  \\
            DTr(X)
          }\\\\\xymatrix{
            DTr(X) \ar@{->}[r]  \ar@{->}[d]^{DTr(h_1)}
              & E'  \\
            DTr(X)
          }}\\\\\\
          Mějme nějaké $E'''\in mod(A)$ spolu s morfismy z $Hom_A(E,E'')$ a $Hom_A(DTr(X),E''')$ 
          takovými, že následující diagram komutuje\\\\
          \centerline{\xymatrix{
            DTr(X) \ar@{->}[r]  \ar@{->}[d]^{DTr(h_1h_2)} 
              & E \ar@{->}[d] \\
            DTr(X) \ar@{->}[r]
              & E'''
          }}\\\\\\
          Protože $DTr(h_1h_2)=DTr(h_1)DTr(h_2)$, 
          existuje z univerzální vlastnosti pushoutu 
          $E'$ jednoznačný homorfismus z $Hom_A(E',E''')$ s odpovídajícími 
          vlastnostmi. A z univerzální vlastnosti $E''$ dostaneme stejným způsobem homomorfismus
          $Hom_A(E'',E''')$. Je zřejmé, že následující diagram komutuje \\\\
          \centerline{\xymatrix{
          DTr(X) \ar@{->}[r]  \ar@{->}[d]^{DTr(h_2)} \ar@/_3pc/[dd]_{DTr(h_1h_2)}
            & E  \ar@{->}[d]  \\
          DTr(X) \ar@{->}[r]  \ar@{->}[d]^{DTr(h_1)}
            & E' \ar@{->}[d] \ar@{.>}[ddr] \\
          DTr(X) \ar@{->}[r] \ar@{->}[drr]
            & E'' \ar@{.>}[dr] \\
          & & E'''
          }}\\\\\\
          a $E''$ je tedy pushoutem diagramu\\\\
          \centerline{\xymatrix{
            DTr(X) \ar@{->}[r]  \ar@{->}[d]^{DTr(h_1h_2)}
              & E  \\
            DTr(X)
          }}\\\\\\
          a asociativita našeho násobení je dokázána.
        \end{proof}
        
        \begin{lem}\label{lem-ext-delta-jsou-moduly}
          R-moduly $Ext^1(L,DTr(X))$ a $\delta_*(DTr(X))$ jsou zároveň $\underline{End}_A(X)$-moduly. 
        \end{lem}
        \begin{proof}
          Dle \hyperref[ekvivalence-upsilon-ext]{Věty \ref*{ekvivalence-upsilon-ext}} máme izomorfismus 
          abelovských grup \\\\ 
          \centerline{$Ext^1(L,DTr(X))\simeq \Upsilon_{DTr(X),L}/\sim $,} \\\\
           který přenáší strukturu 
          $\underline{End}_A(X)$-modulu i na $Ext^1(L,DTr(X))$.          
          Dále je \\\\
          \centerline{$\delta_*(DTr(X)) \subseteq Ext^1(L,DTr(X))$}\\\\
           jakožto $R$-modul. Potřebujeme dokázat, že 
          $\delta_*(DTr(X))$ je $\underline{End}_A(X)$-podmodul 
          $Ext^1(L,DTr(X))$. Postupovat budeme tak, že identifikujeme každý 
          prvek modulu $\delta_*(DTr(X))$ s prvkem $\Upsilon_{DTr(X),L}/\sim$ a poté 
          dokážeme, že násobením výsledného prvku prvkem $\bar{h}\in\underline{End}_A(X)$ 
          dostaneme opět prvek $\Upsilon_{DTr(X),L}/\sim$ korespondující s nějakým prvkem 
          $\delta_*(DTr(X))$. Tím bude dána struktura $\underline{End}_A(X)$-modulu 
          i na $\delta_*(DTr(X))$.
            
          Mějme tedy libovolný prvek $\bar{y}\in\delta_*(DTr(X))$. Protože z definice 
          máme \\\\
          \centerline{$\delta_*(DTr(X))=Hom_A(M,DTr(X))/Im((-\circ f )_I)$,} \\\\
          můžeme zvolit $y\in Hom_A(M,DTr(X))$ takové, že \\\\
          \centerline{$\bar{y}=y+Im((-\circ f )_I)$.} \\\\
          Dle \cite{5} Proposition 5.13 pushout v kategorii modulů vždy existuje. 
          Pak nám pushout $E$ homomorfismů $f$ a $y$ dává následující komutativní diagram v 
          $mod(A)$:  \\\\
          \centerline{\xymatrix{
          & 0 \ar@{->}[r]        
            & M \ar@{->}[r]^f \ar@{->}[d]^y
            & N \ar@{->}[r]^g \ar@{->}[d]
            & L \ar@{->}[r] \ar@{=}[d]        
            & 0  \\
          & 0 \ar@{->}[r]  
            & DTr(X) \ar@{->}[r] 
            & E \ar@{->}[r] 
            & L \ar@{->}[r]       
            & 0
          }}\\\\
          Spodní řádek diagramu je krátká exaktní posloupnost, kterou 
          identifikujeme s prvkem $\bar{y}\in\delta_*(DTr(X))$. Je možné dokázat, 
          že je určena jednoznačně až na ekvivalenci $\sim$ (viz. [5, Ch.7]).
         
         Vynásobíme-li tuto posloupnost prvkem $\bar{h}\in \underline{End}_A(X)$, dostaneme exaktní 
         posloupnost   \\\\
          \centerline{\xymatrix{
          0 \ar@{->}[r]  
            & DTr(X) \ar@{->}[r] 
            & E' \ar@{->}[r] 
            & L \ar@{->}[r]       
            & 0
          },}\\\\
          která při našem ztotožnění odpovídá prvku $DTr(h)y\in Hom_A(M,DTr(X))$, který je reprezentant 
          třídy $\bar{DTr(h)y}\in\delta_*(DTr(X))$. Vše ilustruje následující 
          diagram:  \\
          \centerline{\xymatrix{
          & 0 \ar@{->}[r]        
            & M \ar@{->}[r]^f \ar@{->}[d]^y
            & N \ar@{->}[r]^g \ar@{->}[d]
            & L \ar@{->}[r] \ar@{=}[d]        
            & 0  \\
          & 0 \ar@{->}[r]  
            & DTr(X) \ar@{->}[r] \ar@{->}[d]^{DTr(h)}
            & E \ar@{->}[r] \ar@{->}[d]
            & L \ar@{->}[r] \ar@{->}[d]      
            & 0  \\
          & 0 \ar@{->}[r]  
            & DTr(X) \ar@{->}[r] 
            & E' \ar@{->}[r] 
            & L \ar@{->}[r] 
            & 0
          }}\\\\\\
          A tedy $\delta_*(DTr(X))$ je $\underline{End}_A(X)$-podmodul 
          $Ext^1(L,DTr(X))\simeq \Upsilon_{DTr(X),L}/\sim$.
        \end{proof}
        
        \begin{thm}\label{izo-gamma-existuje}
           Nechť $(-\circ \iota )_I$ je kanonická projekce \\\\
           \centerline{$D(Tr(X)\otimes_A M)\rightarrow DKer(1_{Tr(X)}\otimes  f)$.} \\\\
           Izomorfismus $\gamma_{\delta, X}:D(Ker(1_{Tr(X)}\otimes f)) \rightarrow \delta_*(DTr(X))$ 
           existuje a je dán předpisem 
           \begin{eqnarray}
             \gamma_{\delta, X}(\bar{z}) &=& \theta_{Tr(X),M,I}(z)+Im((-\circ f)_{DTr(X)}) \nonumber \\
             &=& [m\mapsto z(-\otimes m)]+Im((-\circ f)_{DTr(X)}), \nonumber
           \end{eqnarray}
           kde $z\in D(Tr(X)\otimes_A M)$ je takové, že $\bar{z}=(-\circ \iota)_I(z)=zi$.
        \end{thm}
        \begin{proof} 
          Nejprve dokážeme existenci tohoto izomorfismu.
          Aplikujeme-li funktor $D$ na exaktní posloupnost z \hyperref[mega-diagram]{Věty \ref*{mega-diagram}} \\\\
          \resizebox{14cm}{!}{\xymatrix{
          0 \ar@{->}[r] 
            & Ker(1_{Tr(X)}\otimes f) \ar@{->}[r]           
            & Tr(X)\otimes_A M \ar@{->}[r]
            & Tr(X)\otimes_A N \ar@{->}[r]          
            & Tr(X)\otimes_A L \ar@{->}[r] 
            & 0      
          }}\\\\
          a $Hom_A(-,DTr(X))$ na posloupnost $\delta$, dostaneme následující diagram \\\\
          \resizebox{14cm}{!}{\xymatrix{
          0 \ar@{->}[r]        
            & D(Tr(X)\otimes_A L) \ar@{->}[r] \ar@{->}[d]_{\simeq}^{\theta_{Tr(X),L,I}} 
            & D(Tr(X)\otimes_A N) \ar@{->}[r] \ar@{->}[d]_{\simeq}^{\theta_{Tr(X),N,I}} 
            & D(Tr(X)\otimes_A M) \ar@{->}[r] \ar@{->}[d]_{\simeq}^{\theta_{Tr(X),M,I}}   
            & D(Ker(1_{Tr(X)}\otimes f)) \ar@{->}[r] \ar@{-->}[d]_{\simeq}^{\gamma_{\delta, X}}            
            & 0  \\
          0 \ar@{->}[r]  
            & Hom_A(L, DTr(X)) \ar@{->}[r]   
            & Hom_A(N, DTr(X)) \ar@{->}[r]   
            & Hom_A(M, DTr(X)) \ar@{->}[r]   
            & \delta_*(DTr(X)) \ar@{->}[r]            
            & 0
          }}\\\\\\
          Pak protože $\theta_{Tr(X),M,I}$, $\theta_{Tr(X),N,I}$ i $\theta_{Tr(X),L,I}$ 
          jsou izomorfismy abelovských grup, existuje dle \hyperref[lemma-five]{Lemma \ref*{lemma-five}} 
          takový   izomorfismus abelovských grup \\\\
          \centerline{$\gamma_{\delta, X}:D(Ker(1_{Tr(X)}\otimes f)) \rightarrow \delta_*(DTr(X))$,} 
          \\\\
          že diagram komutuje.
          
          
          Pro $\bar h \in \underline{End}_A(X)$ máme
          \begin{eqnarray}
            \gamma_{\delta, X}(\bar h \bar z)&=& \gamma_{\delta, X}(z\circ (Tr(h)\otimes 1_M)\circ i) \nonumber \\
            &=& [a\mapsto z(Tr(h)(-)\otimes a)]+Im((-\circ f)_{DTr(X)}) \nonumber
          \end{eqnarray}      
         Ukážeme, že stejný prvek $\Upsilon_{DTr(X),L}/\sim$ obdržíme
         \begin{description}
           \item[(1)] identifikováním $[a\mapsto z(Tr(h)(-)\otimes a)]+Im((-\circ f)_{DTr(X)}) 
           $
           s třídou ekvivalence v $\Upsilon_{DTr(X),L}/\sim$.
           \item[(2)] identifikováním $[a\mapsto z(-\otimes a)]+Im((-\circ f)_{DTr(X)})$ 
            s třídou ekvivalence v $\Upsilon_{DTr(X),L}/\sim$ a poté 
            vynásobením $\bar h$, jako v  \hyperref[lem-ext-delta-jsou-moduly]{Lemma \ref*{lem-ext-delta-jsou-moduly}}.
         \end{description}
         
         Poznamenejme, že první případ vede ke krátké exaktní posloupnosti 
         vzniklé jako pushout $f$ a $[a\mapsto z(Tr(h)(-)\otimes a)]$, zatímco 
         druhý případ k pushoutu $f$ a složeného zobrazení \\\\
         \centerline{$DTr(h)\circ[a\mapsto z(-\otimes a)]=[a\mapsto DTr(h)(z(-\otimes a))]$.}\\\\
         Protože
         \begin{eqnarray}
           DTr(h)(z(-\otimes a)) &=& (-\circ Tr(h))_I(z(-\otimes a)) \nonumber 
           \\
           &=& z(-\otimes a)\circ Tr(h) \nonumber \\
           &=& z(Tr(h)(-)\otimes a), \nonumber 
         \end{eqnarray}
         vidíme, že \\\\
         \centerline{$
         DTr(h)\circ [a\mapsto z(-\otimes a)]=[a\mapsto z(Tr(h)(-)\otimes a)].
         $}\\\\
         Právě jsme ukázali, že $\gamma_{\delta, X}(\bar h\bar z)=\bar h \gamma_{\delta, X}(\bar z)$ 
         pro všechny $\bar z\in D(Ker(1_{Tr(X)}\otimes f))$ a $\bar h\in 
         \underline{End}_A(X)$. Tedy, že jde o izomorfismus 
         $\underline{End}_A(X)$-modulů.
         
         
          Přirozenost našeho izomorfismu dokazovat nebudeme. Důkaz je možné 
          nalézt v \cite{3} na stranách 109-113.
        \end{proof}
        
    \subsection{Izomorfismus $\omega_{\delta,X}$} 
    
      \begin{thm}\label{thm-omega}
        Položme \\\\
        \centerline{$\omega_{\delta,X}:=\gamma_{\delta,X}D\sigma^{-1}_{\delta,X}$.}  \\\\        
        Pak $\omega_{\delta,X}$ je izomorfismem $\underline{End}_A(X)$-modulů  \\\\
        \centerline{$\omega_{\delta,X}:D\delta^*(X)\rightarrow \delta_*(DTr(X))$,} \\\\
        který je přirozený v $\delta$ a $X$.        
      \end{thm}
      \begin{proof}
        Protože \\\\
        \centerline{$\sigma^{-1}_{\delta,X}:Ker(1_{Tr(X)}\otimes f)\rightarrow \delta^*(X)$} 
        \\\\
        je izomorfismem $\underline{End}_A(X)$-modulů, je \\\\
        \centerline{$D\sigma^{-1}_{\delta,X}:D\delta^*(X) \rightarrow DKer(1_{Tr(X)}\otimes f)$} 
        \\\\ 
        také izomorfismem $\underline{End}_A(X)$-modulů. Pak je 
        $\gamma_{\delta,X}D\sigma^{-1}_{\delta,X}$ jakožto homomorfismus vzniklý 
        složením dvou izomorfismů $\underline{End}_A(X)$-modulů také
        izomorfismem $\underline{End}_A(X)$-modulů. 
        
        Protože $D$ je funktor, plyne přirozenost $D\sigma^{-1}_{\delta,X}$ v $\delta$ a $X$ z 
        přirozenosti $\sigma^{-1}_{\delta,X}$ ve stejných proměnných.
        
        Navíc protože je složení dvou přirozených 
        transformací opět přirozená transformace, je i 
        $\omega_{\delta,X}:=\gamma_{\delta,X}D\sigma^{-1}_{\delta,X}$
        přirozená v proměnných  v $\delta$ a $X$.
      \end{proof}
      \clearpage
      
  \section{Algoritmus pro nalezení skoro štěpitelné posloupnosti} 
  
    \paragraph{ } Nyní popíšeme samotný algoritmus a dokážeme jeho správnost.
    Nadále budeme namísto  okruhu R pracovat s libovolným komutativním tělesem K. 
    Funktor $D$ tedy bude dle \hyperref[lem-dual-teleso]{Lemma \ref*{lem-dual-teleso}} 
    tvaru $D=Hom_K(-,K):mod(K)\rightarrow mod(K)$.      
    Navíc budeme pracovat s jednou pevně zvolenout exaktní 
    poslupností $\delta$ \\\\
      \centerline{\xymatrix{
        0 \ar@{->}[r] 
          & \Omega \ar@{->}[r]^i 
          & P_0 \ar@{->}[r]^t 
          & X \ar@{->}[r] 
          & 0
      },}\\\\
    kde $(P_0, t)$ je projektivní pokrytí modulu $X$, $\Omega:=Ker(t)$ a $i$ kanonické vnoření. Nechť navíc $(P_1, \omega)$ 
    je projektivní pokrytí $\Omega$ a $s:=i\omega$.  Pak máme projektivní 
    prezentaci modulu $X$:\\\\
      \centerline{\xymatrix{
           & P_1 \ar@{->}[r]^s
          & P_0 \ar@{->}[r]^t 
          & X \ar@{->}[r] 
          & 0
      },}\\\\
    Pevná volba posloupnosti $\delta$ je možná, protože moduly $P_0$, $P_1$ a $\Omega$ 
    jsou určeny modulem $X\in mod(A)$ jednoznačně až na izomorfismus.
    
    Navíc si vzledem k pevné volbě $\delta$ jednodušeji označme 
    homomorfismy z předchozí kapitoly. Položme:
    \begin{description}
      \item[(a)] $\sigma_X:=\sigma_{\delta,X}:\delta^*(X)\rightarrow Ker(1_{Tr(X)}\otimes i)$ 
      \item[(b)] $\gamma_X:=\gamma_{\delta,X}: DKer(1_{Tr(X)\otimes i})\rightarrow \delta_*(DTr(X))$ 
      \item[(c)]  $\omega_X:=\omega_{\delta,X}: D\delta^*(X) \rightarrow \delta_*(DTr(X))$ 
    \end{description}
    
    V následujících několika tvrzeních zjednodušíme výsledky z předchozích 
    částí, poté již zformulujeme algoritmus.
  
      \begin{lem}\label{lem-delta-as-hom}
        Nechť $Y\in mod(A)$, pak $\delta^*(Y)=\underline{Hom}_A(Y,X)$.
      \end{lem}
      \begin{proof}
        Připomeňme, že pro $Y\in mod(A)$ je $\delta^*(Y)$ definováno exaktností 
        následující posloupnosti: \\
        \centerline{\xymatrix{
          0 \ar@{->}[r] 
            & Hom_A(Y,\Omega) \ar@{->}[r]^{(i\circ-)_Y} 
            & Hom_A(Y,P_0) \ar@{->}[r]^{(t\circ-)_Y}
            & Hom_A(Y,X) \ar@{->}[r] 
            & \delta^*(Y) \ar@{->}[r] 
            & 0
        }}\\\\      
        Tedy $\delta^*(Y)$ je kojádro $(t\circ-)_Y$, neboli \\\\
        \centerline{$\delta^*(Y)=Hom_A(Y,X)/Im((t\circ-)_Y)$}* \\\\
        Zbývá dokázat, že $Im(t\circ-)_Y=P(Y,X)$.
        
        Pokud $u\in Im(t\circ-)_Y$, tak se $u$ faktorizuje skrze projektivní 
        A-modul $P_0$. Opačně, pokud se $u$ faktorizuje skrze nějaký projektivní 
        A-modul $P$, tak dle \hyperref[lem-faktorizuje-skrze-proj]{Lemma \ref*{lem-faktorizuje-skrze-proj}}
        se $u$ také faktorizuje skrze
        $P(X)$. Pak dle (*) dostáváme \\\\
        \centerline{$\delta^*(Y)=Hom_A(Y,X)/P(Y,X)=\underline{Hom}_A(Y,X)$.}
      \end{proof}
      
      \begin{thm}\label{thm-omega-x}
        $\omega_X$ je izomorfismem $\underline{End}_A(X)$-modulů: \\\\
        \centerline{$\omega_X: D\underline{End}_A(X) \rightarrow Ext_A^1(X,DTr(X))$}
      \end{thm}
      \begin{proof}
        Připomeňme z \hyperref[thm-omega]{Věty \ref*{thm-omega}}, že \\\\
        \centerline{$\omega_X: D\delta^*(X) \rightarrow\delta_*(DTr(X))$} \\\\
        \hyperref[lem-delta-as-hom]{Lemma \ref*{lem-delta-as-hom}} 
        imlikuje, že $\delta^*(X)=\underline{End}_A(X)$ a tedy \\\\
        \centerline{$D\delta^*(X)=D\underline{End}_A(X)$}. \\
        
        Navíc protože $X\in mod(A)$, máme $Tr(X)\in mod(A^{op})$ a $DTr(X)\in 
        mod(A)$.
        Aplikováním kontravariatního funktoru $Hom_A(-,DTr(X))$ na $\delta$ 
        dostaneme dle \cite{5} Theorem 7.3 následující exaktní 
        posloupnost: \\\\
        \centerline{$
          0\rightarrow 
          Hom_A(X,DTr(X))\rightarrow 
          Hom_A(P_0,DTr(X))\rightarrow
          Hom_A(\Omega(X),DTr(X))\rightarrow\ldots 
        $}
        \centerline{$
          \ldots\rightarrow 
          Ext^1_A(X,DTr(X))\rightarrow 
          Ext^1_A(P_0,DTr(X))
        $}\\
        
        Protože je $P_0$ projektivní, je $Ext_A^1(P_0,DTr(X))=0$ a my dostáváme 
        posloupnosti z \hyperref[def-delta-*]{Definice \ref*{def-delta-*}}, kde je $X$ 
        nahrazeno $DTr(X)$. A tedy: 
        \\\\
        \centerline{$\delta_*(DTr(X))=Ext^1_A(X,DTr(X))$.}
      \end{proof} 
    
      \paragraph{ }\label{phi-omega-nenul}
      Připomeňme \hyperref[lem-ext-delta-jsou-moduly]{Lemma \ref*{lem-ext-delta-jsou-moduly}}, 
      že $Ext^1_A(X,DTr(X)$ má strukturu konečně generovaného $\underline{End}_A(X)$-modulu 
      (tu jsme přenesli ztotožněním jeho prvků s třídami ekvivalence krátkých exaktních posloupností).
      Dále dle věty \hyperref[ekvivalence-upsilon-ext]{Věty \ref*{ekvivalence-upsilon-ext}} 
      máme izomorfismy:
      \begin{eqnarray}
         Soc_\Gamma(\Upsilon_{DTr(X),X}/\sim)&\simeq&(\hat{\Upsilon}_{DTr(X),X}/\sim) \nonumber \\
         Ext_A^1(V,U)&\simeq& (\Upsilon_{U,V}/\sim) \nonumber 
      \end{eqnarray}      
      Tedy $Soc_\Gamma(Ext^1_A(X,DTr(X))$ koresponduje s množinou $\tilde{\Upsilon}_{DTr(X),X}/\sim$ tříd 
      ekvivalence skoro štěpitelných posloupností $mod(A)$ končících v $X$.
      
      Navíc $Soc_\Gamma(Ext^1_A(X,DTr(X))$ je jako $\underline{End}_A(X)$-modul 
      jednoduchý a tedy dle \hyperref[lem-jednoduchy-modul-gen]{Lemma \ref*{lem-jednoduchy-modul-gen}}
      může být vygenerován každým svým prvkem. Díky izomorfismu $\omega_X$ ve tvaru z
      \hyperref[thm-omega-x]{Věty \ref*{thm-omega-x}} \\\\
        \centerline{$\omega_X: D\underline{End}_A(X) \rightarrow Ext_A^1(X,DTr(X))$,} 
        \\\\
      vidíme, že každý nenulový prvek $e\in Soc_\Gamma(D\underline{End}_A(X))$ 
      může být použit k vygenerování celého $\tilde{\Upsilon}_{DTr(X),X}/\sim$, jelikož
      \\\\
      \centerline{$\omega_X(e)\in Soc_\Gamma(Ext^1_A(X,DTr(X))\simeq (\tilde{\Upsilon}_{DTr(X),X}/\sim)$} \\\\
      bude nenulový.     
    
      \begin{lem}\label{lem-B-ker}
        Uvažujme identitu $\bar{1_X}\in\underline{End}_A(X)$. Nechť \\\\
        \centerline{$B_{Ker(1_{Tr(X)}\otimes i)}:=\{\sigma_X(\bar{1_X}),\omega_2\,\ldots,\omega_l\}$} \\\\
        je K-báze $Ker(1_{Tr(X)}\otimes i)$. Pak \\\\
        \centerline{$\gamma_X(d_{B_{Ker(1_{Tr(X)}\otimes i)}}(\sigma_X(\bar{1_X})))
        \in Soc_\Gamma(Ext^1_A(X,DTr(X)))$} \\\\
        je generátor.
      \end{lem}
      
      \begin{proof}
        Nechť \\\\
        \centerline{$\sigma^{-1}_X(B_{Ker(1_{Tr(X)}\otimes i)}):=\{\bar 1_X,\sigma^{-1}_X(\omega_2),\ldots,\sigma^{-1}_X(\omega_l)\}$} 
        \\\\
         je $K$-báze $\underline{End}_A(X)$ korespondující s $B_{Ker(1_{Tr(X)}\otimes 
         i)}$. Víme, že \\\\
         \centerline{$\bar 1_X\in Top_{\Gamma^{op}}(\underline{End}_A(X))$}\\\\
         je nenulový prvek, pak dle \hyperref[lem-soc-top]{Věty \ref*{lem-soc-top}}  je\\\\
         \centerline{$(d_{\sigma_X^{-1}(B_{Ker(1_{Tr(X)}\otimes i)})})(\bar 1_X)\in Soc_\Gamma(D\underline{End}_A(X))$}\\\\
         nenulový prvek a tedy protože $\omega_X$ je izomorfismem $\underline{End}_A(X)$-modulů, je 
         \begin{eqnarray}
           \omega_X ((d_{\sigma_X^{-1}(B_{Ker(1_{Tr(X)}\otimes i)})})(\bar 1_X)) 
           &=& 
           \underbrace{
             \gamma_X(D\sigma_X^{-1}) ((d_{\sigma_X^{-1}(B_{Ker(1_{Tr(X)}\otimes i)})})(\bar 1_X))
           }_{\in Soc_\Gamma(Ext_A^1(X,DTr(X)))}   
           \nonumber
         \end{eqnarray}
         nenulový prvek. Navíc dle \hyperref[lem-baze-dual-xi]{Lemma \ref*{lem-baze-dual-xi}} 
         víme, že \\\\
         \centerline{$
           (D\sigma_X^{-1}) (d_{\sigma_X^{-1}(B_{Ker(1_{Tr(X)}\otimes i)})}(\bar 1_X))
           = 
           d_{B_{Ker(1_{Tr(X)}\otimes i)}}(\sigma_X(\bar 1_X)) 
         $}\\\\
         a tedy \\\\
         \centerline{$
           \omega_X ((d_{\sigma_X^{-1}(B_{Ker(1_{Tr(X)}\otimes i)})})(\bar 1_X)) 
           = 
           \gamma_X( d_{B_{Ker(1_{Tr(X)}\otimes i)}}(\sigma_X(\bar 1_X))  )
         $.}\\\\
         Potom dle \hyperref[lem-jednoduchy-modul-gen]{Lemma \ref*{lem-jednoduchy-modul-gen}} 
         je tento nenulový prvek generátorem $Soc_\Gamma(Ext_A^1(X,DTr(X)))$.
      \end{proof}
      
      \begin{lem}\label{alg-vraci}
        Algoritmus výpočtu $\sigma_{\delta,X}$ nám při našem pevně zvoleném $\delta$ a 
        vstupu $\bar{1}_X$ vrátí \\
        \centerline{ $\sigma_{X}(\bar{1}_X)=\phi_\Omega(\omega)$,}\\\\
        kde $\omega$ je projektivní pokrytí $\Omega$. ($\phi_\Omega$
        jsme zavedli v \hyperref[def-phi]{Definici \ref*{def-phi}}) 
      \end{lem}
      \begin{proof}
        Podívejme se znovu na diagram z \hyperref[mega-diagram]{Věty \ref*{mega-diagram}} a 
        upravme ho dle naší posloupnosti $\delta$: \\\\
      \centerline{\xymatrix{
        & 
          & 
          & 
          & 0 \ar@{->}[d] 
          & \\
        & 0 \ar@{->}[d]
          & 0 \ar@{->}[d] 
          & 0 \ar@{->}[d] 
          & Ker(1_{Tr(X)}\otimes i) \ar@{_{(}->}[d]
          & \\
        0 \ar@{->}[r]
          & Hom_A(X,\Omega) \ar@{->}[r]^{(-\circ t)_\Omega} \ar@{->}[d]_{(i\circ-)_X} 
          & Hom_A(P_0,\Omega) \ar@{->}[r]^{(-\circ s)_\Omega} \ar@{->}[d]_{(i\circ-)_P_0} 
          & Hom_A(P_1,\Omega) \ar@{->}[r]^{\phi_\Omega} \ar@{->}[d]_{(i\circ-)_P_1} 
          & Tr(X)\otimes_A \Omega \ar@{->}[r] \ar@{->}[d]_{1_{Tr(X)}\otimes i} 
          & 0 \\
        0 \ar@{->}[r]
          & Hom_A(X,P_0) \ar@{->}[r]^{(-\circ t)_{P_0}} \ar@{->}[d]_{(t\circ-)_X} 
          & End_A(P_0) \ar@{->}[r]^{(-\circ s)_{P_0}} \ar@{->}[d]_{(t\circ-)_P_0} 
          & Hom_A(P_1,P_0) \ar@{->}[r]^{\phi_{P_0}} \ar@{->}[d]_{(t\circ-)_P_1} 
          & Tr(X)\otimes_A {P_0} \ar@{->}[r] \ar@{->}[d]_{1_{Tr(X)}\otimes t} 
          & 0 \\
        0 \ar@{->}[r]
          & End_A(X) \ar@{->}[r]^{(-\circ t)_X} \ar@{->}[d] 
          & Hom_A(P_0,X) \ar@{->}[r]^{(-\circ s)_X} \ar@{->}[d] 
          & Hom_A(P_1,X) \ar@{->}[r]^{\phi_X} \ar@{->}[d] 
          & Tr(X)\otimes_A X \ar@{->}[r] \ar@{->}[d] 
          & 0 \\
        & \underline{End}_A(X) \ar@{->}[d] 
          & 0
          & 0
          & 0
          & \\
        & 0
      }}\\\\
      Provedeme $\sigma_X$-algoritmus pro prvek $\bar 1_X\in \underline{End}_A(X)$. 
      Jeho vzorem je prvek $1_X\in {End}_A(X)$. Jako $u\in {End}_A(P_0)$ takový, 
      že $tu=t$, zvolíme jednoduše $1_{P_0}$. Dále hledáme prvek $v\in Hom_A(P_1,\Omega)$ 
      takový, že $iv=s$, což z definice splňuje projektivní pokrytí $\omega$ 
      modulu $\Omega$. To pak zobrazíme na $\phi_\Omega(\omega)$.
      \end{proof}
      
      \paragraph{Algoritmus pro výpočet generátoru $\tilde{\Upsilon}_{DTr(X),X}/\sim$}     
        \begin{description}
          \item[\space\space\space Vstup:] $X\in mod(A)$ nerozložitelný a 
          neprojektivní.
          \item[\space\space\space Výstup:] Generátor $0 \rightarrow DTr(X) \rightarrow E \rightarrow X \rightarrow 0$} 
            množiny $\tilde{\Upsilon}_{DTr(X),X}/\sim$.
         \item[\space\space\space Průběh:] 
            \begin{description} 
               \item[]
               \item[(a)] Spočtěme projektivní pokrytí ($P_0$, $t$) modulu $X$  \\\\
                   \centerline{\xymatrix{
                      Ker(t) \ar@{^{(}->}[r]^i & P_0 \ar@{->}[r]^t & X  
                    },}\\\\  a položme $\Omega:=Ker(t)$.
               \item[(b)] Vezměme $\phi_\Omega(\omega)\in Ker(1_{Tr(X)}\otimes i)$, kde $(P_1,\omega)$ 
                    je projektivní pokrytí $\Omega$ a rozšiřme ho na K-bázi: \\\\
                    \centerline{$B_{Ker(1_{Tr(X)}\otimes i)}:=
                    \{\phi_\Omega(\omega),\omega_2,\ldots,\omega_l\}$.}  \\\\
                   \centerline{\xymatrix{
                      P_1 \ar@{->}[r]^\omega & \Omega \ar@{^{(}->}[r]^i & P_0 \ar@{->}[r]^t & X  
                    },}\\
               \item[(c)] Rozšiřme $B_{Ker(1_{Tr(X)}\otimes i)}$ na K-bázi \\\\
                    \centerline{$B_{Tr(X)\otimes\Omega}:=
                    \{\phi_\Omega(\omega),\omega_2,\ldots,\omega_l,\omega_{l+1},\ldots,\omega_{l+m}\}$.} 
                    \\
               \item[(d)] Definujme homomorfismus A-modulů $\xi:\Omega\rightarrow DTr(X)$  
                    následovně. Pro $a\in \Omega$ definujme $\xi(a):Tr(X)\rightarrow K$ 
                    předpisem \\\\
                    \centerline{$q\mapsto$[první K-koeficient $q\otimes a$ vzhledem k bázi 
                    $B_{Tr(X)\otimes_A\Omega}]$.}\\
               \item[(e)] Položme $E$ rovno pushoutu $i$ a $\xi$. 
            \end{description}     
        \end{description}          
        \centerline{ \xymatrix{
        \Omega \ar[r]^\xi \ar[d]_i & DTr(X) \ar@{..}[d] \\ 
        P_0 \ar@{..}[r] & E
      } }\,
      
      \begin{thm}
        Algoritmus vrací generátor \\\\
        \centerline{$0 \rightarrow DTr(X) \rightarrow E \rightarrow X \rightarrow 0$} 
        \\\\
        všech skoro štěpitelných posloupností v $mod(A)$ končících v $X$.
      \end{thm}
      \begin{proof}
        Dle \hyperref[alg-vraci]{Lemma \ref*{alg-vraci}} máme
        \\\\\centerline{$\sigma_X(\bar 1_X)=\phi_\Omega(w)$}\\\\
        a $B_{Ker(1_{Tr(X)}\otimes i)}$ je jako v  
        \hyperref[lem-B-ker]{Lemma \ref*{lem-B-ker}} a tedy 
        \\\\\centerline{$\gamma_X \underbrace{( d_{B_{Ker(1_{Tr(X)}\otimes i)}}( \phi_\Omega(w) ) )}_{a}=: \bar y$}\\\\
        generuje $Soc_\Gamma(Ext_A^1(X,DTr(X)))$.
        
        Stejně jako v důkazu \hyperref[lem-ext-delta-jsou-moduly]{Lemma \ref*{lem-ext-delta-jsou-moduly}} 
        je prvek $\hat\Upsilon/\sim$ obdržen jako pushout $E$ morfismů $i$ a 
        libovolného reprezentantu $y\in Hom_A(\Omega, DTr(X))$ prvku $\bar y$.
        Protože $\gamma_X$ je kojádro zobrazení $\theta_{Tr(X),\Omega,K}$, tak 
        pokud je 
        \\\\\centerline{$y:=\theta_{Tr(X),\Omega,K}(z)$,}\\\\
        pro libovolný reprezentant $z\in (Tr(X)\otimes_A\Omega)$ prvku $\bar z$, 
        pak je $y$ reprezentant prvku $\gamma_X(\bar z)=\bar y$. Nechť $\mu$ 
        značí inkluzi 
        \\\\\centerline{$\mu:Ker(1_{Tr(X)}\otimes i)\to Tr(X)\otimes_A\Omega$.}\\\\
        Pak 
        \\\\\centerline{$D\mu=(-\circ\mu)_K:D(Tr(X)\otimes_A\Omega)\to DKer(1_{Tr(X)}\otimes i)$}\\\\
        je kanonická projekce a pro každé $z\in D(Tr(X)\otimes_A\Omega)$ takové, že
        \\\\\centerline{$z\mu=\bar z$,}\\\\
        je reprezentantem $\bar z$. Ukážeme, že volba
        \\\\\centerline{$z:=d_{B_{Ker(1_{Tr(X)}\otimes i)}}( \phi_\Omega(w) )$}\\\\
        tuto podmínku splňuje.
        
        Nechť $q\otimes a\in Ker(1_{Tr(X)}\otimes i)$. Aplikace našeho zvoleného $z$ 
        na $q\otimes a$ koresponduje s vyjádřením $q\otimes a$ vzhledem k bázi  
        $B_{Ker(1_{Tr(X)}\otimes i)}$
        a extrakcí prvního $K$-koeficientu. Protože $B_{Tr(X)\otimes_A\Omega}$ je 
        pouze rozšířením $B_{Ker(1_{Tr(X)}\otimes i)}$, tak se výsledek nezmění, 
        pokud $q\otimes a$ vnoříme do $Tr(X)\otimes_A \Omega$ a vyjádříme ho 
        vzhledem k bázi $B_{Tr(X)\otimes_A\Omega}$ před extrakcí prvního 
        $K$-koeficientu, což odpovídá aplikaci $d_{B_{Tr(X)\otimes_A \Omega)}}( \phi_\Omega(w) ) \mu$ 
        na $q\otimes a$.
        
        Pak musíme spočíst pushout $i$ a 
        $\theta_{Tr(X),\Omega,K}(d_{B_{Tr(X)\otimes_A \Omega)}}( \phi_\Omega(w) 
        ))$, abychom obdrželi hledanou skoro štěpitelnou posloupnost. Připomeňme 
        \hyperref[thm-adjunkce]{Větu \ref*{thm-adjunkce}}, že
        \begin{eqnarray}
          \theta_{Tr(X),\Omega,K}(d_{B_{Tr(X)\otimes_A \Omega)}}( \phi_\Omega(w) ))(a)
            &=& [q\mapsto d_{B_{Tr(X)\otimes_A \Omega)}}( \phi_\Omega(w) )(q\otimes a)]\nonumber \\
            &=& [q\mapsto 1.\, koef.\, q\otimes a\,vzhledem\,k\,B_{Tr(X)\otimes_A \Omega)}]\nonumber \\
            &=& \xi(a) \nonumber 
        \end{eqnarray}
        pro každé $a\in\Omega$. A tedy 
        \\\\\centerline{$\theta_{Tr(X),\Omega,K}(d_{B_{Tr(X)\otimes_A \Omega)}}( \phi_\Omega(w) ))=\xi$,}\\\\
        tím je důkaz hotov a my se můžeme pustit do jeho implementace.
      \end{proof}












  \renewcommand{\section}[1]{ 
  \oldsection{#1}
}


\addtocontents{toc}{\protect\setcounter{tocdepth}{1}} 

\chapter{Implementace}\label{implementace}

  \paragraph{ } V této části implementujeme Algoritmus pro nalezení generátoru skoro štěpitelných 
  posloupností modulu $X$ nad algebrou cest $KQ$ toulce $Q$, kde $K$ je libovolné těleso.    
  Algoritmus implementujeme v sytému \cite{GAP4} (Groups, Algorithms, Programming -
a System for Computational Discrete Algebra) s využitím balíku \cite{QPA} (Quivers and path 
algebras).
  
  Algoritmem projdeme krok po kroku. Většina částí obsahuje nejprve teoretický popis
  a poté algoritmus implementovaný v 
  knihovně \cite{QPA}.  
  Syntaxe skriptovacího jazyka užitého v systému \cite{GAP4} je podobná mnoha 
  jiným  jazykům a nepotřebuje podrobnější výklad. Čtenář by měl být 
  schopen většině komentovaných ukázek kódu porozumět bez větších problémů.

    Teorie vychází převážně z \hyperref[teorie-reprezentaci]{části \ref*{teorie-reprezentaci}} 
    a tedy $Mod(A)$ bude značit kategorií pravých modulů, což je navíc  
    v souladu s balíkem \cite{QPA}. O levých modulech budeme tedy referovat jako 
    o $A^{op}$-modulech.
    
  \section{Značení v kódu QPA}
    \paragraph{ } Moduly budeme značit velkými písmenem s prefixem $m$. Tak například 
    snadno odlišíme $A$ 
    jako algebru a $A$ jako modul - značíme $mA$.    
    Morfismy budeme psát malými písmeny a řecké znaky jejich přepisem latinkou 
    (používaným v \LaTeX). Například rho, pi, ... .  
        
    Dále se budeme držet značení, které je využívano v celém \cite{QPA} balíku, 
    tedy například jako $PP$ budeme značit pole nerozložitelných projektivních 
    modulů tvaru $e_iA$ pro primitivní idempotent $e_i$ algebry $A$.

  \section{Vstup a výstup algoritmu}
  
    %\subsection*{Teorie}    

      \paragraph { } Nechť $K$ je komutativní těleso. 
      Mějme konečný toulec $Q$ a $K$-algebru cest $KQ$.
      Toulec je konečný, jde tedy o algebru s jednotkou (\hyperref[quiver-kq-lemma]{Lemma \ref*{quiver-kq-lemma}}). 
      Dále mějme libovolný přípustný ideál $I$ algebry $KQ$. Pokuď je $Q$ bez 
      cyklů, pak můžeme zvolit $I=0$. Dále položme $A=KQ/I$.
      
      Vstupem algoritmu je  $K$-algebra $A$  
      a $X\in mod(A)$ nerozložitelný, konečně generovaný a neprojektivní $A$-modul.
      
      Výstupem algoritmu bude $0\rightarrow DTr(X)\rightarrow E\rightarrow X\rightarrow 0$ 
      generátor všech skoro štěpitelných posloupnosté v $mod(A)$ končících v $X$.
    
      \subsection*{Příklad 1}
        \subparagraph{}
        \centerline{
          Toulec $Q$: \xymatrix{
            \circ^1 \ar@{->}[r]_a \ar@/^3pc/[rr]^c
              & \circ^2 \ar@{->}[r]_b
              & \circ^3
          }
          \rightaligned{
            \space\space\space\space $I=\emptyset$ \space\space\space\space
             $A$-modul $X$: \xymatrix{
               K^2 
                   \ar@{->}[r]_{\left[\begin{smallmatrix}
	               1 & 0 \\
	               0 & 1 \\
                      \end{smallmatrix}\right]} 
                   \ar@/^3pc/[rr]^{\left[\begin{smallmatrix}
	               1 \\
	               0 \\
                      \end{smallmatrix}\right]}
                 & K^2 
                   \ar@{->}[r]_{\left[\begin{smallmatrix}
	               0 \\
	               1 \\
                      \end{smallmatrix}\right]}   
                 & K^1
             }
           }
         }
  
      \begin{Verbatim}[frame=single,numbers=left]
K := Rationals;
Q := Quiver(3, [ [1, 2, "a"], 
                 [2, 3, "b"],
                 [1, 3, "c"] ]);
KQ := PathAlgebra(K,Q);
A := KQ;
matrices := [ ["a", [[1,0],[0,1]]], 
              ["b", [[0],[1]]], 
              ["c", [[1],[0]]] ];
mX := RightModuleOverPathAlgebra(A,matrices);
        \end{Verbatim}
        
     \subsection*{Příklad 2}
      
        \subparagraph{}
        \centerline{
          $Q$: \xymatrix{
            \circ^1 
                \ar@/^1pc/[r]^a 
                \ar@/_1pc/[r]_b 
              & \circ^2 
                \ar@{->}[r]_d
                \ar@(lu,ru)[]^c
              & \circ^3
                \ar@/^3pc/[ll]^e
          }
          \rightaligned{
            \space\space\space\space $I=\{c^2,acd-bd,ea,eb\}$ \space\space\space\space
             $X$: \xymatrix{
            K^2 
                \ar@/^1pc/[r]^{\left[\begin{smallmatrix}
	               0 & 1 \\
	               1 & 1 \\
                      \end{smallmatrix}\right]} 
                \ar@/_1pc/[r]_{\left[\begin{smallmatrix}
	               1 & 0 \\
	               1 & 0 \\
                      \end{smallmatrix}\right]} 
              & K^2 
                \ar@{->}[r]_{\left[\begin{smallmatrix}
	               1 & 1 \\
	               0 & 1 \\
                      \end{smallmatrix}\right]}
                \ar@(lu,ru)[]^{\left[\begin{smallmatrix}
	               0 & 0 \\
	               1 & 0 \\
                      \end{smallmatrix}\right]}
              & K^2
                \ar@/^3pc/[ll]^{\left[\begin{smallmatrix}
	               0 & 0 \\
	               0 & 0 \\
                      \end{smallmatrix}\right]}             
             }
           }
         }

        \begin{Verbatim}[frame=single,numbers=left]
K := Rationals;
Q := Quiver(3, [ [1, 2, "a"], 
                 [1, 2, "b"], 
                 [2, 2, "c"], 
                 [2, 3, "d"], 
                 [3, 1, "e"] ]);
KQ := PathAlgebra(K, Q);
gen := GeneratorsOfAlgebra(KQ);
a := gen[4];
b := gen[5];
c := gen[6];
d := gen[7];
e := gen[8];
rels := [c^2,a*c*d-b*d,e*a,e*b];
A := KQ/rels;
mat :=[["a", [[0,1],[1,1]]],
       ["b", [[1,0],[1,0]]],
       ["c", [[0,0],[1,0]]],
       ["d", [[1,1],[0,1]]],
       ["e", [[0,0],[0,0]]]
      ];
mX := RightModuleOverPathAlgebra(A,mat);
        \end{Verbatim}
    
  \section{Modul $\Omega$}
  
    %\subsection*{Teorie} 
      
      \paragraph{ } Spočteme projektivní pokrytí $P_0$ modulu $X$ \\\\
      \centerline{$Ker(t)\xrightarrow{i} P_{0}\xrightarrow{t}X$,} \\\\
      položíme $\Omega:=Ker(t)$. Dostaneme krátkou exaktní posloupnost \\\\
      \centerline{$0\rightarrow\Omega\xrightarrow{i} P_{0}\xrightarrow{t}X\rightarrow0$.}  \\\\
      Dále označme $P_1$ projektivní pokrytí $\Omega$. Výsledkem je minimální 
      projektivní prezentace modulu $X$: \\
      \centerline{ \xymatrix{
        P_1 \ar@{->}[rd]_w \ar@{->}[r]^{s=iw} &P_0 \ar@{->}[r]^t &X  \ar@{->}[r] &0 \\
        &\Omega \ar@{->}[u]_i
      } }      \\\\      
      Kanonickou inkluzi $i:\Omega\rightarrow P_0$ budeme v kódu značit 
      $kernel\_inc$, aby nedocházelo k záměně s  iterační proměnnou $i$.

    %\subsection*{QPA kód}
      \begin{Verbatim}[frame=single,numbers=left]
t          := ProjectiveCover(mX);
mP0        := Source(t);          
mOmega     := Kernel(t);
omega      := ProjectiveCover(mOmega);
kernel_inc := KernelInclusion(t); 
s          := omega * kernel_inc;  
mP1        := Source(omega);
       \end{Verbatim}
       
       Dále budeme potřebovat algebru $A^{op}$ a pole [$e_1A,e_2A,\ldots,e_mA$] nerozložitelných 
       projektivních $A$-modulů, kde $\{e_1,e_2,\ldots,e_m\}$ je úplná množina 
       primitivních ortogonálních idempotentů algebry $A$, a jemu korespondující 
       pole algebry $A^{op}$. Toto pole spočteme s pomocí funkce 
       $IndecProjectiveModules$.
       
      \begin{Verbatim}[frame=single,numbers=left]
A_op  := OppositePathAlgebra(A);
PP    := IndecProjectiveModules(A);
PP_op := IndecProjectiveModules(A_op);
      \end{Verbatim}
      
      Navíc ještě zkonstruujeme $A$ jako pravý $A$-modul. Ten je dle 
      \hyperref[rozklad-A-na-proj]{Věty \ref*{rozklad-A-na-proj}} direktním součtem 
      $A=e_1A\oplus e_2A\oplus \ldots \oplus e_mA$.
    
      \begin{Verbatim}[frame=single,numbers=left]
mA := DirectSumOfModules(PP);      
    \end{Verbatim}
    
    

  \section{Rozložení $A^n\simeq P_1\oplus P_1'$}
  
    %\subsection*{Teorie}  
      
      \paragraph{ } Algebra $A$ může být zapsána dle \hyperref[rozklad-A-na-proj]{Věty \ref*{rozklad-A-na-proj}} jako direktní součet 
      $A=e_1A\oplus e_2A\oplus \ldots \oplus e_mA$ 
      a tedy dle \hyperref[rozklad-proj]{Věty \ref*{rozklad-proj}} lze projektivní modul $P_1$ vyjádřit jako
      $P_1\simeq(e_1A)^{n_1}\oplus(e_2A)^{n_2}\oplus \ldots \oplus(e_mA)^{n_m}$.
       Položme $n:=max_{i=1,\ldots,m}(n_i)$, pak \\\\
       \centerline{$A^n\simeq P_1\oplus(e_1A)^{n-n_1}\oplus(e_2A)^{n-n_2}\oplus \ldots \oplus(e_mA)^{n-n_m}$.} \\\\ 
       Definujeme-li \\\\
       \centerline{$P_1':=(e_1A)^{n-n_1}\oplus(e_2A)^{n-n_2}\oplus \ldots \oplus(e_mA)^{n-n_m}$,}  
       \\\\
       dostáváme hledaný vztah  \\\\
       \centerline{$A^n\simeq P_1\oplus P_1'$.}   

    %\subsection*{QPA kód}
      \paragraph{ }  S pomocí funkce $IndecProjectiveModules(A);$ spočteme 
      moduly $e_iA$ a následně spočteme, kolikrát je každý z nich obsažen v 
      rozkladu modulu $P$ na direktní součet nerozložitelných projektivních podmodulů. 
      Číslo $n$ bude maximum z těchto čísel.
      Výsledkem tedy bude čtveřice \\\\
        \centerline{
          [ $n$, $P'$, [$n_1,n_2,\ldots,n_m$], [$n-n_1,n-n_2,\ldots,n-n_m$] ]
        }\\\\
      taková, že
      \begin{eqnarray}
        A^n &\simeq&
        \underbrace{((e_1A)^{n_1}\oplus \ldots \oplus(e_mA)^{n_m})}_{=P_1}
            \oplus
        \underbrace{((e_1A)^{n-n_1}\oplus \ldots \oplus(e_mA)^{n-n_m}).}_{=P_1'} 
        \nonumber
      \end{eqnarray}
      
      Algoritmus níže je zapsaný obecně pro libovolný projektivní $A$-modul. V našem 
      případě $mP:=P_1$.
        
      \begin{Verbatim}[frame=single,numbers=left]
SuppProjModule := function(mP)
  local A, PP, mPs, n, common, i, j, diff,
        in_multiplicities, ou_multiplicities;

  A := RightActingAlgebra(mP);
  PP:= IndecProjectiveModules(A);
  in_multiplicities := [];
  ou_multiplicities := [];
  n := 0;

  # Přes všechny moduly e_iA.
  for i in [1..Length(PP)] do
    Add(in_multiplicities, 0);

    # Zkoušíme kolikrát je daný e_iA obsažen v P.
    repeat
      common := CommonDirectSummand(mP, PP[i]);
      if IsList(common) then
        in_multiplicities[i] := in_multiplicities[i] + 1;
        mP := common[2];
      fi;
    until IsList(common) = false;

    n := Maximum([n, in_multiplicities[i]]);
  od;

  # Spočteme P'.
  mPs := [];
  for i in [1..Length(PP)] do
    diff := n - in_multiplicities[i];
    ou_multiplicities[i] := diff;

    for j in [1..diff] do
      Add(mPs, PP[i]);
    od;
  od;
  mPs := DirectSumOfModules(mPs);

  return [n, mPs, in_multiplicities, ou_multiplicities];
end;  
      \end{Verbatim}      
      Tuto funkci využijeme a dopočítáme další moduly potřebné pro náš výpočet.
      
      \begin{Verbatim}[frame=single,numbers=left]
supp := SuppProjModule(mP1);
mP1s := supp[3];                                    # P_1'
n    := supp[2];
mP1_mP1s := DirectSumOfModules([mP1, mP1s]);           
mAn := DirectSumOfModules( List([1..n], i -> mA) ); # A^n  
      \end{Verbatim}
      
      
  \section{Pomocné funkce}\label{alg-opposite}
 
          \paragraph{ } Nejprve definujeme pomocnou funkci, kterou použijeme 
          ještě několikrát později. Funkce vrátí
          pole $as\_algebra\_element$ velikosti $m$. Každé $as\_algebra\_element[i]$ spočteme následovně:
          
          Uvažujme bázi $B_i$ modulu $e_iA$. Modul $e_iA$ je 
          generovaný jedním prvkem a to primitivním idempotentem $e_i$, 
          takže všechny ostatní bázové prvky dostaneme 
          působením algebry $A$ na něj. Položme $as\_algebra\_element[i][j]=\lambda\in A$, 
          kde $\lambda$ je prvkem algebry $A$ takovým, že $e_i\lambda=B_i[j]$. \\
 
    \begin{Verbatim}[frame=single,numbers=left,numbers=left] 
ProjectiveBasisVectorGens := function(PP)
  local A;

  A := RightActingAlgebra(PP[1]);

  return List(BasisOfProjectives(A), b -> Flat(b));
end;
    \end{Verbatim}         
      
      Druhá funkce bude konstruovat matice homomorfismu z obrazů báze modulu, 
      jež je jeho definičním oborem. 
      
      Mějme například dva $A$-moduly $M$, $N$ a homomorfismus $f:M\to N$. 
       Označme vektory dimenze reprezentací $M$ a $N$ po řadě 
       $[m_1,m_2,\ldots,m_k]$ a $[n_1,n_2,\ldots,n_k]$ a položme $m:=\sum_{i=1}^k m_i$ 
       a  $n:=\sum_{i=1}^k n_i$. Dále mějme bázi $B_M=\{u_1,u_2,\ldots,u_m\}$ modulu $M$, 
       bázi $B_N=\{v_1,v_2,\ldots\,v_n\}$ modulu $N$ a matici $i$ velikosti $m\times n$, kde 
       se na $j$-tém řádku nachází $f(m_j)$ vzhledem k bázi $B_N$.
       Potom matice homomorfismu $f$ 
       jsou následující:\\\\


\hspace{-0.15\textwidth}     
\resizebox{17cm}{!}{       
\begin{array}{lccr}
    \underbrace{\begin{bmatrix}
      i[1,1] & \cdots & i[1,d_1] \\
      \vdots & \ddots & \vdots \\
      i[c_1,1] & \cdots & i[c_1,d_1)]
    \end{bmatrix} }_{f_1}
    &
    &
    &
    \begin{array}{ccc}
      \quad & \cdots & i[1,n] \\
      & & \vdots \\
      & & \quad
    \end{array}
    \\    
    & 
    \underbrace{\begin{bmatrix}
      i[c_1+1,d_1+1] & \cdots & i[c_1+1,d_1+d_2] \\
      \vdots & \ddots & \vdots \\
      i[c_1+c_2,d_1+1] & \cdots & i[c_1+c_2,d_1+d_2)]
    \end{bmatrix}}_{f_2}
    \\ 
    &
    &
    \ddots
    \\ 
    \begin{array}{ccc}
      \quad \\ 
      \vdots \\ 
      i[m,1] & \cdots & \quad
    \end{array}  
    & 
    & 
    &       
    \underbrace{\begin{bmatrix}
      i[m-c_{k-1}+1,n-d_{k-1}+1] & \cdots & i[m-c_{k-1}+1,n] \\
      \vdots & \ddots & \vdots \\
      i[m,n-d_{k-1}+1] & \cdots & i[m,n]
    \end{bmatrix}}_{f_k}
\end{array}
}}\\\\\\
     Naše funkce pak vátí pole $[f_1,f_2,\ldots,f_k]$ obsahující $k$ matic. Ty slouží 
     jako vstupní parametr pro výpočet homomorfismu funkcí 
     $RightModuleHomOverAlgebra$. \\
     
      \begin{Verbatim}[frame=single,numbers=left] 
ExtractHomMatrices := function(matrix, mM, mN)
  local A, Q, d_mM, d_mN, used_x, used_y, i, j, k, dx, dy;

  A := RightActingAlgebra(mM);
  Q := QuiverOfPathAlgebra(A);

  matrices := [];
  d_mM := DimensionVector(mM);
  d_mN := DimensionVector(mN);
  used_x := 0;
  used_y := 0;
  for i in [1..NumberOfVertices(Q)] do
    dx := d_mN[i];
    dy := d_mM[i];

    matrices[i] := [];

    if dy = 0 and dx = 0 then
      Add(matrices[i], [0]);
    elif dy = 0 then
      Add(matrices[i], List([1..dx], j -> 0));
    elif dx = 0 then
      Add(matrices[i], List([1..dy], j -> [0]));
    else
      for j in [1+used_y..dy+used_y] do
        Add(matrices[i], 
            List([1+used_x..dx+used_x], k -> matrix[j][k])
          );
      od;
    fi;

    used_x := used_x + dx;
    used_y := used_y + dy;
  od;

  return matrices;
end;        
      \end{Verbatim}
      
      
      
      

  \section{Dualita $P_1^*$}\label{dualita-p-hom}
  
    %\subsection*{Teorie}    
      \paragraph{ }Nyní nám nastává problém jak vyjádřit libovolný homomorfismus $f:P_1\rightarrow A$ 
      jakožto prvek reprezentace $P_1^*:=Hom_A(P_1,A)$. 
      Využijeme dvou izomorfismů z \hyperref[izo-hom-aei]{Věty \ref*{izo-hom-aei}} 
      a \hyperref[rozklad-proj]{Věty \ref*{rozklad-proj}}: \\\\
         \centerline{
           $Hom_A (e_iA, A)\simeq Ae_i$ daný předpisem $f \mapsto f(e_i)$
         }*
         \\\\
         \centerline{$P_1\simeq\bigoplus_{i=1}^m\bigoplus_{j=1}^{n_i}e_iA$}**
         \\\\
       Z definice $P_1^*$, (**) a \hyperref[dir-sum-hom]{Věty \ref*{dir-sum-hom}} dostáváme vztah: 
         \begin{eqnarray}           
           P_1^* &\equiv& 
             Hom_A (P_1,A) \nonumber \\
             &\simeq& 
             Hom_A (\bigoplus_{i=1}^m\bigoplus_{j=1}^{n_i}e_iA,A) \nonumber \\
             &\simeq&
             \bigoplus_{i=1}^m\bigoplus_{j=1}^{n_i}Hom_A (e_iA,A) \nonumber
         \end{eqnarray}
         }      
      Na tuto posloupnost použijeme ještě izomorfismus (*) a dostaneme 
       izomorfismus, který homomorfismu $P_1\rightarrow A$ přiřadí 
       korespondující prvek reprezentace $P_1^*$:
         \begin{eqnarray} 
             Hom_A (P_1,A) &\simeq&  \bigoplus_{i=1}^m\bigoplus_{j=1}^{n_i}Ae_i \nonumber \\
             f &\mapsto& \sum_{i=1}^m\sum_{j=1}^{n_i}f(e_i) \nonumber
         \end{eqnarray}
         
       Samotná implemetace v knihovně QPA bude složitější a provedena v 
       několika krocích. To je způsobeno tím, že $f(e_i)$ není automaticky prvkem $A^{op}$-modulu $e_iA$, 
       ale z definice homomorfismu $f$ stále prvkem $A$-modulu $A$. Opačný jemu korespondující prvek 
       musíme teprve spočíst. Stejně tak musíme prvky $e_i$ nejprve vnořit z modulů $e_iA$ do $P_1$. 
       Pro každý sčítanec z rozkladu $P_1$ na direktní součet 
       modulů $e_iA$ zobrazíme $e_i\in e_iA$ následujícím řetězcem zobrazení: 
       \\\\
       \centerline{\xymatrix{ 
         & & & e_1A \\
            e_iA 
              \ar@{^{(}->}[r]^{\eta_{i,j}} 
            & P_1  
              \ar@{->}[r]^f
            & A 
              \ar@{->>}[ru]^{\beta_1}
              \ar@{->>}[rd]_{\beta_m}
            & \vdots \\            
         & & & e_mA
        }} \\\\\\     
      kde $\eta_{i,j}$ je kanonické vnoření $e_iA\rightarrow P_1$ a $\beta_i$ 
      kanonická projekce $A\rightarrow e_iA$. Prvek $\beta_k f \eta_{i,j}(e_i)$ je prvkem modulu 
      $e_kA$ a je tedy tvaru $e_k \lambda_{i,j,k}$ pro nějaké $\lambda_{i,j,k}\in A$ a 
      $e_k$ generátor modulu $e_kA$ odpovídající indempotentu $e_k$ algebry $A$. My 
      vyjádříme prvky $\lambda_{i,j,k}$ ($i=1,2,\ldots,m$) jako prvky algebry $A$ a následně je všechny sečteme 
      na 
      $\tilde\lambda_{i,j}=\sum_{i=1}^m e_k\lambda_{i,j,k} \in A$. 
      
      Spočteme 
      opačný prvek $\tilde\lambda_{i,j}^{op}\in A^{op}$, ten musí být ze vztahu výše z 
      ideálu $Ae_i$, přesněji v \cite{QPA} implementaci z ideálu $e_i^{op} A^{op}$.
      Dále spočteme prvek $e_i^{op}\tilde\lambda_{i,j}^{op}$ jakožto prvek reprezentace 
      $e_i^{op}A^{op}$ a vnoříme ho kanonickým vnořením $\eta_{i,j}':Ae_i\to P^*$ do $P_1^*$.
      
      Všechny takto vnořené prvky sečteme a dostaneme tak hledaný prvek odpovídající 
      homomorfismu $f$.
        
      Samotný izomorfismus  $Hom_A (P_1,A) \rightarrow \bigoplus_{i=1}^m\bigoplus_{j=1}^{n_i}Ae_i$
      bude tedy, pomineme-li přechody mezi reprezentací $A$ a algebrou $A$ v samotné implementaci, 
      dán předpisem: \\\\
         \centerline{
           $f\mapsto\sum_{i=1}^m\sum_{j=1}^{n_i} \eta_{i,j}'(e_i^{op} (\sum_{k=1}^m\beta_kf\eta_{i,j}(e_i))^{op})$
         } \\\\


    %\subsection*{QPA kód}    
     
     Níže je algoritmus zapsán obecně pro $P$  projektivní modul a homomorfismus $f:P\to A$. 
     Prvním parametrem je $f:P\to A$ a druhým modul $P^*$.   
         
      \begin{Verbatim}[frame=single,numbers=left] 
FromHomToProjRep := function(f, mP_star)
  local i, j, incl, incl2, proj, mu, mu_f, pi, mu2,
    A, e_i, e_i_op, fe_i, result, mP, me_iA, PP, mA,
    as_algebra_element, lambda, lambda_op, pi_f_ei, coeffs;

  mP := Source(f);
  mA := Range(f);
  A := RightActingAlgebra(mP);

  # Moduly e_iA.
  PP := IndecProjectiveModules(A);

  incl := DirectSumInclusions(mP);
  incl2:= DirectSumInclusions(mP_star);
  proj := DirectSumProjections(mA);
  as_algebra_element := ProjectiveBasisVectorGens(PP);

  result := Zero(mP_star);

  # Přes všechny inkluze z direktních sčítanců P tedy moduly e_iA.
  for i in [1..Length(incl)] do
    # Slozime s inkluzi na homomorfismus e_iA --> P -> A.
    mu := incl[i];
    mu_f := mu * f;

    # Spočteme e_iA a e_i.
    me_iA := Source(mu_f);
    e_i := MinimalGeneratingSetOfModule(me_iA)[1];

    # Zobrazíme e_i homomorfismem f.
    fe_i := ImageElm(mu_f, e_i);

    # Nyní budeme prvek f(e_i) projektovat do modulů e_iA ...
    # ... k jeho obrazům najdeme korespondující prvek ...
    # ... algebry A a výsledky sečteme.
    lambda := Zero(A);
    for j in [1..Length(proj)] do
      pi := proj[j];
      pi_f_ei := ImageElm(pi, fe_i);
      coeffs  := Coefficients(Basis(PP[j]), pi_f_ei);
      lambda  := lambda + coeffs * as_algebra_element[j];
    od;

    # Spočteme opposite prvek.
    lambda_op := OppositePathAlgebraElement(lambda);

    # Vnoříme do P*.
    mu2 := incl2[i];
    e_i_op := MinimalGeneratingSetOfModule(Source(mu2))[1];
    result := result + ImageElm(mu2, e_i_op ^ lambda_op);
  od;

  return result;
end;      \end{Verbatim}
      
      Tento izomorfismus budeme potřebovat i v opačném směru. Z důkazu věty 
      \hyperref[izo-hom-aei]{Věty \ref*{izo-hom-aei}} máme inverzní 
      izomorfismus k (*) a to:
      \begin{eqnarray}
        Ae_i &\to& Hom_A(e_iA,A) \nonumber \\
        \lambda e_i &\mapsto& [e_i\lambda'\mapsto \lambda e_i\lambda']  \nonumber
      \end{eqnarray}
      Podobně jako v prvním případě, pak dostáváme izomorfismus:
      \begin{eqnarray}
        P_1^* &\to& Hom_A(P,A) \nonumber \\
        \sum_{i=1}^m\sum_{j=1}^{n_i} \lambda_{i,j} e_i &\mapsto& \sum_{i=1}^m\sum_{j=1}^{n_i} [e_i\lambda'\mapsto \lambda_{i,j} e_i\lambda']  \nonumber      
      \end{eqnarray}
      Implementace bude opět složitější.
      Budeme-li pro nějaký prvek $p\in P_1^*$ hledat odpovídající homomorfismus $P_1\to A$, budeme postupovat
      následovně. Pro každé $i=1,2,\ldots,m$ a $j=1,2,\ldots,n_i$ zobrazíme $p$ na $Ae_i$ 
      kanonickou projekcí $\varrho'_{i,j}:P_1^*\to Ae_i$. Poté spočteme prvek $\lambda\in A^{op}$, pro 
      který je $\varrho'_{i,j}(e_i)=\lambda e_i$. K němu spočteme opačný prvek $\lambda^{op}\in 
      A$. Výsledný homomorfismus $P_1\to A$ je tvaru \\\\
      \centerline{$\alpha_i[e_i\lambda'\mapsto e_i\lambda^{op}\lambda']\varrho_{i,j}$,}\\\\
      kde $\alpha_i$ je inklize $e_iA\to A$ a $\varrho_{i,j}$ je projekce $P_1\to e_iA$. 
      Všechna tato zobrazení následně sečteme.

      A zde je již výsledná funkce hledající korespondující homomorfismus 
      k prvku $x$ projektivního $A^{op}$-modulu $P^*$,
      dalšími parametry jsou $A$-modul $P$, $A^{op}$-modul $P^*$, $A$ jako 
      $A$-modul a prvek $1_A$.\\
     
           
      \begin{Verbatim}[frame=single,numbers=left] 
FromProjRepToHom := function(p, mP, mP_star, mA, 1_mA)
  local proj, proj2, i, j, k, Ae_i, e_iA, proj_p, coeffs,
    as_algebra_element, PP, PP_op, A, A_op, lambda,
    lambda_op, v, proj2_v, matrix, as_algebra_element2,
    lambda2, result, K, image, matrices;

  A    := RightActingAlgebra(mP);
  A_op := RightActingAlgebra(mP_star);
  K    := LeftActingDomain(A);

  # Moduly e_iA resp. Ae_i.
  PP    := IndecProjectiveModules(A);
  PP_op := IndecProjectiveModules(A_op);

  # Projekce P*->Ae_i resp. P->e_iA
  proj  := DirectSumProjections(mP_star);
  proj2 := DirectSumProjections(mP);

  as_algebra_element := ProjectiveBasisVectorGens(PP_op);
  as_algebra_element2 := ProjectiveBasisVectorGens(PP);

  result := [];

  # Přes všechny projekce Ae_i->P*.
  for j in [1..Length(proj)] do
    Ae_i := Range(proj[j]);

    # Zjistíme na které Ae_i projektujeme.
    for i in [1..Length(PP_op)] do
      if (IsomorphicModules(PP_op[i], Ae_i)) then
        break;
      fi;
    od;

    # Projektujeme p do Ae_i, spočteme korespondující prvek
    # algebry A_op a k němu opposite prvek algebry A.
    proj_p := ImageElm(proj[j], p);
    coeffs := Coefficients(Basis(Ae_i), proj_p);
    lambda := coeffs * as_algebra_element[i];
    lambda_op := OppositePathAlgebraElement(lambda);

    # Sestavíme matici odpovídající zobrazení P->A.
    matrix := [];
    for v in BasisVectors(Basis(mP)) do
      e_iA := Range(proj2[j]);

      # Projektujeme v do e_iA.
      proj2_v := ImageElm(proj2[j], v);

      # Spočteme odpovídající prvek algebry A.
      coeffs := Coefficients(Basis(e_iA), proj2_v);
      lambda2 := coeffs * as_algebra_element2[i];
      image := (1_mA ^ lambda_op) ^ lambda2;

      Add(matrix, Coefficients(Basis(mA), image));
    od;

    result := result + matrix;
  od;

  matrices := ExtractHomMatrices(result, mP, mA);

  return RightModuleHomOverAlgebra(mP, mA, matrices * One(K));
end;      
\end{Verbatim}
    
    
    
    
  \section{Modul $Tr(X)$}
  
    %\subsection*{Teorie}    
      Připomeňme následující definice:
      \begin{description}
        \item[(a)] $()^*:=Hom(-,A)$ kontravariantní funktor $mod(A)\rightarrow mod(A^{op})$
        \item[(b)]
          Použijeme funktor $()^*$ na následující krátkou exaktní posloupnost v $mod(A)$ \\\\
          \centerline{$0\rightarrow\Omega\xrightarrow{i} P_{0}\xrightarrow{t}X\rightarrow0$,\\}  \\\\     
          dostaneme exaktní posloupnost v $mod(A^{op})$\\\\
          \centerline{ \xymatrix{ 
            X^* 
              \ar@{->}[r]^{t^*} &P_0^* 
              \ar@{->}[r]^{s^*} &P_1^* 
              \ar@{->}[r]^{\^{t}}
            & Cok(s^*)  
              \ar@{->}[r]
            &0 & & }} \\\\     
          Definujme funktor $Tr(X):=Cok(s^*):\underline{mod}(A)\rightarrow \underline{mod}(A^{op})$.
          \item
      \end{description}

    Než se pustíme do funktoru $Tr$, tak
       budeme nejprve potřebovat spočíst moduly $P_0^*$ a $P_1^*$. 
      Z \hyperref[dualita-p-hom]{Části \ref*{dualita-p-hom}}  již víme, že pro $i=1,2$ existují 
      $m,n_1, n_2, \ldots, n_m \in \mathbb N$ taková, že: 
          \begin{eqnarray} 
             P_i^* &\simeq&  \bigoplus_{i=1}^m\bigoplus_{j=1}^{n_i}Ae_i \nonumber 
             \\
             P_i &\simeq&  \bigoplus_{i=1}^m\bigoplus_{j=1}^{n_i}e_iA. \nonumber 
         \end{eqnarray}
       Výpočet je tedy jednoduchý:
      \begin{Verbatim}[frame=single,numbers=left]
multiplicities0 := SuppProjModule(mP0)[3];
mP0_star := [];
for i in [1..Length(multiplicities0)] do
  for j in [1..multiplicities0[i]] do
    Add(mP0_star, PP_op[i]);
  od;
od;
mP0_star := DirectSumOfModules(mP0_star);

multiplicities1 := SuppProjModule(mP1)[3];
mP1_star := [];
for i in [1..Length(multiplicities1)] do
  for j in [1..multiplicities1[i]] do
    Add(mP1_star, PP_op[i]);
  od;
od;
mP1_star := DirectSumOfModules(mP1_star);        
      \end{Verbatim}     
      
      Dále budeme potřebovat spočítat prvek $1_A$ jakožto prvek  pravého $A$-modulu 
      $A$.   Toulec $Q$ je konečný, jednotka $1_A$ dle 
   \hyperref[quiver-kq-lemma]{Lemma \ref*{quiver-kq-lemma}} 
   a \hyperref[mega-veta-toulec-dsl]{Důsledku \ref*{mega-veta-toulec-dsl}} existuje a 
   navíc je tvaru: \\
   \centerline{$\sum_{i\in Q_0}e_i$}\\\\
      Budeme počítat přes všechny nerozložitelné 
      projektivní moduly $e_iA$, které získáme z rozkladu modulu $A$ na nerozložitelné direktní sčítance.     
      Projektivní modul $e_iA$ je jako vektorový prostor
      dle věty \hyperref[lem-proj-prezentace]{Věty \ref*{lem-proj-prezentace}}
      generován množinou všech cest z vrcholu $i\in Q_0$. První dimenze je dána 
      triviální cestou $(i||i)$ odpovídající idempotentu $e_i$ algebry $A$. 
      Prvek modulu $e_iA$ korespondující s idempotentem $e_i$ tedy získáme 
      takto:
      \begin{Verbatim}[frame=single,numbers=left]
e_i := MinimalGeneratingSetOfModule(e_iA)[1];         
      \end{Verbatim}            
      A zde je již celý výpočet:
    
      \begin{Verbatim}[frame=single,numbers=left]
# Spočteme prvek 1_A reprezentace A  jakožto součet prvků
# e_i z reprezentací e_iA  vnořených do A.
1_mA := Zero(mA);
incl := DirectSumInclusions(mA);
for i in [1..Length(incl)] do
  e_iA := Source(incl[i]);
  e_i := MinimalGeneratingSetOfModule(e_iA)[1];
  1_mA := 1_mA + ImageElm(incl[i], e_i);
od;
      \end{Verbatim}

      
      Nyní spočteme $Tr(X)$. V \cite{QPA} obsažená funkce $TransposeOfModule$ pro výpočet $Tr(X)$ vrací pouze modul $Tr(X)$.
      My ale potřebujeme pracovat i s projekcí $\hat t$. 
      Proto vytvoříme funkci  $S\_Star$, která nám vrátí homomorfismus $s^*:P_0^*\to 
      P_1^*$, jehož je $Tr(X)$ kojádro.
      
      Postupovat budeme, tak že si každý prvek báze $P_0^*$ vyjádříme s pomocí 
      duality $P_0^*$ a $Hom_A(P_0,A)$ (\hyperref[dualita-p-hom]{Část \ref*{dualita-p-hom}}) 
      jako homomorfismus $P_0\to A$. Ten složíme s $s:P_1\to P_0$ na homomorfismus z 
      $Hom_A(P_1,A)$ a ten si s pomocí inverzní duality vyjádříme jako prvek 
      $P_1^*$. Z takto získaných obrazů báze $P_0^*$ již snadno spočteme výsledný 
      homomorfismus. Postup ilustuje následující diagram:
      \\\\\centerline{\xymatrix{
        P_0^* \ar@{.>}[r]^{s^*} \ar[d]^\simeq & P_1^* \\
        Hom_A(P_0,A) \ar[r]^{(-\circ s)_A} & Hom_A(P_1,A) \ar[u]_\simeq
      }}\\\\
      
      \begin{Verbatim}[frame=single,numbers=left]
S_Star := function(s, mP0, mP1, mP0_star, mP1_star, mA, 1_mA)
  local v, f, fs, image, matrix, matrices, A, K;

  A := RightActingAlgebra(mP1);
  K := LeftActingDomain(A);

  matrix := [];
  for v in BasisVectors(Basis(mP0_star)) do
    f  := FromProjRepToHom(v, mP0, mP0_star, mA, 1_mA);
    fs := s * f;
    image := FromHomToProjRep(fs, mP1_star);
    Add(matrix, Coefficients(Basis(mP1_star), image));
  od;

  matrices := ExtractHomMatrices(matrix, mP0_star, mP1_star);

  return RightModuleHomOverAlgebra(mP0_star, mP1_star, matrices);
end;        
      \end{Verbatim}
      
      
      Nyní již můžeme spočíst $s^*$, $\hat t$, $Tr(X)$ i $DTr(X)$. Navíc si spočteme 
      bázi $K$-vektorového prostoru $Hom_A(\Omega, DTr(X))$, kterou budeme 
       později potřebovat při výpočtu tenzorového součinu $Tr(X)\otimes_A \Omega$. 
       Tuto bázi spočteme jako pole složené z jednotlivých homomorfismů.
    
      \begin{Verbatim}[frame=single,numbers=left]
s_star := S_Star(s, mP0, mP1, mP0_star, mP1_star, mA, 1_mA);
t_hat  := CoKernelProjection(s_star);
mTrX   := Range(t_hat);
mDTrX  := DualOfModule(mTrX);
B_hom_mDTrX_mOmega := HomOverAlgebra(mOmega, mDTrX);
      \end{Verbatim}      
                  
  \section{Izomorfismus $P_1\oplus P_1' \simeq A^n$}\label{algoritmus-P1P1sA-izo}
  
    %\subsection*{Teorie}    
      \paragraph{ } Máme následující dva izomorfismy:  \\\\
      \centerline{$A^n \simeq (e_1A\oplus\ldots\oplus e_mA)^n$} \\\\
      \centerline{$P_1\oplus P_1' 
            \simeq 
            ((e_1A)^{n_1}\oplus \ldots \oplus(e_mA)^{n_m})
            \oplus
            ((e_1A)^{n-n_1}\oplus \ldots \oplus(e_mA)^{n-n_m})
          $} \\\\
      Pro každé $i=1,2,\ldots,m$ je počet modulů $e_iA$ v obou rozkladech stejný. Uvažujme následující 
      kononické projekce:      
      \begin{eqnarray}        
      \pi&:&P_1\oplus P_1' \rightarrow P \nonumber \\      
      \pi'&:&P_1\oplus P_1' \rightarrow P' \nonumber \\      
      \varrho_{i,j}&:&P_1 \rightarrow e_iA,  \quad i=1,2,\ldots,m, \quad j=1,2,\ldots,n_i \nonumber \\      
      \varrho'_{i,j}&:&P_1' \rightarrow e_iA, \quad i=1,2,\ldots,m, \quad j=1,2,\ldots,n-n_i \nonumber 
      \end{eqnarray}      
      a následující kononická vnoření:      
      \begin{eqnarray}             
        \nu_j&:&A \rightarrow A^n, \quad i=1,2,\ldots,m  \nonumber \\
        \alpha_{i,j}&:&e_iA \rightarrow A,    \quad i=1,2,\ldots,m, \quad j=1,2,\ldots,n   \nonumber 
      \end{eqnarray}  
      Ty nám znázorňuje následující diagram: \\\\
         \centerline{ \xymatrix{ 
            &  & e_1A 
              \ar@{->}[rd]^{\alpha{1,j}}  \\ 
            & P_1 
              \ar@{->}[ru]^{\varrho_{1,j}} 
              \ar@{->}[rd]_{\varrho_{n,j}}  
            & \ldots 
            & A 
              \ar@/^1pc/[rdd]^{\nu_j}  \\
            & & e_mA
              \ar@{->}[ru]_{\alpha{n,j}}  \\
            P_1\oplus P_1' 
              \ar@/^1pc/[ruu]^{\pi}
              \ar@/_1pc/[rdd]_{\pi'} 
            & & & & \quad A^n\quad \\
            & & e_1A 
              \ar@{->}[rd]^{\alpha{1,j}}  \\ 
            & P_1' 
              \ar@{->}[ru]^{\varrho'_{1,j}} 
              \ar@{->}[rd]_{\varrho'_{n,j}}
             & \ldots 
            & A 
              \ar@/_1pc/[ruu]_{\nu_j} \\
            & & e_mA
              \ar@{->}[ru]_{\alpha{n,j}}  \\
           }} \\\\\\
      Náš hledaný izomorfismus $\psi$ pak bude součtem \\\\
      \centerline{$
        \psi:=          
          \sum_{i=1,\ldots,m}\sum_{j=1,\ldots,n_i}\nu_j\alpha_{i,j}\varrho_{i,j}\pi
          +
          \sum_{i=1,\ldots,m}\sum_{j=1,\ldots,n-n_i}\nu_{n-j}\alpha_{i,n-j}\varrho'_{i,j}\pi'
      $,} \\\\
      což je součet $m\times n$ izomorfismů jednotlivých podmodulů $e_iA$ z rozkladů $A^n$ 
      a $P_1\oplus P_1'$ na direktní součty nerozložitelných podmodulů.

    %\subsection*{QPA kód}
      \begin{Verbatim}[frame=single,numbers=left]
IsomorphismProjAndAn := function(mP1_P2, mAn)
  local iso, used, i, j,
        proj_P1_P2, proj_P1_P2_fin, proj_PX,
        incl_An, incl_An_fin, incl_A,
        f, g;

  # Projekce P1_P2 ->> P1 resp. P1_P2 ->> P2 ...
  proj_P1_P2 := DirectSumProjections(mP1_P2);

  # ... složíme s projekcemi P1 ->> e_iA resp. P2 ->> e_iA.
  proj_P1_P2_fin := [];
  for f in proj_P1_P2 do
    proj_PX := DirectSumProjections( Range(f) );

    for g in proj_PX do
      Add(proj_P1_P2_fin, f * g);
    od;
  od;

  # Inkluze A -> A^n ...
  incl_An := DirectSumInclusions(mAn);

  # ... složíme s inkluzemi e_iA ->> A.
  incl_An_fin := [];
  for f in incl_An do
    incl_A := DirectSumInclusions( Source(f) );

    for g in incl_A do
      Add(incl_An_fin, g * f);
    od;
  od;

  # Nyní spárujeme spočtené projekce a inkluze
  # a vzniklá zobrazení P1_P2 -> A^n sečteme.
  iso := ZeroMapping(mP1_P2, mAn);
  used := [1..Length(incl_An_fin)];
  for g in proj_P1_P2_fin do
    for i in [1..Length(incl_An_fin)] do
      f := incl_An_fin[i];

      if (not used[i] = true) and Source(f) = Range(g) then
        iso := iso + g * f;
        used[i] := true;
        break;
      fi;
    od;
  od;

  return iso;
end;    
      \end{Verbatim}
      
     Kromě izomorfismu $\psi$ budeme potřebovat jeho inverzi. Tu můžeme v \cite{QPA} 
     jednoduše spočíst následujícím příkazem:
     
     \begin{Verbatim}[frame=single,numbers=left]
psi_inv := InverseOfIsomophism(psi);       
     \end{Verbatim}
     
     To je ale poměrně neefektivní, o mnoho rychleší bude v tomto jednoduchém si zavést funkci 
     $IsomorphismAnAndProj$, která bude počítat přesně opačně k naší funkci
     $IsomorphismProjAndAn$. 
     
     Namísto sčítání projekcí $P_1\oplus P_1'\to e_iA$ složených
     s vnořeními $e_iA\to A^n$ sečteme projekce $A^n\to e_iA$ složené s patřičnými vnořeními
     $e_iA\to P_1\oplus P_1'$. Vstup inverzní funkce bude tedy stejný jako u 
     funkce původní.
     
     \begin{Verbatim}[frame=single,numbers=left]
  IsomorphismAnAndProj := function(mP1_P2, mAn)
       
  ...
       
  return iso;
end;
     \end{Verbatim}
     
     Všimněme si, že naše funkce $IsomorphismAnAndProj$ zpracovává oba parametry 
     stejným způsobem. Platí tedy vztah
     \begin{Verbatim}[frame=single,numbers=left]
IsomorphismAnAndProj(nAn,mP1_P2)=IsomorphismProjAndAn(mAn,mP1_P2);
     \end{Verbatim}
     a obě tyto funkce jsou totožné, mající pouze jinak pojmenované proměnné. 
     Navíc není použitelná pouze pro dvojici s $A^n$,
     ale jediná podmínka nalezení izomorfismu je, že 
     oba parametry musí být direktní sumou 
     direktních sum nerozložitelných projektivních $A$-modulů $e_iA$ a 
     ty musejí být v obou rozkladech ve stejném počtu.
           
  \section{Izomorfismus $\varphi_{P_1, \Omega}}\label{vypocet-varphi}
  
    %\subsection*{Teorie}   
      \paragraph{ } Dle \hyperref[varphi-izomorfismus]{Věty \ref*{varphi-izomorfismus}} máme izomorfismus \\\\
      \centerline{$\varphi_{P_1,\Omega}:Hom_A(P_1,\Omega)\rightarrow Hom_A(P_1,A)\otimes_A \Omega$} 
      \\\\
      daný předpisem pro $h\in Hom_A(P_1,\Omega)$ následovně \\\\
       \centerline{$h\mapsto\sum_{i=1}^n \rho_i\psi\mu\otimes h\pi\psi^{-1}\nu_i(1_A)$,} 
       \\\\
       kde jednotlivá zobrazení jsou: \\
         
         $ \mu: P_1\rightarrow P_1\oplus P_1'$ \quad kanonická inkluze
         
         $ \psi: P_1\oplus P_1' \simeq A^n $ 
            \quad izomorfismus konstruovaný v 
            \hyperref[algoritmus-P1P1sA-izo]{Sekci \ref*{algoritmus-P1P1sA-izo}}
       
         $ \rho_i: A^n\rightarrow A $ \quad kanonická projekce
         
         $ \nu_i: A\rightarrow A^n $ \quad kanonická inkluze
         
         $ \pi: P_1\oplus P_1'\rightarrow P_1$ \quad kanonická projekce\\\\
      Spočteme jednotlivé homomorfismy nutné pro výpočet $\varphi_{P_1,\Omega}$:
      \begin{Verbatim}[frame=single,numbers=left]
# Levá strana tenzorového součinu
mu     := DirectSumInclusions(mP1_mP1s)[1];
psi    := IsomorphismProjAndAn(mP1_mP1s, mAn);
rho    := DirectSumProjections(mAn);

# Pravá strana tenzorového součinu
nu     := DirectSumInclusions(mAn);
psi_inv:= IsomorphismAnAndProj(mP1_mP1s, mAn);
pi     := DirectSumProjections(mP1_mP1s)[1];
      \end{Verbatim}      
      Zatím si ale ponecháme $\varphi_{P_1,\Omega}$ jako
      jednotlivé složky. Výsledné zobrazení spočteme později, jelikož zatím 
      neumíme spočíst tenzorový součin obou stran.

              
  \section{Dualita $DTr(X)$}\label{dualita-DTrX}
  
    %\subsection*{Teorie}   
      \paragraph{ } Nyní budeme řešit podobný problém jako v případě duality $P_1^*$ 
      (\hyperref[dualita-p-hom]{Sekce \ref*{dualita-p-hom}}). 
      Jak vyjádřit libovolný homomorfismus z $Hom_K(Tr(X), K)$ jako prvek 
      reprezentace $DTr(X)$?
      
      Připomeňme, že $Tr(X)\in mod(A^{op})$. 
      Pojmenujme vrcholy našeho toulce $Q=(Q_0, Q_1,s,t)$ čísly $1,2,\ldots,m$ (tedy $Q_0=\{1,\ldots,m\}$).
      Uvažujme o $DTr(X)$ nejprve jako o $Hom_K(Tr(X), 
      K)$. Ze vztahu modulů a reprezentací 
      (\hyperref[ekvivalence-rep-a-mod]{Věta \ref*{ekvivalence-rep-a-mod}}) 
      vypadá reprezentace $(DTr(X)_i,\varphi_\alpha)_{i\in Q_0,\alpha\in Q_1}$ toulce $Q$ korespondující s modulem $DTr(X)$ následovně:
      \begin{description}
      \item[(a)] Vektorový prostor $DTr(X)_i$ (pro $i\in Q_0$) je generovaný homomorfismy z množiny: \\\\ 
        \centerline{$DTr(X)e_i=\{fe_i=f(e_i^{op}\cdot-)| f\in DTr(X)\}$} \\\\
        Bází vektorového prostoru $DTr(X)_i$  jsou tedy homomorfismy z $DTr(X)$, 
        které jsou nenulové právě na $e_i^{op}Tr(X)\subseteq Tr(X)$.
      \item[(b)] Homomorfismus $f_\alpha$ (kde $\alpha\in Q_1$ je šipka $i \rightarrow j$, $i,j\in Q_0$) 
      je homomorfismus \\\\
      \centerline{$f_\alpha:DTr(X)_i\rightarrow DTr(X)_j$}\\\\
      dáný předpisem $f \mapsto f(e_i^{op}v^{op}e_j^{op}\cdot-)$.
      \end{description}
      
      
       
      \quad\\
      
      Nyní buď $(Tr(X)_i, \psi_\alpha)_{i\in Q_0,\alpha\in Q_1}$ reprezentace $Q$ korespondující s modulem 
      $Tr(X)$, $n_i:=dim_K(Tr(X)_i)$ a mějme kanonickou bázi \\\\
      \centerline{
        $B_{Tr(X)}=\{v_1^1,v_1^2,\ldots,v_1^{n_1},
                            v_2^1,v_2^2,\ldots,v_2^{n_2},
                            \ldots,
                            v_m^1,v_m^2,\ldots,v_m^{n_m}
        \}$,} \\\\ 
      kde $v_i^j$ je vektor s $1$ na $j$-té souřadnici vektorového prostoru $Tr(X)_i$
      a nulami všude jinde. Báze $B_{Tr(X)}$ je kanonická báze vektorového  prostoru  
      $(Tr(X)_i, \psi_\alpha)$. Pak máme duální bázi\\\\
      \centerline{$
        B_{DTr(X)}=\{g_i:Tr(X)\rightarrow K|g_i^j(v_l^k)=\delta_{i,l}\delta_{j,k}
        $, $i,l\in Q_0
        $, $j=1,\ldots,n_i
        $, $k=1\ldots n_l\}
      $} \\\\ 
      reprezentace $DTr(X)$ o velikosti $\sum_{i\in Q_0}dim_K(Tr(X)_i)$, kde $\delta$ 
      značí Kroneckerovu deltu.
      
      Máme-li homomorfismus $f:Tr(X)\rightarrow  K$, pak jemu 
      korespondujícím elementem reprezentace $DTr(X)$ vzhledem k bázi $B_{DTr(X)}$ bude prvek \\\\
      \centerline{$(
      (f(v_1^1), f(v_1^2), \ldots, f(v_1^{n_1})), 
      (f(v_2^1), f(v_2^2), \ldots, f(v_2^{n_2})), 
      \ldots,
      (f(v_m^1), f(v_m^2), \ldots, f(v_m^{n_m}))
      )$.}

    \subsection*{Příklad}
        Nechť toulec $Q$, algebra $A$ a $A$-modul $X$ jsou dány následovně: \\\\
        \centerline{
          $Q$: \xymatrix{
            \circ^1 \ar@{->}[r]_a \ar@/^3pc/[rr]^c
              & \circ^2 \ar@{->}[r]_b
              & \circ^3
          }
          \rightaligned{
            ,\space\space\space\space $A=KQ$ \space\space,\space\space
             $X$: \xymatrix{
               K^2 
                   \ar@{->}[r]_{\left[\begin{smallmatrix}
	               1 & 0 & 0 \\
	               0 & 1 & 0 \\
                      \end{smallmatrix}\right]} 
                   \ar@/^3pc/[rr]^{\left[\begin{smallmatrix}
	               0 & 0 \\
	               1 & 0 \\
                      \end{smallmatrix}\right]}
                 & K^3 
                   \ar@{->}[r]_{\left[\begin{smallmatrix}
	               0 & 1 \\
	               1 & 0 \\
	               0 & 1 \\
                      \end{smallmatrix}\right]}   
                 & K^2
             }
           }
         }\\\\
         Pak $A^{op}$-modul $Tr(X)$ pak vypadá následovně \\\\
        \centerline{\xymatrix{
               K^3 
                   \ar@{->}[r]_{\left[\begin{smallmatrix}
	               1 & 0 & 0 \\
	               0 & 0 & 1 \\
                      \end{smallmatrix}\right]} 
                   \ar@/^3pc/[rr]^{\left[\begin{smallmatrix}
	               0 & 0 & 0 \\
	               0 & 1 & 0 \\
	               0 & 0 & 1 \\
                      \end{smallmatrix}\right]}
                 & K^2 
                   \ar@{->}[r]_{\left[\begin{smallmatrix}
	               1 & 0 \\
	               0 & 0 \\
	               0 & 1 \\
                      \end{smallmatrix}\right]}   
                 & K^3
         }}\\\\\\
         kde jsou parametry reprezentace a jednotlivých vektorových prostorů 
         $m=3$, $n_1=3$, $n_2=2$, $n_3=3$. Báze reprezentace $Tr(X)$ jako 
         vektorového prostoru je:
         \begin{eqnarray}
         B_{Tr(X)}=&\{&\nonumber\\  
           v_1^1&=&((1,0,0),(0,0),(0,0,0)),\nonumber\\
           v_1^2&=&((0,1,0),(0,0),(0,0,0)),\nonumber\\     
           v_1^3&=&((0,0,1),(0,0),(0,0,0)),\nonumber\\         
           v_2^1&=&((0,0,0),(1,0),(0,0,0)),\nonumber\\         
           v_2^2&=&((0,0,0),(0,1),(0,0,0)),\nonumber\\         
           v_3^1&=&((0,0,0),(0,0),(1,0,0)),\nonumber\\         
           v_3^2&=&((0,0,0),(0,0),(0,1,0)),\nonumber\\         
           v_3^3&=&((0,0,0),(0,0),(0,0,1)) \nonumber\\    
         \}\quad\quad\quad\nonumber
         \end{eqnarray}
         Což nám dává duální bázi reprezentace $DTr(X)$:
         \\\\      
         \centerline{$B_{DTr(X)}=\{g_i^j:Tr(X)\rightarrow K\,|\, i\in\{1,2,3\},\, j\in{1,\ldots,n_i},\, g_i^j(v_l^k)=\delta_{i,l}\delta_{j,k} 
         \}$}
         \\
         
         Homomorfismu $f:Tr(X)\rightarrow K$ z $A$-modulu $DTr(X)$, pak v 
         $DTr(X)$ jakožto reprezentaci $Q$ odpovídá prvek (zapsaný vzhledem k bázi $B_{DTr(X)}$):\\
         \begin{eqnarray}
            (\underbrace{ \, \left(f(v_1^1),f(v_1^2),f(v_1^3),\right)}_{\in DTr(X)_1},
            \underbrace{\left(f(v_2^1),f(v_2^2),\right)}_{\in DTr(X)_2},
            \underbrace{\left(f(v_3^1),f(v_3^2),f(v_3^3) \right) \, \right)}_{\in DTr(X)_3})
          \nonumber
         \end{eqnarray}
          
  \section{Tenzorový součin $Tr(X)\otimes_A \Omega$}
  
    %\subsection*{Teorie}    
      \paragraph{ } Dle \hyperref[thm-adjunkce]{Věty \ref*{thm-adjunkce}}
      máme následující izomorfismus \\\\
      \centerline{$Hom_K(M\otimes_A N,L)\simeq Hom_A(N, Hom_K(M,L))$}
      \\\\
      daný předpisem \\\\
      \centerline{$f \mapsto [n \mapsto f(-\otimes n)]$.}
      \\\\
      Ten nám pro $M=Tr(X)$, $N=\Omega$ a $L=K$ dává izomorfismus \\\\
      \centerline{$Hom_K(Tr(X)\otimes_A \Omega,K)\simeq Hom_A(\Omega, 
      Hom_K(Tr(X),K))$,}
      \\\\      
      který můžeme dále upravit z defnice funktoru $D=Hom_K(-,K)$ na  \\\\
      \centerline{$D(Tr(X)\otimes_A \Omega)\simeq Hom_A(\Omega,DTr(X))$.}
      \\\\    
      Aplikujeme-li funktor $D$ ještě jednou, dostaneme izomorfismus:
      \begin{eqnarray}
        Tr(X)\otimes_A\Omega &\simeq& DHom_A(\Omega,DTr(X))  \nonumber \\
        t \otimes\omega &\mapsto& \left[f\mapsto f(\omega)(t)\right] \nonumber
      \end{eqnarray}
      Ten využijeme k našemu výpočtu.
            
      Tenzorový součin $Tr(X)\otimes_A \Omega$ je tedy izomorfní jako $K$-vektorový prostor s 
      $DHom_A(\Omega,DTr(X))$, s nímž budeme v dalších částech pracovat. Otázkou ale 
      zůstává, jak libovolné dvojici $(t,\omega)\in Tr(X)\times\Omega$ přiřadit $K$-homomorfismus 
      \begin{eqnarray}
        f &\in& DHom_A(\Omega,DTr(X)) \nonumber \\
        f &:& Hom_A(\Omega,DTr(X))\rightarrow K \nonumber
      \end{eqnarray}
      odpovídající 
      prvku tenzorového součinu $t\otimes\omega\in Tr(X)\otimes_A\Omega$.      
      V \cite{QPA} spočteme poměrně jednoduše bázi\\\\
      \centerline{$B_{Hom_A(\Omega,DTr(X))}=\{f_1,f_2,\ldots,f_n\}$,} \\\\
      spočíst jednoduše $DHom_A(\Omega,DTr(X))$ už ale nejde. 
      Proto použijeme následující algoritmus, počítající na základě výše 
      uvedeného izomorfismu:
      
      \paragraph{Vstup:} Báze $B_{Hom_A(\Omega,DTr(X))}=\{f_1,f_2,\ldots,f_n\}$ a  
      dvojice $(t,\omega)\in Tr(X)\times\Omega$.
      \paragraph{Výstup:} Vektor  K-homomorfismu 
      $Hom_A(\Omega,DTr(X))\rightarrow K$ (vzhledem k bázi  
      $B_{Hom_A(\Omega,DTr(X))}$) odpovídající 
      prvku tenzorového součinu $t\otimes\omega$.
      \paragraph{Průběh:}
      \begin{description}
        \item[(1)]Pro každý $f_i\in B_{Hom_A(\Omega,DTr(X))}$:  
          \begin{description} 
             \item[(a)] Dosadíme $\omega$, čímž dostaneme $f_i(\omega)\in DTr(X)$ 
             jakožto prvek reprezentace.
             \item[(b)] Dle \hyperref[dualita-DTrX]{sekce \ref*{dualita-DTrX}} 
             spočteme obrazy báze $B_{Tr(X)}$ při zobrazení $f_i(\omega)$. 
             \item[(c)] Spočteme koeficienty prvku $t\in Tr(X)$ vzhledem bázi 
             $B_{Tr(X)}$ a na základě obrazů bázových vektorů spočtených v 
             předchozím bodě spočteme i $f_i(\omega)(t)\in K$. 
          \end{description}
        \item[(2)] Hledaný vektor je $(f_1(\omega)(t),f_2(\omega)(t),\ldots,f_n(\omega)(t))\in K^n$. 
      \end{description} 
                 
      Algoritmus je zde zapsaný oběcně, v našem případě je: 
      \begin{eqnarray}
        mM &:=& Tr(X)   \nonumber \\
        mN &:=&  \Omega \nonumber \\
        m &:=& t   \nonumber \\
        n &:=& \omega   \nonumber \\
        mDM &:=& DTr(X) \nonumber \\
        B\_hom\_mN\_mDM) &:=& Hom_A(\Omega, DTr(X)) \nonumber
      \end{eqnarray}
      
    %\subsection*{QPA kód}
      \begin{Verbatim}[frame=single,numbers=left]
 TensorProductMap := function(m, n, mM, mN, mDM, B_hom_mN_mDM)
  local coeffs_m, coeffs_f_i_n, i, B_hom_images, f_i_n;

  coeffs_m := Coefficients(Basis(mM), m);
  B_hom_images := [];

  f_i_n := List(B_hom_mN_mDM, f_i -> ImageElm(f_i, n));

  for i in [1..Length(f_i_n)] do
    coeffs_f_i_n := Coefficients(Basis(mDM), f_i_n[i]);

    B_hom_images[i] := coeffs_m * coeffs_f_i_n;
  od;

  return B_hom_images;
end;
      \end{Verbatim}

  \section{Hlavní bázový prvek $\phi_\Omega(\omega)$}
  
    %\subsection*{Teorie}    
      \paragraph{ } V této části spočteme prvek $\psi_\Omega(\omega)\in Tr(X)\otimes_A \Omega$, 
      který bude hlavním prvkem báze tohoto tenzorového součinu, kterou budeme vzápětí konstruovat. 
      V \hyperref[vypocet-varphi]{Části \ref*{vypocet-varphi}} jsme 
      spočetli jdenotlivé složky izomorfismu \\\\
      \centerline{$\varphi_{P_1,\Omega}:Hom_A(P_1,\Omega)\rightarrow Hom_A(P_1,A)\otimes_A \Omega$} 
      \\\\
      daného předpisem \\\\
      \centerline{$h\mapsto\sum_{i=1}^n \rho_i\psi\mu\otimes h\pi\psi^{-1}\nu_i(1_A)$.} 
      \\\\      
      Připomeňme ještě komutativní diagram s exaktními řádky z
      \hyperref[alg-sigma]{Části \ref*{alg-sigma}}:
      \\\\
        \centerline{ \xymatrix{
          0  \ar@{->}[r] 
          &Hom(X,\Omega) \ar@{->}[r]^{(-\circ t)_\Omega} 
          &Hom(P_0,\Omega) \ar@{->}[r]^{(-\circ s)_\Omega}  \ar@{->}[d]^{\varphi_{P_0, \Omega}}
          &Hom(P_1,\Omega)  \ar@{->}[d]^{\varphi_{P_1, \Omega}} \\
          &
          &Hom(P_0,A)\otimes_A\Omega \ar@{->}[r]^{(-\circ s)_A\otimes1_\Omega}
          &Hom(P_1,A)\otimes_A\Omega \ar@{->}[r]^{\hat{t}\otimes1_\Omega}
          &Tr(X)\otimes_A\Omega \ar@{->}[r]
          &0
        }}\\\\\\
        Definujme homomorfismus \\\\        
        \centerline{$\phi_\Omega:=[\hat{t}\otimes1_\Omega]\varphi_{P_1,\Omega}:\, Hom_A(P_1,\Omega)\rightarrow Tr(X)$}\\\\
        a zobrazme jím homomorfismus
        $\omega\in Hom_A(P_1,\Omega)$
        projektivního pokrytí modulu $\Omega$. Dostaneme: 
         \begin{eqnarray}
           \phi_\Omega(w)
           &=& [\hat{t}\otimes1_\Omega]\varphi_{P_1, \Omega}(w)  \nonumber \\
           &=& [\hat{t}\otimes1_\Omega]\sum_{i=1}^n \rho_i\psi\mu\otimes w\pi\psi^{-1}\nu_i(1_A)  \nonumber \\
           &=& \sum_{i=1}^n \hat{t}(\rho_i\psi\mu)\otimes w\pi\psi^{-1}\nu_i(1_A) \nonumber 
         \end{eqnarray}   \\

    %\subsection*{QPA kód}   
      \begin{Verbatim}[frame=single,numbers=left]
mu_psi_rho := mu * psi * rho;
omega_pi_psi_inv_nu := nu * psi_inv * pi * omega;

# omega * pi * psi^(-1) * nu(1_A).
omega_pi_psi_inv_nu_1_A := List(omega_pi_psi_inv_nu, 
    f -> ImageElm(f, 1_mA)
  );

# Spočteme zobrazení mu_psi_rho jako elementy modulu P1*.
mu_psi_rho_el := List(mu_psi_rho, 
    f -> FromHomToProjRep(f, mP1_star)
  );
t_mu_psi_rho_el := List(mu_psi_rho_el, el -> ImageElm(t_hat, el));

# Spočteme náš hlavní bázovy prvek.
psi_omega := [];
for i in [1..Length(t_mu_psi_rho_el)] do
  m := t_mu_psi_rho_el[i];
  n := omega_pi_psi_inv_nu_1_A[i];

  Add(psi_omega, TensorProductMap(
      m, n, mTrX, mOmega, mDTrX, B_hom_mDTrX_mOmega)
    );
od;
psi_omega := Sum(psi_omega);
      \end{Verbatim}
       
  \section{Báze $Tr(X)\otimes_A \Omega$}\label{alg-baze}
  
    %\subsection*{Teorie}    
      \paragraph{ }Naším cílem je nyní nalézt prvky $g_1,g_2,\ldots,g_n\in Tr(X)\otimes_A\Omega$
      tak, aby\\\\
      \centerline{$B_{Tr(X)\otimes_A\Omega}=\{\phi_\Omega(w),g_1,g_2,\ldots,g_n\}$}\\\\ 
      byla bází $Tr(X)\otimes_A\Omega$. Nechť $B_{\Omega}$ je báze $\Omega$ 
      a $B_{Tr(X)}$ je báze $Tr(X)$. Spočteme množinu  \\
      \centerline{$\{t\otimes\omega|t\in B_{Tr(X)},\omega\in B_{\Omega}\}$.} \\\\
      Ta nám zaručeně generuje
      celý K-vektorový prostor  $Tr(X)\otimes_A\Omega$. Spočteme tedy $Tr(X)\otimes_A\Omega$ 
      jako K-vektorový prostor a vezmeme jeho kakonickou bázi 
      $B_c=\{g_1,g_2,\ldots,g_{n+1}\}$. Víme, že dle \hyperref[phi-omega-nenul]{úvahy v sekci 2.2}
      $\phi_\Omega(w)$ je nenulový vektor. Nechť je $i$-tá pozice vektoru $\phi_\Omega(w)$ 
      nenulová. Pak můžeme zvolit \\\\
      \centerline{
        $B_{Tr(X)\otimes_A\Omega}=\{\phi_\Omega(w),g_1,g_2,\ldots,g_{i-1},g_{i+1},\ldots,g_n\}$.
      }\\     

    %\subsection*{QPA kód}
      \begin{Verbatim}[frame=single,numbers=left]
# Pole obsahující na i-té pozici pole
# [t_1\otimes\omega_i, ... ,t_n\otimes\omega_i].
# To využijeme v následující části. Z důvodu
# efektivnosti si ho ale předpočítáme nyní.
mTrX_mOmega := [];

# Pole nenulových prvků t_i\otimes\omega_i.
B_mTrX_mOmega := [];

# Spočteme bázi tenzorového součinu TrX a Omega.
B_mOmega := BasisVectors(Basis(mOmega));
B_mTrX := BasisVectors(Basis(mTrX));
for n in B_mOmega do
  n_images := [];

  for m in B_mTrX do
    m_n := TensorProductMap(
        m, n, mTrX, mOmega, mDTrX, B_hom_mDTrX_mOmega
      );

    Add(n_images, m_n);
    if not Sum(m_n) = 0 and not Sum(m_n) = Zero(K) then
      Add(B_mTrX_mOmega, m_n);
    fi;
  od;

  Add(mTrX_mOmega, n_images);
od;

# Jeden bázový prvek nahradíme význačným prvkem z přechozí sekce.
V := VectorSpace(K, B_mTrX_mOmega);
B_V := CanonicalBasis(V);
B_V_new := [psi_omega];
added := false;
for i in [1..Length(B_V)] do
  if not psi_omega[i] = Zero(K) and not added then
    added := true;
  else
    Add(B_V_new, B_V[i]);
  fi;
od;
      \end{Verbatim}
          
  \section{Homomorfismus $\xi:\Omega\rightarrow DTr(X)$} 
  
    %\subsection*{Teorie}    
      \paragraph{ } Naším cílem je spočíst matice homomorfismu \\\\
      \centerline{$\xi: \Omega\rightarrow DTr(X)$} \\
      daného předpisem \\ \\
      \centerline{$\omega\mapsto [t\mapsto$(první koeficient
       $t\otimes\omega$ vzhledem k bázi $B_{Tr(X)\otimes_A \Omega})]$.}\\\\
       Při \hyperref[alg-baze]{výpočtu báze \ref*{alg-baze}} $Tr(X)\otimes_A \Omega$ 
       jsme  již spočetli pole $mTrX\_mOmega$ tvaru \\\\
       \centerline{$mTrX\_mOmega[i]=[t_1\otimes\omega_i,\ldots,t_n\otimes\omega_i]$,} \\\\ 
       kde
       $\{t_1,t_2,\ldots,t_n\}$ je báze $Tr(X)$ a $\{\omega_1,\omega_2,\ldots,\omega_m\}$ 
       je báze $\Omega$. Každý prvek $t_i\otimes\omega_i$ si vyjádříme vzhledem bázi 
       $Tr(X)\otimes_A \Omega$ a první souřadnici si uložíme do pole $images$ na pozici 
       $images[i][j]$.

      \begin{Verbatim}[frame=single,numbers=left]
images := [];
for i in [1..Length(B_mOmega)] do
  Add(images, List(mTrX_mOmega[i], 
      v -> Coefficients(Basis(V, B_V_new), v)[1] 
    ));
od;
      \end{Verbatim}             
A zkonstruujeme hledaný homomorfismus:       
       
    %\subsection*{QPA kód}
      \begin{Verbatim}[frame=single,numbers=left]
matrices := ExtractHomMatrices(images, mOmega, mDTrX);
xi := RightModuleHomOverAlgebra(mOmega, mDTrX, matrices * One(K));
      \end{Verbatim}
    
  \section{Hledaný generátor $E$}
  
    %\subsection*{Teorie}   
        E spočteme jako pushout homomorfismů $i$ a $\xi$. \\\\
        \centerline{ \xymatrix{
          \Omega \ar[r]^\xi \ar[d]_i & DTr(X) \ar@{..}[d] \\ 
          P_0 \ar@{..}[r] & E
        } }
        
    %\subsection*{QPA kód}
      \begin{Verbatim}[frame=single,numbers=left]
mE := PushOut(kernel_inc, xi);
      \end{Verbatim}
      
      Stejně jako u funkce $AlmostSplitSequence$ vrátíme pole obsahující na 
      prvním místě  homomorfismus $DTr(X)\to E$ a na druhém jeho kojádro, 
       tedy homomorfismus $E\to X'$, kde $X'$ nemusí být přímo $X$, 
      ale nějaký modul $X$ izomorfní. 
      
      \begin{Verbatim}[frame=single,numbers=left]
return [mE[1], CoKernelProjection(mE[1])];
      \end{Verbatim} 
      
      \begin{description}
        \item[Tím jsme hotovi.]
      \end{description}
      
    








  \addtocontents{toc}{\protect\setcounter{tocdepth}{0}} 

\chapter{Příklady použití a testování časové náročnosti} 

  \paragraph{ } Vyzkoušíme nyní algoritmus na několika příkladech. Poté provedeme srovnání 
  času potřebného pro výpočet s algoritmem $AlmostSplitSequence$ implementovaným v \cite{QPA}.
  Náš algoritmus pojmenujeme jednoduše $AlmostSplitSequence2$. 
  
  Pro testování 
  času necháme každý algoritmus provést výpočet několikrát následujícím 
  skriptem, kde ve funkci $GetModule$ znovu vytvoříme modul pro který počítáme. 
  Modul, jeho algebru i quiver je 
  třeba vždy vytvořit znova, aby GAP nemohl jeho generátor skoro štěpitelných 
  posloupností a další vlastnosti, které mohou zkreslit konečný čas, cacheovat (neboli pamatovat si předchozí 
  výpočty pro urychlení budoucích operací).
  \begin{Verbatim}[frame=single,numbers=left]
GetModule := function(i)
  local K, Q, KQ, A, matrices, mX;

  K := Rationals;
  Q := Quiver(3, [[1, 2, "a"], [2, 3, "b"],[1, 3, "c"]]);
  A := PathAlgebra(K,Q);
  matrices := [ ["a", [[1,0,0],[0,1,0]]], 
                ["b", [[0,1],[1,0],[0,1]]], 
                ["c", [[0,0],[1,0]]] ];
  mX := RightModuleOverPathAlgebra(A,matrices);

  return mX;
end;

TestPerformance := function(iter)
  local i, time;

  time := Runtime();
  for i in [1..iter] do
    AlmostSplitSequence( GetModule(i) );
  od;
  time := Float((Runtime() - time) / iter / 1000);
  Print("Execution time for AlmostSplitSequence: ", time, "\n");

  time := Runtime();
  for i in [1..iter] do
    AlmostSplitSequence2( GetModule(i) );
  od;
  time := Float((Runtime() - time) / iter / 1000);
  Print("Execution time for AlmostSplitSequence2: ", time, "\n");
end;

TestPerformance(100);
  \end{Verbatim} 
  
      \subsection*{Příklad 1}
        \paragraph{ } Nechť $K$ je těleso racionálních čísel a toulec $Q$, přípustný ideál $I$ a $A$-modul $X$, 
        kde algebra $A=KQ/I$, jsou dány následovně: \\\\
        \centerline{
          Toulec $Q$: \xymatrix{
            \circ^1 \ar@{->}[r]_a \ar@/^3pc/[rr]^c
              & \circ^2 \ar@{->}[r]_b
              & \circ^3
          }
          \rightaligned{
            \space\space\space\space $I=\emptyset$ \space\space\space\space
             A-modul $X$: \xymatrix{
               K^2 
                   \ar@{->}[r]_{\left[\begin{smallmatrix}
	               1 & 0 \\
	               0 & 1 \\
                      \end{smallmatrix}\right]} 
                   \ar@/^3pc/[rr]^{\left[\begin{smallmatrix}
	               1 \\
	               0 \\
                      \end{smallmatrix}\right]}
                 & K^2 
                   \ar@{->}[r]_{\left[\begin{smallmatrix}
	               0 \\
	               1 \\
                      \end{smallmatrix}\right]}   
                 & K^1
             }
           }
         }\\\\\\
         Výsledný generátor množiny skoro štěpitelných posloupností je: 
         \\\\
         \centerline{\xymatrix{
            DTr(X) \ar@{->}[ddddd]
                 & 
                 & K^4
                   \ar@{.>}[ddddd]_{\left[\begin{smallmatrix}
	               1 & 0 & 0 & 0 & 0 & 0\\
	               0 & 1 & 0 & 0 & 0 & 0\\
	               0 & 0 & 1 & 0 & 0 & 0 \\       
	               0 & 0 & 0 & 1 & 0 & 0 \\       
	               \end{smallmatrix}\right]}
                   \ar@{->}[r]_{\left[\begin{smallmatrix}
	               0 & 0 & 0\\
	               1 & 0 & 0\\
	               0 & 0 & 1\\   
	               0 & 1 & 0 \\    
                      \end{smallmatrix}\right]} 
                   \ar@/^3pc/[rr]^{\left[\begin{smallmatrix}
	               1 & 0 & 0 \\
	               0 & 0 & -1 \\
	               0 & 1 & 0 \\
	               0 & 0 & 0 \\
                      \end{smallmatrix}\right]}
                 & K^3
                   \ar@{.>}[ddddd]|-{\left[\begin{smallmatrix}
	               1 & 0 & 0 & 0 & 0\\
	               0 & 1 & 0 & 0 & 0\\
	               0 & 0 & 1 & 0 & 0\\       
	               \end{smallmatrix}\right]}
                   \ar@{->}[r]_{\left[\begin{smallmatrix}
	               1 & 0 & 0 \\
	               0 & 1 & 0 \\
	               0 & 0 & 1 \\
                      \end{smallmatrix}\right]}   
                 & K^3
                   \ar@{.>}[ddddd]^{\left[\begin{smallmatrix}
	               1 & 0 & 0 & 0\\
	               0 & 1 & 0 & 0\\
	               0 & 0 & 1 & 0\\              
	               \end{smallmatrix}\right]}
                 \\ \\ \\ \\ \\
            E  \ar@{->}[dddd]
                 & 
                 & K^6
                   \ar@{.>}[dddd]
                   \ar@{->}[r]_{\left[\begin{smallmatrix}
	               0 & 0 & 0 & 0 & 0\\
	               1 & 0 & 0 & 0 & 0\\
	               0 & 0 & 1 & 0 & 0\\
	               0 & 1 & 0 & 0 & 0\\
	               0 & 0 & 0 & 1 & 0\\
	               0 & 0 & 0 & 0 & 1\\                      
	               \end{smallmatrix}\right]} 
                   \ar@/^3pc/[rr]|-{\left[\begin{smallmatrix}
	               1 & 0 & 0 & 0 \\
	               0 & 0 & -1 & 0 \\
	               0 & 1 & 0 & 0 \\
	               0 & 0 & 0 & 0 \\
	               0 & 0 & 0 & 1 \\        
	               0 & 0 & 0 & 0 \\                   
	               \end{smallmatrix}\right]}
                 & K^5  
                   \ar@{.>}[dddd]
                   \ar@{->}[r]_{\left[\begin{smallmatrix}
	               1 & 0 & 0 & 0\\
	               0 & 1 & 0 & 0\\
	               0 & 0 & 1 & 0\\
	               1 & 0 & 0 & 0\\
	               0 & 0 & 0 & 1\\                      
                      \end{smallmatrix}\right]}   
                 & K^4
                   \ar@{.>}[dddd] 
                   \\ \\ \\ \\
            X
                 & 
                 & K^2 
                   \ar@{->}[r]_{\left[\begin{smallmatrix}
	               1 & 0 & 0 \\
	               0 & 1 & 0 \\
                      \end{smallmatrix}\right]} 
                   \ar@/_3pc/[rr]_{\left[\begin{smallmatrix}
	               0 & 0 \\
	               1 & 0 \\
                      \end{smallmatrix}\right]}
                 & K^3 
                   \ar@{->}[r]_{\left[\begin{smallmatrix}
	               0 & 1 \\
	               1 & 0 \\
	               0 & 1 \\
                      \end{smallmatrix}\right]}   
                 & K^2
         }} \\\\\\         
       \paragraph{ }Náš algoritmus byl v tomto poměrně jednoduchém případě pomalejší. Integrovaná 
       funkce $AlmostSplitSequence$ běžela v průměru ze 100 běhů 0,171s a naše 
       funkce $AlmostSplitSequence2$ průměrně 0.238s.
       
       \subsection*{Příklad 2}
        \paragraph{ }Otestujeme nyní algoritmus na mírně složitějším příkladě.
        Nechť $K$ je těleso racionálních čísel a toulec $Q$, přípustný ideál $I$ a $A$-modul $X$, kde algebra $A=KQ/I$, 
        jsou dány následovně: \\\\
        \centerline{
          $Q$: \xymatrix{
            \circ^1 
                \ar@/^1pc/[r]^a 
                \ar@/_1pc/[r]_b 
              & \circ^2 
                \ar@{->}[r]_d
                \ar@(lu,ru)[]^c
              & \circ^3
                \ar@/^3pc/[ll]^e
          }
          \rightaligned{
            \space\space\space\space $I=\{c^2,acd-bd,ea,eb\}$ \space\space\space\space
             $X$: \xymatrix{
            K^2 
                \ar@/^1pc/[r]^{\left[\begin{smallmatrix}
	               0 & 1 \\
	               1 & 1 \\
                      \end{smallmatrix}\right]} 
                \ar@/_1pc/[r]_{\left[\begin{smallmatrix}
	               1 & 0 \\
	               1 & 0 \\
                      \end{smallmatrix}\right]} 
              & K^2 
                \ar@{->}[r]_{\left[\begin{smallmatrix}
	               1 & 1 \\
	               0 & 1 \\
                      \end{smallmatrix}\right]}
                \ar@(lu,ru)[]^{\left[\begin{smallmatrix}
	               0 & 0 \\
	               1 & 0 \\
                      \end{smallmatrix}\right]}
              & K^2
                \ar@/^3pc/[ll]^{\left[\begin{smallmatrix}
	               0 & 0 \\
	               0 & 0 \\
                      \end{smallmatrix}\right]}             
             }
           }
         }\\\\
         Výsledný generátor množiny skoro štěpitelných posloupností je: 
         \\\\
         \centerline{\xymatrix{
            DTr(X) \ar@{->}[dd]
                & 
                & K^{10} 
                  \ar@/^1pc/[r]
                  \ar@/_1pc/[r]
                  \ar@{.>}[dd]
                & K^{6} 
                  \ar@{->}[r]
                  \ar@(lu,ru)[]
                  \ar@{.>}[dd]
                & K^{4}
                  \ar@/^3pc/[ll]   
                  \ar@{.>}[dd] 
                 \\ \\
            E  \ar@{->}[dd]
                & 
                & K^{12} 
                  \ar@/^1pc/[r]
                  \ar@/_1pc/[r]
                  \ar@{.>}[dd]
                & K^{8} 
                  \ar@{->}[r]                  
                  \ar@(lu,ru)[]
                  \ar@{.>}[dd]
                & K^{6}
                  \ar@/^3pc/[ll]   
                  \ar@{.>}[dd]
                 \\ \\
            X
                & 
                & K^2
                  \ar@/^1pc/[r]
                  \ar@/_1pc/[r]
                & K^2 
                  \ar@{->}[r]
                  \ar@(lu,ru)[]
                & K^2
                  \ar@/^3pc/[ll]   
         }} \\\\\\         \\\\
       \paragraph{ }Náš algoritmus již byl v tomto mírně složitějším případě rychlejší. Integrovaná 
       funkce $AlmostSplitSequence$ běžela v průměru ze 20 běhů 25s a naše 
       funkce $AlmostSplitSequence2$ průměrně 4s.
       \\\\
                      
       \subsection*{Příklad 3}
       
         \paragraph{ }        Nyní si ukážeme, jak je možné algoritmus 
         u algeber nad konečným tělesem, 
         které mají až na izomorfismus konečně mnoho nerozložitelných 
         reprezentací, využít k jejich výpočtu. Začneme s libovolným 
         neprojektivním a nerozložitelným modulem $X$. Spočteme jeho skoro 
         štepitelnou posloupnost \\\\
         \centerline{$0 \to DTr(X) \to E \to X \to 0$.} \\\\
         Poté modul $E$ rozložíme na nerozložitelné direktní sčítance a postup 
         opakujeme s každým z nich. Po konečném počtu kroků obdržíme 
         již pouze moduly izomorfní dříve nalezeným. Užitý postup nebudeme 
         dokazovat ani dopodrobna rozebírat, jde jen o základní ilustraci 
         využití zkonstruovaného algoritmu.
                  
         Nechť $K=GF(13)$ (Galois Field), $K$-algebra $A=KQ$, kde toulec $Q$ je dán následovně:\\\\
         \centerline{
         \xymatrix{
            & \circ^2  \\
            & \circ^1 \ar[rd] \ar[ld] \ar[u]\\
            \circ^3 & & \circ^4
          }}\\\\\\
          Začněme s neprojektivním jednoduchým nerozložitelným modulem $S(1)$:\\\\
         \centerline{
         \xymatrix{
           & 0  \\
            & K \ar[rd] \ar[ld] \ar[u] \\
            0 & & 0
          }}\\\\\\
          Ten i všechny ostatní moduly budeme pro jednoduchost zapisovat následovně s vynecháním šipek: \\\\
          \centerline{\begin{matrix}
               & 0       \\
               & 1 &    \\
            0 &    & 0   
          \end{matrix}} \\\\\\
          Máme jeho skoro štěpitelnou posloupnost\\\\
         \centerline{
         \xymatrix{
            DTr(X)={\begin{matrix} & 1 \\ & 2 & \\ 1 &  & 1\end{matrix}} \ar[rr]
            & & E_1= {\begin{matrix} & 1 \\ & 3 & \\ 1 &  & 1\end{matrix}} \ar[rr]
            & & S(0)={\begin{matrix} & 0 \\ & 1 & \\ 0 &  & 0\end{matrix}}
          },}\\\\\\
          kde $E_1$ rozložíme na 3 nerozložitelné direktní sčítance $N_1$, $N_2$, $N_3$ 
          a dostaneme diagram:\\\\
         \centerline{
         \xymatrix{
            & &  N_1={\begin{matrix} & 0 \\ & 1 & \\ 1 &  & 0\end{matrix}} \ar[rrd] \\
            DTr(X)={\begin{matrix} & 1 \\ & 2 & \\ 1 &  & 1\end{matrix}} \ar[rr] \ar[rru] \ar[rrd]
            & &  N_2={\begin{matrix} & 0 \\ & 1 & \\ 0 &  & 1\end{matrix}} \ar[rr]
            & & S(1)={\begin{matrix} & 0 \\ & 1 & \\ 0 &  & 0\end{matrix}} \\
            & &  N_3={\begin{matrix} & 1 \\ & 1 & \\ 0 &  & 0\end{matrix}} \ar[rru]
          }}\\\\\\
          Dále postup zopakujeme pro $DTr(X)$ a dostaneme diagram:\\\\
         \centerline{
         \xymatrix{
            & N_4={\begin{matrix} & 1 \\ & 1 & \\ 0 &  & 1\end{matrix}} \ar[rd]
              & &  {\begin{matrix} & 0 \\ & 1 & \\ 1 &  & 0\end{matrix}} \ar[rd] \\
            P(1)={\begin{matrix} & 1 \\ & 1 & \\ 1 &  & 1\end{matrix}} \ar[r] \ar[ru] \ar[rd]
              & N_5={\begin{matrix} & 1 \\ & 1 & \\ 1 &  & 0\end{matrix}} \ar[r]
              & {\begin{matrix} & 1 \\ & 2 & \\ 1 &  & 1\end{matrix}} \ar[r] \ar[ru] \ar[rd]
              & {\begin{matrix} & 0 \\ & 1 & \\ 0 &  & 1\end{matrix}} \ar[r]
              & {\begin{matrix} & 0 \\ & 1 & \\ 0 &  & 0\end{matrix}} \\
            & N_6={\begin{matrix} & 0 \\ & 1 & \\ 1 &  & 1\end{matrix}} \ar[ru]
              & &  {\begin{matrix} & 1 \\ & 1 & \\ 0 &  & 0\end{matrix}} \ar[ru]
          }}\\\\\\
          Pokud nyní spočteme skoro štěpitelné posloupnosti pro moduly $N_4$, 
          $N_5$ a $N_6$, tak všechny budou procházet projektivním modulem $P(1)$ 
          a začínat v jednom ze tří zbylých nerozložitelných projektivních 
          modulů. Tím jsme hotovi a náš diagram je kompletní:\\\\
         \centerline{
         \xymatrix{
            {\begin{matrix} & 0 \\ & 0 & \\ 1 &  & 0\end{matrix}} \ar[rd]
              & & {\begin{matrix} & 1 \\ & 1 & \\ 0 &  & 1\end{matrix}} \ar[rd]
              & &  {\begin{matrix} & 0 \\ & 1 & \\ 1 &  & 0\end{matrix}} \ar[rd] \\
            {\begin{matrix} & 0 \\ & 0 & \\ 0 &  & 1\end{matrix}} \ar[r]
              & {\begin{matrix} & 1 \\ & 1 & \\ 1 &  & 1\end{matrix}} \ar[r] \ar[ru] \ar[rd]
              & {\begin{matrix} & 1 \\ & 1 & \\ 1 &  & 0\end{matrix}} \ar[r]
              & {\begin{matrix} & 1 \\ & 2 & \\ 1 &  & 1\end{matrix}} \ar[r] \ar[ru] \ar[rd]
              & {\begin{matrix} & 0 \\ & 1 & \\ 0 &  & 1\end{matrix}} \ar[r]
              & {\begin{matrix} & 0 \\ & 1 & \\ 0 &  & 0\end{matrix}} \\
           {\begin{matrix} & 1 \\ & 0 & \\ 0 &  & 0\end{matrix}} \ar[ru]
              & & {\begin{matrix} & 0 \\ & 1 & \\ 1 &  & 1\end{matrix}} \ar[ru]
              & &  {\begin{matrix} & 1 \\ & 1 & \\ 0 &  & 0\end{matrix}} \ar[ru]
          }}\\\\\\
         Při našem pojmenování \\\\
         \centerline{
         \xymatrix{
            P(3) \ar[rd]
              & & N_4 \ar[rd]
              & & N_1 \ar[rd] \\
            P(4) \ar[r]
              &P(1) \ar[r] \ar[ru] \ar[rd]
              & N_5 \ar[r]
              & DTr(S(1)) \ar[r] \ar[ru] \ar[rd]
              & N_2 \ar[r]
              & S(1) \\
           P(2) \ar[ru]
              & & N_6 \ar[ru]
              & & N_3 \ar[ru]
          }}\\\\\\
          máme následující posloupnosti skoro štěpitelné posloupnosti \\\\
          \centerline{$   0\to P(3)\to P(1)\to N_4\to 0     $}
          \centerline{$   0\to P(4)\to P(1)\to N_5\to 0     $}
          \centerline{$   0\to P(2)\to P(1)\to N_6\to 0     $}
          \centerline{$   0\to N_4\to DTr(S(1))\to N_1\to 0     $}
          \centerline{$   0\to N_5\to DTr(S(1))\to N_2\to 0     $}
          \centerline{$   0\to N_6\to DTr(S(1))\to N_3\to 0     $}
          \centerline{$   0\to DTr(S(1))\to N_1\oplus N_2 \oplus N_3\to S(1)\to 0     $}
          \centerline{$   0\to P(1)\to N_4\oplus N_5\oplus N_6\to DTr(S(1)) \to 0     $}
          \\\\
          a navíc platí, že
          \begin{eqnarray}
            P(4) &=& S(4) \, = \, DTr(N_5)\nonumber \\
            P(3) &=& S(3) \, = \, DTr(N_6) \nonumber \\
            P(2) &=& S(2) \, = \, DTr(N_4) \nonumber \\
            P(0) &=& DTrDTr(S(1)) \nonumber \\
            N_4 &=& DTr(N_3) \nonumber \\
            N_5 &=& DTr(N_2) \nonumber \\
            N_6 &=& DTr(N_1). \nonumber
          \end{eqnarray}
          
          Měli jsme tedy více možností, jak při výpočtu postupovat a například 
          modul $N_6$ jsme mohli spočíst jako první modul skoro štěpitelné 
          posloupnosti končící v modulu $N_3$.
          
       \subsection*{Příklad 4}
       
         \paragraph{ } Zkusme nyní na srovnat rychlost obou algoritmů ve vztahu k počtu vrcholů a hran.
                   
          Nejprve budeme zvyšovat počet vrcholů a hran současně. 
          Opět položme $K=GF(13)$ (Galois Field) a $K$-algebru $A=KQ$, kde toulci $Q$ budeme 
          postupně navyšovat počet vrcholů od 2 do 20:\\\\
         \centerline{
         \xymatrix{
            \circ^2  \\
            \circ^1 \ar[u]
          }\\\\\\\\
         \xymatrix{
            \circ^2  \\
            \circ^1 \ar[d] \ar[u]\\
            \circ^3
          }\\\\
         \xymatrix{
            & \circ^2  \\
            & \circ^1 \ar[rd] \ar[ld] \ar[u]\\
            \circ^3 & & \circ^4
          }\\
         \xymatrix{
            \circ^2 & & \circ^3  \\
            & \circ^1 \ar[rd] \ar[ld] \ar[lu]  \ar[ru]\\
            \circ^ & & \circ^5
          }\\\\
         \xymatrix{
            \\
            \cdots
          }\\
          }\\\\
          
          Jako modul $X$, jehož skoro štěpitelnou posloupnost budeme hledat, zvolíme 
          neprojektivní, nerozložitelný a jednoduchý modul $S(1)$. Tedy modul, který 
          vrcholu $1$ přiřazuje vektorový prostor $K$, ostatním vrcholům nulový 
          vektorový prostor a šipkám nulová zobrazení.
          
          Nejprve se podíváme na rychlost v případě nízkého početu vrcholů toulce $Q$ a to $1,2,\ldots,7$:
        \\\\
\centerline{\begin{tikzpicture}
\begin{axis}[
xlabel={Počet vrcholů $Q$},
ylabel={Čas výpočtu [ms]},
grid=major, 
height=12cm,
width=12cm, ]
\addplot coordinates {
(2,33)
(3,62)
(4,81)
(5,100)
(6,176)
(7,302)
};
\addlegendentry{QPA algoritmus}
\addplot coordinates {
(2,76)
(3,123)
(4,198)
(5,285)
(6,412)
(7,611)
};
\addlegendentry{Náš algoritmus} 
\end{axis}
\end{tikzpicture}}
\paragraph{ }Jak vidíme, pro nízký počet vrcholů je náš algoritmus méně efektivní. Zhruba od 
velikosti toulce $|Q_0|=10$ ale začne původní algoritmus zaostávat  a to exponenciální 
rychlostí, jak je vidět na následujícím grafu:\\\\\\       
\centerline{\begin{tikzpicture}
\begin{axis}[
xlabel={Počet vrcholů $Q$},
ylabel={Čas výpočtu [ms]},
grid=major, 
height=12cm,
width=12cm, ]
\addplot coordinates {
(2,33)
(3,62)
(4,81)
(5,100)
(6,176)
(7,302)
(8,528)
(9,881)
(10,1505)
(11,2429)
(12,3843)
(13,5986)
(14,8987)
(15,12902)
(16,19127)
(17,26604)
(18,36819)
(19,51028)
(20,68455)
};
\addlegendentry{QPA algoritmus}
\addplot coordinates {
(2,76)
(3,123)
(4,198)
(5,285)
(6,412)
(7,611)
(8,856)
(9,1196)
(10,1716)
(11,2307)
(12,2801)
(13,3712)
(14,4808)
(15,6196)
(16,7892)
(17,9877)
(18,11546)
(19,14637)
(20,17713)
};
\addlegendentry{Náš algoritmus} 
\end{axis}
\end{tikzpicture}}\\
    
Zkusme nyní zafixovat počet vrcholů na počtu 3 a zvyšovat počet hran. 
Toulec $Q$ budeme postupně ve 4 krocích konstruovat následovně:     \\\\
         \centerline{
         \xymatrix{
            \circ^2  \\
            \circ^1 \ar[d] \ar[u]\\
            \circ^3
          }\\\\
         \xymatrix{
            \circ^2  \\
            \circ^1 \ar@/^1pc/[d] \ar@/_1pc/[d]  \ar@/^1pc/[u] \ar@/_1pc/[u]\\
            \circ^3
          }\\\\
         \xymatrix{
            \circ^2  \\
            \circ^1 \ar[d] \ar@/^1pc/[d] \ar@/_1pc/[d] \ar[u] \ar@/^1pc/[u] \ar@/_1pc/[u]\\
            \circ^3
          }\\\\
         \xymatrix{
            \circ^2  \\
            \circ^1 \ar@/^1pc/[d] \ar@/_1pc/[d]  \ar@/^1pc/[u] \ar@/_1pc/[u] \ar@/^2pc/[d] \ar@/_2pc/[d]  \ar@/^2pc/[u] \ar@/_2pc/[u]\\
            \circ^3
          }
          }\\\\\\
    Jako modul $X$ opět jednoduše zvolíme modul $S(1)$. Z následujícího grafu je vidět ještě markatnější rozdíl
    v růstu výpočetního času mezi oběma algoritmy:
    \\\\\\       
\centerline{\begin{tikzpicture}
\begin{axis}[
xlabel={Počet hran vedoucích z vrcholu 1 do každého dalšího},
ylabel={Čas výpočtu [ms]},
grid=major, 
height=12cm,
width=12cm, ]
\addplot coordinates {
(1,52)
(2,253)
(3,4770)
(4,56743)q
};
\addlegendentry{QPA algoritmus}
\addplot coordinates {
(1,119)
(2,267)
(3,750)
(4,2505)
};
\addlegendentry{Náš algoritmus} 
\end{axis}
\end{tikzpicture}}\\
          
                 
  \chapter{Závěr}

\paragraph{ } V této práci jsme prošli konstrukcí algoritmu pro výpočet generátoru skoro štěpitelných posloupností 
od základů dané problematiky až po důkaz jeho správnosti. V následujících částech jsme 
provedli implementaci v teorii reprezentací, uvedli několik příkladů a srovnali 
algoritmus s jiným již implementovaným.

Podívejme se nyní na implementaci algoritmu. Samotný kód je poměrně rozsáhlý a konstrukce složitá, 
ale velké množství pomocných funkcí použitých během výpočtu nám dává mnoho možností pro pozdější
optimalizaci za účelem zvýšení rychlosti. Přesto je již nyní tato implementace v mnoha případech složitějšího
toulce rychlejší než aktuální algoritmus v balíku \cite{QPA}. 

Samotná práce by při svojí podrobnosti měla obsahovat vše potřebné pro pochopení algoritmu i jeho implementace 
a může sloužit jako vstupní brána do teorie reprezentací a systému \cite{GAP4} bez 
jakýchkoliv
předchozích znalostí dané problematiky.
  \thebibliography*{
\addcontentsline{toc}{chapter}{Seznam použité literatury}
\bibitem{1}
  { I. Assem, D. Simson, A. Skowroński:}
  {\em Elements of the representation theory of associative algebras.} \\
  Vol. 1,
  CUP 2006.
  
\bibitem{2}
  { M. Auslander, I. Reiten, S. O. Smalo:}
  {\em Representation theory of Artin algebras.} \\
  CUP 2005.
  
\bibitem{3}
  { Tea Sormbroen Lian:}
  {\em Computing almost split sequences.} \\
  Vol. 1,
  CUP 2006.
  
\bibitem{4}
  { Frank W. Anderson, Kent R. Fuller:}
  {\em Rings and Categories of Modules.} \\
  Second edition, 
  1992.
  
\bibitem{5}
  { J. Rotman:}
  {\em An Introduction To Homological Algebra.} \\
  Number v. 85 in Pure and Applied Mathematics. 
  Academic Press, 1979.
           
\bibitem[GAP]{GAP4}
  The GAP~Group, \emph{GAP -- Groups, Algorithms, and Programming}.; \\
  Version 4.6.4, 2013, \\
  \verb+http://www.gap-system.org+.
           
\bibitem[QPA]{QPA}
  The QPA-Team, \emph{QPA -- Quivers and path algebras}. \\
  Version 1.11, 2013, \\
  \verb+http://www.math.ntnu.no/~oyvinso/QPA/+,\\
  \verb+http://sourceforge.net/projects/quiverspathalg/+.
}
  \chapter*{Přílohy}\addcontentsline{toc}{chapter}{Přílohy}

\begin{description}
  \item[(1)] Kód algoritmu - umístěn na přiloženém kompaktním (CD). 
  \item[(2)] Kód příkladů použití a testování časové náročnosti - umístěn na přiloženém kompaktním (CD). 
\end{description}
\end{document}