\thispagestyle{empty}

\vbox to 0.5\vsize{
\setlength\parindent{0mm}
\setlength\parskip{5mm}

Název práce:
Algoritmy v teorii reprezentací

Autor:
Marek Trunkát

Katedra:  % Případně Ústav:
Katedra algebry
% dle Organizační struktury MFF UK

Vedoucí diplomové práce:
RNDr. Jan Šťovíček, Ph.D., Katedra algebry
% dle Organizační struktury MFF UK, případně plný název pracoviště mimo MFF UK

Abstrakt:
% abstrakt v rozsahu 80-200 slov; nejedná se však o opis zadání diplomové práce
Práce se zabývá implementací algoritmu pro nalezení generátoru
skoro štěpitelných posloupností nerozložitelného a neprojektivního 
modulu algebry cest nad konečným toulcem.
Algoritmus je zde implementován v 
algebraickém systému GAP (Groups, Algorithms, Programming) 
s využitím doplňujícího balíku QPA (Quivers and Path Algebras).

Klíčová slova:
% 3 až 5 klíčových slov
skoro štěpitelné posloupnosti, teorie reprezentací, algoritmus, QPA

\vss}\nobreak\vbox to 0.49\vsize{
\setlength\parindent{0mm}
\setlength\parskip{5mm}

Title:
Algorithms in Representation Theory

Author:
Marek Trunkát

Department:
Department of Algebra
% dle Organizační struktury MFF UK v angličtině

Supervisor:
RNDr. Jan Šťovíček, Ph.D., Department of Algebra
% dle Organizační struktury MFF UK, případně plný název pracoviště
% mimo MFF UK v angličtině

Abstract:
% abstrakt v rozsahu 80-200 slov v angličtině; nejedná se však o překlad
% zadání diplomové práce
This thesis deals with the implementation of an algorithm to compute a generator of almost split sequences which end in an indecomposable nonprojective module of path algebra over finite quiver. The algorithm is implemented in the algebra system GAP (Groups, Algorithms, Programming) with the additional package QPA (Quivers and Path Algebras).

Keywords:
% 3 až 5 klíčových slov v angličtině
almost split sequences, representation theory, algorithm, QPA

\vss}